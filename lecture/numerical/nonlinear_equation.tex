前面在引入复数的时候,我们已经知道如何求解一元二次方程,其解为解析的。
经过数学家们的不断努力,三次方程及四次方程在16世纪中有了解答。
如一元三次方程 
$
a x^3+b x^2+c x+d=0, a \neq 0
$
的解也可解析求得。但是,一般五次及其以上的一元多项式方程无解析解,
称为Abel–Ruffini 定理。也就是说诸如
$x^5 - x -1 = 0$这样的方程也无法解析求出。
%伽罗华利用群论证明了一般,即。
此外,现实情况常常需要我们求解如$ e^x  =\cos x $这样的超越方程。因此,我们需要
依赖数值方法来求解这样的问题。这一部分,我们将主要探讨解决这类问题的方法。

我们以一个三次方程为例,看看我们有什么办法来求解。该方程为
$$
\frac{1}{1+x^2} = x 
$$
由代数基本定理,我们知道它必然有三个解,例如使用WolframAlpha可以得到实数解
为
$$
x=\frac{\sqrt[3]{2(9+\sqrt{93})}-2 \sqrt[3]{\frac{3}{9+\sqrt{93}}}}{6^{2 / 3}}
$$
数值解为
$$
 x \approx 0.6823278038280193273694837。
$$
下面我们将使用试错法来看看是否可以给出比较满意的答案。
我们将试错的过程记录下来做成一个表格方便分析。为此,
方程左侧记为left-hand side (L.H.S.),右侧记为
right hand side (R.H.S.)。
\begin{table}[h]
    \centering
    \begin{tabular}{p{1cm} p{8cm}p{1cm}p{1cm}p{1cm}p{1cm}}
        \hline
        步骤 & 思考 & $x$ & L.H.S. & R.H.S & 差 \\ \hline
        1.& 尝试$x=0$  & 0 & 1.0 & 0.0 &  1.0 \\ \hline
        2.& 不好,试一下$x=1$? & 1 & 0.5 &  1.0 & -0.5 \\ \hline
        3.& 也不是很妙! 但是差或正或负,也许结果在中间?
        不如试一下$x=0.5$
         & 0.5 & 0.8 & 0.5 &  0.3 \\ \hline
        4.& 差更接近了,不错,试一试$x=0.7$
         & 0.7 & 0.671 & 0.7 & −0.029 \\ \hline
        5.& 3和4的结果也有正有负,试一试二者之间插值?
        $x = 0.7-0.2\frac{0.029}{0.3+0.029} \approx 0.6824$  
        & 0.6824 & 0.6823 & 0.6824 & -0.001 \\ \hline
    \end{tabular}
    \caption{A 5-column table example}
    \label{tab:5column}
\end{table}
