\subsection{傅里叶变换}
\label{subsec:fourier_transform}
前面我们讨论了周期性体系的傅里叶级数展开.现在我们希望研究非周期性函数的傅里叶展开问题.
假设$f(x)$是定义在$-\infty < x < \infty$的实函数,我们依旧对该函数进行傅里叶级数展开,
该级数必须理解为周期$2\ell$扩展至$\infty$的极限情况.
由于三角函数族的变量为
\[
 \frac{n\pi x}{\ell}    
\]
引入非连续变量
\[\omega_n = \frac{n\pi} {\ell}, \quad k = 0,1,2,\cdots
    \]
可知$\omega_n = n \omega_0$, 其中$\omega_0 = \frac{\pi}{\ell}$为基础频率(Fundamental Frequency),
\[\Delta \omega_n = \omega_n - \omega_{n-1} = \omega_0 . \] 
有
\begin{equation}
    f(x) = \lim_{\ell\to \infty} \sum_{n=-\infty}^{\infty} \left[
        c_n e^{\imath \omega_n x}
        % \frac{a_0}{2} + \sum_{n=1}^{\infty} \left( a_n \cos \omega_n x + b_n \sin \omega_n x \right)
    \right]
\end{equation}
其中系数
\begin{equation}
    c_n = \frac{1}{2\ell} \int_{-\ell}^{\ell} f(x) e^{-\imath \omega_n x} dx 
\end{equation}
取$\ell\to \infty$的极限,则
\[
  \sum_n \to  \frac{ \ell}{\pi}  \int d\omega 
\]
傅里叶系数记为$F(\omega)$,
有
\begin{equation}
    f(x) = \int_{-\infty}^{\infty} d \omega F(\omega) e^{\imath \omega x}, % \quad F (\omega) \equiv 2\ell c_n
\end{equation}
上式的积分称为\textbf{傅里叶积分}(Fourier integral),是$f(x)$的\textbf{傅里叶变换}(Fourier transform).
其中
\begin{equation}
    F (\omega) \equiv \lim_{\ell\to \infty} \frac{ 2\ell c_n}{2\pi}  = \frac{1}{2\pi} \int_{-\infty}^{\infty} e^{-\imath \omega x} f(x) dx,
\end{equation}
这被称为\textbf{逆傅里叶变换}(inverse Fourier transform). 这里积分前的系数并不要紧,更关键的是积分关于$\omega$的形式.

傅里叶积分存在的条件由\textbf{傅里叶积分定理}判定: 
若函数 $f(x)$ 在区间 $(-\infty, \infty)$ 上满足条件: $(1) f(x)$ 在 任一有限区间上满足狄里希利条件; (2) $f(x)$ 在 $(-\infty, \infty)$ 
上绝对可积 (即 $\int_{-\infty}^{\infty}|f(x)| dx$ 收敛), 则 $f(x)$ 可表示成傅里叶积分, 且
$$
\textrm{傅里叶积分值} = \lim_{\epsilon\to 0} \half \left[ f(x+\epsilon) + f(x-\epsilon) . \right]
$$
% $F (\omega) \equiv 2\ell c_n$.

% \begin{align}
%     a_n =  \lim_{\ell\to \infty} \frac{1}{\ell} \int_{-\ell}^{\ell} f(x) \cos {  \omega_n x  } dx
%     \\
%     b_n = \frac{1}{\ell} \int_{-\ell}^{\ell} f(x) \sin {   \omega_n x } dx
% \end{align}

若采用正余弦展开,

\begin{equation}
    f(x) =\int_{0}^{\infty} A(\omega) \cos {\omega x} d\omega + \int_{0}^{\infty} B(\omega) \sin {\omega x} d\omega
\end{equation}
其中
\begin{equation}
    \left\{\begin{array}{l}
    A(\omega)=\frac{1}{\pi} \int_{-\infty}^{\infty} f(x) \cos \omega x d x, \\
    B(\omega)=\frac{1}{\pi} \int_{-\infty}^{\infty} f(x) \sin \omega x d x .
    \end{array}\right.
\end{equation}

上式还可以写成

\begin{equation}
    f(x)=\int_0^{\infty} C(\omega) \cos [\omega x-\varphi(\omega)] d \omega,
\end{equation}

其中
$$
\begin{gathered}
C(\omega)=\sqrt{[A(\omega)]^2+[B(\omega)]^2} \\
\varphi(\omega)=\arctan [B(\omega) / A(\omega)] .
\end{gathered}
$$
$C(\omega)$ 称为 $f(x)$ 的振幅谱, $\varphi(\omega)$ 称为 $f(x)$ 的相位谱.如果$f(x)$是奇函数或偶函数,
对应的级数为\textbf{傅里叶余弦积分}或\textbf{傅里叶正弦积分}.

复数形式的傅里叶积分可以写成对称的形式,
\begin{equation}
    \begin{aligned}
    f(x) & =\frac{1}{\sqrt{2 \pi}} \int_{-\infty}^{\infty} F(\omega)e^{\imath \omega x} d \omega, \\
    F(\omega) & =\frac{1}{\sqrt{2 \pi}} \int_{-\infty}^{\infty} f(x)\left[e^{\imath \omega x}\right]^* d x .
    \end{aligned}
\end{equation}
并常用符号简写为
$$
F(\omega)=\mathcal{F}[f(x)], \quad f(x)=\mathcal{F}^{-1}[F(\omega)] .
$$
$f(x)$ 和 $F(\omega)$ 分别称为傅里叶变换的\textbf{原函数}和\textbf{像函数}.

\begin{examplebox}{
求以下函数的复数傅里叶变换.
\begin{enumerate}
    \item $f(x) = e^{-\alpha |x|}, \alpha > 0$;
    \item $f(x) = \delta (x)$;
    \item $f(x) = e^{-\alpha x^2}, \alpha > 0$.
\end{enumerate}
}
\begin{enumerate}
    \item 对$x$分段进行处理,利用对称形式的傅里叶变换有
    \begin{align}
        F(\omega) &=
        \sqrt{\frac{1}{2 \pi}} \int_{-\infty}^0 e^{\alpha x+i \omega t} dx \nonumber
        % \\ &
        +
        \sqrt{\frac{1}{2 \pi}} \int_0^{\infty} e^{-\alpha x+i \omega t} dx  \nonumber
        \\
        &= \sqrt{\frac{1}{2 \pi}} \left[\frac{1}{\alpha + \imath \omega } + \frac{1}{\alpha - \imath \omega } \right] \nonumber
        \\
        &= \sqrt{\frac{1}{2 \pi}} \frac{2\alpha}{\alpha^2 + \omega^2}, \nonumber
    \end{align}
    越大的$\alpha$则$f(x)$越窄,集中在$x=0$附近,也就是越局域,而其傅里叶变换则越弥散.当$f(x)$是偶函数时,傅里叶变换的像函数为实函数.

    \item 代入
    \begin{align}
        F(\omega) &=
        \sqrt{\frac{1}{2 \pi}} \int_{-\infty}^{\infty} \delta(x) e^{+\imath \omega x} d x = \sqrt{\frac{1}{2 \pi}} ,
    \end{align}
    $f(x)$无限局域对应的傅里叶变换为无限弥散.

    \item 类似的,  
    \begin{align}
        F(\omega) &=\frac{1}{\sqrt{2 \pi}} \int_{-\infty}^{\infty} e^{-\alpha x^2} e^{\imath \omega x} d x \nonumber
        % \\ &
        % \\
        % & 
        = \frac{ e^{-\omega^2 / 4\alpha }}{\sqrt{2 \pi}} \int_{-\infty}^{\infty} e^{-\alpha  \left( x - \frac{\imath \omega}{2\alpha} \right)^2}  d x
        \nonumber
        % \\ & 
        \\
        & = \frac{ e^{-\omega^2 / 4\alpha }}{\sqrt{2 \pi}} \int_{-\infty - \imath \omega / 2\alpha}^{\infty - \imath \omega / 2\alpha} e^{-\alpha t^2} dt
        \nonumber
        % \\ &
        \\
        & = \frac{ e^{-\omega^2 / 4\alpha }}{\sqrt{2 \pi}} \sqrt{\frac{\pi}{\alpha}} 
        \nonumber
        % \\ &
        \\
            & = \frac{1}{\sqrt{2\alpha}} e^{- \frac{\omega^2}{4\alpha}} \nonumber .
        % \\ &
    \end{align}
    上式用了变量代换$t = x - \imath \omega /2 \alpha$,高斯函数的像函数还是高斯函数.
\end{enumerate}
\end{examplebox}


\subsection{傅里叶变换基本性质}
傅里叶变换满足以下基本性质.假定$f(x)$的傅里叶变换存在,记为$\mathcal{F}[f(x)] = F(\omega)$.

\begin{enumerate}
    \item 导数定理
    \begin{equation}
        \mathcal{F} [f'(x)] = \imath \omega F(\omega)
    \end{equation}

    \item 积分定理
    \begin{equation}
        \mathcal{F} [ \int^{x} f(x) dx ] = \frac{1}{\imath \omega} F(\omega)
    \end{equation}

    \item 相似性定理
    \begin{equation}
        \mathcal{F} [ f(ax) ] = \frac{1}{a} F(\frac{\omega}{a}),
    \end{equation}
    \item 延迟定理
    \begin{equation}
        \mathcal{F} [ f(x - x_0 ) ] = e^{-\imath \omega x_0} F(\omega),
    \end{equation}
    \item 位移定理
    \begin{equation}
        \mathcal{F} [ e^{\imath \omega_0 x} f(x) ] = F(\omega - \omega_0),
    \end{equation}
    \item 卷积定理, 如果
    \[
        \mathcal{F} [f_1(x)] =  F_1(\omega), \mathcal{F} [f_2(x)] =  F_2(\omega), 
    \]
    有
    \begin{equation}
        \mathcal{F} [f_1(x)\star f_2(x) ] = F_1(\omega) F_2(\omega).
    \end{equation}
    其中 $f_1(x) \star f_2(x)=\int_{-\infty}^{\infty} f_1(\alpha) f_2(\alpha-x) d \alpha$ 称为 $f_1(x)$ 与 $f_2(x)$ 的卷积.
\end{enumerate}
以上部分定理作为作业由大家完成.
\subsection{高维傅里叶变换}

二维连续函数 $f(x, y)$ 的傅里叶变换定义如下:
设 $f(x, y)$ 是两个独立变量 $x, y$ 的函数, 且在 $\pm \infty$ 上绝对可积, 则定义积分
$$
F\left(k_1, k_2\right)=\frac{1}{2 \pi} \int_{-\infty}^{\infty} \int_{-\infty}^{\infty} f(x, y) e^{-\imath\left(k_1 x+k_2 y\right)} d x d y
$$
为二维连续函数 $f(x, y)$ 的傅里叶变换,并定义
$$
f(x, y)=\int_{-\infty}^{\infty} \int_{-\infty}^{\infty} F\left(k_1, k_2\right) e^{\imath\left(k_1 x+k_2 y\right)} d k_1 d k_2
$$
为 $F\left(k_1, k_2\right)$ 的逆变换.
$f(x, y)$ 和 $F\left(k_1, k_2\right)$ 称为傅里叶变换对.

\begin{examplebox}{
    求函数 $f(x, y)= \begin{cases}A, & |x| \leq X,|y| \leq Y \\ 0, & |x|>X,|y|>Y\end{cases}$ 的傅里叶变换 (矩孔费琅和夫衍射).
}
由傅里叶变换关系
$$
F\left(k_1, k_2\right)=\frac{1}{(2 \pi)^2} \int_{-\infty}^{\infty} \int_{-\infty}^{\infty} f(x, y) e^{-i\left(k_1 x+k_2 y\right)} d x d y
$$
有
$$
\begin{aligned}
F\left(k_1, k_2\right) & =\frac{A}{2 \pi} \int_{-X}^X e^{-i k_1 x} d x \int_{-Y}^Y e^{-i k_2 y} d y \\
& =\left.\left.\frac{A}{2 \pi} \frac{1}{-i k_1} e^{-i k_1 x}\right|_{-X} ^X \frac{1}{-i k_2} e^{-i k_2 y}\right|_{-Y} ^Y \\
& =\frac{2A}{\pi} \frac{1}{i 2 k_1}\left(e^{i k_1 X}-e^{-i k_1 X}\right) \frac{1}{i 2 k_2}\left(e^{i k_2 Y}-e^{-i k_2 Y}\right) \\
& =\frac{2A X Y}{\pi} \frac{\sin \left(k_1 X\right)}{k_1 X} \frac{\sin \left(k_2 Y\right)}{k_2 Y}
\end{aligned}
$$
\end{examplebox}

对于三维情况,$\vec{k}=\left(k_1, k_2, k_3\right), \vec{r}=(x, y, z)$,
\begin{equation}
    F(\vec{k})=\frac{1}{(2 \pi)^3} \iiint_{-\infty}^{\infty} f(\vec{r}) e^{-\imath  \vec{k} \cdot \vec{r}} d^3 \vec{r} \\
\end{equation}
或者
\begin{equation}
    F\left(k_1, k_2, k_3\right)=\frac{1}{(2 \pi)^3} \iiint_{\infty}^{\infty} f(x, y, z) e^{-\imath\left(k_1 x+k_2 y+k_3 z\right)} d x d y d z 
\end{equation}
逆变换为
\begin{equation}
     f(\vec{r})=\iiint_{-\infty}^{\infty} F(k) e^{\imath \vec{k} \cdot \vec{r}} d^3 \vec{k} 
\end{equation}  
或
\begin{equation}
f(x, y, z)=\iiint_{-\infty}^{\infty} F\left(k_1, k_2, k_3\right) e^{\imath\left(k_1 x+k_2 y+k_3 z\right)} d k_1 d k_2 d k_3 
\end{equation} 