\documentclass[
    %twoside,
]{article}
\usepackage[
% paperheight=16cm, 
% paperwidth=12cm,% Set the height and width of the paper
% includehead,
% nomarginpar,% We don't want any margin paragraphs
% % textwidth=10cm,% Set \textwidth to 10cm
% headheight=14pt,% Set \headheight to 10mm
]{geometry}
\usepackage{ctex}
\usepackage{xcolor}

\usepackage{graphicx}% Include figure files
\usepackage{braket}% Dirac notations
\usepackage{enumitem}
\usepackage{fancyhdr}
\usepackage{fontspec}
\usepackage{titlesec}
\usepackage{titletoc}  
\usepackage{fourier-orns}

\renewcommand{\headrule}{%
\vspace{-8pt}\hrulefill
\raisebox{-2.1pt}{\quad\decofourleft\decotwo\decofourright\quad}\hrulefill}

\begin{document}

\pagestyle{fancy}
% \fancyhf{}
% \fancyhead[LE]{\nouppercase{\rightmark\hfill\leftmark}}
% \fancyhead[RO]{\nouppercase{\leftmark\hfill\rightmark}}

%... then configure it.
\fancyhead{} % clear all header fields
% \fancyhead[LE]{\textbf{PHYS}}
\fancyhead[RO]{\textbf{Many-Body Physics}}
% \fancyhead[LO]{\textbf{Kai Luo}}
% \fancyhead[LO]{罗凯}

% \fancyhead[LE]{\nouppercase{\rightmark\hfill\leftmark}}
% \fancyhead[RO]{\nouppercase{\leftmark\hfill\rightmark}}

\fancyfoot{} % clear all footer fields
\fancyfoot[CE,CO]{\thepage}
% \fancyfoot[LO,CE]{}
% \fancyfoot[CO,RE]{Nanjing University of Science and Technology}
% \fancyfoot[CO,RE]{\thepage}
% \fancyfoot[CE,O]{\thechapter}


% This is page 1.\newpage
% This is page 2.

% \section{围棋的棋具}
% \begin{enumerate}
%     \item 棋盘。\\
%     棋盘由纵横各19条等距离、垂直交叉的平行线构成。形成361个交叉点,简称为“点”。\\
%     棋盘整体形状以及每个格子纵、横向相比,横向稍短,通常每格分别为2.4厘米:2.3厘米。\\
%     在棋盘上标有9个小圆点,称作“星”。中央的星又称“天元”(见图1)。
%     \item 棋子。\\
%     棋子分黑白两色,形状为扁圆形体。\\
%     棋子的数量应能保证顺利终局。正式比赛以黑、白各180子为宜。
% \end{enumerate}
% \section{围棋的棋具}
% \begin{enumerate}
%     \item 棋盘。\\
%     棋盘由纵横各19条等距离、垂直交叉的平行线构成。形成361个交叉点,简称为“点”。\\
%     棋盘整体形状以及每个格子纵、横向相比,横向稍短,通常每格分别为2.4厘米:2.3厘米。\\
%     在棋盘上标有9个小圆点,称作“星”。中央的星又称“天元”(见图1)。
%     \item 棋子。\\
%     棋子分黑白两色,形状为扁圆形体。\\
%     棋子的数量应能保证顺利终局。正式比赛以黑、白各180子为宜。
% \end{enumerate}\section{围棋的棋具}
% \begin{enumerate}
%     \item 棋盘。\\
%     棋盘由纵横各19条等距离、垂直交叉的平行线构成。形成361个交叉点,简称为“点”。\\
%     棋盘整体形状以及每个格子纵、横向相比,横向稍短,通常每格分别为2.4厘米:2.3厘米。\\
%     在棋盘上标有9个小圆点,称作“星”。中央的星又称“天元”(见图1)。
%     \item 棋子。\\
%     棋子分黑白两色,形状为扁圆形体。\\
%     棋子的数量应能保证顺利终局。正式比赛以黑、白各180子为宜。
% \end{enumerate}\section{围棋的棋具}
% \begin{enumerate}
%     \item 棋盘。\\
%     棋盘由纵横各19条等距离、垂直交叉的平行线构成。形成361个交叉点,简称为“点”。\\
%     棋盘整体形状以及每个格子纵、横向相比,横向稍短,通常每格分别为2.4厘米:2.3厘米。\\
%     在棋盘上标有9个小圆点,称作“星”。中央的星又称“天元”(见图1)。
%     \item 棋子。\\
%     棋子分黑白两色,形状为扁圆形体。\\
%     棋子的数量应能保证顺利终局。正式比赛以黑、白各180子为宜。
% \end{enumerate}\section{围棋的棋具}
% \begin{enumerate}
%     \item 棋盘。\\
%     棋盘由纵横各19条等距离、垂直交叉的平行线构成。形成361个交叉点,简称为“点”。\\
%     棋盘整体形状以及每个格子纵、横向相比,横向稍短,通常每格分别为2.4厘米:2.3厘米。\\
%     在棋盘上标有9个小圆点,称作“星”。中央的星又称“天元”(见图1)。
%     \item 棋子。\\
%     棋子分黑白两色,形状为扁圆形体。\\
%     棋子的数量应能保证顺利终局。正式比赛以黑、白各180子为宜。
% \end{enumerate}
\chapter{Introduction}
\label{ch:intro}


\section{Many-body problem}
\chapter{Second Quantization}
\label{ch:2ndquan}


\section{Quantum Mechanics Review}
\label{sec:qmreview}
\chapter{Green's Function}
\label{ch:gf}


\section{Green's Function}

\end{document}