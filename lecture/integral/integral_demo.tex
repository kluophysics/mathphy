\section{积分求解实例}

\begin{itemize}
    \item 计算定积分 
    $$
    I=\int_0^{\infty} x^{\alpha-1} \frac{1}{1+x} d x \quad(0<\alpha<1).
    $$

    将被积函数 $x^{\alpha-1} /(1+x)$ 从实轴延拓到复数 $z$ 平面得到 
    $f(z)=z^{\alpha-1} /(1+z)$. 由于 $f(z)$ 含 有 $z^{\alpha-1}$ ,
    而 $\alpha$ 不是整数,所以 $f(z)$ 是多值函数, 它有两个支点:原点和无限远点. 
    $z$ 每绕原点或 无限远点一圈, 辐角增加 $2 \pi, z^{\alpha-1}$ 多出因子 
    $e^{\imath 2 \pi(\alpha-1)}$ 亦即 $e^{\imath 2 \pi \alpha}$,
     从而 $f(z)$ 也多出这么一个 因子.
    从原点沿着正实轴直至无限远作割线.   
    
    $$
\oint_l f(z) d z=\int_{\epsilon}^R \frac{x^{\alpha-1}}{1+x} d x+\int_{C_R} f(z) d z+\int_R^{\epsilon} \frac{x^{\alpha-1} e^{\imath 2 \pi \alpha}}{1+x} d x+\int_{C_{\epsilon}} f(z) d z .
$$
令 $R \rightarrow \infty, \epsilon \rightarrow 0$.
 上式左边按照留数定理应为 $2 \pi i\{f(z)$ 在有限远各奇点留数之和\}
  右边第一个积分成为所求的 $I$, 第三个积分则成为 $-e^{\imath 2 \pi \alpha} I$. 
  可以证明第 二个和第四个积分则成为零. 事实上,
  $$
\begin{aligned}
\left|\int_{C_R} \frac{z^{\alpha-1}}{1+z} d z\right| & =\left|\int_{C_R} \frac{z^\alpha}{1+z} \frac{d z}{z}\right| \leqslant \max _{\left(C_R \text { 上 }\right)}\left|\frac{z^\alpha}{1+z}\right| \frac{\int|d z|}{|z|} \\
= & \max \frac{R^\alpha}{|1+z|} \cdot \frac{2 \pi R}{R}=2 \pi \max \frac{R^\alpha}{|1+z|} \\
& \sim 2 \pi \frac{1}{R^{1-\alpha} \rightarrow 0} \quad(\text { 于 } R \rightarrow \infty) .
\end{aligned}
$$

$$
\begin{aligned}
\left|\int_{C_{\epsilon}} \frac{z^{\alpha-1}}{1+z} d z\right|= & \left|\int_{C_{\epsilon}} \frac{z^\alpha}{1+z} \frac{d z}{z}\right| \leqslant \max _{\left(C_{\epsilon} \text { 上 }\right)}\left|\frac{z^\alpha}{1+z}\right| \frac{\int|d z|}{|z|} \\
= & \max \frac{\epsilon^\alpha}{|1+z|} \cdot \frac{2 \pi \epsilon}{\epsilon}=2 \pi \max \frac{\epsilon^\alpha}{|1+z|} \\
& \sim 2 \pi \frac{\epsilon^\alpha}{1} \rightarrow 0 \quad(\text { 于 } \epsilon \rightarrow 0) .
\end{aligned}
$$
于是 $\left(1-e^{\imath 2 \pi \alpha}\right) I=2 \pi \imath\{f(z)$ 在有限远各奇点留数之和 $\}$.
$f(z)=z^{\alpha-1}(1+z)^{-1}$ 只有一个单极点 $z_0=-1=e^{\imath \pi}$, 而
$$
\Res f(-1)=\lim _{z \rightarrow-1}[(z+1) f(z)]=\lim _{z \rightarrow-1}\left[z^{\alpha-1}\right]=e^{\imath(\alpha \pi-\pi)}=-e^{\imath \alpha \pi} \text {. }
$$
因此
$$
\begin{aligned}
I & =-\frac{2 \pi \imath e^{\imath \pi \alpha}}{1-e^{\imath 2 \pi \alpha}}=-\frac{2 \pi \imath e^{\imath \pi \alpha}}{e^{\imath \pi \alpha}\left(e^{-\imath \pi \alpha}-e^{\imath \pi \alpha}\right)} \\
& =\frac{2 \pi \imath}{\left(e^{-\imath \pi \alpha}-e^{\imath \pi \alpha}\right)}=\frac{2 \pi \imath}{2 \sin \pi \alpha}=\frac{\pi}{\sin \pi \alpha} .
\end{aligned}
$$

    \item 计算菲涅耳积分(Fresnel integrals)
    $$
    I_1=\int_0^{\infty} \sin \left(x^2\right) d x \text { 及 } I_2=\int_0^{\infty} \cos \left(x^2\right) d x \text {. }
    $$

    由于 $\sin \left(x^2\right)=\operatorname{Im} e^{\imath x^2}$, 而 $\cos \left(x^2\right)=\Re e^{\imath x^2}$, 所以
$$
I_2+\imath I_1=\int_0^{\infty} e^{\imath x^2} d x .
$$
取图 4-10 所示回路 $l$. 由于 $e^{\mathrm{iz}^2}$ 没有有限远奇点, 所以根据留数定理得
$$
\oint_l e^{\imath z^2} d z=0,
$$
即 $\int_0^R e^{\imath x^2} d x+\int_{C_R} e^{\imath z^2} d z+\int_R^0 e^{\imath\left(\rho e^{\imath \pi / 4}\right)^2} d\left(\rho e^{\imath \pi / 4}\right)=0$,

令 $R \rightarrow \infty$. 第一个积分即所求的 $I_2+\imath I_1$. 第三个积分不难如下算出 :
$$
\begin{aligned}
\lim _{R \rightarrow \infty} \int_R^0 e^{\imath\left(\rho^2 \imath\right)} e^{\imath \pi / 4} d \rho & =\lim _{R \rightarrow \infty}\left(-e^{\imath \pi / 4}\right) \int_0^R e^{-\rho^2} d \rho=-e^{\imath \pi / 4} \int_0^{\infty} e^{-\rho^2} d \rho \\
& =-\frac{\sqrt{\pi}}{2} e^{\imath \pi / 4}=-(1+\imath) \sqrt{\frac{\pi}{8}} .
\end{aligned}
$$
可以证明第二个积分成为零. 为此, 先作一次分部积分,
$$
\int_{C_R} e^{\imath z^2} d z=\left.\frac{e^{\imath z^2}}{2 \imath z}\right|_{z=R} ^{R e^{\imath \pi / 4}}+\int_{C_R} e^{\imath z^2} \frac{d z}{2 \imath z^2},
$$
其中已积出部分的模
$$
\left|\frac{e^{-R^2}}{2 \imath R e^{\imath \pi / 4}}-\frac{e^{\imath R^2}}{2 \imath R}\right| \leqslant \frac{e^{-R^2}}{2 R}+\frac{1}{2 R} \rightarrow 0 \quad(\text { 于 } R \rightarrow \infty),
$$
未积出部分的模
$$
\begin{aligned}
\left|\int_{C_R} \frac{e^{\imath 2^2}}{2 \imath {z}^2} d z\right| & =\left|\int_{C_R} \frac{e^{-R^2 \sin 2 \varphi+\imath^2 \cos 2 \varphi}}{2 \imath R^2 e^{\imath 2 \varphi}} R e^{\imath \varphi} \imath d \varphi\right| \\
& \leqslant \int_{C_R} \frac{e^{-R^2 \sin 2 \varphi}}{2 R^2} R d \varphi \leqslant \max \left(\frac{e^{-R^2 \sin 2 \varphi}}{2 R}\right) \frac{\pi}{4}
\\
&=\frac{1}{2 R} \frac{\pi}{4} \rightarrow 0 \quad(\text { 于 } R \rightarrow \infty) .
\end{aligned}
$$
于是
$$
\begin{gathered}
I_2+\imath I_1-\sqrt{\frac{\pi}{8}}(1+\imath)=0, \\
I_1=\sqrt{\frac{\pi}{8}}, \quad I_2=\sqrt{\frac{\pi}{8}} .
\end{gathered}
$$
\item 考虑用Feynman技巧来计算
$$
I = \int_0^1 \frac{x^2-1}{\log x} d x
$$
设计这样的函数
$$
G(t):=\int_0^1 \frac{x^t-1}{\log x} d x
$$
$G(0) = 0$,于是问题变为求$G(2)$.不难验证
$$
G^{\prime}(t)=\int_0^1 x^t d x=\frac{1}{t+1}
$$
对$t$求积分后得到
$$
G(2)=\int_0^2 G^{\prime}(t) d t=\int_0^2 \frac{d t}{t+1}=\log 3.
$$

\item 考虑用Feynman技巧来验证高斯积分:
$$
\int_0^{\infty} e^{-x^2} d x = \frac{\sqrt{\pi}}{2}.
$$
\\
定义一个$t$的函数
$$
I(t):=\int_0^{\infty} \frac{e^{-x^2}}{1+(x / t)^2} d x, t>0. 
$$
于是菲涅尔积分就是求$I_1 = I(\infty)$.
做代换$x/t=y$后,并换回积分变量为$x$有 $$
I(t)=t \int_0^{\infty} \frac{e^{-t^2 x^2}}{1+x^2} d x
$$
可以得到 $$
\lim _{t \rightarrow 0} \frac{I(t)}{t}=\frac{\pi}{2}.
$$
为了能利用上面的技巧,我们将考虑
$$
e^{-t^2} I(t)=t \int_0^{\infty} \frac{e^{-t^2\left(1+x^2\right)}}{1+x^2} d x
$$
对于
$$
\frac{d}{d t}\left(t^{-1} e^{-t^2} I(t)\right)=\int_0^{\infty}-2 t e^{-t^2\left(1+x^2\right)} d x=-2 e^{-t^2} \int_0^{\infty} e^{-u^2} d u=-2 e^{-t^2} I(\infty)
$$

两边积分
$$
\underbrace{\int_0^{\infty} \frac{d}{d t}\left(t^{-1} e^{-t^2} I(t)\right) d t}_{=-\lim _{t \rightarrow 0} \frac{I(t)}{t}}=\underbrace{\int_0^{\infty}-2 e^{-t^2} I(\infty) d t}_{=-2 I(\infty)^2}
$$
得到
$$
I(\infty) = \sqrt{\frac{\pi}{4}} = \frac{\sqrt{\pi}}{2}.
$$

\end{itemize}


% \textbf{加分! 利用Feynman技巧完成  
% $$
% I=\int_0^{\infty} x^{\alpha-1} \frac{1}{1+x} d x \quad(0<\alpha<1).
% $$
% 该积分计算的, 最终成绩可以加{\color{red}{5分}}, 并可以共同发{\color{red}{一篇学术论文}}, 欢迎大家尝试.}