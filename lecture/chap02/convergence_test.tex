\subsection{级数的收敛判定法}

\subsubsection{柯西判据}
柯西判据(Cauchy criterion)说的是,对于$\epsilon>0$, 总存在固定的$N$使得$|s_j - s_i|< \epsilon$, 其中$i,j$是任意大于$N$的整数。也就是说,若$j>i$,
\begin{equation}
    | w_{i+1} + w_{i+2} + \cdots + w_{j} | < \epsilon ,
\end{equation}
直观地理解,也就是说某项以后的所有求和可以忽略不计,即部分求和趋于某一值,级数收敛。对于复数项级数,一样成立。如果将复数项级数的每一项都取模组成新的级数,
记为
\begin{equation}
    \sum_{k=1} |w_k| = \sum_{k=1}\sqrt { u_k^2 + v_k^2},
\end{equation}
若该级数收敛,则称原级数\textbf{绝对收敛}。绝对收敛的级数必然是收敛的。如果级数收敛,但非绝对收敛,那么我们称它为\textbf{条件收敛}。
\subsubsection{比较判定法}
如果我们有某一已知正项级数 $\sum_k a_k$收敛,若级数的每一项都满足 $0 \leq w_k \leq a_k$, 那么可以判定$\sum_k w_k$收敛。
这可以利用柯西判据进行证明。相反的, 若同发散级数$\sum_{k} a_k$比较有,$0 \leq a_k  \leq w_k$, 那么可以判定$\sum_k w_k$发散。
对于复数项级数,比较判据可以表示为$0 \leq |w_k| \leq a_k$。

\begin{examplebox}{试判定级数$\sum_{k=1} k^{-p}, p\leq 1$收敛还是发散。}
    已知调和级数$\sum_{k=1}1/k$是发散的,而 $k^{-p} > k^{-1}$, 根据比较判定法,可知该级数发散。
\end{examplebox}

\subsubsection{达朗贝尔判据(D'Alembert Ratio Criterion)}
% 级数$\sum_{n} w_n$,对足够大的$N$和与$N$无关的常数$r$,若满足条件$w_{n+1} / w_{n} \leq r < 1$,那么该级数收敛。反之,若$w_{n+1} / w_{n} \geq >= 1$,
% ,该级数发散。
该方法又称比值判定法。
若任意项级数 $\sum_{n=1}^{\infty} w_n$ 通项满足:
$$
\lim _{n \to \infty}\left|\frac{w_{n+1}}{w_n}\right|=q ,
$$
\begin{enumerate}
    \item 当 $q<1$ 时, 级数绝对收敛;
    \item 当 $q>1$ 时, 级数发散;
    \item 当 $q=1$ 时, 此方法无效,需要其他方法判定。
\end{enumerate}

\begin{examplebox}{试判断级数$\sum_{k=1} k/2^k$的收敛性。}
    利用比值判定法
    \[ 
        \lim_{k\to \infty} \frac{w_{k+1}}{w_k}=\lim_{k\to \infty} \frac{k+1}{2^{k+1}} / \frac{k}{2^{k}}=\frac{1}{2} 
        = \lim_{k\to \infty} \frac{1}{2}\left(1+\frac{1}{k}\right) 
        = \half < 1
    \]
    可见该级数是收敛的。
\end{examplebox}
% $\frac{w_{k+1}}{w_k}=r , r 5 k$ 死关,
% 世東 $r<1$ ,那昍数收斂。
% 当 $r=1$ ,榇弅
% $$
% \frac{a_{k+1}}{a_k}=\frac{k}{k+1}<
% $$
% $$
% \begin{gathered}
% \frac{w_{k+1}}{w_k}=\frac{k+1}{2^{k+1}} / \frac{k}{2^{k+}}=\frac{1}{2} \frac{k+1}{k} \Rightarrow \frac{1}{2}<1 \\
% =\frac{1}{2}\left(1+\frac{1}{k}\right) \leqslant \frac{3}{4} \quad(k \geqslant 2)
% \end{gathered}
% $$

% \subsubsection{莱布尼兹判据(Leibniz Criterion)}
此外,还有很多其他判定方法,如对于交错级数可以利用莱布尼兹判据(Leibniz Criterion)。
% \begin{enumerate}
%     \item 
% \end{enumerate}
