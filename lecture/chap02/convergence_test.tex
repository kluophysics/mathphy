\subsection{级数的收敛判据}

\subsubsection{柯西判据}
柯西判据(Cauchy criterion)说的是,对于$\epsilon>0$, 总存在固定的$N$使得$|s_j - s_i|< \epsilon$, 其中$i,j$是任意大于$N$的整数。也就是说,若$j>i$,
\begin{equation}
    | w_{i+1} + w_{i+2} + \cdots + w_{j} | < \epsilon ,
\end{equation}
直观地理解,也就是说某项以后的所有求和可以忽略不计,即部分求和趋于某一值,级数收敛。对于复数项级数,一样成立。如果将复数项级数的每一项都取模组成新的级数,
记为
\begin{equation}
    \sum_{k=1} |w_k| = \sum_{k=1}\sqrt { u_k^2 + v_k^2},
\end{equation}
若该级数收敛,则称原级数\textbf{绝对收敛}。绝对收敛的级数必然是收敛的。如果级数收敛,但非绝对收敛,那么我们称它为\textbf{条件收敛}。
\subsubsection{比较判据}
如果我们有某一已知正项级数 $\sum_k a_k$收敛,若级数的每一项都满足 $0 \leq w_k \leq a_k$, 那么可以判定$\sum_k w_k$收敛。
这可以利用柯西判据进行证明。相反的, 若同发散级数$\sum_{k} a_k$比较有,$0 \leq a_k  \leq w_k$, 那么可以判定$\sum_k w_k$发散。
对于复数项级数,比较判据可以表示为$0 \leq |w_k| \leq a_k$。

\begin{examplebox}{试判定级数$\sum_{k=1} k^{-p}, p\leq 1$收敛还是发散。}
    
\end{examplebox}

\subsubsection{达朗贝尔判据(D'Alembert Ratio Test)}

\subsubsection{莱布尼兹判据(Leibniz Criterion)}

\begin{enumerate}
    \item 
\end{enumerate}
