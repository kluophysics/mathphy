\subsection{级数的基本性质}
物理学和工程学中的一些函数常常可以用无穷级数来表示。一个很有用的例子,
\begin{equation}
    1+ x + x^2 + x^3 + \cdots = \frac{1}{1-x} \textrm{。}
\end{equation}
有了该式,我们可以处理更复杂的级数,如
\begin{equation}
    1 + a \cos \theta + a^2 \cos 2\theta + a^3 \cos 3\theta + \cdots = ? 
\end{equation}
有了前面的复数概念,我们有
\begin{equation}
    \cos \theta = \Re e^{\imath \theta} = \frac{e^{\imath \theta} +e^{-\imath \theta} }{2} \textrm{。}
\end{equation}
于是,
\begin{align*}
   & 1 + a \cos \theta + a^2 \cos 2\theta + a^3 \cos 3\theta + \cdots 
    \\  
 = &  1 + \half a e^{\imath \theta} + \half a e^{- \imath \theta} + \half a^2 e^{2\imath \theta} + \half a^2 e^{-2\imath \theta}  + \cdots 
 \\  
 =   & \half \left( 1 + a e^{\imath \theta} + a^2 e^{2\imath \theta} + \cdots \right)  
+ \half \left( 1 + a e^{- \imath \theta} + a^2 e^{-2\imath \theta}  + \cdots \right) 
\\  
= &  \half \frac{1}{1 - a e^{\imath \theta} } + \half \frac{1}{1 - a e^{-\imath \theta} }
\\  
= &  \half \left[ \frac{1-a\cos\theta + \imath a \sin \theta }{(1-a\cos\theta)^2 + (a\sin \theta)^2} + \frac{1-a\cos\theta - \imath a \sin \theta}{(1-a\cos\theta)^2 + (a\sin \theta)^2} \right]
 \\  
= &  \frac{1-a\cos\theta}{1-2 a \cos\theta + a^2} \textrm{。}
\end{align*}
现在我们要处理一个稍微复杂的级数,
\begin{equation}
    S(x) = 1 - \frac{x}{2} + \frac{x^2}{3} - \frac{x^3}{4} + \cdots
\end{equation}
为了将上式化简,我们需要把它转换成一个我们熟知的形式。由于分母比较特殊,我们想办法摆脱这些数字。为此,我们对$x S(x)$求导,得到
\begin{align}
    \frac{d}{dx} ( x S(x))  &=   \frac{d}{dx} \left( x- \frac{x^2}{2} + \frac{x^3}{3} - \frac{x^4}{4} \right)
    \nonumber \\ 
    &= 1 - x^2 + x^3 - x^4 + \cdots 
    \nonumber \\ 
    & = \frac{1}{1+x} \,
\end{align}
于是我们有,$
    x S(x) = \ln (1 + x) + C \textrm{。}$
当$x=0$, $S(x) = 1, \ln (1+x) = 0$, 有$C=0$。最终我们得到了
\begin{equation}
    \ln (1+x) = x -  \frac{x^2}{2} + \frac{x^3}{3} - \frac{x^4}{4} = \sum_{k=1}^{\infty} \frac{(-1)^{k+1} x^k}{k} \textrm{。}
\end{equation}
我们还可以利用指数的级数表示
\begin{equation}
    e^{x} = \sum_{n=0}^{\infty} \frac{x^n}{n!} = 1 + \frac{x}{1} + \frac{x^2}{2} + \frac{x^3}{3} + \cdots 
\end{equation}
来求解正余弦函数的级数表示。
\begin{enumerate}
    \item 几何级数: $ 1 + x + x^2 + x^3 + \cdots + x^n$, 对$|x|<1$,收敛于$\frac{1}{1-x}$。
    \item 调和级数: $ 1 + \half + \frac{1}{3} + \cdot +\frac{1}{k} = \sum_{k=1} \frac{1}{k}$,发散。
\end{enumerate}

值得注意的是,以上的求和表示实际上假定了$|x|<1$这个条件。容易验证,$|x|\geq 1$,级数是发散的。一般的,我们将复数的概念拓展到级数,
\begin{equation}
    s_n = \sum_{k=1}^{n} w_{k}
\end{equation}
若级数通项每一个都是函数形式$w_k(z)$,我们称之为\textbf{函数项级数}。
随着,$n\to \infty$,部分求和趋于一个定值$S$,即
\begin{equation}
    \lim_{n\to \infty} s_n = S ,
\end{equation}
我们称无穷级数 $\sum_{k=1}^{n} w_{k}${\bf 收敛}并趋于$S$。如果级数的求和趋于$\pm \infty$,级数{\bf 发散}。对于如
\begin{equation}
    \sum_{k=1}^{\infty} (-1)^k = 1 - 1 + 1 - 1 \cdots 
\end{equation}
这样的级数取值在$\pm 1$之间振荡,我们也称其发散。级数收敛的必要条件很显然是$\lim_{k\to \infty} w_k = 0$。对于级数是否收敛,在什么条件下收敛显得十分重要。
对级数收敛的充分条件寻找,派生出了各种各样的判据。



