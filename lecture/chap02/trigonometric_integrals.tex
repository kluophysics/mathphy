\section{三角函数的积分}
考虑积分区间为$\left[ 0, 2\pi \right]$,被积函数为三角函数有理式的积分
\begin{equation}
    \int_{0}^{2\pi} R(\cos{x}, \sin{x}) dx,
\end{equation}
当实变数 $x$ 从 0 变到 $2 \pi$ 时, 复变数 $z=\mathrm{e}^{\mathrm{i} x}$ 从 $z=1$ 出发沿单位圆 $|z|=1$ 逆时针 走一圈又回到 $z=1$,
实变定积分化为复变回路积分, 就可以应用留数定理了. 至于实变定积分里的 $\cos x, \sin x$ 和 $\mathrm{d} x$, 作如下变换:
$$
\cos x=\frac{1}{2}\left(z+z^{-1}\right), \quad \sin x=\frac{1}{2 \mathrm{i}}\left(z-z^{-1}\right), \quad \mathrm{d} x=\frac{1}{\mathrm{i} z} \mathrm{~d} z .
$$
于是, 原积分化为
$$
I=\oint_{|z|=1} R\left(\frac{z+z^{-1}}{2}, \frac{z-z^{-1}}{2 \mathrm{i}}\right) \frac{\mathrm{d} z}{\mathrm{i} z}
$$
利用留数定理即可求得。

\begin{examplebox}{求定积分\[I=\int_0^{2 \pi} \frac{d \theta}{1+a \cos \theta}, \quad|a|<1 \]}
    根据上面的方法,可得
    \[
        \begin{aligned}
        I & =-\imath \oint_{|z|=1} \frac{d z}{z\left[1+(a / 2)\left(z+z^{-1}\right)\right]} \\
        & =-\imath \frac{2}{a} \oint \frac{d z}{z^2+(2 / a) z+1} .
        \end{aligned}
    \]
    两个极点分别为
    \[
        z_1=-\frac{1+\sqrt{1-a^2}}{a} \quad \text {和} \quad z_2=-\frac{1-\sqrt{1-a^2}}{a}
    \]
    不难看出,$z_1$在单位圆外,$z_2$在单位圆内。积分可以写成
    \[
      \oint \frac{dz}{(z-z_1)(z-z_2)}   
    \]
    留数则为$\frac{1}{z_2 - z_1}$, 利用留数定理可得 
    \[
      I=  -\imath \frac{2}{a} 2\pi\imath \frac{1}{z_2 - z_1} = \frac{2}{\sqrt{1 - a^2}} . 
    \]
\end{examplebox}




