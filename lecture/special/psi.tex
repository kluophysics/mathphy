\section{$\psi$函数}
$\psi$ 函数是 $\Gamma$ 函数的对数微商
\begin{equation}
    \psi(z)=\frac{d \ln \Gamma(z)}{d z}=\frac{\Gamma^{\prime}(z)}{\Gamma(z)}
\end{equation}
根据 $\Gamma$ 函数的性质,可以得出 $\psi(z)$ 的下列性质:

\begin{enumerate}
    \item $z=0,-1,-2, \cdots$ 都是 $\psi(z)$ 的一阶极点, 留数均为 -1 ; 除了这些点以外, $\psi(z)$ 在全平面解析.
    \item $$
    \begin{aligned}
    \psi(z+1) & =\psi(z)+\frac{1}{z} \\
    \psi(z+n) & =\psi(z)+\frac{1}{z}+\frac{1}{z+1}+\cdots+\frac{1}{z+n-1}, \quad n=2,3, \cdots
    \end{aligned}
    $$
    \item   $\psi(1-z)=\psi(z)+\pi \cot \pi z$.
    \item  $\psi(z)-\psi(-z)=-\frac{1}{z}-\pi \cot \pi z$.
    \item  $\psi(2 z)=\frac{1}{2} \psi(z)+\frac{1}{2} \psi\left(z+\frac{1}{2}\right)+\ln 2$.
    \item  $\lim _{n \rightarrow \infty}[\psi(z+n)-\ln n]=0$.
\end{enumerate}

$\psi$ 函数的特殊值有
$$
\begin{array}{ll}
\psi(1)=-\gamma, & \psi^{\prime}(1)=\frac{\pi^{2}}{6}, \\
\psi\left(\frac{1}{2}\right)=-\gamma-2 \ln 2, & \psi^{\prime}\left(\frac{1}{2}\right)=\frac{\pi^{2}}{2} \\
\psi\left(-\frac{1}{2}\right)=-\gamma-2 \ln 2+2, & \psi^{\prime}\left(-\frac{1}{2}\right)=\frac{\pi^{2}}{2}+4, \\
\psi\left(\frac{1}{4}\right)=-\gamma-\frac{\pi}{2}-3 \ln 2, & \psi\left(\frac{3}{4}\right)=-\gamma+\frac{\pi}{2}-3 \ln 2, \\
\psi\left(\frac{1}{3}\right)=-\gamma-\frac{\pi}{2 \sqrt{3}}-\frac{3}{2} \ln 3, & \psi\left(\frac{2}{3}\right)=-\gamma+\frac{\pi}{2 \sqrt{3}}-\frac{3}{2} \ln 3 .
\end{array}
$$
其中 $\gamma=-\psi(1)$ 是数学中的一个基本常数, 称为 Euler 常数.


利用 $\psi$ 函数, 可以方便地求出通项为有理式的无穷级数
$$
\sum_{n=0}^{\infty} u_{n}=\sum_{n=0}^{\infty} \frac{p(n)}{d(n)}
$$
之和, 其中 $p(n)$ 和 $d(n)$ 都是 $n$ 的多项式. 为了保证级数收敛, $p(n)$的次数至少要比 $d(n)$ 的次数低 2 , 即

$$
\lim _{n \rightarrow \infty} u_{n}=\lim _{n \rightarrow \infty} n \cdot u_{n}=0
$$

如果 $d(n)$ 是 $n$ 的 $m$ 次多项式, 并且全部零点都是一阶零点,

$$
d(n)=\left(n+\alpha_{1}\right)\left(n+\alpha_{2}\right) \cdots\left(n+\alpha_{m}\right)
$$

即 $u_{n}$ 只有一阶极点, 则可部分分式为

$$
u_{n}=\frac{p(n)}{d(n)}=\sum_{k=1}^{m} \frac{a_{k}}{n+a_{k}}
$$

利用 $\psi$ 函数的递推关系, 即可求得

$$
\begin{aligned}
\sum_{n=0}^{N} u_{n} & =\sum_{k=1}^{m} a_{k}\left[\psi\left(\alpha_{k}+N\right)-\psi\left(\alpha_{k}\right)\right] \\
& =\sum_{k=1}^{m} a_{k}\left[\psi\left(\alpha_{k}+N\right)-\ln N-\psi\left(\alpha_{k}\right)\right]
\end{aligned}
$$

其中利用了 $\sum_{k=1}^{m} a_{k}=0$. 取极限 $N \rightarrow \infty$, 即得

$$
\begin{aligned}
\sum_{n=0}^{\infty} u_{n} & =\lim _{N \rightarrow \infty} \sum_{k=1}^{m} a_{k}\left[\psi\left(\alpha_{k}+N\right)-\ln N-\psi\left(\alpha_{k}\right)\right] \\
& =\lim _{N \rightarrow \infty} \sum_{k=1}^{m} a_{k}\left[\psi\left(\alpha_{k}+N\right)-\ln N\right]-\sum_{k=:}^{m} a_{k} \psi\left(\alpha_{k}\right) \\
& =-\sum_{k=1}^{m} a_{k} \psi\left(\alpha_{k}\right)
\end{aligned}
$$

\begin{example}
求无穷级数 $\sum_{n=0}^{\infty} \frac{1}{(3 n+1)(3 n+2)(3 n+3)}$ 之和.
\end{example}
\begin{solution}
因为
$$
\frac{1}{(3 n+1)(3 n+2)(3 n+3)}=\frac{1}{6} \frac{1}{n+i / 3}-\frac{1}{3} \frac{1}{n+2 / 3}+\frac{1}{6} \frac{1}{n+1}
$$

所以, 根据上面给出的求和公式, 有

$$
\sum_{n=0}^{\infty} \frac{1}{(3 n+1)(3 n+2)(3 n+3)}=-\frac{1}{6}\left[\psi\left(\frac{1}{3}\right)-2 \psi\left(\frac{2}{3}\right)+\psi(1)\right]
$$

代入 $\psi$ 函数的特殊值, 即得

$$
\sum_{n=0}^{\infty} \frac{1}{(3 n+1)(3 n+2)(3 n+3)}=\frac{1}{4}\left[\frac{\pi}{\sqrt{3}}-\ln 3\right]
$$
\end{solution}

\begin{example}
求无穷级数 $\sum_{n=0}^{\infty} \frac{1}{n^{2}+a^{2}}$ 之和, 其中 $a>0$.
\end{example}
\begin{solution}
因为
$$
\frac{1}{n^{2}+a^{2}}=\frac{\imath}{2 a}\left(\frac{1}{n+\imath a}-\frac{1}{n-\imath a}\right),
$$
所以
$$
\sum_{n=0}^{\infty} \frac{1}{n^{2}+a^{2}}=-\frac{\imath}{2 a}[\psi(\imath a)-\psi(-\imath a)]
$$

利用上面列出的 $\psi$ 函数的性质,
$$
\psi(\imath a)-\psi(-\imath a)=-\frac{1}{\imath a}-\pi \cot \imath \pi a=\imath\left[\frac{1}{a}+\pi \coth \pi a\right]
$$

就可以求得
$$
\sum_{n=0}^{\infty} \frac{1}{n^{2}+a^{2}}=\frac{1}{2 a^{2}}[1+\pi a \coth \pi a]
$$
\end{solution}