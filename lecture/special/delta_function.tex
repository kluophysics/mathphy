\section{$\delta$函数}
\label{sub:delta_function}

\subsection{$\delta$函数的概念}

物理学常常要研究一个物理量在空间或时间中分布的密度, 例如质量密度 (通常简称为密度)、电荷密度、每单位时间传递的动量 (即力) 等等. 但是物 理学又常常运用质点、点电荷、瞬时力等抽象模型, 它们不是连续分布于空间 或时间中, 而是集中在空间的某一点或时间的某一瞬时. 它们的密度又该如何 描述呢?
我们知道, 若质量 $m$ 均匀分布在长为 $l$ 的线段 $[-l / 2, l / 2]$ 上, 则其线密 度 $\rho_l(x)$ 可表为
$$
\rho_l(x)=\left\{\begin{array}{ll}
0 & (|x|>l / 2), \\
m / l & (|x| \leqslant l / 2),
\end{array} \text { 即 } \rho_l(x)=\frac{m}{l} \operatorname{rect}\left(\frac{x}{l}\right) .\right.
$$
将 $\rho_l(x)$ 对 $x$ 积分, 则得到总质量
$$
\int_{-\infty}^{\infty} \rho_l(x) d x=\int_{-\frac{l}{2}}^{\frac{l}{2}} \frac{m}{l} d x=m
$$

如果让上述线段的长度 $l \rightarrow 0$, 我们将得到位于坐标原点质量为 $m$ 的质点, 而线密度函数就成为质点的线密度函数. 将它记为 $\rho(x)$, 则
$$
\lim _{l \rightarrow 0} \int_{-\infty}^{\infty} \rho_l(x) d x=\int_{-\infty}^{\infty} \rho(x) d x=m .
$$
若不求积分而先取极限, 则有
$$
\rho(x)=\lim _{l \rightarrow 0} \rho_l(x)=\lim _{l \rightarrow 0} \frac{m}{l} \operatorname{rect}\left(\frac{x}{l}\right)= \begin{cases}0 & (x \neq 0), \\ \infty & (x=0) .\end{cases}
$$

由此可以看出质点线密度分布函数的直观图像, 它在 $x=0$ 处为 $\infty$, 在 $x \neq$ 0 处为零. 它的积分为 $m$.
对于质点、点电荷、瞬时力这类集中于空间某一点或时间的某一瞬时的抽 象模型, 在物理学中引人 $\delta$ 函数以描述其密度:
$$
\begin{gathered}
\delta(x)= \begin{cases}0 & (x \neq 0), \\
\infty & (x=0) ;\end{cases} \\
\int_a^b \delta(x) d x= \begin{cases}0 & (\text{其他情况}), \\
1 & (a<0<b) .\end{cases}
\end{gathered}
$$
\subsection{$\delta$函数的性质}
$\delta$函数的很多性质可以用在实际的计算中,方便实用. 首先它是满足积分为单位1.
$$
\int_{-\infty}^{\infty} \delta(x) d x=1
$$

并且作为被积函数的乘积因子,它具有挑选性特点

$$
\int_{-\infty}^{\infty} f(x) \delta(x-a) d x=f(a).
$$

$\delta(x)$ 是偶函数, 它的导数是奇函数,
$$
\begin{gathered}
\delta(-x)=\delta(x), \\
\delta^{\prime}(-x)=-\delta^{\prime}(x) .
\end{gathered}
$$

对于变量为函数的情况,有
\begin{equation}
    \delta[\varphi(x)]=\sum_k \frac{\delta\left(x-x_k\right)}{\left|\varphi^{\prime}\left(x_k\right)\right|}
\end{equation}
其中$x_k(k=1,2,3, \cdots)$是
$\varphi(x)=0$ 的实根.
特例包括
\begin{equation}
    \begin{gathered}
    \delta(a x)=\frac{\delta(x)}{|a|}, \\
    \delta\left(x^2-a^2\right)=\frac{\delta(x+a)+\delta(x-a)}{2|a|}=\frac{\delta(x+a)+\delta(x-a)}{2|x|} .
    \end{gathered}
\end{equation}

$\delta$函数是一种广义函数,常常可以看作是以下函数的极限.
\begin{equation}
    \begin{aligned}
    & \delta(x)=\lim _{l \rightarrow 0} \frac{1}{l} \operatorname{rect}\left(\frac{x}{l}\right), \\
    & \delta(x)=\lim _{K \rightarrow \infty} \frac{1}{\pi} \frac{\sin K x}{x}, \\
    & \delta(x)=\lim _{\varepsilon \rightarrow 0} \frac{1}{\pi} \frac{\varepsilon}{\varepsilon^2+x^2} .
    \end{aligned}
\end{equation}

\subsection{其他}
$\delta$ 函数的傅里叶变换: 用复数形式的傅里叶积分表示$\delta$函数,
\begin{equation}
\delta(x)=\int_{-\infty}^{\infty} C(\omega) e^{\imath\omega x} d \omega,
\end{equation}
其中变换
$$
C(\omega)=\frac{1}{2 \pi} \int_{-\infty}^{\infty} \delta(x) e^{-\imath\omega x} d x=\frac{1}{2 \pi} e^{-\imath\omega \cdot 0}=\frac{1}{2 \pi} .
$$
因此,有$\delta$函数的傅里叶积分为
\begin{equation}
\delta(x)=\frac{1}{2 \pi} \int_{-\infty}^{\infty} e^{i \omega x} d \omega .
\end{equation}

多维的 $\delta$ 函数:
$$
\begin{aligned}
& \br=(x,y,z), \quad \delta (\br)= \begin{cases}0,  & \br \neq 0) \\
\infty, & (\br=0)\end{cases} \\
& \iiint_V \delta(\br)d^3 \br=\iiint_{-\infty}^{\infty} \delta(\br) dx dy dz=1
\end{aligned}
$$
在直角坐标系中, 这样的三维 $\delta$ 函数往往用三个一维 $\delta$ 函数的乘积表示:
\begin{equation}
    \delta(\br)=\delta(x) \delta(y) \delta(z),
\end{equation}
在柱坐标系中,
\begin{equation}
    \delta(\br)=\frac{1}{\rho} \delta(\varphi)\delta(z),
\end{equation}
在球坐标系中,
\begin{equation}
    \delta(\br)=\frac{1}{r^2\sin{\theta}} \delta(r) \delta(\theta) \delta(\varphi).
\end{equation}