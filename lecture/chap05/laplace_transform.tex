\subsection{拉普拉斯变换}
\label{subsec:laplace_transform}

傅里叶变换在分析信号的频谱等方面是十分有效的,但在系统分析方面有不足之处:

\begin{itemize}
    \item 对时间函数限制严格,绝对可积是充分条件,不少函数不能直接按定义求,如$e^{\alpha t}, \alpha > 0$,其傅里叶变换不存在。

    \item  
\end{itemize}
为了解决上述问题拓宽应用范围,人们发现适当地改造满足傅里叶变换的条件。
拉普拉斯变换常用于初始值问题, 即已知某个物理量在初始时刻 $t=0$ 的 值 $f(0)$, 而求解它在初始时刻之后的变化情况 $f(t)$。
 至于它在初始时刻之前 的值, 我们置
$$
f(t)=0 \quad(t<0) .
$$
构造一个带收敛因子$e^{-\sigma t}, \sigma > 0$的函数
$$
g(t) =e^{-\sigma t} f(t)
$$
则$g(t)$绝对可积。于是对$g(t)$进行傅里叶变换:
\begin{equation}
    G(\omega) = \sqrt{\frac{1}{2\pi} } \int_{-\infty}^{\infty} g(t) e^{-\imath \omega t} dt 
    = \sqrt{\frac{1}{2\pi} } \int_{0}^{\infty} f(t) e^{-(\sigma + \imath \omega) t} dt  
\end{equation}
将$\sigma + \imath \omega$记作$p$,令$F(p)=\sqrt{2\pi}G(\omega)$,则有
函数$f(t)$的拉普拉斯变换$F(p)$为
\begin{equation}
    F(p) = \mathcal{L} \{ f(t) \} = \int_0 ^{\infty} e^{-pt} f(t) dt \textrm{。}
\end{equation}
该积分称为拉普拉斯积分,$\mathcal{L}$称为\textbf{拉普拉斯变换算符}。$e^{-pt}$称为拉普拉斯变换的\textbf{核}。注意
这里的积分上下限。