\subsection{拉普拉斯变换}
\label{subsec:laplace_transform}

傅里叶变换在分析信号的频谱等方面是十分有效的,但在系统分析方面有不足之处:

\begin{itemize}
    \item 对时间函数限制严格,绝对可积是充分条件,不少函数不能直接按定义求,如$e^{\alpha t}, \alpha > 0$,其傅里叶变换不存在.

    \item  
\end{itemize}
为了解决上述问题拓宽应用范围,人们发现适当地改造满足傅里叶变换的条件.
拉普拉斯变换常用于初始值问题, 即已知某个物理量在初始时刻 $t=0$ 的 值 $f(0)$, 而求解它在初始时刻之后的变化情况 $f(t)$.
 至于它在初始时刻之前 的值, 我们置
$$
f(t)=0 \quad(t<0) .
$$
构造一个带收敛因子$e^{-\sigma t}, \sigma > 0$的函数
$$
g(t) =e^{-\sigma t} f(t)
$$
则$g(t)$绝对可积.于是对$g(t)$进行傅里叶变换:
\begin{equation}
    G(\omega) = \sqrt{\frac{1}{2\pi} } \int_{-\infty}^{\infty} g(t) e^{-\imath \omega t} dt 
    = \sqrt{\frac{1}{2\pi} } \int_{0}^{\infty} f(t) e^{-(\sigma + \imath \omega) t} dt  
\end{equation}
将$\sigma + \imath \omega$记作$p$,令$F(p)=\sqrt{2\pi}G(\omega)$,则有
函数$f(t)$的拉普拉斯变换$F(p)$为
\begin{equation}
    F(p) = \mathcal{L} \{ f(t) \} = \int_0 ^{\infty} e^{-pt} f(t) dt .
\end{equation}
该积分称为拉普拉斯积分,$\mathcal{L}$称为\textbf{拉普拉斯变换算符}.$e^{-pt}$称为拉普拉斯变换的\textbf{核}.注意
这里的积分上下限.

拉普拉斯变换存在的条件是 
\begin{enumerate}
    \item[(1)]  在 $0 \leqslant t<\infty$ 的任一有限区间上; 除了有限个间断点外, 函数 $f(t)$ 及其导数是处处连续的;
    \item[(2)]  存在常数 $M>0$ 和 $\sigma \geqslant 0$, 使对任何 $t$ 值 $(0 \leqslant t<\infty)$, 有
    $$
    |f(t)|<M \mathrm{e}^{\sigma t} \text {. }
    $$
\end{enumerate}
$\sigma$ 的下界称为收敛横标, 用 $\sigma_0$ 表示. 在实际应用中, 大多数函数都满足 这个充分条件.

\begin{examplebox}{求以下函数的拉普拉斯变换
    \begin{enumerate}
        \item $f(t) = 1$,
        \item $f(t) = t$,
        \item $f(t) =e^{s t}, \textrm{s为常数}$,
        \item $f(t) = t e^{s t}, \textrm{s为常数}$,
        \item $f(t) = \cosh {k t}$, $g(t)= \sinh {kt}$.
    \end{enumerate}}

    \begin{enumerate}
        \item $\Re p > 0$, 
            $$\int_{0}^{\infty} 1 \cdot e^{-p t} dt = \frac{1}{p}$$
        \item  $\Re p > 0$, 
            $$\int_{0}^{\infty} t \cdot e^{-p t} dt = \frac{1}{p^2}$$
            因此类似有$\mathcal{L}[t^n] = \frac{n!}{p^{n+1}}$.
        \item $$
            \int_{0}^{\infty} t \cdot e^{s t} e^{-p t} dt = \frac{1}{p-s} , 
        $$
        要求$\Re (p-s) > 0$.
        \item 
        $$
        \int_{0}^{\infty} t \cdot t e^{s t} e^{-p t} dt = \frac{1}{(p-s)^2} ,
        $$
        要求$\Re (p-s) > 0$.  
    类似有 $\mathcal{L}[t^n e^{s t}] = \frac{n!}{(p-s)^{n+1}}$.
        \item 利用
        $$
         \cosh{k t}  = \frac{1}{2} \left[ e^{kt} + e^{-kt}\right] \quad 
         \sinh{k t}  = \frac{1}{2} \left[ e^{kt} - e^{-kt}\right] 
        $$

        $$
        \mathcal{L}[ \cosh{k t} ] = \half \left( \frac{1}{s-k} + \frac{1}{s+k} \right) = \frac{s}{s^2 - k^2} ,
        $$
        $$
        \mathcal{L}[ \sinh{k t} ] = \half \left( \frac{1}{s-k} - \frac{1}{s+k} \right) = \frac{k}{s^2 - k^2} .
        $$
    \end{enumerate}
\end{examplebox}

\subsection{拉普拉斯变换的性质}
