\subsection{拉普拉斯变换}
\label{subsec:laplace_transform}

傅里叶变换在分析信号的频谱等方面是十分有效的,但在系统分析方面有不足之处:

\begin{itemize}
    \item 对时间函数限制严格,绝对可积是充分条件,不少函数不能直接按定义求,如$e^{\alpha t}, \alpha > 0$,其傅里叶变换不存在.

    \item  求解傅里叶反变换比较麻烦.
\end{itemize}
为了解决上述问题拓宽应用范围,人们发现适当地改造满足傅里叶变换的条件.
拉普拉斯变换常用于初始值问题, 即已知某个物理量在初始时刻 $t=0$ 的 值 $f(0)$, 而求解它在初始时刻之后的变化情况 $f(t)$.
 至于它在初始时刻之前 的值, 我们置
$$
f(t)=0, \quad(t<0) .
$$
构造一个带收敛因子$e^{-\sigma t}, \sigma > 0$的函数
$$
g(t) =e^{-\sigma t} f(t)
$$
则$g(t)$绝对可积.于是对$g(t)$进行傅里叶变换:
\begin{equation}
    g(\omega) = \frac{1}{2\pi}  \int_{-\infty}^{\infty} g(t) e^{-\imath \omega t} dt 
    = \frac{1}{2\pi}  \int_{0}^{\infty} f(t) e^{-(\sigma + \imath \omega) t} dt  
\end{equation}
将$\sigma + \imath \omega$记作$p$,令$\bar{f}(p)=2\pi g(\omega)$,则有
函数$f(t)$的拉普拉斯变换$\bar{f}(p)$为
\begin{equation}
    \bar{f}(p) = \mathcal{L} \{ f(t) \} = \int_0 ^{\infty} e^{-pt} f(t) dt .
\end{equation}
该积分称为\textbf{拉普拉斯积分},$\mathcal{L}$称为\textbf{拉普拉斯变换算符}.$e^{-pt}$称为拉普拉斯变换的\textbf{核}.注意
这里的积分上下限.

$g(\omega)$ 的傅里叶逆变换是
$$
g(t)=\int_{-\infty}^{\infty} g(\omega) e^{\imath \omega t} d \omega=\frac{1}{2 \pi} \int_{-\infty}^{\infty} \bar{f}(\sigma+\imath \omega) e^{\imath \omega t} d \omega,
$$
即
$$
f(t)=\frac{1}{2 \pi} \int_{-\infty}^{\infty} \bar{f}(\sigma+\imath \omega) e^{(\sigma+\imath \omega)t} d \omega
$$
由 $\sigma+\imath \omega=p$, 有 $d \omega=\frac{1}{\imath} d p$. 所以
\begin{equation}
    \label{eq:fourier_mellin_integral}
    f(t)=\frac{1}{2 \pi \imath} \int_{\sigma-\imath \infty}^{\sigma+\imath \infty} \bar{f}(p) e^{ p t} d p .
\end{equation}
$\bar{f}(p)$ 又称为\textbf{像函数}, 而 $f(t)$ 称为\textbf{原函数}, 它们之间的关系常用简单的符号写为
\begin{equation}
\begin{aligned}
& \bar{f}(p)=\mathcal{L}[f(t)], \\
& f(t)=\mathcal{L}^{-1}[\bar{f}(p)]
\end{aligned}
\end{equation}
或
\begin{equation}
\begin{aligned}
& \bar{f}(p) \risingdotseq f(t), \\
& f(t) \fallingdotseq \bar{f}(p) .
\end{aligned}
\end{equation}


拉普拉斯变换存在的充分条件是 
\begin{enumerate}
    \item[(1)]  在 $0 \leqslant t<\infty$ 的任一有限区间上; 除了有限个间断点外, 函数 $f(t)$ 及其导数是处处连续的;
    \item[(2)]  存在常数 $M>0$ 和 $\sigma \geqslant 0$, 使对任何 $t$ 值 $(0 \leqslant t<\infty)$, 有
    $$
    |f(t)|<M e^{\sigma t} \text {. }
    $$
\end{enumerate}
$\sigma$ 的下界称为收敛横标, 用 $\sigma_0$ 表示. 在实际应用中, 大多数函数都满足 这个充分条件.
但一个反例是$e^{t^2}$. 对于满足拉普拉斯变换存在条件的函数,通过证明可以知道$\lim_{p\to \infty} \bar{f}(p) = 0$.
也就意味着,若$\bar{f}(p)$在$p\to \infty$的渐近行为是$p$的正幂次,那么逆变换不存在. 且可以证明$\bar{f}(p)$在$\Re \sigma > \sigma_0$的半平面是解析的(具体证明
见梁昆淼书).
不难看出拉普拉斯变换为\textbf{线性变换}即满足
$$
\mathcal{L}\{a f(t)+b g(t)\}=a \mathcal{L}\{f(t)\}+b \mathcal{L}\{g(t)\} \,.
$$

对于$\delta$函数,易知
$$
\mathcal{L}\left\{\delta\left(t-t_0\right)\right\}=\int_0^{\infty} e^{-p t} \delta\left(t-t_0\right) d t=e^{-p t_0},  \quad t_0>0 .
$$
% 通过反复迭代
% 可以验证
% $$
% \mathcal{L}\left\{F^{(n)}(t)\right\}=s^n \mathcal{L}\{F(t)\}-s^{n-1} F(+0)-\cdots-F^{(n-1)}(+0) .
% $$
\begin{examplebox}{求以下函数的拉普拉斯变换
    \begin{enumerate}
        \item $f(t) = 1$,
        \item $f(t) = t$,
        \item $f(t) =e^{s t}, \textrm{s为常数}$,
        \item $f(t) = t e^{s t}, \textrm{s为常数}$,
        \item $f(t) = \cosh {k t}$, $g(t)= \sinh {kt}$.
    \end{enumerate}}

    \begin{enumerate}
        \item $\Re p > 0$, 
            $$\int_{0}^{\infty} 1 \cdot e^{-p t} dt = \frac{1}{p}$$
        \item  $\Re p > 0$, 
            $$\int_{0}^{\infty} t \cdot e^{-p t} dt = \frac{1}{p^2}$$
            因此类似有$\mathcal{L}[t^n] = \frac{n!}{p^{n+1}}$.
        \item $$
            \int_{0}^{\infty} e^{s t} e^{-p t} dt = \frac{1}{p-s} , 
        $$
        要求$\Re (p-s) > 0$.
        \item 
        $$
        \int_{0}^{\infty} t e^{s t} e^{-p t} dt = \frac{1}{(p-s)^2} ,
        $$
        要求$\Re (p-s) > 0$.  
    类似有 $\mathcal{L}[t^n e^{s t}] = \frac{n!}{(p-s)^{n+1}}$.
        \item 利用
        $$
         \cosh{k t}  = \frac{1}{2} \left[ e^{kt} + e^{-kt}\right] \quad 
         \sinh{k t}  = \frac{1}{2} \left[ e^{kt} - e^{-kt}\right] 
        $$

        $$
        \mathcal{L}[ \cosh{k t} ] = \half \left( \frac{1}{p-k} + \frac{1}{p+k} \right) = \frac{p}{p^2 - k^2} ,
        $$
        $$
        \mathcal{L}[ \sinh{k t} ] = \half \left( \frac{1}{p-k} - \frac{1}{p+k} \right) = \frac{k}{p^2 - k^2} .
        $$
    \end{enumerate}
\end{examplebox}

\subsection{拉普拉斯变换的性质}
与傅里叶变换类似,拉氏变换$\bar{f}(p)$也满足很多基本性质.
\begin{enumerate}
    \item 导数定理
        \begin{equation}
        f^{\prime}(t) \fallingdotseq  p \bar{f}(p)-f(0),
        \end{equation}
        对于高阶导数有递推关系
        \begin{equation}
            f^{(n)}(t) \fallingdotseq p^n \bar{f}(p)-p^{n-1} f(0)-p^{n-2} f^{\prime}(0)-\cdots-p f^{(n-2)}(0)-f^{(n-1)}(0)
        \end{equation}
        \label{eq:lapl_trans_derivative}
    \item 积分定理
        \begin{equation}
        \int_0^t \psi(\tau) d \tau \fallingdotseq  \frac{1}{p} \mathcal{L}[\psi(t)]
        \end{equation}
    \item 相似性定理 
    \begin{equation}
         f(a t) \fallingdotseq  
         \frac{1}{a} \bar{f}\left(\frac{p}{a}\right)
    \end{equation}
        
    \item 位移定理 
        \begin{equation}       
         e^{-\lambda t} f(t) \fallingdotseq  \bar{f}(p+\lambda)
        \end{equation}

    \item 延迟定理 
    \begin{equation}       
        f\left(t-t_0\right) \fallingdotseq  e^{-p t_0} \bar{f}(p).
    \end{equation}

    \item 卷积定理 若  $f_1(t) \fallingdotseq  \bar{f}_1(p), f_2(t) \fallingdotseq  \bar{f}_2(p)$ , 则 
    \begin{equation}       
        f_1(t) * f_2(t) \fallingdotseq  \bar{f}_1(p) \bar{f}_2(p),
    \end{equation}
    其中
    \begin{equation}        
    f_1(t) * f_2(t) \equiv \int_0^t f_1(\tau) f_2(t-\tau) d \tau
    \end{equation}
    称为 $f_1(t)$ 与 $f_2(t)$ 的\textbf{卷积}.
\end{enumerate}

% 若 $f(t)$ 满足 Laplace 变换存在的充分条件, 则
% $$\bar{f}(p) \rightarrow 0, \Re p=s \rightarrow+\infty$$.



% \begin{table}
%     \centering
% \begin{tabular}{|c|c|c|}
%     \hline No & 原 函 数 & 像 函 数 \\
%     \hline 1 & 1 & $\frac{1}{p}$ \\
%     3 & $t^n(n$ 为整数 $)$ & $\frac{n !}{p^{n+1}}$ \\
%     4 & $t^a(a>-1)$ & $\frac{\Gamma(a+1)}{p^{a+1}}$ \\
%     5 & $\sin \omega t$ & $\frac{1}{p-\lambda}$ \\
%     6 & $\cos \omega t$ & $\frac{\omega}{p^2+\omega^2}$ \\
%     7 & $\operatorname{sh} \omega t$ & $\frac{p}{p^2+\omega^2}$ \\
%     8 & $\operatorname{ch} \omega t$ & $\frac{\omega}{p^2-\omega^2}$ \\
%     & & $\frac{p}{p^2-\omega^2}$ \\
%     9 & $\frac{\omega}{(p+\lambda)^2+\omega^2}$ \\
%     10 & $e^{-\lambda t} \sin \omega t$ & $\frac{p+\lambda}{(p+\lambda)^2+\omega^2}$ \\
%     11 & $e^{-\lambda t} t^a$ & $\frac{\Gamma(a+1)}{(p+\lambda)^{a+1}}$ \\
%     12 & $\frac{1}{\sqrt{\pi t}}$ & $\frac{1}{\sqrt{p}}$ \\
%     13 & $\frac{1}{\sqrt{\pi t}} e^{-a^{2 / 4 t}}$ & $\frac{e^{-a \sqrt{p}}}{\sqrt{p}}$ \\
%     14 & $\frac{1}{\sqrt{\pi t}} e^{-2 a \sqrt{t}}$ & $\frac{1}{\sqrt{p}} e^{\frac{a^2}{p}} erfc\left(\frac{a}{\sqrt{p}}\right)$ \\
%     15 & $\frac{1}{\sqrt{\pi a}} \sin 2 \sqrt{a t}$ & $\frac{1}{p \sqrt{p}} e^{-\frac{a}{p}}$ \\
%     16 & $\frac{1}{\sqrt{\pi a}} \cos 2 \sqrt{a t}$ \\
%     17 & $erf(\sqrt{a t})$ & $\frac{1}{\sqrt{p}} e^{-\frac{a}{p}}$ \\
%     18 & $erfc\left(\frac{a}{2 \sqrt{t}}\right)$ & $\frac{\sqrt{a}}{p \sqrt{p+a}}$ \\
%     19 & $e^t erfc(\sqrt{t})$ & $\frac{1}{p} e^{-a \sqrt{p}}$ \\
%     20 & $\frac{1}{\sqrt{\pi t}}-e^t erfc(\sqrt{t})$ & $\frac{1}{p+\sqrt{p}}$ \\
%     21 & $\frac{1}{\sqrt{\pi t}} e^{-a t}+\sqrt{a} \operatorname{erf} \sqrt{a t}$ & $\frac{1}{1+\sqrt{p}}$ \\
%     & $\frac{\sqrt{p+a}}{p}$ \\
%     22 & $J_0(t)$ & $\frac{1}{\sqrt{p^2+1}}$ \\
%     23 & $J_n(t)$ & $\frac{\left(\sqrt{p^2+1}-p\right)^n}{\sqrt{p^2+1}}$ \\
%     24 & $\frac{J_n(a t)}{t}$ & $\frac{1}{n a^n}\left(\sqrt{p^2+a^2}-p\right)^n$ \\
%     26 & $e^{-a t} I_0(b t)$ & $\frac{1}{\sqrt{(p+a)^2-b^2}}$ \\
%     27 & $\lambda^n e^{-\lambda t} I_n(\lambda t)$ & $\frac{\left\{\sqrt{p^2+2 \lambda p}-(p+\lambda)\right\}^n}{\sqrt{p^2+2 \lambda p}}$ \\
%     28 & $t^n J_n(t)\left(n>-\frac{1}{2}\right)$ & $\frac{2^n \Gamma\left(n+\frac{1}{2}\right)}{\sqrt{\pi}} \frac{1}{\left(p^2+1\right)^{n+\frac{1}{2}}}$ \\
%     29 & $t^{\frac{n}{2}} J_n(2 \sqrt{t})$ & $\frac{1}{p^{n+1}} e^{-\frac{1}{p}}$ \\
%     30 & $J_0\left(a \sqrt{t^2-\tau^2}\right) H(t-\tau)$ & $\frac{1}{\sqrt{p^2+a^2}} e^{-\tau \sqrt{p^2+a^2}}$ \\
%     31 & $\frac{J_1\left(a \sqrt{t^2-\tau^2}\right)}{\sqrt{t^2-\tau^2}} H(t-\tau)$ & $\frac{e^{-\tau p}-e^{-\tau \sqrt{p^2+a^2}}}{a \tau}$ \\
%     32 & $\int_t^{\infty} \frac{J_0(t)}{t} d t$ & $\frac{1}{p} \ln \left(p+\sqrt{1+p^2}\right)$ \\
%     33 & $\frac{e^{b t}-e^{a t}}{t}$ & $\ln \frac{p-a}{p-b}$ \\
%     34 & $\frac{1}{\sqrt{\pi t}} \sin \frac{1}{2 t}$ & $\frac{1}{\sqrt{p}} e^{-\sqrt{p}} \sin \sqrt{p}$ \\
%     35 & $\frac{1}{\sqrt{\pi t}} \cos \frac{1}{2 t}$ & $\frac{1}{\sqrt{p}} e^{-\sqrt{p}} \cos \sqrt{p}$ \\
%     \hline
% \end{tabular}
% \end{table}
