\subsection{拉普拉斯积分变换的应用}
\label{subsec:applications}

\subsubsection{计算级数和}
有时,拉普拉斯变换可以用来计算某些级数的和.以下面的例子说明.

如计算级数和
$$
    \sum_{n=1} ^{\infty}  \frac{1}{n^2},
$$
由前面例题
$$
\int_0^{\infty} t \mathrm{e}^{-p^t} \mathrm{~d} t=\frac{1}{p^2}, \quad \operatorname{Re} p>0
$$

将级数化为
$$
\begin{aligned}
\sum_{n= 1}^{\infty} \frac{1}{n^2} & =\sum_{n=1}^{\infty} \int_0^{\infty} t e^{-n t} \mathrm{~d} t \\
& =\int_0^{\infty} t\left[\sum_{n=1}^{\infty} \mathrm{e}^{-n t}\right] \mathrm{d} t=\int_0^{\infty} \frac{t}{\mathrm{e}^t-1} \mathrm{~d} t
\end{aligned}
$$
查表可得
\begin{equation}
    \sum_{n=1} ^{\infty}  \frac{1}{n^2} = \frac{\pi^2}{6}.
\end{equation}

\subsubsection{求解定积分}
如果 $\int_v^{\infty} \bar{f}(q) d q$ 存在, 且当$t\to 0$时, $|f(t)/t|$有界,则
\begin{equation}
    \int_{p}^{\infty} \bar{f}(q) dq \risingdotseq \frac{f(t)}{t} . 
\end{equation}
比如
$$
\frac{\sin \omega t}{t} \fallingdotseq \int_p^{\infty} \frac{\omega}{q^2+\omega^2} d q=\frac{\pi}{2}-\arctan \frac{p}{\omega}.
$$

当$p\to 0$时, 有
\begin{equation}
    \int_0^{\infty} \bar{f}(p) d p=\int_0^{\infty} \frac{f(t)}{t} d t
\end{equation}
一个例子如
$$
\int_0^{\infty} \frac{\sin t}{t} d t=\int_0^{\infty} \frac{1}{p^2+1} d p=\frac{\pi}{2}
$$
不仅如此, 有些积分无法用留数定理计算,如
$$
\int_0^{\infty} \frac{\cos a t-\cos b t}{t} d t \quad a>0, b>0
$$
使用以上等式可得

$$
\begin{gathered}
\int_0^{\infty} \frac{\cos a t-\cos b t}{t} d t \fallingdotseq \int_0^{\infty}\left(\frac{p}{p^2+a^2}-\frac{p}{p^2+b^2}\right) d p \\
=\left.\frac{1}{2} \ln \frac{p^2+a^2}{p^2+b^2}\right|^{\infty}=\ln b-\ln a .
\end{gathered}
$$

\subsubsection{求解微分方程}
给一个例子,利用拉普拉斯变换求解简谐振子方程的解.
% \begin{examplebox}{利用拉普拉斯变换求解简谐振子方程的解.}
    质量为$m$的质点在弹性系数为$k$的弹簧牵引下做简谐运动,满足方程为
    $$
m \frac{d^2 X(t)}{d t^2}+k X(t)=0
$$
初始条件取
$$
X(0)=X_0, \quad X^{\prime}(0)=0
$$
应用拉氏变换到该方程上得到
$$
m \mathcal{L}\left\{\frac{d^2 X}{d t^2}\right\}+k \mathcal{L}\{X(t)\}=0
$$
用$x(s)$表示未知变换$\mathcal{L}\{ X(t)\}$,于是有
$$
m s^2 x(s)-m s X_0+k x(s)=0 ,
$$
化简为
$$
x(s)=X_0 \frac{s}{s^2+\omega_0^2}, \quad \text { with } \omega_0^2 \equiv \frac{k}{m} .
$$
查表得
$$
X(t)=X_0 \cos \omega_0 t .
$$
% \end{examplebox}