
\section{二阶线性常微分方程(齐次)}
$$
y''+p(x) y '+q(x) y=0
$$
$x_0$ 时, $p\left(x_0\right), q\left(x_0\right)$ 有限,称 $x_0$为ODE的常点
$p\left(x_0\right), q\left(x_0\right) \rightarrow \infty$称 $x_0$为ODE的奇点

常奇点(或正则奇点) $$\left\{\begin{array}{l}
    \left(x-x_0\right) p\left(x_0\right) \text{有限}
    \\
    \left(x-x_0\right)^2 q(x) \text{有限}
    \\ p\left(x_0\right), q(x) \rightarrow \infty .\end{array}
    \right.$$
其他情况为非常奇点.


下面给出几种重要的ODE的常奇点
\begin{enumerate}
    \item 超几何(hypergeometic):
    $$
    x(x-1) y''+[(1+a+b) x+c] y'+a b y= 0
    $$
    常奇点 $0,1, \infty$\\
    \item 勒让德 (Legendre):
    $$
    \left(1-x^2\right) y''-2 x y'+l(l+1) y=0
    $$
    常奇点. $-1,1, \infty$ .\\
    \item  切比雪夫. (Chebyshev):
    $$
    \left(1-x^2\right) y''-x y'+n^2 y=0 
    $$
    常奇点 $-1,1, \infty$.\\

    \item  合流超几何 (confluent hypergeometic):
    $$
    x y''+(c-x) y'-a y=0,
    $$
    \\
    \item 拉盖尔(Laguerre)
    $$
    x y''+(1-x) y'+a y=0,
    $$
    常奇点 $0$.
    \item 贝塞尔(Bessel)
    $$
    x^2 y'' + xy' + (x^2 - n^2) y = 0
    $$
    常奇点$0$.
\end{enumerate}
\begin{note}
    注意这里的$\infty$为常奇点的情况具体
    解释我们并没有给出, 它的解决需要用$z=1/x$的代换研究$z=0$时的情形, 具体讨论这里略去.
\end{note}

\subsection{几类特殊的二阶方程}
\subsubsection{ $y'' = f(x)$ 型}
此类方程只需要积分两次就可以得到通解
\begin{equation}
    y = \int F(x) dx + C_1 x + C_2
\end{equation}
其中$F(x) = \int f(x) dx$, $C_1, C_2$是积分常数.
\subsubsection{ $y'' = f(x, y')$ 型}
利用换元$u = y'$, 则有$u' = y''$, 原方程化为一阶方程
$$ u' = f(x,u)$$
该解为$u = \varphi(x, C_1)$, 原方程解为
\begin{equation}
    y = \int \varphi(x, C_1) dx  + C_2.
\end{equation}
% \subsubsection{ $y'' = f(y, y')$ 型}


\subsection{常点邻域上的级数解}
$\left|x-x_0\right|<R$ 上.$x_0$为常点.

$$
y(x)=\sum_{k=0} a_k\left(x-x_0\right)^k, a_1, a_2, a_k \cdots \text { 待定. }
$$
这里以谐振子为例
$$
\text { 令 } \begin{aligned}
y(x) & =x^s\left(a_0+a_1 x+c_2 x^2+\cdots\right) \\
& =\sum_{j=0}^{\infty} a_j x^{s+j}, a_0 \neq 0 .
\end{aligned}
$$

\begin{note}
这里课堂上采用了直接令$s=0$的讲法.
\end{note}
$$
\begin{aligned}
& \frac{d y}{d x}=\sum_{j=0}^{\infty} a_j(j+s) x^{s+j-1} \\
& \frac{d^2 y}{d x^2}=\sum_{j=0}^{\infty} a_j(j+s)(j+s-1) x^{s+j-2}
\end{aligned}
$$

代入原方程.
$$
\sum_{j=0}^{\infty} a_j(s+j)(s+j-1) x^{s+j-2}+\omega^2 \sum_{j=0}^{\infty} a_j x^{s+j=0} \text {. }
$$

$$
\text { 任意阶 } x^{s+j} \text { 都满足 }
$$
,系数必为零.

$$
\begin{aligned}
& x^{s-2} a_0(s)(s-1)=0 \quad\left(a_0 \neq 0\right) \\
& x^{s-1}: a_1(s+1)(s)=0 \\
& x^s: a_2(s+2)(s+1)+a_0 w^2=0
\end{aligned}
$$

$\Rightarrow$ 若 $s=0, y \sim  a_0+a_1 x+\cdots$
若 $s=1, y \sim a_0 x+a_1 x^2+\cdots$
当 $s=1$时, $a_1=0$.
当 $s=0$ 时, $a_1$ 任意, 因此我们取$a_1=0$.
$$
\begin{gathered}
\Rightarrow a_{j+2}(s+j+2)(s+j+1)+w^2 a_j=0 \\
\Rightarrow a_{j+2}=\frac{-w^2}{(s+j+2)(s+j+1)} a_j
\end{gathered}
$$

递推关系.
$$
\begin{aligned}
& \text { 若 } s=0 \Rightarrow a_{j+2}=\frac{-\omega^2}{(j+2)(j+1)} a_j \\
& a_2=-\frac{w^2}{2 !} a_0 \\
& a_4=-\frac{a_2}{3^3 4}=+\frac{w^4}{4 !} a_0 \\
& a_6=-\frac{a_0^{3.4}}{5.6}=-\frac{w 6}{6 !} a_0 \\
&
\end{aligned}
$$

$$
a_{2 n}=(-1)^n \frac{w^{2 n}}{2 n !} a_0
$$

$$
\begin{aligned}
y & \left.=a_0\left[1-\frac{(\omega x)^2}{2 !}+\frac{(\omega x}{4}\right)^4-\frac{\omega x x^6}{6 !}+\omega\right] \\
& =a_0 \cos \omega x .
\end{aligned}
$$

$$
\begin{aligned}
& \text { 若 } s=1 . \Rightarrow \\
& a_{j+2}=-a_j \frac{\omega^2}{(j+3) (j+2)} \\
& \Rightarrow a_2=-a_0 \frac{\omega^2}{2 \cdot 3}=\frac{\omega^2}{3 !} a_0 \\
& a_4=-a_2 \frac{\omega^2}{5 \cdot 4}=\frac{\omega^4}{5 !} a_0 \\
& a_6=-a_4 \frac{\omega^2}{6\cdot 7}=\frac{w^6}{7 !} a_0 \\
\end{aligned}
$$

$$
\begin{aligned}
& \Rightarrow \quad a_0 x\left[1-\frac{(\omega x)^2}{3 !}+\frac{(\omega x)^{4}}{5 !}-\frac{(\omega x)^6}{7 !} \cdots\right] \\
& =\frac{a_0}{\omega}\left[(\omega x)-\frac{(\omega x)^3}{3 !}+\frac{(\omega x)^5}{5 !} \cdots\right] \\
& =\frac{a_0}{\omega} \sin \omega x
\end{aligned}
$$
此方法称为Frobenius方法,上面关于$x_0$上展开的,一般的
我们可以在 $x_0$处展开
$$
y(x)=\sum_{j=0}^{\infty} a_j\left(x-x_0\right)^{s+j}, a_0 \neq 0 .
$$


\subsection{正则奇点邻域上的级数求解}
级数解有限个负幂项称为正则奇点 $p(x): \sum_{k=-1}^{\infty}$
$q(x) : \sum_{k=-2}^{\infty}$

有判定方程,
$$
s(s-1)+s p_{-1}+q_{-2}=0
$$
其中$s_1, s_2$ 为两根($s_1> s_2$).

$$
\begin{aligned}
y_1(x) & =\sum_{k=0}^{\infty} a_k\left(x-x_0\right)^{s_1+k} \cdot \\
y_2(x) & =\sum_{k=0}^{\infty} b_k\left(x-x_0\right)^{s_2+k} . \\
\text{或} y_2(x) & =A y_1(x) \ln \left(x-x_0\right) \\
& +\sum_{k=0}^{\infty} b_k\left(x-x_0\right)^{s_2+k}
\end{aligned}
$$


$\nu$阶贝塞尔方程的级数求解参考书上.




Frobenius 方法并不总是能给出两个解,当特征根的差为整数时,级数解法会给出两个线性相关的, 因此需要其他方法.
一般会给出一个级数解,找出第二个解的方式可以通过如下方法获得.

这里引入朗斯基行列式(Wronskian)
$$
\Delta(x)=\left|\begin{array}{ll}
y_1(x) & y_2(x) \\
y_1'(x) & y_2'(x)
\end{array}\right|= y_1(x) y_2'(x) - y_2(x) y_1'(x)
$$

由于$y_1, y_2$满足微分方程,
$$
\begin{aligned}
& \left(y_1 y_2''-y_1'' 
y_2\right)+p\left(y_1 y_2'-y_1' y_2\right) = 0\\
& \frac{d \Delta}{d x} - p \Delta(x)=0
\end{aligned}
$$

$$
\Delta(x)=\Delta_0 e^{-\int p(x) d x}
$$

$$
\begin{aligned}
& \frac{d}{d x}\left(\frac{y_{2}}{y_1}\right)=\frac{y_{1} y_{2}'-y_{1}' y_{2}}{y_{1}^2}=\frac{\Delta(x)}{y_{1}^2} \\
& y_{2}=y_{1} \int \frac{\Delta(x)}{\left(y_{1}(x)\right)^2} d x
\end{aligned}
$$

