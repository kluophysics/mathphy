\chapter{偏微分方程}
超过一个独立变量的称为偏微分方程, partial differential equation (PDE). 有且仅有一个独立变量的称为常微分方程, 
ordinary differential equation (ODE).

线性算子
$\frac{d}{d x}$ 是线性算子.
$$
\frac{d}{d x}(\alpha f+\beta g)=\alpha \frac{d}{d x} f +\beta \frac{d}{d x} g
$$
线性算子的通用形式可表示为.
$$
L=\sum_{\nu=0}^n p_\nu(x) \frac{d^\nu}{d x^\nu}, P_\nu(x) \text { 任意 }
$$
类似的 $\frac{d^2}{d x^2}$ 也是线性算子.
\section{一阶常微分方程}
下面讨论一阶微分方程
\begin{equation}
    y' = f(x,y),
\end{equation}
或写成对称形式的一阶微分方程
\begin{equation}
    p(x,y) dx + q(x,y) dy = 0,
\end{equation}
这里$f(x,y), p(x,y), q(x,y)$均表示含$x,y$两个变量的解析表达式. 当微分方程写为这种对称形式时,
变量$x,y$处于平等地位, 可将$x$或$y$看成是未知函数.
\subsection{可分离变量的方程.}
$$
\begin{gathered}
\frac{d y}{d x}=-\frac{p(x)}{q(y)} \\
\Rightarrow p(x) d x+q(y) d y=0 . \\
\int p(x) d x+\int q(y) d y=C .
\end{gathered}
$$
这里的积分常数已单独写出.两个不定积分的原函数分别是$P(x), Q(y)$, 则上式的通解为
$P(x) + Q(y) = C$.

分离变量的方程其实不需要微分方程线性.
考虑全微分形式
$$
\begin{aligned}
& p(x, y) d x+q(x, y) d y=0 \\
& d \varphi=\frac{\partial \varphi}{\partial x} d x+\frac{\partial \varphi}{\partial y} d y=0 .
\end{aligned}
$$

$$
\left\{\begin{array}{l}
\frac{\partial \varphi}{\partial x}=p(x, y) . \\
\frac{\partial \varphi}{\partial y}=q(x, y) .
\end{array}\right.
$$
若 $\varphi$ 满足上式,则 $\varphi(x, y)=C$.
当且仅当$\frac{\partial p(x, y)}{\partial y} = \frac{\partial q(x, y)}{\partial x}$
\begin{equation}
\varphi(x, y)=\int_{x_0}^x p(x, y) d x+\int_{y_0}^y q(x_0, y) \equiv C  
\end{equation}

$$
\begin{aligned}
    & \text { 可分离 } \Rightarrow \text { 全微分 } \\
    & \text { 但全微分} \nRightarrow \text{可分离} \\
\end{aligned}
$$
% \subsection{}

\begin{example}
$$
y'+\left(1+\frac{y}{x}\right)=0
$$
\end{example}
\begin{solution}
$$
\Rightarrow(x+y) d x+x d y=0
$$
$$
\text { 由于 } \frac{\partial p}{\partial y}=1, \frac{\partial q}{\partial x}=1
$$

可得 $d \varphi=0$
$\Rightarrow \varphi=\int_{x_0}^x(x+y) d x+\int_{y_0}^y x_0 d y$

\begin{equation}
    \begin{aligned}
    \varphi & =\left(\frac{x^2}{2}+x y\right)-\left(\frac{x_0^2}{2}-x_0 y\right)+x_0 y-x_0 y_0 \\
    & =\frac{x^2}{2}+x y+c \\
    & \Rightarrow \frac{x^2}{2}+x y=c
    \end{aligned}
    \end{equation}
\end{solution}

下面考虑可以换元法解决的方程, 它的形式为
\begin{equation}
    \frac{dy}{dx} = f(a x + by + c), b\neq 0
\end{equation}
我们可以采用换元$u = ax + by$, 将$u$作为未知函数,有$a dx + b dy = du $,
可以化为
\begin{equation}
    du  = ( a + b f(u+c) ) dx,
\end{equation}
变成了可分离变量形式. 我们可得$u  = \varphi (x, C)$, 原方程解为
\begin{equation}
    y = \frac{1}{b} \left(\varphi(x, C) - a x \right).
\end{equation}

\subsection{权重法}
令 $y=x^{m}v$
$$
\begin{aligned}
& \left(x^2-y\right) d x+x d y=0 . \\
& x \text { 权重为 } 1, y\text { 权重为 }m \\
& x^2 d x \text { 权重为 } 3:
\end{aligned}
$$
$-ydx,  x d y$ 权重为 $1+m$.为了配平权重
$$
1+m=3 \Rightarrow m=2 \text { . }
$$


$$
\begin{aligned}
& y=x^2 v \text { 代入. } \\
& (1-v) d x+x d v=0 \\
& \frac{d x}{x}+\frac{d v}{v+1}=0 . \\
& \Rightarrow \ln x+\ln (v+1)=c \\
& \Rightarrow x(v+1)=c \\
\end{aligned}
$$
得到最终解
\[
    y=x^2 v=x^2\left(\frac{c}{x}-1\right) = -x^2+c x 
\]

\subsection{一阶线性方程}
一般一阶线性常微分方程的标准形式为
\begin{equation}
    \frac{d y}{d x}+p(x) y=q(x).
\end{equation}
当$q(x) \equiv 0$时, 上式为特殊形式
\begin{equation}
    \frac{d y}{d x}+p(x) y = 0.
\end{equation}
成为\textbf{一阶线性齐次方程}

两边同时乘以 $\alpha(x)$ .
\begin{equation}
     \alpha \frac{d y}{d x}+\alpha p y=\alpha q 
\end{equation}
而对$\alpha y$求导有
\begin{equation}
    \frac{d}{d x}(\alpha y)=\frac{d \alpha}{d x} y+\alpha \frac{d y}{d x} .
\end{equation}
为了使左式是一个全微分, 比较两式要求
$\frac{d \alpha}{d x}=\alpha p $, 即 $\frac{d \alpha}{\alpha}=p d x $. 于是有
\begin{equation}
    \alpha(x)=e^{\int p(x) d x}.
\end{equation}

回到原问题,需求
$$\frac{d}{d x}(\alpha y)=\alpha q $$
即
\begin{equation}
    y=\frac{1}{\alpha(x)}\left[\int^x \alpha(t) q(t) dt+C\right] \equiv y_1(x)+y_2(x)
\end{equation}
这里,我们将该解分成两个部分
\begin{equation}
\begin{aligned}
& y_1(x)=\frac{1}{\alpha(x)} \int^x \alpha(t) q(t) d t \\
& y_2(x)=\frac{C}{\alpha(x)}
\end{aligned}
\end{equation}
可以看出, $y_2(x)=\frac{c}{\alpha(x)}$是齐次方程的通解,
而 $y_1(x)=\frac{1}{\alpha(x)} \int^x \alpha(t) q(t) d t$是方程的特解.
一般地, 微分方程的解是\textbf{特解}加\textbf{通解}的形式.

\subsection{几类特殊的二阶方程}
\subsubsection{ $y'' = f(x)$ 型}
此类方程只需要积分两次就可以得到通解
\begin{equation}
    y = \int F(x) dx + C_1 x + C_2
\end{equation}
其中$F(x) = \int f(x) dx$, $C_1, C_2$是积分常数.
\subsubsection{ $y'' = f(x, y')$ 型}
利用换元$u = y'$, 则有$u' = y''$, 原方程化为一阶方程
$$ u' = f(x,u)$$
该解为$u = \varphi(x, C_1)$, 原方程解为
\begin{equation}
    y = \int \varphi(x, C_1) dx  + C_2.
\end{equation}
% \subsubsection{ $y'' = f(y, y')$ 型}


\subsection{ 常系数的常微分方程 (ODE)}
常系数的常微分方程的标准形式为
\begin{equation}
\frac{d^n y}{d x^n}+a_{n-1}\frac{d^{n-1} }{d x^{n-1}} y  + \cdots +a_{1} \frac{d}{d x} y+a_0 y=F(x)
\end{equation}
其中$a_i$为常数.
$$
\Rightarrow \text { 解的形式为 } y=e^{m x}
$$

其中$m$需满足代数方程
$$
m^n+a_{n-1} m^{n-1}+\cdots a_1 m+a_0=0
$$

对于二阶常系数线性齐次方程, $$
y'' + a y' + by = 0,
$$
代数方程化成
\begin{equation}
    m^2 + a m + b = 0
\end{equation}
该方程成为微分方程的\textbf{特征方程},其解为\textbf{特征根}. 如果两根不同
则有通解
\begin{equation}
    y = C_1 e^{m_1 x} +  C_2 e^{m_2 x} 
\end{equation}
若特征根为二重实根$m$, 则解为 
\begin{equation}
    y = (C_1 + C_2  x) e^{m x}
\end{equation}
% 如果有两个
\begin{example}
    谐振子方程
$$
\begin{aligned}
& m \frac{d X^2}{d t^2}=-k X(t) \\
& X''+a_0 X=0, a_0=\frac{k}{m} . \\
& m^2+a_0=0 \\
& \Rightarrow m= \pm i \sqrt{a_0}= \pm i \omega \quad \omega=\sqrt{\frac{k}{m}} \\
& y(t)=C_1 ^{+i \omega t}+D_{1} e^{-i \omega t} . \\
& \text { 或 } C_1 \cos \omega t+D_1 \sin \omega t
\end{aligned}
$$
\end{example}


\section{二阶线性常微分方程(齐次)}
$$
y''+p(x) y '+q(x) y=0
$$
$x_0$ 时, $p\left(x_0\right), q\left(x_0\right)$ 有限,称 $x_0$为ODE的常点
$p\left(x_0\right), q\left(x_0\right) \rightarrow \infty$称 $x_0$为ODE的奇点

常奇点(或正则奇点) $$\left\{\begin{array}{l}
    \left(x-x_0\right) p\left(x_0\right) \text{有限}
    \\
    \left(x-x_0\right)^2 q(x) \text{有限}
    \\ p\left(x_0\right), q(x) \rightarrow \infty .\end{array}
    \right.$$
其他情况为非常奇点.


(1) 超几何(hypergeometic):
$$
x(x-1) y''+[(1+a+b) x+c] y'+a b y= 0
$$

常奇点 $0,1, \infty$\\
(2) 勒让德 (Legendre):
$$
\left(1-x^2\right) y''-2 x y'+l(l+1) y=0
$$
常奇点. $-1,1, \infty$ .\\
(3) 切比雪夫. (Chebyshev):
$$
\left(1-x^2\right) y''-x y'+n^2 y=0 
$$
常奇点 $-1,1, \infty$.\\

(4) 合流超几何 (confluent hypergeometic):
$$
x y''+(c-x) y'-a y=0,
$$
\\
(5) 拉盖尔(Laguerre)
$$
x y''+(1-x) y'+a y=0,
$$
常奇点 $0$.

\subsection{常点邻域上的级数解}
$\left|z-z_0\right|<R$ 上.$z_0$为常点.

$$
y(z)=\sum_{k=0} a_k\left(z-z_0\right)^k, a_1, a_2, a_k \cdots \text { 待定. }
$$


以谐振子为例
$$
\text { 令 } \begin{aligned}
y(x) & =x^s\left(a_0+a_1 x+c_2 x^2+\cdots\right) \\
& =\sum_{j=0}^{\infty} a_j x^{s+j}, a_0 \neq 0 .
\end{aligned}
$$

$$
\begin{aligned}
& \frac{d y}{d x}=\sum_{j=0}^{\infty} a_j(j+s) x^{s+j-1} \\
& \frac{d^2 y}{d x^2}=\sum_{j=0}^{\infty} a_j(j+s)(j+s-1) x^{s+j-2}
\end{aligned}
$$

代入原方程.
$$
\sum_{j=0}^{\infty} a_j(s+j)(s+j-1) x^{s+j-2}+\omega^2 \sum_{j=0}^{\infty} a_j x^{s+j=0} \text {. }
$$

$$
\text { 任意阶 } x^{s+j} \text { 都满足 }
$$
,系数必为零.

$$
\begin{aligned}
& x^{s-2} a_0(s)(s-1)=0 \quad\left(a_0 \neq 0\right) \\
& x^{s-1}: a_1(s+1)(s)=0 \\
& x^s: a_2(s+2)(s+1)+a_0 w^2=0
\end{aligned}
$$

$\Rightarrow$ 若 $s=0, y \sim  a_0+a_1 x+\cdots$
若 $s=1, y \sim a_0 x+a_1 x^2+\cdots$
当 $s=1$时, $a_1=0$.
当 $s=0$ 时, $a_1$ 任意, 因此我们取$a_1=0$.
$$
\begin{gathered}
\Rightarrow a_{j+2}(s+j+2)(s+j+1)+w^2 a_j=0 \\
\Rightarrow a_{j+2}=\frac{-w^2}{(s+j+2)(s+j+1)} a_j
\end{gathered}
$$

递推关系.
$$
\begin{aligned}
& \text { 若 } s=0 \Rightarrow a_{j+2}=\frac{-\omega^2}{(j+2)(j+1)} a_j \\
& a_2=-\frac{w^2}{2 !} a_0 \\
& a_4=-\frac{a_2}{3^3 4}=+\frac{w^4}{4 !} a_0 \\
& a_6=-\frac{a_0^{3.4}}{5.6}=-\frac{w 6}{6 !} a_0 \\
&
\end{aligned}
$$

$$
a_{2 n}=(-1)^n \frac{w^{2 n}}{2 n !} a_0
$$

$$
\begin{aligned}
y & \left.=a_0\left[1-\frac{(\omega x)^2}{2 !}+\frac{(\omega x}{4}\right)^4-\frac{\omega x x^6}{6 !}+\omega\right] \\
& =a_0 \cos \omega x .
\end{aligned}
$$

$$
\begin{aligned}
& \text { 若 } s=1 . \Rightarrow \\
& a_{j+2}=-a_j \frac{\omega^2}{(j+3) (j+2)} \\
& \Rightarrow a_2=-a_0 \frac{\omega^2}{2 \cdot 3}=\frac{\omega^2}{3 !} a_0 \\
& a_4=-a_2 \frac{\omega^2}{5 \cdot 4}=\frac{\omega^4}{5 !} a_0 \\
& a_6=-a_4 \frac{\omega^2}{6\cdot 7}=\frac{w^6}{7 !} a_0 \\
\end{aligned}
$$

$$
\begin{aligned}
& \Rightarrow \quad a_0 x\left[1-\frac{(\omega x)^2}{3 !}+\frac{(\omega x)^{4}}{5 !}-\frac{(\omega x)^6}{7 !} \cdots\right] \\
& =\frac{a_0}{\omega}\left[(\omega x)-\frac{(\omega x)^3}{3 !}+\frac{(\omega x)^5}{5 !} \cdots\right] \\
& =\frac{a_0}{\omega} \sin \omega x
\end{aligned}
$$
此方法称为Frobenius方法,上面关于$x_0$上展开的,一般的
我们可以在 $x_0$处展开
$$
y(x)=\sum_{j=0}^{\infty} a_j\left(x-x_0\right)^{s+j}, a_0 \neq 0 .
$$


\subsection{奇点邻域上的级数求解}
$$
\begin{aligned}
y_1(z) & =\sum_{k=-\infty}^{\infty} a_k\left(z-z_0\right)^{s_1+k} \cdot \\
y_2(z) & =\sum_{k=-\infty}^{\infty} b_k\left(z-z_0\right)^{s_2+k} . \\
\text{或} y_2(x) & =A y_1(z) \ln \left(z-z_0\right) \\
& +\sum_{k=\infty}^{\infty} b_k\left(z-z_0\right)^{s_2+k}
\end{aligned}
$$


级数解有限个负幂项称为正则奇点 $p(z): \sum_{k=-1}^{\infty}$
$q(z) : \sum_{k=-2}^{\infty}$

有判定方程,
$$
s(s-1)+s p_{-1}+q_{-2}=0
$$
其中$s_1, s_2$ 为两根($s_1> s_2$).
朗斯基行列式(Wronskian)
$$
\Delta(z)=\left|\begin{array}{ll}
y_1(z) & y_2(z) \\
y_1'(z) & y_2'(z)
\end{array}\right|= y_1(z) y_2'(z) - y_2(z) y_1'(z)
$$

$$
\begin{aligned}
& \left(y_1 y_2''-y_1'' 
y_2\right)+p\left(y_1 y_2'-y_1' y_2\right) = 0\\
& \frac{d \Delta}{d z} - p \Delta(z)=0
\end{aligned}
$$

$$
\Delta(z)=\Delta_0 e^{-\int p(x) d x}
$$

$$
\begin{aligned}
& \frac{d}{d z}\left(\frac{y_{2}}{y_1}\right)=\frac{y_{1} y_{2}'-y_{1}' y_{2}}{y_{1}^2}=\frac{\Delta(z)}{y_{1}^2} \\
& y_{2}=y_{1} \int \frac{\Delta(z)}{\left(y_{1}(z)\right)^2} d z
\end{aligned}
$$

$\nu$阶贝塞尔方程的级数求解参考书上.