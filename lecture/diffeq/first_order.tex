\section{一阶常微分方程}
下面讨论一阶微分方程
\begin{equation}
    y' = f(x,y),
\end{equation}
或写成对称形式的一阶微分方程
\begin{equation}
    p(x,y) dx + q(x,y) dy = 0,
\end{equation}
这里$f(x,y), p(x,y), q(x,y)$均表示含$x,y$两个变量的解析表达式. 当微分方程写为这种对称形式时,
变量$x,y$处于平等地位, 可将$x$或$y$看成是未知函数.
\subsection{可分离变量的方程.}
$$
\begin{gathered}
\frac{d y}{d x}=-\frac{p(x)}{q(y)} \\
\Rightarrow p(x) d x+q(y) d y=0 . \\
\int p(x) d x+\int q(y) d y=C .
\end{gathered}
$$
这里的积分常数已单独写出.两个不定积分的原函数分别是$P(x), Q(y)$, 则上式的通解为
$P(x) + Q(y) = C$.

分离变量的方程其实不需要微分方程线性.
考虑全微分形式
$$
\begin{aligned}
& p(x, y) d x+q(x, y) d y=0 \\
& d \varphi=\frac{\partial \varphi}{\partial x} d x+\frac{\partial \varphi}{\partial y} d y=0 .
\end{aligned}
$$

$$
\left\{\begin{array}{l}
\frac{\partial \varphi}{\partial x}=p(x, y) . \\
\frac{\partial \varphi}{\partial y}=q(x, y) .
\end{array}\right.
$$
若 $\varphi$ 满足上式,则 $\varphi(x, y)=C$.
当且仅当$\frac{\partial p(x, y)}{\partial y} = \frac{\partial q(x, y)}{\partial x}$
\begin{equation}
\varphi(x, y)=\int_{x_0}^x p(x, y) d x+\int_{y_0}^y q(x_0, y) \equiv C  
\end{equation}

$$
\begin{aligned}
    & \text { 可分离 } \Rightarrow \text { 全微分 } \\
    & \text { 但全微分} \nRightarrow \text{可分离} \\
\end{aligned}
$$
% \subsection{}

\begin{example}
$$
y'+\left(1+\frac{y}{x}\right)=0
$$
\end{example}
\begin{solution}
$$
\Rightarrow(x+y) d x+x d y=0
$$
$$
\text { 由于 } \frac{\partial p}{\partial y}=1, \frac{\partial q}{\partial x}=1
$$

可得 $d \varphi=0$
$\Rightarrow \varphi=\int_{x_0}^x(x+y) d x+\int_{y_0}^y x_0 d y$

\begin{equation}
    \begin{aligned}
    \varphi & =\left(\frac{x^2}{2}+x y\right)-\left(\frac{x_0^2}{2}-x_0 y\right)+x_0 y-x_0 y_0 \\
    & =\frac{x^2}{2}+x y+c \\
    & \Rightarrow \frac{x^2}{2}+x y=c
    \end{aligned}
    \end{equation}
\end{solution}

下面考虑可以换元法解决的方程, 它的形式为
\begin{equation}
    \frac{dy}{dx} = f(a x + by + c), b\neq 0
\end{equation}
我们可以采用换元$u = ax + by$, 将$u$作为未知函数,有$a dx + b dy = du $,
可以化为
\begin{equation}
    du  = ( a + b f(u+c) ) dx,
\end{equation}
变成了可分离变量形式. 我们可得$u  = \varphi (x, C)$, 原方程解为
\begin{equation}
    y = \frac{1}{b} \left(\varphi(x, C) - a x \right).
\end{equation}

\subsection{权重法}
令 $y=x^{m}v$
$$
\begin{aligned}
& \left(x^2-y\right) d x+x d y=0 . \\
& x \text { 权重为 } 1, y\text { 权重为 }m \\
& x^2 d x \text { 权重为 } 3:
\end{aligned}
$$
$-ydx,  x d y$ 权重为 $1+m$.为了配平权重
$$
1+m=3 \Rightarrow m=2 \text { . }
$$


$$
\begin{aligned}
& y=x^2 v \text { 代入. } \\
& (1-v) d x+x d v=0 \\
& \frac{d x}{x}+\frac{d v}{v+1}=0 . \\
& \Rightarrow \ln x+\ln (v+1)=c \\
& \Rightarrow x(v+1)=c \\
\end{aligned}
$$
得到最终解
\[
    y=x^2 v=x^2\left(\frac{c}{x}-1\right) = -x^2+c x 
\]

\subsection{一阶线性方程}
一般一阶线性常微分方程的标准形式为
\begin{equation}
    \frac{d y}{d x}+p(x) y=q(x).
\end{equation}
当$q(x) \equiv 0$时, 上式为特殊形式
\begin{equation}
    \frac{d y}{d x}+p(x) y = 0.
\end{equation}
成为\textbf{一阶线性齐次方程}

两边同时乘以 $\alpha(x)$ .
\begin{equation}
     \alpha \frac{d y}{d x}+\alpha p y=\alpha q 
\end{equation}
而对$\alpha y$求导有
\begin{equation}
    \frac{d}{d x}(\alpha y)=\frac{d \alpha}{d x} y+\alpha \frac{d y}{d x} .
\end{equation}
为了使左式是一个全微分, 比较两式要求
$\frac{d \alpha}{d x}=\alpha p $, 即 $\frac{d \alpha}{\alpha}=p d x $. 于是有
\begin{equation}
    \alpha(x)=e^{\int p(x) d x}.
\end{equation}

回到原问题,需求
$$\frac{d}{d x}(\alpha y)=\alpha q $$
即
\begin{equation}
    y=\frac{1}{\alpha(x)}\left[\int^x \alpha(t) q(t) dt+C\right] \equiv y_1(x)+y_2(x)
\end{equation}
这里,我们将该解分成两个部分
\begin{equation}
\begin{aligned}
& y_1(x)=\frac{1}{\alpha(x)} \int^x \alpha(t) q(t) d t \\
& y_2(x)=\frac{C}{\alpha(x)}
\end{aligned}
\end{equation}
可以看出, $y_2(x)=\frac{c}{\alpha(x)}$是齐次方程的通解,
而 $y_1(x)=\frac{1}{\alpha(x)} \int^x \alpha(t) q(t) d t$是方程的特解.
一般地, 微分方程的解是\textbf{特解}加\textbf{通解}的形式.




\subsection{ 常系数的常微分方程 (ODE)}
常系数的常微分方程的标准形式为
\begin{equation}
\frac{d^n y}{d x^n}+a_{n-1}\frac{d^{n-1} }{d x^{n-1}} y  + \cdots +a_{1} \frac{d}{d x} y+a_0 y=F(x)
\end{equation}
其中$a_i$为常数.
$$
\Rightarrow \text { 解的形式为 } y=e^{m x}
$$

其中$m$需满足代数方程
$$
m^n+a_{n-1} m^{n-1}+\cdots a_1 m+a_0=0
$$

对于二阶常系数线性齐次方程, $$
y'' + a y' + by = 0,
$$
代数方程化成
\begin{equation}
    m^2 + a m + b = 0
\end{equation}
该方程成为微分方程的\textbf{特征方程},其解为\textbf{特征根}. 如果两根不同
则有通解
\begin{equation}
    y = C_1 e^{m_1 x} +  C_2 e^{m_2 x} 
\end{equation}
若特征根为二重实根$m$, 则解为 
\begin{equation}
    y = (C_1 + C_2  x) e^{m x}
\end{equation}
% 如果有两个
\begin{example}
    谐振子方程
$$
\begin{aligned}
& m \frac{d X^2}{d t^2}=-k X(t) \\
& X''+a_0 X=0, a_0=\frac{k}{m} . \\
& m^2+a_0=0 \\
& \Rightarrow m= \pm i \sqrt{a_0}= \pm i \omega \quad \omega=\sqrt{\frac{k}{m}} \\
& y(t)=C_1 ^{+i \omega t}+D_{1} e^{-i \omega t} . \\
& \text { 或 } C_1 \cos \omega t+D_1 \sin \omega t
\end{aligned}
$$
\end{example}

