\subsection{泰勒级数(Taylor Series)和洛朗级数(Laurent Series)}
\label{subsec:taylor_laurent_series}
同实变函数泰勒级数展开一样,复变函数中解析函数自然也可以展为复变项的\textbf{泰勒级数}(Taylor series).
若$f(z)$在以$z_0$为圆心半径为$R$的圆$C_R$内(含圆上)解析,对圆内任意$z$,$f(z)$可以展为
\begin{equation}
    f(z) = \sum_{k=0}^{\infty} a_k (z-z_0)^k,
\end{equation}
其中
\begin{equation}
    a_k=\frac{1}{2 \pi \imath} \oint_{C_{R}} \frac{f(\zeta)}{\left(\zeta-z_0\right)^{k+1}} d \zeta
    =\frac{f^{(k)}\left(z_0\right)}{k !} ,
\end{equation}
且该展开唯一.

以上的证明如下.利用$1/(\zeta -z)$围绕$z_0$为圆心的几何级数,构造
\[
\frac{1}{\zeta -z} = \frac{1}{(\zeta -z_0) - (z - z_0)} = \frac{1}{\zeta - z_0}\frac{1}{ 1 - \frac{z-z_0}{\zeta - z_0}}
 = \sum_{k=0}^{\infty}\frac{(z-z_0)^k}{(\zeta - z_0)^{k+1}}.    
\]
将上式带入柯西公式有
\begin{equation}
    \begin{aligned}
        f(z) &= \frac{1}{2\pi \imath} \oint_C \frac{f(\zeta)}{\zeta - z} d \zeta
        \\
        &= \frac{1}{2\pi \imath} \sum_{k=0} ^{\infty} \oint_C \frac{f(\zeta)(z-z_0)^k}{(\zeta - z_0)^{k+1}} d \zeta
        \\
        &= \frac{1}{2\pi \imath} \sum_{k=0} ^{\infty}(z-z_0)^k 2\pi \imath \frac{f^{(k)}(z_0)}{k!}
        \\
        & = \sum_{k=0} ^{\infty}  \frac{f^{(k)}(z_0)}{k!} (z-z_0)^k
    \end{aligned}
\end{equation}
不难证明泰勒展开的唯一性.

以下是一个重要推论.若$f(z)$, $g(z)$在区域$R$解析,且在某些子区域$S\subset R$有$f(z)=g(z)$,那么全区域$R$内都有$f(z)=g(z)$.
该证明可通过构造$h(z) = f(z) - g(z)$然后利用该式在区域$R$内任意一点的的泰勒展开进行系数比较,可得$h(z)$在全区域内为零.
% \subsection{洛朗级数(Laurent Series)}
% \label{subsec:laurent_series}

\begin{examplebox}{在$z_0=1$的邻域上将$f(z) = \ln{z}$进行泰勒展开.}
    首先,$f(z) = \ln z $的支点为$0,\infty$.这里的展开中心为$z_0=1$非支点,各个单值分支互相独立.
    根据泰勒展开公式,我们需要将各个系数计算出来.
    \[
        \begin{array}{ll}
            f(z)=\ln z, & f(1)=\ln 1=n 2 \pi \imath \quad(n \text { 为整数 }) \text {; } \\
            f^{\prime}(z)=\frac{1}{z}, & f^{\prime}(1)=+1 ; \\
            f^{\prime \prime}(z)=-\frac{1 !}{z^2}, & f^{\prime \prime}(1)=-1 ! ; \\
            f^{(3)}(z)=\frac{2 !}{z^3}, & f^{(3)}(1)=+2 ! ; \\
            f^{(4)}(z)=-\frac{3 !}{z^4}, & f^{(4)}(1)=-3 ! ; \\
            \ldots & \cdots
            \end{array}
    \]
    于是,可以得到
    \[
    \ln z = 2 n \pi \imath + (z-1) - \frac{(z-1)^2}{2} +  \frac{(z-1)^3}{3} -  \frac{(z-1)^4}{4} \cdots
    \]
    按照比值判定法,该级数的收敛半径为$1$,需附加上$|z-1|< 1$这个收敛条件.我们称$n=0$的单值分支为$\ln z $的\textbf{主值}.
\end{examplebox}

目前为止,泰勒级数的展开的前提条件为区域内解析.然而,如果$f(z)$在区域内$z=z_0$的有奇点,那么泰勒展开就无法进行.假设$f(z)$仅在$z_0$有
一$p$阶\textbf{极点}(pole),其他区域均解析,我们有$g(z) = (z-z_0)^p f(z)$在区域内解析.因此,我们可以在$z_0$处进行泰勒展开
\begin{equation}
    g(z) = \sum_{k=0}^{\infty} b_k (z-z_0)^{k} .
\end{equation}
于是,我们可以将$f(z)$表示为
\begin{equation}
    f(z) = a_{-p} (z-z_0)^{-p} + \cdots a_{-1}(z-z_0)^{-1} + a_0 + a_{1} (z-z_0) + a_{2} (z-z_0)^2 + \cdots 
\end{equation}
作为泰勒级数的延伸,这样的级数我们称为\textbf{洛朗级数}(Laurent series).比较展开系数,不难得到$a_k = b_{k+p}$,
\begin{equation}
    a_k = b_{k+p} = \frac{1}{2\pi \imath} \oint \frac{g(\zeta)}{(\zeta - z_0)^{k+1+p}} d\zeta 
    = \frac{1}{2\pi \imath} \oint \frac{f(\zeta)}{(\zeta - z_0)^{k+1}} d\zeta
\end{equation}

注意:尽管含$z-z_0$的负幂项,$z_0$可能也可能不是函数$f(z)$的奇点,如$f(z) = 1/(z^2-1)$的洛朗级数展开为
$$
\frac{1}{z^2} + \frac{1}{z^4} + \frac{1}{z^6}\cdots,
$$
包含了无数多负幂项,但展开中心$z=0$本身却不是函数的奇点,函数的奇点在$z=\pm 1$处. 此外注意,虽然洛朗级数展开的系数同泰勒展开的系数公式
相同但不论$z_0$是否为$f(z)$的奇点,
$$a_k \neq \frac{f^{(k)}(z_0)}{k!}.
$$
如果$z_0$是奇点,$f^{(k)}(z_0)$不存在;若$z_0$不是奇点, $f^{(k)}(z_0)$存在, 上式成立的条件是在以$C$为边界的区域内解析,而现在该区域(环状)内是有
奇点的(否则就不需要进行洛朗展开了).

洛朗级数所有幂次$k\geq 0$的项称为\textbf{解析部}(analytic part), 其余由$z-z_0$负幂次项称为\textbf{主部}(principal part).解析部的收敛半径记为
$R_1$.如果
主部收敛,那么对$|(z-z_0)^{-1}|$比某半径小的圆内收敛.若该圆半径为$1/R_2$,可以知当$|z-z_0|> R_2$,主部收敛.
如果$R_2 < R_1$那么级数在环域$R_2 < |z- z_0| < R_1$内绝对一致收敛,级数求和为解析函数,级数可以逐项求导.该环域称为\textbf{收敛环}.
如果$R_2 > R_1$那么该级数发散.

\begin{examplebox}{在$z_0 = 0$的邻域上将$e^{1/z}$展开.}
指数$e^z$的泰勒展开为
\[
    \mathrm{e}^z=\sum_{k=0}^{\infty} \frac{1}{k !} z^k=1+\frac{1}{1 !} z+\frac{1}{2 !} z^2+\frac{1}{3 !} z^3+\cdots \quad(|z|<\infty),    
\]
替换$z$成$1/z$得到
\[
    \mathrm{e}^{\frac{1}{z}}=\sum_{k=0}^{\infty} \frac{1}{k !} z^k=1+\frac{1}{1 !} \frac{1}{z}+\frac{1}{2 !} \frac{1}{z^2}+\frac{1}{3 !} \frac{1}{z^3}+\cdots \quad(|\frac{1}{z}|<\infty),    
\]
即有
\[
e^{\frac{1}{z}} = \sum_{k=-\infty}^{0} \frac{1}{(-k)!} z^k ( |z| > 0) .
\]
\end{examplebox}