\subsection{柯西积分公式}
\label{subsec:cauchy_formula}
复变函数理论中最重要的一个公式是柯西积分公式.若$f(z)$在一闭合回路围成的区域内解析,$z_0$为区域内一点,
则有
\begin{equation}
    f\left(z_0\right)=\frac{1}{2 \pi i} \oint_C \frac{f(z)}{z-z_0} d z .  
\end{equation}
柯西公式将解析函数在任何一内点 $z_0$ 的值 $f(z_0)$ 用沿边界线 $l$ 的回路积分 表示了出来.
也就是说,一个解析函数在闭合回路$C$内任意点$z_0$的值完全由该路径上的值决定的. 
这看起来不可思议.从物理上说, 解析函数紧密联系于平面标量场, 而平面场的边界条件决定着区域内部的场.
我们可以通过前面提到的柯西定理来证明.我们需要围绕$z_0$选取一个半径为$r$的内圆$\gamma$,由于$f(z)/(z-z_0)$在
内圆与$C$形成的区域解析,故我们有
\[ \frac{1}{2 \pi \imath} \oint_C \frac{f(z)}{z-z_0} d z = \frac{1}{2 \pi \imath} \oint_\gamma \frac{f(z)}{z-z_0} d z
    \]
有$z-z_0 = re^{\imath \theta}$,于是
\begin{equation}
    \begin{aligned}
        I &= \oint_\gamma \frac{f(z)}{z-z_0} d z
        \\
        & = \int _0 ^{2\pi} \frac{f(z_0 + re^{\imath \theta})}{re^{\imath \theta}} ire^{\imath \theta} d\theta
        \\
        & = \imath \int _0 ^{2\pi} f(z_0 + re^{\imath \theta}) d\theta
    \end{aligned}
\end{equation}
现令内圆半径$r\to 0$,则有$I\to 2\pi\imath f(z_0)$,得证.
由于$z_0$的任意性,我们可以改写成$z$,而把积分变量改成$\zeta$,则有
\begin{equation}
    f(z) = \frac{1}{2\pi \imath} \oint_C \frac{f(\zeta)}{\zeta - z} d \zeta.
    \label{eq:cauchy_formula}
\end{equation}

对柯西公式(\ref{eq:cauchy_formula})求导,我们有
\begin{equation}
    f'(z) = \frac{1!}{2\pi \imath} \oint_C \frac{f(\zeta)}{(\zeta - z)^2} d \zeta.
    \label{eq:cauchy_formula_1st_derivative}
\end{equation}
反复求导则有
\begin{equation}
    f^{(n)}(z) = \frac{n!}{2\pi \imath} \oint_C \frac{f(\zeta)}{(\zeta - z)^{n+1}} d \zeta.
    \label{eq:cauchy_formula_nth_derivative}
\end{equation}
% 这是因为解析函数在各点的值通过柯西 - 黎曼方程相互联系着. 

下面介绍柯西公式的重要推论.

\textbf{模数原理} $f(z)$在闭区域上解析,$|f(z)|$只能在边界线上取极大值.

由$f(z)^n = \frac{1}{2\pi \imath} \oint_C \frac{f(\zeta)^n}{\zeta - z} d \zeta$,若$|f(\zeta)|$在$C$上极大值为$M$,
$|\zeta - z|$的极小值为$\delta$, $C$的长度为$s$,则
\[
  |f(z)|^n \leq \frac{1}{2\pi} \frac{M^n}{\delta} s  ,
\]
即
\[
    |f(z)| \leq M \left( \frac{s}{2\pi \delta} \right)^{\frac{1}{n}},
\]
令$n\to \infty$,$|f(z)| \leq M$.证毕.

\textbf{刘维尔(Liouville)定理} \quad 如$f(z)$在全平面上解析且有界,则$|f(z)|$必为常数.

\textbf{证} \quad $f$有界,即$|f(z)| \leq N$, 对$f'(z)$取模,取以$z$为圆心半径为$R$的圆周,可得
\[
  |f'(z)| \leq \frac{1}{2\pi} \frac{N} {R^2} 2\pi R = \frac{N}{R},
\]
由于$R$任意选定,令$R\to \infty$,有$f'(z) \equiv 0$,所以$f(z)$为常数.