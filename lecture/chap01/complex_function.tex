
% \subsubsection{}
\subsection{复变函数}
\label{sub:complexfunctions}

\subsubsection{定义}
\label{subsub:cmplx_func_def}

存在复数平面的点集$Z$,每一点$z\in Z$有一个或多个复数值$w$与之对应,则称$w$为$z$的函数--复变函数。$z$称为$w$的宗量,定义域为$Z$,记作
\begin{equation}
    w = w(z)\textrm{,} z\in Z \textrm{。}
\end{equation}
任意一个复变函数$w(z)$,$z=x + \imath y$,我们可以写称实部和虚部的组合,
\begin{align}
    w(z) = u(x,y) +\imath \; v(x,y) \textrm{,}
\end{align}
其中$u(x,y), v(x,y)$为纯实函数。它们可以类似的写成
\begin{align}
    \Re w(z) = u(x,y)\textrm{,} \quad \Im w(z) = v(x,y) \textrm{,}
\end{align}
$w(z)$的复共轭为$u(x,y) - \imath \; v(x,y)$。取决于$w(z)$,二者可能相等也可能不等。

这里我们列举一些常见复变函数。
\begin{itemize}
    \item 多项式:
        \begin{equation}
            a_0 + a_1 z + a_2 z^2 + \cdots + a_n z^n \textrm{,} \quad n\in \mathbb{Z}^+ \textrm{,}
        \end{equation}
    \item 有理分式:       
         \begin{equation}
        \frac{a_0 + a_1 z + a_2 z^2 + \cdots + a_n z^n}{{b_0 + b_1 z + b_2 z^2 + \cdots + b_m z^m}} \textrm{,} \quad  n,m\in \mathbb{Z}^+ \textrm{,}
        \end{equation}
    \item 根式:
        \begin{equation}
            (z-a)^{m/n} \textrm{,} \quad  n,m\in \mathbb{Z}^+ \textrm{,}
        \end{equation}
    \item 对数、指数
        \begin{equation}
            \ln z = \ln |z| + \imath \Arg z, \quad z^s = e^{s\ln z} \textrm{,}
        \end{equation}
    \item 正余弦,正余切函数 
        \begin{equation}
            \sin z , \cos z , \tan z, \cot z \textrm{,}
        \end{equation}
    \item 双曲正余弦, 双曲正余切函数
        \begin{equation}
            \sinh z , \cosh z , \tanh z, \coth z  \textrm{。}
        \end{equation}
\end{itemize}
以上所有出现的常数均为复数。

\begin{examplebox}{验证\begin{equation*}
    |\sin z|=\frac{1}{2} \sqrt{\left(e^{2 y}+e^{-2 y}\right)+2\left(\sin ^2 x-\cos ^2 x\right)} .
    \end{equation*}
    }
    由正弦函数定义得
    \begin{align*}
        \sin z &= \frac{e^{\imath z} - e^{-\imath z}}{2\imath} 
        \\ 
        & = \frac{e^{\imath x - y} - e^{-\imath x + y}}{2\imath}
        \\
        & = \frac{1}{2\imath}\left( e^{-y} (\cos x + \imath \sin x ) - e^{y} (\cos x - \imath \sin x ) \right) 
        \\
        & = \frac{1}{2\imath} \left( \cos x (e^{-y} - e^{y}) + \imath \sin x (e^{-y} + e^{y}) \right)
    \end{align*}
    取模后可得,
    \begin{align*}
        |\sin z | &= \frac{1}{2}\sqrt{ \cos^2x (e^{2y} + e^{-2y} -2) + \sin^2 x (e^{2y} + e^{-2y} +2) }
        \\
        &=\frac{1}{2}\sqrt{\left(e^{2 y}+e^{-2 y}\right)+2\left(\sin ^2 x-\cos ^2 x\right)}
    \end{align*}
    可见,与实函数不同的是,$|\sin z|$的取值完全可以大于$1$。
\end{examplebox}

复数
\begin{figure}
    \centering

    \input{tikz/branch_cut.tex}

\end{figure}
\section{解析函数}