\subsection{幂级数(Power Series)}
本讲专门讨论在函数项级数中非常重要的一类级数,叫做幂级数.幂级数的各项都是幂函数,
\begin{equation}
    \sum_{k=0}^{\infty} a_k\left(z-z_0\right)^k=a_0+a_1\left(z-z_0\right)+a_2\left(z-z_0\right)^2+\cdots
\end{equation}
其中$z_0, a_i$都是复常数.这样的级数称为以 $z_0$ 为中心的幂级数.
该幂级数可能收敛,也可能发散.如果幂级数是收敛的,称$z$为该级数的\textbf{收敛点};反之若它是发散的,称$z$为该级数的\textbf{发散点}.
函数项级数式的所有收敛点的集合称为其\textbf{收敛域},所有发散点的集合称为其\textbf{发散域}.

现在利用达朗贝尔比值判定法,如果
\begin{equation}
    \lim _{k \rightarrow \infty} \frac{\left|a_{k+1}\right|\left|z-z_0\right|^{k+1}}{\left|a_k\right|\left|z-z_0\right|^k}
    =\lim _{k \rightarrow \infty}\left|\frac{a_{k+1}}{a_k}\right|\left|z-z_0\right|<1
\end{equation}
则级数绝对收敛.若记
\begin{equation}
    % \lim_{k \rightarrow \infty} \left|\frac{a_{k+1}}{a_k}\right| = R^{-1},
    \lim_{k \rightarrow \infty} \left|\frac{a_{k}}{a_{k+1}}\right| = R,
\end{equation}
那么当$|z-z_0| < R$时,级数绝对收敛.若$|z-z_0| > R$时, 级数的模越来越大,级数发散.对于$|z-z_0| = R$的时候,无法简单判定.
例如幂级数$\sum_{k=0} \frac{1}{k} |z-z_0|^k$,可知$R=1$.$|z-z_0|=1$有两种情况:当$z-z_0 = +1$时, 由调和级数知道,该级数发散;而
当$z-z_0 = -1$时,由莱布尼兹判定法可知该级数收敛.
对于有些幂级数,根式判别法的使用有时更加简便,应当在这两种方法灵活选择.
根式判别法的收敛半径可由
\begin{equation}
    R = \lim_{k \rightarrow \infty} \frac{1}{\sqrt[k]{a_k}}
\end{equation}
得到.

以$z_0$ 为圆心作一个半径为$R$的圆$C_R$.幂级数在圆的内部绝对收敛, 在圆外发散. 这个圆因而称为幂级数的\textbf{收敛圆}, 
它的半径则称为\textbf{收敛半径}.至于在收敛圆周上的收敛情况则需要具体分析.

\begin{examplebox}{求幂级数$1 - z^2 + z^4 - z^6\cdots$的收敛圆,$z$为复变数.}
    将$z^2$记作$t$,则本例的级数化成$1-t + t^2 - t^3\cdots$.系数$a_k =(-1)^k$, 因此t平面上收敛半径$R=1$. 
    $z$平面上以$z=0$为圆心,收敛半径为$\sqrt{R}=1$.
\end{examplebox}

% 利用柯西公式(\ref{eq:cauchy_formula}), 我们将$\frac{1}{2\pi \imath} \frac{1}{\zeta - z}$遍乘幂级数的每一项得到
% \[
%     \frac{1}{2 \pi \imath} \frac{a_0}{\zeta-z}+\frac{1}{2 \pi \imath} \frac{a_1\left(\zeta-z_0\right)}{\zeta-z}+\frac{1}{2 \pi \imath} \frac{a_2\left(\zeta-z_0\right)^2}{\zeta-z}+\cdots 
% \]
% 在收敛圆上逐项积分得到
