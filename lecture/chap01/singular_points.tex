\subsection{奇点的分类}
\label{subsec:singular_points}
利用函数$f(z)$的在$z=z_0$处的洛朗级数展开,我们可以刻画函数在此处的行为.
容易知道,若$f(z)$在$z=z_0$处解析,那么所有负幂项系数为零,即$a_k = 0$, $k<0$.
若$f(z)$在$z=z_0$处非解析,那么两种情况会出现.
\begin{enumerate}
    \item 能找到整数$p$使得$a_{-p} \neq 0$, 对所有$k>0$有$a_{-p - k}=0$.
    \item 无法找到上述最小的$-p$.
\end{enumerate}
对于第一种情况,$f(z)$在$z=z_0$处有$p$阶极点.$a_{-1}$称为$f(z)$在极点$z=z_0$处的\textbf{留数}(residue).
对于第二种情况,如果有无穷多负幂项,我们称之为\textbf{本性奇点}(essential singularity).

\begin{examplebox}{找出\[f(z) = \frac{1}{z( z- 2)^3}\]
    的在$z=0$和$z=2$处的洛朗级数展开,并验证$z=0$时一阶极点,$z=2$是三阶极点,并找出两个极点的留数.}
对于极点$z=0$处的洛朗级数展开,我们可以利用$(1-\alpha z)^n$的泰勒展开.有
\[
    \begin{aligned}
    f(z) &= -\frac{1}{8z(1-z/2)^3}
    \\
     &= -\frac{1}{8z}\left[ 1 + (-3) (-\frac{z}{2}) + \frac{(-3)(-4)}{2!} \left( -\frac{z}{2}\right)^2 + \cdots \right] 
    \\
     &= -\frac{1}{8z} - \frac{3}{16} - \frac{3z}{16} + \cdots   
    \end{aligned}
\]
由定义可知,$z=0$为一阶奇点,留数为$-1/8$.对于$z=2$的洛朗级数展开,我们将函数写成
\[
    \begin{aligned}
        f(z) &= \frac{1}{(z-2)^3} \frac{1}{z-2 + 2}
        \\
        &= \frac{1}{2(z-2)^3} \frac{1}{1+\frac{z-2}{2}}
        \\
        &= \frac{1}{2(z-2)^3} \left[ 1 - \left(\frac{z-2}{2} \right) + \left(\frac{z-2}{2}\right)^2 - \left(\frac{z-2}{2}\right)^3 \cdots \right] 
        \\
        &= \frac{1}{2(z-2)^3} - \frac{1}{4(z-2)^2} + \frac{1}{8(z-2)} - \frac{1}{16} + \cdots
    \end{aligned}
\]
由定义可知,$z=2$为三阶极点,留数为$1/8$.
\end{examplebox}
% 若函数$f(z)$在某点$z_0$不可导,但在$z_0$的任意小邻域内除$z_0$外处处可导,我们称$z_0$为$f(z)$的\textbf{孤立奇点}.
% 若在$z_0$的邻域内可以找到其他不可导的点,则称之为\textbf{非孤立奇点}.
% 去除$z_0$,在某环域上解析函数$f(z)$的洛朗级数展开,可以分为三种情况.
% \begin{enumerate}
%     \item 没有负幂项,称为\textbf{可去奇点}, 如$\sin{z}/z$.
%     \item 有限个负幂项,称为\textbf{极点}, 如$1/(z^2-1)$.
%     \item 无穷多个负幂项,称为\textbf{本性奇点}, 如$e^{1/z}$.
% \end{enumerate}

% 如果 $z_0$ 是 $f(z)$ 的可去奇点, 则在以 $z_0$ 为圆心而内半径为零的圆环域 $0<$ $\left|z-z_0\right|<R$ ( $R$ 是某个有限或无限的数值) 上的洛朗展开为
% $$
% f(z)=a_0+a_1\left(z-z_0\right)+a_2\left(z-z_0\right)^2+\cdots \quad\left(0<\left|z-z_0\right|<R\right) .
% $$
% 据此显然有
% $$
% \lim _{z \rightarrow z_0} f(z)=a_0
% $$
% 是有限的. 即函数在可去奇点的邻域上是有界的.

% 其实, 如果定义函数 $g(z)$ 以代替 $f(z)$,
% $$
% g(z)= \begin{cases}f(z) & \left(z \neq z_0\right), \\ a_0 & \left(z=z_0\right),\end{cases}
% $$
% 则由 (3.6.2) 得
% $$
% g(z)=a_0+a_1\left(z-z_0\right)+a_2\left(z-z_0\right)^2+\cdots \quad\left(\left|z-z_0\right|<R\right) .
% $$
% 这就是 $g(z)$ 在 $z_0$ 邻域上的泰勒展开, $z_0$ 不再是函数 $g(z)$ 的奇点. 这正是 “可去奇点” 一词的来历. 可去奇点今后将不作为奇点看待.