\documentclass[tikz,border=10pt]{standalone}
\usepackage{ctex}

\begin{document}
\begin{tikzpicture}[scale=1.5]
  \draw[very thick, ->] (-2,0) -- (2,0) node[right] {$\Re(z)$};
  \draw[very thick, ->] (0,-2) -- (0,2) node[above] {$\Im(z)$};
  \draw[black, thick] (0,0) circle (1.5);
  \node[black, right] at (0.7,0.7) {$C$};

% Inner circles to avoid the singular points
\draw[black, thick, dotted] (0.5,0) circle (0.1);
\draw[black, thick, dotted] (-0.8,0.8) circle (0.1);
\draw[black, thick, dotted] (0,-1) circle (0.1);


%   \draw[black, thick, ->] (0.4,1.3) arc (120:240:1.4);
  \node[black, below] at (-0.6,-1.2) {$\gamma$};
  \node[black, above] at (1.5,0.8) {$f(z)$};
%   \node[black, below] at (-0.5,0.5) {$z_0$};
  \node[black, above] at (0,2.5) {数学物理方法};
%   \node[black, above] at (0,1.8) {\large using the Residue theorem};
%   \node[black, above] at (0,-1) {\large Your Name};
%   \node[black, above] at (0,-1.5) {\large Date};
  \node[black, below] at (-0.8,-2) {\large$\oint_C f(z) dz = 2\pi i\sum_{k=1}^{n} Res(f, z_k)$};
  
  \filldraw[black] (0.5,0) circle (0.05);
  \node[black, below] at (0.5,0) {$z_1$};
  \filldraw[black] (-0.8,0.8) circle (0.05);
  \node[black, above] at (-0.8,0.8) {$z_2$};
  \filldraw[black] (0,-1) circle (0.05);
  \node[black, below] at (0,-1) {$z_3$};
\end{tikzpicture}
\end{document}
