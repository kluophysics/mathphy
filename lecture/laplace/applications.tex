\subsection{拉普拉斯积分变换的应用}
\label{subsec:applications}
拉普拉斯变换有很多应用,这里从三个方面的例子来说明.
\subsubsection{计算级数和}
有时,拉普拉斯变换可以用来计算某些级数的和.以下面的例子说明.

如计算级数和
$$
    \sum_{n=1} ^{\infty}  \frac{1}{n^2},
$$
由前面例题
$$
\int_0^{\infty} t e^{-p t} d t=\frac{1}{p^2}, \quad \Re p>0
$$

将级数化为
$$
\begin{aligned}
\sum_{n= 1}^{\infty} \frac{1}{n^2} & =\sum_{n=1}^{\infty} \int_0^{\infty} t e^{-n t} d t \\
& =\int_0^{\infty} t\left[\sum_{n=1}^{\infty} e^{-n t}\right] d t=\int_0^{\infty} \frac{t}{e^t-1} d t
\end{aligned}
$$
查表可得
\begin{equation}
    \sum_{n=1} ^{\infty}  \frac{1}{n^2} = \frac{\pi^2}{6}.
\end{equation}

\begin{note}
    Use Parseval's identity (applied to the function $f(x)=x$ ) to obtain

    $$
    \sum_{n=-\infty}^{\infty}\left|c_n\right|^2=\frac{1}{2 \pi} \int_{-\pi}^\pi x^2 d x
    $$
    
    where
    
    $$
    \begin{aligned}
    c_n & =\frac{1}{2 \pi} \int_{-\pi}^\pi x e^{-i n x} d x \\
    & =\frac{n \pi \cos (n \pi)-\sin (n \pi)}{\pi n^2} i \\
    & =\frac{\cos (n \pi)}{n} i \\
    & =\frac{(-1)^n}{n} i
    \end{aligned}
    $$
    
    for $n \neq 0$, and $c_0=0$. Thus,
    
    $$
    \left|c_n\right|^2= \begin{cases}\frac{1}{n^2}, & \text { for } n \neq 0 \\ 0, & \text { for } n=0\end{cases}
    $$
    
    and
    
    $$
    \sum_{n=-\infty}^{\infty}\left|c_n\right|^2=2 \sum_{n=1}^{\infty} \frac{1}{n^2}=\frac{1}{2 \pi} \int_{-\pi}^\pi x^2 d x
    $$
    
    
    Therefore,
    
    $$
    \sum_{n=1}^{\infty} \frac{1}{n^2}=\frac{1}{4 \pi} \int_{-\pi}^\pi x^2 d x=\frac{\pi^2}{6}
    $$
    
    as required.
\end{note}

\subsubsection{求解定积分}
如果 $\int_v^{\infty} \bar{f}(q) d q$ 存在, 且当$t\to 0$时, $|f(t)/t|$有界,则
\begin{equation}
    \int_{p}^{\infty} \bar{f}(q) dq \risingdotseq \frac{f(t)}{t} . 
\end{equation}
比如
$$
\frac{\sin \omega t}{t} \fallingdotseq \int_p^{\infty} \frac{\omega}{q^2+\omega^2} d q=\frac{\pi}{2}-\arctan \frac{p}{\omega}.
$$

当$p\to 0$时, 有
\begin{equation}
    \int_0^{\infty} \bar{f}(p) d p=\int_0^{\infty} \frac{f(t)}{t} d t
\end{equation}
一个例子如
$$
\int_0^{\infty} \frac{\sin t}{t} d t=\int_0^{\infty} \frac{1}{p^2+1} d p=\frac{\pi}{2}
$$
不仅如此, 有些积分无法用留数定理计算,如
$$
\int_0^{\infty} \frac{\cos a t-\cos b t}{t} d t \quad a>0, b>0
$$
使用以上等式可得

$$
\begin{gathered}
\int_0^{\infty} \frac{\cos a t-\cos b t}{t} d t \fallingdotseq \int_0^{\infty}\left(\frac{p}{p^2+a^2}-\frac{p}{p^2+b^2}\right) d p \\
=\left.\frac{1}{2} \ln \frac{p^2+a^2}{p^2+b^2}\right|^{\infty}=\ln b-\ln a .
\end{gathered}
$$

\subsubsection{求解微分方程}
给一个例子,利用拉普拉斯变换求解简谐振子方程的解.
\begin{example}
利用拉普拉斯变换求解简谐振子方程的解.
\end{example}
\begin{solution}
    质量为$m$的质点在弹性系数为$k$的弹簧牵引下做简谐运动,满足方程为
    $$
m \frac{d^2 X(t)}{d t^2}+k X(t)=0
$$
初始条件取
$$
X(0)=X_0, \quad X^{\prime}(0)=0
$$
应用拉氏变换到该方程上得到
$$
m \mathcal{L}\left\{\frac{d^2 X}{d t^2}\right\}+k \mathcal{L}\{X(t)\}=0
$$
用$x(p)$表示未知变换$\mathcal{L}\{ X(t)\}$,于是根据导数定理\eqref{eq:lapl_trans_derivative}有
$$
m p^2 x(p)-m p X_0+k x(p)=0,
$$
化简为
$$
x(p)=X_0 \frac{p}{p^2+\omega_0^2}, \quad \omega_0^2 \equiv \frac{k}{m} .
$$
查表得
$$
X(t)=X_0 \cos \omega_0 t.
$$
\end{solution}