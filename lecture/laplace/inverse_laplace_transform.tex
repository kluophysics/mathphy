\subsection{拉普拉斯反变换}
\label{subsec:inverse_laplace_transform}
前面对于拉普拉斯反变换(亦称逆变换)
\begin{equation}
    f(t) = \mathcal{L}^{-1} [ \bar{f}(p)],
\end{equation}
其具体表达式为
\begin{equation}
    f(t)=\frac{1}{2 \pi \imath} \int_{\sigma-\imath \infty}^{\sigma+\imath \infty} \bar{f}(p) e^{ p t} dp, 
\end{equation}
见式\eqref{eq:fourier_mellin_integral}.该式是著名的Bromwich积分(也叫做Riemann-Mellin积分).
由于像函数$\bar{f}(p)$是解析函数是$p$的解析函数,上述积分可以通过留数定理求得.这里需要用到推广的
约旦定理(具体证明略去).

\textbf{推广的约当引理} 设 $C_R$ 是以 $p=0$ 为圆心, 以 $R$ 为半径的圆周在直线 $\Re p=a(>0)$ 左侧的圆弧. 若当 $|p| \rightarrow \infty$ 时, $\bar{f}(p)$ 在 $\frac{\pi}{2}-\delta \leqslant \operatorname{Arg} p \leqslant \frac{3}{2} \pi+\delta$ 中一致趋于零 ( $\delta$ 是小于 $\frac{\pi}{2}$ 的任意正数), 则
$$
\lim _{R \rightarrow \infty} \int_{C_R} \bar{f}(p) e^{p t} d p=0 \quad(t>0) .
$$
由于所作路径是在$\Re p = a$左半平面,而右半平面$\bar{f}(p)$解析, 最终我们得到
\begin{equation}
    f(t) = \sum_{p_j \in \text{全平面}} Res [ \bar{f}(p_j) e^{p_j t}].
\end{equation}

\begin{example}
利用Bromwich积分求解$1/\sqrt{p}$的拉普拉斯变换原函数.
\end{example}
\begin{solution}
    求解参考梁昆淼书.
    被积函数有唯一的奇点 $p=0$, 这是极点型支点, 在考虑回路积分时, 
    不能像简单地加上圆弧 $C_R$, 因为在函数的多值区域不能作积分运算及应用留数定理. 
    为此, 先自 $p=0$ 沿负实轴至 $\infty$ 将 $p$ 平面切割开,
    再作如图回路, 然后在被积函数的一个单值分支上作积分运算,并应用留数定理. 
    被积函数在回路所围区域上无奇点,
    % 按柯西定理
    % 其中 $l_1$ 和 $l_2$ 分别沿着割线的下沿和上沿,
     圆弧 $C_R$ 和圆 $C_{\varepsilon}$均以原点为圆心, 
     半径分别为 $R$ 和 $\varepsilon$. 令 $R \rightarrow \infty$, 
     由推广的约当引理知, $\lim _{R \rightarrow \infty} \int_{C_R}=0$, 
     同时不难验证 $\lim _{s \rightarrow 0} \int_{C_s}=0$.
     于是,
$$
f(t)=\int_{a-\imath \infty}^{a+\imath \infty} \frac{e^{p t}}{\sqrt{p}} d p=-\frac{1}{2 \pi \imath} \lim _{R \rightarrow \infty}\left(\int_{l_1}+\int_{l_2}\right) \frac{e^{p t}}{\sqrt{p}} d p .
$$
$p$ 在割线下沿 $l_1$ 和上沿 $l_2$ 上的辐角分别为 $-\pi$ 和 $\pi$. 这样, 在 $l_1$ 上, $\sqrt{p}=\sqrt{|p|} e^{-\imath \pi / 2}=$ $-\imath \sqrt{|p|}$; 在 $l_2$ 上 $\sqrt{p}=\sqrt{|p|} e^{\imath \pi / 2}=\imath \sqrt{|p|}$. 于是,
$$
\begin{aligned}
f(t) & =-\frac{1}{2 \pi \imath} \int_0^{-\infty} e^{p t} \frac{1}{-\imath \sqrt{|p|}} d p-\frac{1}{2 \pi \imath} \int_{-\infty}^0 e^{p t} \frac{1}{\imath \sqrt{|p|}} d p \\
& =\frac{1}{\pi} \int_{-\infty}^0 e^{p t} \frac{1}{\sqrt{|p|}} d p .
\end{aligned}
$$
改用 $y=\sqrt{|p| t}$ 作为积分变数,
$$
\begin{aligned}
f(t) & =\frac{1}{\pi} \int_{\infty}^0 e^{-y^2} \frac{\sqrt{t}}{y}\left(-\frac{2 y}{t} d y\right) \\
& =\frac{2}{\pi \sqrt{t}} \int_0^{\infty} e^{-y^2} d y=\frac{2}{\pi \sqrt{t}} \cdot \frac{\sqrt{\pi}}{2}=\frac{1}{\sqrt{\pi t}} .
\end{aligned}
$$
\end{solution}

很多时候,我们并不总需要用上述方法求解,对于有些问题,可以使用利用拉普拉斯变换的性质结合常见函数的表达式来求解.
对于像函数的导数, $f(t)$是满足Laplace变换的充分条件,那么$\bar{f}(p)$在
$\Re p > \sigma_0$的半平面解析,
于是求$n$阶导数得
\begin{equation}
\bar{f}^{(n)}(p)=\frac{d^{n}}{d p^n} \int_0^{\infty} f(t) e^{-p t} d t=\int_0^{\infty}(-t)^n f(t) e^{-p t} d t
\end{equation}
所以有
\begin{equation}
    \bar{f}^{(n)}(p)\risingdotseq (-t)^n f(t)
\end{equation}
不难发现,

$$    \begin{aligned}
    & \frac{1}{p^2}=-\frac{d}{d p} \frac{1}{p} \risingdotseq  t \\
    & \frac{1}{p^3}=\frac{1}{2} \frac{d^2}{d p^2} \frac{1}{p} \risingdotseq \frac{1}{2} t^2 .
    \end{aligned}$$


例如, 利用留数性质可以用待定系数法分解
$$
    \begin{aligned}
    \frac{1}{p^3(p+\alpha)} & =\frac{1}{\alpha} \frac{1}{p^3}-\frac{1}{\alpha^2} \frac{1}{p^2}+\frac{1}{\alpha^3} \frac{1}{p}-\frac{1}{\alpha^3} \frac{1}{p+\alpha} \\
    & \risingdotseq \frac{1}{2 \alpha} t^2+\frac{1}{\alpha^2} t+\frac{1}{\alpha^3}-\frac{1}{\alpha^3} e^{-\alpha t}
    \end{aligned}
$$

\begin{example}
求 $\frac{e^{-\alpha p}}{p(p+b)}$ 的原函数.
\end{example}
\begin{solution}
    $\frac{1}{p} \risingdotseq \Theta(t)$, 应用延迟定理,
    $$
e^{-\alpha p} / p \risingdotseq \Theta(t-\alpha) \text {. 又 } 1 /(p+b) \risingdotseq e^{-b t} \text {. }
$$

这样, 本题给的像函数可以看作 $e^{-\alpha p} / p$ 与 $1 /(p+b)$ 的乘积, 应用卷积定理,即得
$$
\begin{aligned}
\frac{e^{-\alpha p}}{p(p+b)} & \risingdotseq \int_0^t \Theta(\tau-\alpha) e^{-b(t-\tau)} d \tau \\
& =\Theta(t-\alpha) \int_\alpha^t e^{-b(t-\tau)} d \tau \\
& =\Theta(t-\alpha)\left[\frac{1}{b} e^{-b(t-\tau)}\right]_\alpha^t \\
& =\frac{1}{b}\left[1-e^{-b(t-\tau)}\right] \Theta(t-\alpha) .
\end{aligned}
$$
读者也可以运用裂项分解分母来求解验证.
\end{solution}

