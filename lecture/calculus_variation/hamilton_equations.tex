\subsection{哈密顿方程}
经典物理里,拉式量$L(x_1, x_2, \cdots, x_n, \dot{x}_1, \dot{x}_2, \cdots, \dot{x}_n)|t)$是位置和速度的函数.\textbf{哈密顿原理}
说的是系统的运动从$t_1$到$t_2$使得拉式量的时间积分的变分为零.即
\begin{equation}
    \delta \int_{t_1}^{t_2} L\left(x_1, x_2, \ldots, x_n, \dot{x}_1, \dot{x}_2, \ldots, \dot{x}_n | t\right) d t=0 .
\end{equation}
于是有拉格朗日运动方程

\begin{equation}
    \frac{d}{d t} \frac{\partial L}{\partial \dot{x}_i}-\frac{\partial L}{\partial x_i}=0, \quad \forall i
\end{equation}

拉格朗日运动方程同牛顿运动方程是等价的.拉格朗日运动方程有一些优点.首先,依赖坐标不必是标准的坐标或长度,
它们可以根据物理问题具体选择,拉式运动方程对坐标系统是不变的.其次,拉式运动方程可以扩展到各个物理分支,如电磁场
和量子力学以及更远的规范场论等.此外,对于拉式量的变换,守恒律和对称性可以很容易通过诺特定理对应起来.


哈密顿注意到拉式运动方程可以简化为一系列偏微分方程,称为\textbf{哈密顿方程}.
正则动量定义为
\begin{equation}
    p_i=\frac{\partial L}{\partial \dot{q}_i}
\end{equation}

\begin{equation}
    d L=\sum_i\left(\frac{\partial L}{\partial q_i} d q_i+\frac{\partial L}{\partial \dot{q}_i} d \dot{q}_i\right)+\frac{\partial L}{\partial t} d t=\sum_i\left(\dot{p}_i d q_i+p_i d \dot{q}_i\right)+\frac{\partial L}{\partial t} d t
\end{equation}

定义体系哈密顿量为
\begin{equation}
    H(q_i, p_i, t)=\sum_i p_i \dot{q}_i-L 
\end{equation}
我们有

\begin{equation}
    d H=\sum_i\left(p_i d \dot{q}_i+\dot{q}_i d p_i\right)-\left(\sum_i\left(\dot{p}_i d q_i+p_i d \dot{q}_i\right)+\frac{\partial L}{\partial t} d t\right)=\sum_i\left(\dot{q}_i d p_i-\dot{p}_i d q_i\right)-\frac{\partial L}{\partial t} d t
\end{equation}

另外根据定义有
\begin{equation}
    d H=\sum_i\left(\frac{\partial H}{\partial p_i} d p_i+\frac{\partial H}{\partial q_i} d q_i\right)+\frac{\partial H}{\partial t} d t .
\end{equation}
于是得到哈密顿方程

\begin{align}
    \frac{\partial H}{\partial p_i}&=\dot{q}_i
    \\
    \frac{\partial H}{\partial q_i}&=-\dot{p}_i
    \\
    \frac{\partial H}{\partial t}&=-\frac{\partial L}{\partial t}
\end{align}
对于保守系统,$\frac{\partial H}{\partial t}=0$, $H$为守恒量对应体系的能量.