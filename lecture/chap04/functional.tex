\subsection{泛函和变分法}
\label{subsec:functional}
变分法在数学、物理、工程和其它学科有广泛应用价值。通过以下的几个例子,我们可以了解变分法具体解决什么问题。

如图所示,现在寻找在两点$(x_1, y_1)$和$(x_2,y_2)$间找一个最短的路径$u(x)$。常识告诉我们,两点之间直线最短。
数学上表示出来为直线方程,
\[
  y = k x + b = \frac{y_2 - y_1}{x_2 - x_1} ( x - x_1) + y_1  
\]
这样的常识其实在数学上等价为一个最小化问题。对于任意路径,我们利用微积分求得该路径的长度。
$ds^2 = dx^2 + dy^2$,
\[
J[u] = \int_{x_1}^{x_2} \sqrt{ 1 + u'(x)^2} dx    
\]
其中 $u'(x) = du(x)/dx$,满足边界条件$u(x_1) = y_1, u(x_2) = y_2$。因此,我们的问题即求如何找到函数$u(x)$使得
$J$最小化。这里的方括号区别于函数的圆括号,用来表示这里的映射是函数$\to$数的。这种函数的函数称为\textbf{泛函}。
\subsection{泛函的导数和应用}
\label{subsec:functional_applications}