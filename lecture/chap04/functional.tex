\subsection{泛函和变分法}
\label{subsec:functional}
变分法在数学、物理、工程和其它学科有广泛应用价值。通过以下的几个例子,我们可以了解变分法具体解决什么问题。

如图所示,现在寻找在两点$(x_1, y_1)$和$(x_2,y_2)$间找一个最短的路径$u(x)$。常识告诉我们,两点之间直线最短。
数学上表示出来为直线方程,
\[
  y = k x + b = \frac{y_2 - y_1}{x_2 - x_1} ( x - x_1) + y_1  
\]
这样的常识其实在数学上等价为一个最小化问题。对于任意路径,我们利用微积分求得该路径的长度。
$ds^2 = dx^2 + dy^2$,
\[
J[u] = \int_{x_1}^{x_2} \sqrt{ 1 + u'(x)^2} dx    
\]
其中 $u'(x) = du(x)/dx$,满足边界条件$u(x_1) = y_1, u(x_2) = y_2$。因此,我们的问题即求如何找到函数$u(x)$使得
$J$最小化。这里的方括号区别于函数的圆括号,用来表示这里的映射是函数$\to$数的。这种函数的函数称为\textbf{泛函}。
具体的求解我们下面介绍。
理论力学上,最小作用量原理可以导出牛顿力学方程。我们定义的体系的动能为$K = \half m \bv^2$,
势能为$V = V(\br)$, 定义作用量
\begin{equation}
  A = \int \left[ \half m \left( \frac{d\br}{dt} \right)^2 - V(\br) \right] dt 
\end{equation}
上式利用了定义$\vec{\bv} = \frac{d\br}{dt}$。 对作用量$A$进行最小化将得到$\br$满足的方程,即牛顿第二定律方程。

对于一般的情况,有函数到数的映射写成
\begin{equation}
  J[u] = \int_{a}^{b}  dx L(u, u', u'', \cdots| x), 
\end{equation}
其中$L(u, u', u'',\cdots | x)$为$u$的局域泛函, $u' \equiv du/dx, u'' \equiv = du^2/dx^2$。
对于不包含一阶以上的导数时,即$L = L(u,u')$,对函数$u(x)$作一微小变化$\delta u(x)$,在积分上下限处,
有$\delta u(a) = \delta u(b) = 0$。$J[u]$取最小值时,需满足$\delta J[u] \equiv = J[u + \delta u] - J[u]$
为零。利用$\delta u' = \funcpd{}{x} \delta u$和$fg' = (fg)' - f'g$,
\begin{align}
  J[u + \delta u] &= \int_{a}^{b}  dx L(u + \delta u, u' + \delta u'| x) 
    \nonumber \\ 
    &= \int_{a}^{b} dx \left[ L(u, u'| x) + \funcpd{L}{u} \delta u + \funcpd{L}{u'} \delta u' \right]  
    \nonumber \\ 
    &= \int_{a}^{b} dx \left[ L(u, u'| x) + \funcpd{L}{u} \delta u + \funcpd{}{x}\left( \funcpd{L}{u'} \delta u \right)
    -  \funcpd{}{x}\left( \funcpd{L}{u'} \right) \delta u  \right]  
    \nonumber \\ 
    &= \int_{a}^{b} dx \left[ L(u, u'| x) + \left( \funcpd{L}{u} -  \funcpd{}{x}\left( \funcpd{L}{u'} \right)   \right) \delta u \right]  
        + \left. \funcpd{L}{u'} \delta u \right|_{a}^{b} 
     \nonumber %\\ 
\end{align}
于是,结合边界条件我们有
\begin{align}
  \delta J[u] = \int_{a}^{b} dx \left[ \funcpd{L}{u} -  \funcpd{}{x}\left( \funcpd{L}{u'} \right)   \right] \delta u 
\end{align}
被积函数对任意变化$\delta u$都需要满足$\delta J[u] = 0$,于是我们得到
\begin{equation}
  \funcpd{L}{u} -  \funcpd{}{x}\left( \funcpd{L}{u'} \right) = 0 \textrm{。}
\end{equation}
这就是著名的\textbf{欧拉-拉格朗日方程}(Euler-Lagrange Equation)。

利用该方程,可以得到两点之间距离最短满足
\[
  0 = - \funcd{}{x} \frac{u'}{\sqrt{1 + u'^2}} = \frac{-u''}{(1+ u'^2)^{3/2}}
\]
也就是
\[ u'' = 0\]
也就是解为直线方程$ u = k x + b$,$k,b$的值由边界条件确定。

对最小作用量原理问题,可以得到
\[
  0 = - \funcd {V(\br)}{\br} - \funcd{}{t} \left[ m \frac{d\br}{dt} \right]
\]
若质量$m$不依赖于时间,得到牛顿方程,
\[
  m \frac{d^2\br}{dt^2} = - \funcd {V(\br)}{\br} \equiv \bF \textrm{。}
\]

\subsubsection{约束问题}
我们常常遇到的一类问题是$u$满足其他的约束条件,比如上述两点之间曲线长度问题可以增加一个约束条件,
所构成的曲线所围面积为定值$A$,即$\int u(x) dx = A$。通常的做法是利用拉格朗日乘子法,即对
目标函数$J[u]$添加一项$\mu\left( \int u(x) dx - A \right)$。 这里的$\mu$称为拉格朗日乘子。
这样拉式量内包含了这一约束。对于该例子,我们得到
\[
  \mu - \funcd{}{x} \frac{u'}{\sqrt{1 + u'^2}} = 0,
\]
可以求得
\[
  \mu x-\frac{u'}{\sqrt{1+u'^2}}=c \quad \textrm{或} u'=\frac{\mu x-c}{\sqrt{1-(\mu x-c)^2}}
 \]
最终得到方程
\[
  (\mu x - c)^2 + (\mu x -d)^2 = 1
\]
其中$c,d,\mu$由$u(x_1)=y_1, u(x_2) = y_2$ 和$\int u(x) dx = A$决定。在这样的约束条件下,最短路径是一段圆弧!
% \subsection{Euler-Lagrange方程}
% 上式的求解其实可以归结

% \subsubsection{二维问题}


\subsection{泛函的导数和应用}
\label{subsec:functional_applications}