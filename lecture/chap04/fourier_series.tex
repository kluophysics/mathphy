\subsection{傅里叶级数的引入}
\label{subsec:fourier_series}
Fourier曾试图解决一个问题,在此过程中发展了傅里叶级数的概念.此问题处理长度为$\ell$的一维均匀棒的热传播过程,即
给定初始温度分布,求解在$t$时刻,$x$处的温度,$T(x,t)$.该温度场满足微分方程
\[
\frac{\partial^2}{\partial x^2} T(x,t) = \frac{\partial}{\partial t} T(x,t), \quad  T(0,t) = T(\ell,t) = 0 .  
\]
该微分方程的一特殊解可以由分离变量法得到,具体方法后面会学习.令$T(x,t) = f(x) g(t)$,代入该方程有
\[
  f''(x) g(t) = f(x) g'(t) ,
\]
两边同除以$f(x)g(t)$得到
\[
\frac{f''(x)}{f(x)} = \frac{g'(t)}{g(t)}   .
\]
由于左式是与$t$无关的,而右边是与$x$无关的,于是唯一可能相等的情况为二者皆为同一个常数$c$.
这时候若$c>0$,可知$\lim_{t\to \infty} g(t) = \infty$,因此我们令$c=-k^2$.
那么可以知道
\[
f(x) =  A \sin{k x} + B \cos{k x},    
\]
由边界条件确定$B = 0$,且$k = \frac{n\pi}{\ell}$.求解$g(t)$得 $g(t) = e^{-k^2 t}$.
该方程的解可以表示为
\[
T(x,t) = A \sin{\left( \frac{n\pi}{\ell} x \right)} e^{-\frac{n^2\pi^2}{\ell^2} t}  
\]
其中$A$由其他条件确定.
可以知道$t=0$时,$T$为一个正弦函数,节点数由$n$决定.但是,更普遍的情况是起始时刻的温度分布并不是正弦函数.
那该如何决定稍后某时刻$t$的分布呢?傅里叶发现我们并不需要对此问题反反复复求解.
对于线性微分方程可以知道若$T_1(x,t)$和$T_2(x,t)$满足微分方程
\[
    \frac{\partial^2}{\partial x^2} T(x,t) = \frac{\partial}{\partial t} T(x,t)
\]
那么二者的任意线性组合
\[
T_3 = \alpha T_1  + \beta T_2  
\]
也是其解.
傅里叶得到结论,对于任意一种光滑的温度分布,其解总可以写成一系列有着不同振幅和模式的正弦函数的线性叠加,即
\[
  T(x,t) = \sum_{n=1}^{\infty} A_n \sin {\left( \frac{n\pi}{\ell} x \right)} e^{-\frac{n^2\pi^2}{\ell^2} t} .     
\]
若初始温度分布函数为$f(x) = T(x,0)$,那么
\[
  f(x) =      \sum_{n=1}^{\infty} A_n \sin {\left( \frac{n\pi}{\ell} x \right)}
\]
上式什么样的函数可以表示为正弦和余弦函数的叠加呢?每个模式的振幅$A_n$该如何计算呢?
这就利用了正余弦函数的特征.

记
\begin{equation}
  \braket{f|g} = \int_0^{2\pi} f^{*}(x) g(x) dx\,,
\end{equation}

可以验证
\[
\braket{\sin{ n x }|\sin{m x }}  =  \braket{\cos{ n x }|\cos {m x }}  =  \delta_{mn}\pi  \; (m\neq 0, n\neq 0)
\]

\[
\braket{\sin{ n x }|\cos{m x }}  =  0
\]
将上式代入,注意修改积分变量上下限可以得到振幅系数
\begin{equation}
  A_n = \frac{2}{\ell} \int_0^{\ell} f(x) \sin{ \left( \frac{n\pi}{\ell} x \right) } dx 
\end{equation}
$A_n$称为傅里叶级数的\textbf{系数}.若所有系数都求得后,我们就得到了$f(x)$.原则上,所有的平滑的单值函数并没有什么问题,
但似乎缺乏数学的严谨.自然,越多的级数项对函数的逼近也就越好.

\subsection{傅里叶定理}
\label{subsec:fourier_theorem}
对于任意一个以$2\ell$为周期的函数,
\[
   f(x + 2\ell) = f(x),  
\]
我们可以通过三角函数族进行级数展开
\begin{equation}
  f(x) = \frac{a_0}{2} + \sum_{n=1}^{\infty} \left[ a_n \cos{ \frac{n\pi}{\ell} x } + b_n \sin{ \frac{n\pi}{\ell} x } \right] 
\end{equation}
利用三角函数族的正交性,可以得到展开系数
\begin{align}
  a_n = \frac{1}{\ell} \int_{-\ell}^{\ell} f(x) \cos {  \left( \frac{n\pi}{\ell} x \right) } dx
  \\
  b_n = \frac{1}{\ell} \int_{-\ell}^{\ell} f(x) \sin {  \left( \frac{n\pi}{\ell} x \right) } dx
\end{align}
这里的积分上下限为一个周期$2\ell$,也可以为$[0,2\ell]$.
上述傅里叶级数展开的成立条件被称为狄里希利条件 (Dirichlet conditions),即$f(x)$在$\left[-\ell, \ell\right]$区间内只有有限个间断点,且每个周期内有有限个极值点.满足
这两个条件的函数称为分段平常.
满足狄里希利条件的函数$f(x)$在点$x_0$出间断,那么其傅里叶级数在此点的值为该函数左右值的算术平均
\begin{equation}
  \textrm{级数和}(x_0) = \lim_{\epsilon \to 0} \half \left[
     f(x_0 + \epsilon) + f(x_0  - \epsilon) \right]
\end{equation}

对于奇函数和偶函数的傅里叶展开,不难发现,奇函数的展开为

\begin{equation}
  f(x) = \sum_{n=1}^{\infty} b_n \sin {  \left( \frac{n\pi}{\ell} x \right) }
\end{equation}
称为\textbf{傅里叶正弦级数}.
而偶函数的展开为
\begin{equation}
  f(x) = \frac{a_0}{2} + \sum_{n=1}^{\infty}  a_n \cos{  \left( \frac{n\pi}{\ell} x \right) }
\end{equation}
称为\textbf{傅里叶余弦级数}.

对于复指数的展开,不难发现可以写成
\begin{equation}
  f(x) = \sum_{n=-\infty}^{\infty} c_n e^{\imath \frac{n\pi}{\ell} x},
\end{equation}
其中
$c_n = \half(a_n - \imath b_n)$, $c_{-n} = \half (a_n + \imath b_n)$, $n>0$, $c_0 = \half a_0$.
写出来为
\begin{equation}
  c_n = \frac{1}{2\ell} \int_{-\ell}^{\ell} f(x) e^{-\imath \frac{n\pi}{\ell} x} dx ,
\end{equation}
尽管$f(x)$是实数,但其傅里叶系数却可能是复数,还可以看出$c_{-n} = c_{n}^*$.
复指数函数族也是正交的
\[
  \braket{ e^{\imath \frac{n\pi}{\ell} x } | e^{\imath \frac{m\pi}{\ell} x } } = \delta_{mn} .
\]

\begin{examplebox}{
对于锯齿函数
\[  
f(x)= \begin{cases}
  x,   &  0 < x \leq  \ell 
  \\
  x - 2 \ell,   & \ell < x \leq 2\ell
\end{cases}
\]
求其傅里叶级数.
}
不难判定通过解析延拓,该函数为奇函数,根据傅里叶展开系数公式,可以得
\begin{align}
b_n &=   \frac{1}{\ell} \int_{0}^{2\ell} f(x) \sin {  \left( \frac{n\pi}{\ell} x \right) } dx \nonumber
\\
&= \frac{1}{\ell} \int_{0}^{2\ell} x  \sin {  \left( \frac{n\pi}{\ell} x \right) } dx \nonumber
  % \\ & \quad 
  +
 \frac{1}{\ell} \int_{0}^{2\ell} (x - 2\ell) \sin {  \left( \frac{n\pi}{\ell} x \right) } dx \nonumber
 \\ 
 &= \frac{2}{\ell} \int_{0}^{2\ell} x \sin {  \left( \frac{n\pi}{\ell} x \right) } dx \nonumber
%  \\ & \quad 
 -2 \int_{0}^{2\ell}   \sin {  \left( \frac{n\pi}{\ell} x \right) } dx \nonumber
 \\
 & = \frac{2\ell}{n\pi} (-1)^{n+1}  \nonumber
%  \\
%  & = \frac{2}{n\pi}  \left[ (-1)^{n+1} (\ell -1)  -1 \right] , \nonumber
\end{align}
因此,该级数为
\begin{align}
f(x) &=  \sum_{n=1}^{\infty} \frac{2 \ell }{n\pi} (-1)^{n+1} \sin {  \left( \frac{n\pi}{\ell} x \right) }  \nonumber
\\
 & =\frac{2\ell}{\pi} \left[ \sin{\frac{\pi x}{\ell} } - \frac{1}{2}\sin{\frac{2\pi x}{\ell} } 
  +  \frac{1}{3}\sin{\frac{3\pi x}{\ell} } + \cdots + \frac{(-1)^{n+1}}{n}\sin{\frac{n\pi x}{\ell} }
 \right] \nonumber
.
\end{align}
容易验证
\[
  f(0) = 0; f(\pi) = 0;
\]
当$x=\ell/2$, 
\[
f(\ell/2) = \ell/2 =   \frac{2\ell}{\pi} \left[ 1 - 0 - \frac{1}{3} + \frac{1}{5} - 0 -\frac{1}{7}+ \cdots \right]
\]
因此,我们的得到莱布尼兹等式,即以下等式
\[
\frac{\pi}{4} = 1-\frac{1}{3} + \frac{1}{5} - \frac{1}{7} + \cdots = \sum_{n=0} \frac{(-1)^n}{2n + 1} .
\]
上式可以通过等式
$\int_0^1 \frac{d x}{1+x^2}=\left.\tan ^{-1} x\right|_0 ^1=\frac{\pi}{4}$,和被积函数的级数展开验证.
其实,这里的锯齿函数可以直接用$f(x) = x, -\ell < x < \ell$来表示,相应的间断点则被移到$\pm \ell$处.
\end{examplebox}


\subsection{能量定理}
由函数的复数傅里叶级数展开可以得到
\begin{equation}
  E = \frac{1}{2\ell} \int_{-\ell}^{\ell} dx f(x)^* f(x) 
 = \frac{1}{2\ell} \int_{-\ell}^{\ell} dx  \sum c_n e^{\imath \frac{n\pi}{\ell} x} \sum c_m e^{-\imath \frac{m\pi}{\ell} x}
 = \sum |c_n|^2
\end{equation}
其物理意义指的是,一复杂波形分解为许多波幅为$c_n$的正(余)弦波的叠加,该波所携带的总能量为各个成分波强度的求和.
作为习题,证明对于矩形波,
$\begin{array}{ll}f(x)=-1 & \text { for } x<0 \\ f(x)=+1 & \text { for } x>0\end{array}$其傅里叶级数为
$$
f(x)=\frac{4}{\pi} \sin \pi x+\frac{4}{3 \pi} \sin 3 \pi+\frac{4}{5 \pi} \sin 5 x+\cdots
$$
可以发现对于矩形波,其对应的能量为$1$,于是我们有
$$
|f(x)|^2=1=\frac{16}{\pi^2}\left[\frac{1}{2} \frac{1}{1^2}+\frac{1}{2} \frac{1}{3^2}+\frac{1}{2} \frac{1}{5^2}+\cdots\right]
$$
即
\[
\sum_{n=0} \frac{1}{(2n+1)^2} = \frac{\pi^2}{8} .
\]