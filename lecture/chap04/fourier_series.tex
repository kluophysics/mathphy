\subsection{傅里叶级数}
\label{subsec:fourier_series}
Fourier曾试图解决一个问题,在此过程中发展了傅里叶级数的概念。此问题处理长度为$\ell$的一维均匀棒的热传播过程,即
给定初始温度分布,求解在$t$时刻,$x$处的温度,$T(x,t)$。该温度场满足微分方程
\[
\frac{\partial^2}{\partial x^2} T(x,t) = \frac{\partial}{\partial t} T(x,t), \quad  T(0,t) = T(\ell,t) = 0 \textrm{。}  
\]
该微分方程的一特殊解可以由分离变量法得到,具体方法后面会学习。令$T(x,t) = f(x) g(t)$,代入该方程有
\[
  f''(x) g(t) = f(x) g'(t) ,
\]
两边同除以$f(x)g(t)$得到
\[
\frac{f''(x)}{f(x)} = \frac{g'(t)}{g(t)}   \textrm{。}
\]
由于左式是与$t$无关的,而右边是与$x$无关的,于是唯一可能相等的情况为二者皆为同一个常数$c$。
这时候若$c>0$,可知$\lim_{t\to \infty} g(t) = \infty$,因此我们令$c=-k^2$。
那么可以知道
\[
f(x) =  A \sin{k x} + B \cos{k x},    
\]
由边界条件确定$B = 0$,且$k = \frac{n\pi}{\ell}$。求解$g(t)$得 $g(t) = e^{-k^2 t}$。
该方程的解可以表示为
\[
T(x,t) = A \sin{\left( \frac{n\pi}{\ell} x \right)} e^{-\frac{n^2\pi^2}{\ell^2} t}  
\]
其中$A$由其他条件确定。
可以知道$t=0$时,$T$为一个正弦函数,节点数由$n$决定。但是,更普遍的情况是起始时刻的温度分布并不是正弦函数。
那该如何决定稍后某时刻$t$的分布呢?傅里叶发现我们并不需要对此问题反反复复求解。
对于线性微分方程可以知道若$T_1(x,t)$和$T_2(x,t)$满足微分方程
\[
    \frac{\partial^2}{\partial x^2} T(x,t) = \frac{\partial}{\partial t} T(x,t)
\]
那么二者的任意线性组合
\[
T_3 = \alpha T_1  + \beta T_2  
\]
也是其解。
傅里叶得到结论,对于任意一种光滑的温度分布,其解总可以写成一系列有着不同振幅和模式的正弦函数的线性叠加,即
\[
  T(x,t) = \sum_{n=1}^{\infty} A_n \sin {\left( \frac{n\pi}{\ell} x \right)} e^{-\frac{n^2\pi^2}{\ell^2} t} \textrm{。}     
\]
若初始温度分布函数为$f(x) = T(x,0)$,那么
\[
  f(x) =      \sum_{n=1}^{\infty} A_n \sin {\left( \frac{n\pi}{\ell} x \right)}
\]
上式什么样的函数可以表示为正弦和余弦函数的叠加呢?每个模式的振幅$A_n$该如何计算呢?
这就利用了正余弦函数的特征。