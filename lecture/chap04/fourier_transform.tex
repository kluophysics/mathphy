\subsection{傅里叶变换}
\label{subsec:fourier_transform}
前面我们讨论了周期性体系的傅里叶级数展开。现在我们希望研究非周期性函数的傅里叶展开问题。
假设$f(x)$是定义在$-\infty < x < \infty$的实函数,我们依旧对该函数进行傅里叶级数展开,
该级数必须理解为周期$2\ell$扩展至$\infty$的极限情况。
由于三角函数族的变量为
\[
 \frac{n\pi x}{\ell}    
\]
引入非连续变量
\[\omega_n = \frac{n\pi} {\ell}, \quad k = 0,1,2,\cdots
    \]
可知$\omega_n = n \omega_0$, 其中$\omega_0 = \frac{\pi}{\ell}$为基础频率(Fundamental Frequency),
\[\Delta \omega_n = \omega_n - \omega_{n-1} = \omega_0 \textrm{。} \] 
有
\begin{equation}
    f(x) = \lim_{\ell\to \infty} \sum_{n=-\infty}^{\infty} \left[
        c_n e^{\imath \omega_n x}
        % \frac{a_0}{2} + \sum_{n=1}^{\infty} \left( a_n \cos \omega_n x + b_n \sin \omega_n x \right)
    \right]
\end{equation}
其中系数
\begin{equation}
    c_n = \frac{1}{2\ell} \int_{-\ell}^{\ell} f(x) e^{-\imath \omega_n x} dx 
\end{equation}
取$\ell\to \infty$的极限,则
\[
  \sum_n \to  \frac{ \ell}{\pi}  \int d\omega 
\]
傅里叶系数记为$F(\omega)$,
有
\begin{equation}
    f(x) = \int_{-\infty}^{\infty} \frac{d \omega }{2\pi}F(\omega) e^{\imath \omega x}, % \quad F (\omega) \equiv 2\ell c_n
\end{equation}
上式的积分称为\textbf{傅里叶积分}(Fourier integral),是$f(x)$的\textbf{傅里叶变换}(Fourier transform)。
% $F (\omega) \equiv 2\ell c_n$。
其中
\begin{equation}
    F (\omega) \equiv 2\ell c_n = \int_{-\infty}^{\infty} e^{-\imath \omega x} f(x) dx,
\end{equation}
这被称为\textbf{逆傅里叶变换}(inverse Fourier transform)。
% \begin{align}
%     a_n =  \lim_{\ell\to \infty} \frac{1}{\ell} \int_{-\ell}^{\ell} f(x) \cos {  \omega_n x  } dx
%     \\
%     b_n = \frac{1}{\ell} \int_{-\ell}^{\ell} f(x) \sin {   \omega_n x } dx
% \end{align}

若采用正余弦展开,

\begin{equation}
    f(x) =\int_{0}^{\infty} A(\omega) \cos {\omega x} d\omega + \int_{0}^{\infty} B(\omega) \sin {\omega x} d\omega
\end{equation}
其中
\begin{equation}
    \left\{\begin{array}{l}
    A(\omega)=\frac{1}{\pi} \int_{-\infty}^{\infty} f(x) \cos \omega x d x, \\
    B(\omega)=\frac{1}{\pi} \int_{-\infty}^{\infty} f(x) \sin \omega x d x .
    \end{array}\right.
\end{equation}

上式还可以写成

\begin{equation}
    f(x)=\int_0^{\infty} C(\omega) \cos [\omega x-\varphi(\omega)] \mathrm{d} \omega,
\end{equation}

其中
$$
\begin{gathered}
C(\omega)=\sqrt{[A(\omega)]^2+[B(\omega)]^2} \\
\varphi(\omega)=\arctan [B(\omega) / A(\omega)] .
\end{gathered}
$$
$C(\omega)$ 称为 $f(x)$ 的振幅谱, $\varphi(\omega)$ 称为 $f(x)$ 的相位谱。如果$f(x)$是奇函数或偶函数,
对应的级数为\textbf{傅里叶余弦积分}或\textbf{傅里叶正弦积分}。

复数形式的傅里叶积分可以写成对称的形式,
\begin{equation}
    \begin{aligned}
    f(x) & =\frac{1}{\sqrt{2 \pi}} \int_{-\infty}^{\infty} F(\omega) \mathrm{e}^{\mathrm{i} \omega x} \mathrm{~d} \omega, \\
    F(\omega) & =\frac{1}{\sqrt{2 \pi}} \int_{-\infty}^{\infty} f(x)\left[\mathrm{e}^{\mathrm{i} \omega x}\right]^* \mathrm{~d} x .
    \end{aligned}
\end{equation}
并常用符号简写为
$$
F(\omega)=\mathcal{F}[f(x)], \quad f(x)=\mathcal{F}^{-1}[F(\omega)] \textrm{。}
$$
$f(x)$ 和 $F(\omega)$ 分别称为傅里叶变换的\textbf{原函数}和\textbf{像函数}。

\begin{examplebox}{
求以下函数的复数傅里叶变换。
\begin{enumerate}
    \item $f(x) = e^{-\alpha |x|}, \alpha > 0$;
    \item $f(x) = \delta (x)$;
    \item $f(x) = e^{-\alpha x^2}, \alpha > 0$。
\end{enumerate}
}
\end{examplebox}