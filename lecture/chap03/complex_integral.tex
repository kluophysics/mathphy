\subsection[定义和性质]{复变函数的积分}
有了复变函数微分的基础,我们现在来讨论积分。复变函数的积分的定义可以同实变函数积分的类比得到。
在复平面上取一个路径$\ell$,起终点为$z_0,z'_0$,沿着该路径定义了一连续函数$f(z)$,用$n-1$个点$z_1, z_2, z_{n-1}$将该路径
$\ell$分成$n$个线段。函数$f(z)$在线段$z_{i-1}\rightarrow z_{i}$上任意一点$\xi_i$的值乘上线段的长度$\Delta z_i = z_i - z_{i-1}$并求和,即
\begin{equation}
    S_n = \sum_{i=1}^{n} f(\xi_i) (z_{i} - z_{i-1}) \textrm{。}
\end{equation}
当$n\to \infty$, 求和则转化为积分。当这个和的极限存在且
与$\xi_i$的选取无关时,这个极限称为
函数$f(z)$沿着路径$\ell$的\textbf{路径积分}(Contour integral), 记作
$\int_{\ell} f(z) dz$,即
\begin{equation}
    \int_{\ell} f(z) dz = \lim_{n\to \infty} \sum_{i=1}^{n} f(\xi_i) (z_{i} - z_{i-1}) \textrm{。} 
\end{equation}

其实,我们还可以将该定义用实虚部的方式表达出来,
\begin{align}
    \int_{\ell} f(z) dz  & = \int_{\ell} \left( u(x,y) + \imath v(x,y) \right) (dx + \imath dy) 
    \\
    & = \int_{\ell} \left( u(x,y) dx  -  v(x,y) dy \right) + \imath \int_{\ell}  \left( v(x,y) dx  + u(x,y)dy \right) 
\end{align}
这样一来,复变函数的路径积分就转化成了两个实变函数的线积分。因此,很多实函数线积分的性质可以应用在路径积分上。