\subsection{约束问题}
我们常常遇到的一类问题是$u$满足其他的约束条件,比如上述两点之间曲线长度问题可以增加一个约束条件,
所构成的曲线所围面积为定值$A$,即$\int u(x) dx = A$.通常的做法是利用拉格朗日乘子法,即对
目标函数$J[u]$添加一项$\mu\left( \int u(x) dx - A \right)$. 这里的$\mu$称为拉格朗日乘子.
这样拉式量内包含了这一约束.对于该例子,我们得到
\[
  \mu - \funcd{}{x} \frac{u'}{\sqrt{1 + u'^2}} = 0,
\]
可以求得
\[
  \mu x-\frac{u'}{\sqrt{1+u'^2}}=c \quad \textrm{或} u'=\frac{\mu x-c}{\sqrt{1-(\mu x-c)^2}}
 \]
最终得到方程
\[
  (\mu x - c)^2 + (\mu u -d)^2 = 1
\]
其中$c,d,\mu$由$u(x_1)=y_1, u(x_2) = y_2$ 和$\int u(x) dx = A$决定.在这样的约束条件下,最短路径是一段圆弧!
% \subsection{Euler-Lagrange方程}
% 上式的求解其实可以归结

% \subsubsection{二维问题}
