\section{定解条件}

2. 定解条件 上面所导出的弦振动方程 (1.14) 包含有未知函数 $u(x, t)$ 和它的关于自变量的偏导数, 所以是偏微分方程。对于一个偏微分方程来说, 如果有一个函数 $u(x, t)$,具有方程中所需要的各阶连续偏导数, 且将它代人方程时能使方程成为恒等式, 就称这个函数为该方程的解。列出微分方程以后, 目的就是要从微分方程中求得解或研究解的性质。例如, 为了了解弦的振动情况, 就应该设法求出相应的弦振动方程的解。

我们看到, 弦振动方程 (1.14) 描述了弦作微小横振动时位移函数 $u(x, t)$ 所应满足的一般性规律, 但仅仅利用它还不能完全确定所考察弦的运动状况。这是因为弦的运动还与其初始状态以及边界所处的状况有关,因此还得给出一些其他条件。
在上述弦振动问题中, 弦的两端被固定在 $x=0$ 及 $x=l$ 两点, 因此有
$$
u(0, t)=0, \quad u(l, t)=0,
$$

称为边界条件。此外, 设弦在初始时刻 $t=0$ 时的位置和速度为
$$
u(x, 0)=\varphi(x), \quad \frac{\partial u(x, 0)}{\partial t}=\psi(x) \quad(0 \leqslant x \leqslant l),
$$

称为初始条件。边界条件与初始条件总称为定解条件。把弦振动方程 (1.14) 
和定解条件 (1.17)、(1.18)结合起来, 就得到如下的定解问题:
$$
\left\{\begin{array}{l}
	\frac{\partial^2 u}{\partial t^2}-a^2 \frac{\partial^2 u}{\partial x^2}=f(x, t), \\
	t=0: u=\varphi(x), \frac{\partial u}{\partial t}=\psi(x), \\
	x=0: \quad u=0, \\
	x=l: \quad u=0 .
\end{array}\right.
$$

要在区域 $(0 \leqslant x \leqslant l, t \geqslant 0)$ 上 (见图 1.2) 求上述定解问题的解, 就是要求这样的连续函数 $u=u(x, t)$, 它在区域 $0<x<l, t>0$ 中满足波动方程 (1.19); 在 $x$ 轴 $(t=0)$ 一段区间

$0 \leqslant x \leqslant l$ 上满足初始条件 (1.20), 并在边界 $x=0$ 及 $x=l$ 上分别满足边界条件 (1.21) 及

一般称形如 (1.17) 的边界条件为第一类边界条件, ( 又称狄利克要 (Dirichlet) 边界条件)。对于弦振动方程的边界条件通常还可以有以下两种:
(a) 弦的一端 (例如 $x=0$ ) 处于自由状态, 即可以在垂直于 $x$ 轴的直线上自由滑动, 未受到垂直方向外力。在边界右端的张力的垂直方向分量是 $T \frac{\partial u}{\partial x}$, 得出此时应成立
$$
\left.\frac{\partial u}{\partial x}\right|_{x=0}=0 \text {. }
$$

也可以考虑更普遍的边界条件
$$
\left.\frac{\partial u}{\partial x}\right|_{x=0}=\mu(t) ,
$$

其中 $\mu(t)$ 是 $t$ 的已知函数。这种边界条件称为第二类边界条件 (又称诺伊曼 (Neumann) 边界条件)。

(b)在应用上还会遇到另一种情形。将弦的一端固定在弹性支承上,也就是说此时支承的伸缩符合胡克定律。如果支承原来的位置为 $u=0$, 则 $u$ 在端点的值表示支承在该点的伸长。例如在 $x=l$ 的一端, 弦对支承拉力的垂直方向分量为 $-T \frac{\partial u}{\partial x}$, 由胡克定律知
$$
-\left.T \frac{\partial u}{\partial x}\right|_{x=\imath}=\left.k u\right|_{x=\imath},
$$

其中 $k$ 为弹性系数。因此在弹性支承的情形,边界条件归结为
$$
\left.\left(\frac{\partial u}{\partial x}+\sigma u\right)\right|_{x=\imath}=0
$$

其中 $\sigma=\frac{k}{T}$ 是已知正数。在数学中也可以考虑更普遍的边界条件
$$
\left.\left(\frac{\partial u}{\partial x}+\sigma u\right)\right|_{x=l}=v(t),
$$

其中 $v(t)$ 是 $t$ 的已知函数。这种边界条件称为第三类边界条件。

下面我们再介绍几个概念。一个偏微分方程所含有的未知函数最高阶导数的阶数称为这个偏微分方程的阶, 例如弦振动方程 (1.14) 就是一个二阶偏微分方程。如果方程对未知函数及其各阶导数总体来说是线性的,则称这个方程是线性方程。否则称这个方程是非线
性方程。进一步, 如果方程对未知函数的所有最高阶导数总体来说是线性的, 则称它为拟线性方程。例如,方程
$$
\frac{\partial u}{\partial t}+u \frac{\partial u}{\partial x}=0
$$

是一阶拟线性方程。如果非线性方程中方程对未知函数的最高阶导数不是线性的, 则称它为完全非线性方程。例如, 方程
$$
\left(\frac{\partial u}{\partial x}\right)^2+\left(\frac{\partial u}{\partial y}\right)^2=u
$$

就是一阶完全非线性方程。
我们看到, 方程 (1.14) 与 (1.11) 不同, 它包含有不含 $u$ 及其偏导数的项 $f(x, t)$ (称为自由项), 这样的方程称为非齐次方程,而 (1.11) 称为齐次方程。类似地,边界条件 (1.21)、 (1.22) 称为齐次边界条件,
相应地, 若边界条件为 $\left.u\right|_{1=0}=\mu_1(t),\left.u\right|_{x^{-1}}=\mu_2(t)$, 则称为非齐次边界条件。同样, 初始条件 (1.20) 称为非齐次初始条件, 而对应于 $\varphi \equiv \psi \equiv 0$ 的初始条件称为齐次初始条件。
\subsection{达朗贝尔公式和定解问题}
