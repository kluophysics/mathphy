\section{非齐次边界条件的处理}
\label{eq:inhomo_boundary}

在 $\S 8.1$ 和 $\S 8.2$ 两节中, 不管是齐次还是非齐次振动方程和输运方程,它们的定解问题的解法都有一个前提:边界条件是齐次的.

但是, 在实际问题中, 常有非齐次边界条件出现, 那么, 这样的定解问题又如何求解呢? 由于定解问题是线性的,处理的原则是利用叠加原理,把非齐次边界条件问题转化为另一未知函数的齐次边界条件问题. 请看例题.

\subsection{(一) 一般处理方法}
例 1 自由振动问题

$$
\begin{gathered}
u_{t t}-a^{2} u_{x x}=0, \\
\left.u\right|_{x=0}=\mu(t),\left.u\right|_{x=l}=\nu(t), \\
\left.u\right|_{t=0}=\varphi(x),\left.u_{t}\right|_{t=0}=\psi(x) .
\end{gathered}
$$

边界条件 (8.3.2) 是非齐次的.
选取一个函数 $v(x, t)$, 使其满足非齐次边界条件 (8.3.2), 为了简单起见, 不妨取 $v(x, t)$ 为 $x$ 的线性函数, 即

$$
v(x, t)=A(t) x+B(t) \text {. }
$$

将 (8.3.4) 代入 (8.3.2), 解得

$$
v(x, t)=\frac{[\nu(t)-\mu(t)]}{l} x+\mu(t)
$$

利用叠加原理, 令

$$
u(x, t)=v(x, t)+w(x, t) .
$$

将 (8.3.5)、(8.3.6) 代入定解问题 (8.3.1) (8.3.3), 得 $w(x, t)$ 的定解问题

$$
\begin{gathered}
w_{t u}-a^{2} w_{\imath \imath}=-v_{\imath t}+a^{2} v_{x x}=\frac{x}{l}\left[\mu^{\prime \prime}(t)-\nu^{\prime \prime}(t)\right]-\mu^{\prime \prime}(t), \\
\left.w\right|_{x=0}=0,\left.w\right|_{x=l}=0, \\
\left.w\right|_{t=0}=\varphi(x)-\left.v\right|_{\imath=0}=\varphi(x)+\frac{1}{l}[\mu(0)-\nu(0)] x-\mu(0), \\
\left.w_{\imath}\right|_{\imath=0}=\psi(x)-\left.v_{t}\right|_{t=0}=\psi(x)+\frac{1}{l}\left[\mu^{\prime}(0)-\nu^{\prime}(0)\right] x-\mu^{\prime}(0) .
\end{gathered}
$$

虽然 $w(x, t)$ 的方程 (8.3.7)一般是非齐次的,但是,定解问题 (8.3.7) (8.3.9) 具有齐次边界条件, 可按 $\S 8.2$ 求解.

这里还要特别说一下 $x=0$ 和 $x=l$ 两端都是第二类非齐次边界条件 $\left.u_{x}\right|_{x=0}$ $=\mu(t),\left.u_{x}\right|_{x=l}=\nu(t)$ 的情况. 如果仍按 (8.3.4) 取 $x$ 的线性函数作为 $v$, 则代入非齐次边界条件得

$$
\left.v_{x}\right|_{x=0}=A(t)=\mu(t),\left.\quad v_{x}\right|_{x=l}=A(t)=\nu(t) .
$$

除非 $\mu(t)=\nu(t)$, 否则这两式互相矛盾. 这时不妨改试

$$
v(x, t)=A(t) x^{2}+B(t) x .
$$

\subsection{(二) 特殊处理方法}
例 2 弦的 $x=0$ 端固定, $x=l$ 端受迫作谐振动 $A \sin \omega t$, 弦的初始位移和初始速度都是零, 求弦的振动. 这个定解问题是

$$
\begin{gathered}
u_{t t}-a^{2} u_{x x}=0 \quad(x<0<l), \\
\left.u\right|_{x=0}=0,\left.\quad u\right|_{x=l}=A \sin \omega t, \\
\left.u\right|_{t=0}=0,\left.\quad u_{t}\right|_{t=0}=0 .
\end{gathered}
$$

$x=l$ 端为非齐次边界条件.

如果按上述一般处理方法, 应取 $v(x, t)=(A \sin \omega t / l) x$, 但是, 相应的 $w(x, t)$ 的定解问题中泛定方程为 $w_{u}-a^{2} w_{x x}=-\left(v_{u t}-a^{2} v_{x x}\right)=\left(A \omega^{2} x / l\right) \sin \omega t$,是非齐次方程, 求解麻烦. 能否有较为简便的方法呢?

由于求解的是弦在 $x=l$ 端受迫作谐振动 $A \sin \omega t$ 情况下的振动, 它一定有一个特解 $v(x, t)$, 满足齐次方程 (8.3.11)、非齐次边界条件 (8.3.12), 且跟 $x$
$=l$ 端同步振动, 即其时间部分的函数亦为 $\sin \omega t$, 就是说, 特解具有分离变数的形式:

$$
v(x, t)=X(x) \sin \omega t
$$

将(8.3.14) 代入 (8.3.11)、

$$
\begin{aligned}
& (8.3 .12), \text { 得 } \\
& \left\{\begin{array}{l}
X^{\prime \prime}+\left(\frac{\omega}{a}\right)^{2} X=0, \\
X(0)=0, X(l)=A .
\end{array}\right.
\end{aligned}
$$

将常微分方程 (8.3.15) 的解 $X(x)=C \cos (\omega x / a)+D \sin (\omega x / a)$ 代入 (8.3.16), 由此确定 $X(x)=[A / \sin (\omega l / a)] \sin (\omega x / a)$, 从而

$$
v(x, t)=\frac{A}{\sin \frac{\omega l}{a}} \sin \frac{\omega x}{a} \sin \omega t
$$

于是令

$$
u(x, t)=v(x, t)+w(x, t),
$$

将 (8.3.17)、(8.3.18) 代入 (8.3.11) (8.3.13), 得 $w(x, t)$ 的定解问题

$$
\begin{gathered}
w_{t t}-a^{2} w_{x x}=-\left(v_{x x}-a^{2} v_{x x}\right)=0 \\
\left.w\right|_{x=0}=0,\left.\quad w\right|_{x=\imath}=0 \\
\left.w\right|_{\imath=0}=0,\left.\quad w_{\imath}\right|_{t=0}=-A \omega \frac{\sin (\omega x / a)}{\sin (\omega l / a)}
\end{gathered}
$$

定解问题 (8.3.19) (8.3.21) 为齐次方程、齐次边界条件, 可用分离变数法求解, 其一般解由 (8.1.14) 给出, 因此,

$$
w(x, t)=\sum_{n=1}^{\infty}\left(A_{n} \cos \frac{n \pi a}{l} t+B_{n} \sin \frac{n \pi a}{l} t\right) \sin \frac{n \pi}{l} x
$$

其中系数 $A_{n}$ 和 $B_{n}$ 可由 (8.1.16) 确定, 得

$$
\begin{aligned}
A_{n} & =0 \\
B_{n} & =\frac{2}{n \pi a} \int_{0}^{l}(-A \omega) \frac{\sin (\omega \xi / a)}{\sin (\omega l / a)} \sin \frac{n \pi \xi}{l} d \xi \\
& =\frac{-2 A \omega}{n \pi a \sin (\omega l / a)}\left[-\frac{\sin (\omega / a+n \pi / l) \xi}{2(\omega / a+n \pi / l)}+\frac{\sin (\omega / a-n \pi / l) \xi}{2(\omega / a-n \pi / l)}\right]_{0}^{l} \\
& =\frac{A \omega}{n \pi a \sin (\omega l / a)}\left[\frac{\sin (\omega l / a+n \pi)}{\omega / a+n \pi / l}-\frac{\sin (\omega l / a-n \pi)}{\omega / a-n \pi / l}\right] \\
& =(-1)^{n} \frac{A \omega}{n \pi a}\left[\frac{1}{\omega / a+n \pi / l}-\frac{1}{\omega / a-n \pi / l}\right] \\
& =(-1)^{n} \frac{2 A \omega}{a l} \cdot \frac{1}{\omega^{2} / a^{2}-n^{2} \pi^{2} / l^{2}} .
\end{aligned}
$$

这样

$$
\begin{aligned}
w(x, t)= & \frac{2 A \omega}{a l} \sum_{n=1}^{\infty} \frac{1}{\omega^{2} / a^{2}-n^{2} \pi^{2} / l^{2}} \sin \frac{n \pi a t}{l} \sin \frac{n \pi x}{l}, \\
u(x, t)= & A \frac{\sin (\omega x / a)}{\sin (\omega l / a)} \sin \omega t \\
& +\frac{2 A \omega}{a l} \sum_{n=1}^{\infty} \frac{1}{\omega^{2} / a^{2}-n^{2} \pi^{2} / l^{2}} \sin \frac{n \pi a t}{l} \sin \frac{n \pi x}{l} .
\end{aligned}
$$