% \part{复变函数论}

\chapter{复变函数论}
\label{chap:complexfunctions}
复变函数论是物理学和工程学广泛使用的强大分析工具,学习掌握这个理论十分必要。
\section{复数和复变函数}
\subsection{复数的定义和基本性质}
\subsubsection{定义}
一元二次方程$a x^2 + b x + c = 0 (a\neq 0, b,c \in \mathbb{R})$的求解大家一定都不陌生。当$\Delta \equiv b^2 - 4 a c \geq 0$时,方程有解,求根公式可得
\begin{equation}
    x = \frac{-b \pm \sqrt{\Delta}}{2a} \textrm{。}
\end{equation}
当$\Delta < 0$时,方程无解。这里的无解其实是指没有实数解。
当我们引入以下这一核心定义后,
\begin{equation}
    \imath ^2 = -1 \quad \textrm{或} \quad \imath = \sqrt{-1} \textrm{。} 
\end{equation}
$\imath$为{\bf 虚数},将定义域扩展到复数,二次方程就有确定的两个解,当$\Delta < 0$时,方程有两个复数解:
\begin{equation}
    x = \frac{-b \pm i \sqrt{-\Delta}}{2a} \textrm{。}
\end{equation}
复数$z$定义为
\begin{equation}
    z = x + \imath \; y \textrm{,}
\end{equation}
$x,y \in {\mathbb{R}}$。$x,y$分别为复数$z$的{\bf 实部}(real part) 和{\bf 虚部}(imaginary part)。
复数域常用$\mathbb{C}$来表示。若将$z$看成是由$x,y$组成的有序对$(x,y)$,记为
\begin{equation}
    z \equiv (x,y)\textrm{,}
\end{equation}
则有$1 = (1,0), \imath = (0, 1)$。如果将$x,y$当做平面上点的坐标,复数$z$就和平面上的点一一对应起来。
形成的平面叫做{\bf 复数平面},坐标轴成为{\bf 实轴}和{\bf 虚轴}。
\begin{figure}[htb]
    \centering
\usetikzlibrary{quotes,angles}
\usetikzlibrary {arrows.meta}
\begin{tikzpicture}[scale=1.5]
    \path (0,0) coordinate (origin);
    \path (4, 0) coordinate (x) ;
    \path (0, 2.5) coordinate (y) ;
    \path (4, 2.5) coordinate (z);
    \path (2.5, 2) coordinate (rho);
  
    \draw[->] (-0.2,0) --(4.2,0) node[right] {$\Re$};
    \draw[->] (0,-0.2) --(0,3.2) node[above] {$\Im$};
    \draw[solid, text=blue, thick, -{Stealth[length=2mm]}] (origin) -- (z) node[above] {$z=\rho e^{\imath \varphi}$};
    \draw[text=red, dashed]  (x) --(z) node[above]  {};
    \draw[text=red, dashed] (y) --(z) node[above] {};
  
    \node at (origin) [below] {$O$};
    \node at (x) [below ]{$x$};
    \node at (y) [left ]{ $y$};
    \node at (rho) [right] {$\rho$};
    \draw[color=red, fill=red] (z) circle (0.05);
  
    \draw pic["$\varphi$",draw, ->, angle eccentricity=1.2, angle radius=0.8cm]{angle = x--origin--z};
  
  \end{tikzpicture}
  \caption{复数的平面表示。} \label{fig:complex_plane}
\end{figure} 
自然我们可以改用极坐标来表示,
\begin{align}
    & \rho = \sqrt{x^2 + y^2}\\
    & \varphi = \arctan y/x
\end{align}
或
\begin{align}
    & x = \rho \cos\varphi \\
    & y = \rho \sin\varphi 
\end{align}
则我们得到复数$z$的三角式
\begin{equation}
    z = \rho (\cos\varphi +  \imath\; \sin\varphi) \textrm{,}
\end{equation}
或指数式
\begin{equation}
    z = \rho e^{\imath \varphi} \textrm{。}
\end{equation}
$\rho = |z|$ 为复数的{\bf 模}(modulus), $\varphi$为复数的{\bf 辐角}(argument),记作$\Arg z$。
对于任意一个$z=\rho e^{\imath \varphi}$,由于恒等式$e^{\imath 2\pi n} = 1$, $n = 0, \pm 1, \pm 2, \dots, \in \mathbb{Z}$,
可以知道辐角$\Arg z$不能唯一确定,它们之间相差$2\pi$的整数倍,其中满足
\begin{align}
    0 \leq \Arg z < 2\pi \textrm{,}
\end{align}
的辐角为$z$的主辐角,记为$\arg z$。$\arg z$ 为$\Arg z$的主值。
\begin{align}
    \Arg z = \arg z + 2 n \pi \quad (n = 0, \pm 1, \pm 2\dots)\textrm{。}
\end{align}
\subsubsection{基本性质}
我们可以利用有序实数对的方式对复数进行基本运算: 加减乘除运算。{\bf 加法}运算可以定义为
\begin{align}
    z_1 + z_2 = (x_1, y_1) + (x_2, y_2) = (x_1 + x_2, y_1 + y_2) \textrm{。}
\end{align}
{\bf 乘法}运算定义为
\begin{align}
    z_1 \cdot z_2 = (x_1, y_1) \cdot (x_2, y_2) = (x_1 x_2 - y_1 y_2, x_1 y_2 + x_2 y_1) \textrm{。}
\end{align}
显然加法和乘法满足{\bf 交换律}和{\bf 结合律},以后乘法运算符号$\cdot$均省略。
同实数一样,根据以上定义我们可以得到复数域中的一些特殊元素。对于任意复数$z$,复数域中存在元素$e$满足以下性质
\begin{align}
    & e + z = z + e = z \textrm{,}\\ 
    & e \cdot z = z \cdot e = z \textrm{,} 
\end{align}
可以得到对应的分别为
\begin{align}
    (0, 0) = 0\\
    (1, 0) = 1 \textrm{,}
\end{align}
通过元素$(0,0)$,我们可以定义$-z$使得 $-z + z = 0$。 于是有,$- z = (-x, -y)$。于是我们定义
{\bf 减法}运算为 
\begin{align}
    z_1 - z_2 \equiv z_1 + (-z_2) = (x_1 - x_2, y_1 - y_2) \textrm{,}
\end{align}
通过元素$(1,0)$,我们定义$z^{-1}$使得
$z^{-1} \cdot z = 1$,于是有 $z^{-1} =e^{-\imath \varphi}/\rho  $。
或$z^{-1} = (\frac{x}{x^2 + y^2}, -\frac{y}{x^2 + y^2})$。{\bf 除法}运算定义为
\begin{align}
    z_1 / z_2 \equiv z_1 \cdot z_2^{-1} = \frac{x_1 x_2 - y_1 y_2} {x_2^2  +  y_2^2 }  + \imath \frac{y_1 x_2 - x_1 y_2} {x_2^2  +  y_2^2 } \textrm{。} 
\end{align}
注意往往乘除写成极坐标表达更简洁:
\begin{align}
    z_1 z_2 = \rho_1 e^{\imath \varphi_1 } \rho_2 e^{\imath \varphi_2 } = \rho_1 \rho_2 e^{\imath (\varphi_1 + \varphi_2)}
\end{align}

此外,复数还有一种运算较为特殊,成为{\bf 共轭}(complex conjugation)运算。共轭运算表示为
\begin{align}
    z^{*} \equiv (x, -y) = x - \imath y \textrm{。}
\end{align}
为了得到$z$的模,我们可以利用共轭运算,$|z| = \sqrt{zz^{*}}$。注意区分$|z|^2$和$z^2$的不同。
\subsection{复数运算的几何表示}

\subsubsection{表示}

\section{复变函数}

\section{解析函数}


