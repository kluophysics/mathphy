\begin{tikzpicture}[scale=2]
    % draw the main circle
    \draw[fill=gray!20,dashed] (0,0) circle (1);
    % draw a point inside the circle
    \filldraw[black] (0.3,0.4) circle (0.03) node[below] {$z_0$};
    \node at (0.3, 0.4) [below left=0.3cm] {内点};
    % draw a point outside the circle
    \filldraw[black] (1.4,0.4) circle (0.03) node[above right] {外点};
    % draw a point on the circle
    \filldraw[black] (-0.6,-0.8) circle (0.03) node[below left] {边界点};
    % draw a small circle around the inside point
    % \draw[dashed] (0.3,0.4) circle (0.4) node[above] {邻域};
    \draw[dashed] (0.3,0.4) circle (0.2) node[above left=0.3cm] {邻域};
    % label the radius of the neighborhood
    \draw[<->] (0.3,0.4) -- node[right]{$\epsilon$} (0.3,0.6);
\end{tikzpicture}