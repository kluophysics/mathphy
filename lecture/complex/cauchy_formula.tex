\subsection{柯西积分公式}
\label{subsec:cauchy_formula}
复变函数理论中最重要的一个公式是柯西积分公式.若$f(z)$在一闭合回路围成的区域内解析,$z_0$为区域内一点,
则有
\begin{equation}
    f\left(z_0\right)=\frac{1}{2 \pi i} \oint_C \frac{f(z)}{z-z_0} d z .  
\end{equation}
柯西公式将解析函数在任何一内点 $z_0$ 的值 $f(z_0)$ 用沿边界线 $l$ 的回路积分 表示了出来.
也就是说,一个解析函数在闭合回路$C$内任意点$z_0$的值完全由该路径上的值决定的. 
这看起来不可思议,但又是必然结果.从物理上说, 解析函数紧密联系于平面标量场, 而平面场的边界条件决定着区域内部的场.
我们可以通过前面提到的柯西定理来证明.我们需要围绕$z_0$选取一个半径为$r$的内圆$\gamma$,由于$f(z)/(z-z_0)$在
内圆与$C$形成的区域解析,故我们有
\[ \frac{1}{2 \pi \imath} \oint_C \frac{f(z)}{z-z_0} d z = \frac{1}{2 \pi \imath} \oint_\gamma \frac{f(z)}{z-z_0} d z
    \]
有$z-z_0 = re^{\imath \theta}$,于是
\begin{equation}
    \begin{aligned}
        I &= \oint_\gamma \frac{f(z)}{z-z_0} d z
        \\
        & = \int _0 ^{2\pi} \frac{f(z_0 + re^{\imath \theta})}{re^{\imath \theta}} ire^{\imath \theta} d\theta
        \\
        & = \imath \int _0 ^{2\pi} f(z_0 + re^{\imath \theta}) d\theta
    \end{aligned}
\end{equation}
现令内圆半径$r\to 0$,则有$I\to 2\pi\imath f(z_0)$,得证.
由于$z_0$的任意性,我们可以改写成$z$,而把积分变量改成$\zeta$,则有
\begin{equation}
    f(z) = \frac{1}{2\pi \imath} \oint_C \frac{f(\zeta)}{\zeta - z} d \zeta.
    \label{eq:cauchy_formula}
\end{equation}

对柯西公式\eqref{eq:cauchy_formula}求导,我们有
\begin{equation}
    f'(z) = \frac{1!}{2\pi \imath} \oint_C \frac{f(\zeta)}{(\zeta - z)^2} d \zeta.
    \label{eq:cauchy_formula_1st_derivative}
\end{equation}
反复求导则有
\begin{equation}
    f^{(n)}(z) = \frac{n!}{2\pi \imath} \oint_C \frac{f(\zeta)}{(\zeta - z)^{n+1}} d \zeta.
    \label{eq:cauchy_formula_nth_derivative}
\end{equation}
% 这是因为解析函数在各点的值通过柯西 - 黎曼方程相互联系着. 

\begin{example}
    计算积分 
    $$ 
    \oint_{\gamma} \frac{1}{z^2 - 1} \, dz  
    $$
    其中 $\gamma$满足$|z| = 2$的闭合路径.
\end{example}
\begin{solution}
    考虑函数 \( f(z) = \frac{1}{z^2 - 1} \),它在 \( z = 1 \) 和 \( z = -1 \) 处有简单极点。我们选择半径为 2 的圆 \( \gamma \) 作为积分路径,
    如下图所示:

    \begin{center}
    \begin{tikzpicture}[scale=1.5]
        % 画复平面
        \draw[->] (-3,0) -- (3,0) node[right] {$\Re(z)$};
        \draw[->] (0,-3) -- (0,3) node[above] {$\Im(z)$};
        
        % 画半径为2的圆
        \draw[blue, thick] (0,0) circle (2);
        \node[blue] at (2.2, 1) {$\gamma$};
        
        % 标出极点
        \filldraw[red] (1,0) circle (0.05) node[below=5pt] {$z_1=1$};
        \filldraw[red] (-1,0) circle (0.05) node[below=5pt] {$z_2=-1$};
        
        % 绘制绕过 z=1 的小圆弧(顺时针)
        \draw[blue, thick] (1,0) circle (0.2);
        \node[blue] at (1.3,0.2) {$C_1$};

        % 绘制绕过 z=-1 的小圆弧(顺时针)
        \draw[blue, thick] (-1,0) circle (0.2);
        \node[blue] at (-1.3,0.2) {$C_2$};
        
        % 指示方向
        % \draw[->, blue, thick] (1.5,0.5) -- (1.6,0.6);
    \end{tikzpicture}
    \end{center}
    由(\ref{eq:cauchy_integral_summation})可以知道,
    关于路径 \( \gamma \)的积分
    可以转换为由以下两个部分的路径积分
    \begin{itemize}
        \item 绕过 \( z_1 = 1 \) 的小圆弧 \( C_1 \) 半径 \( \epsilon \),逆时针方向。
        \item 绕过 \( z_2 = -1 \) 的小圆弧 \( C_2 \) 半径 \( \epsilon \),逆时针方向。
    \end{itemize}
    由前面例题,并利用$\frac{1}{z^2 -1} = \frac{1}{2}\left( \frac{1}{z-1} - \frac{1}{z+1}\right)$
    \[
    \oint_{C_1} f(z) \, dz =\frac{1}{2}\oint_{C_1} \frac{1}{z-1} \, dz =  \frac{1}{2} \cdot 2\pi i  = \pi i
    \]
    类似地,有
    \[
        \oint_{C_2} f(z) \, dz =-\frac{1}{2}\oint_{C_2} \frac{1}{z+1} \, dz =  -\frac{1}{2} \cdot 2\pi i  = -\pi i
    \]
    综合起来:
    \[
        \oint_{\gamma} \frac{1}{z^2 - 1} \, dz = 0.
    \]
    注意,也可以用后面要学到的留数定理来求解.
\end{solution}

\begin{example}
    计算 
    $$
    I = \oint_\gamma \frac{e^z}{z^n}, 
    $$
    其中, $n\in \mathbb{Z}$, $\gamma$为满足$|z| = 1$的圆.
\end{example}
\begin{solution}
    我们需要计算复变函数 \( \frac{e^z}{z^n} \) 在单位圆 \( \gamma \) 上的闭合积分,
    其中 \( n \in \mathbb{Z} \).
函数 \( f(z) = \frac{e^z}{z^n} \) 的奇点位于 \( z = 0 \).
当 \( n \leq 0 \) 时,\( z = 0 \) 是解析点, 根据柯西定理可得积分为零.
当 \( n \geq 1 \) 时,\( z = 0 \) 是奇点, 根据柯西公式(\ref{eq:cauchy_formula_nth_derivative})可计算积分,
$$ I = \frac{2 \pi \imath }{(n-1)!}$$

综上所述,积分的结果为:

\[
I = \oint_\gamma \frac{e^z}{z^n} \, dz =
\begin{cases}
\frac{2\pi i}{(n-1)!}, & \text{如果 } n \geq 1, \\
0, & \text{如果 } n \leq 0.
\end{cases}
\]

\end{solution}


下面介绍柯西公式的重要推论.

\textbf{模数原理} $f(z)$在闭区域上解析,$|f(z)|$只能在边界线上取极大值.

由$f(z)^n = \frac{1}{2\pi \imath} \oint_C \frac{f(\zeta)^n}{\zeta - z} d \zeta$,若$|f(\zeta)|$在$C$上极大值为$M$,
$|\zeta - z|$的极小值为$\delta$, $C$的长度为$s$,则
\[
  |f(z)|^n \leq \frac{1}{2\pi} \frac{M^n}{\delta} s  ,
\]
即
\[
    |f(z)| \leq M \left( \frac{s}{2\pi \delta} \right)^{\frac{1}{n}},
\]
令$n\to \infty$,$|f(z)| \leq M$.证毕.

\textbf{刘维尔(Liouville)定理} \quad 如$f(z)$在全平面上解析且有界,则$f(z)$必为常数.

\textbf{证} \quad $f$有界,即$|f(z)| \leq N$, 对$f'(z)$取模,取以$z$为圆心半径为$R$的圆周,可得
\[
  |f'(z)| \leq \frac{1}{2\pi} \frac{N} {R^2} 2\pi R = \frac{N}{R},
\]
由于$R$任意选定,令$R\to \infty$,有$f'(z) \equiv 0$,所以$f(z)$为常数.

刘维尔定理的一个应用是可以证明代数基本定理,即对任意$n$阶多项式(Polynomial)
\begin{equation}
    P(z) = \sum_{k=0}^{n} a_k z^k (n>0, a_n \neq 0)
    \label{eq:poly}
\end{equation}
有$n$个根满足$P(z) = 0$. 它的证明可以通过反证的方法. 假设$P(z)$没有零点,即$P(z)\neq 0$, 则$1/P(z)$是解析
且有界的,根据刘维尔定理可知$1/P(z)$为常数,即$P(z)$为常数,与$a_n\neq 0$矛盾.可以知道,$P(z)$至少有一个根,记为$\lambda_1$,
那么对$P(z)/(z-\lambda_1)$这一$n-1$阶多项式进行上述论证,我们可以降次直至一阶多项式,共计$n$个根,因此$n$阶多项式
有$n$个根.


