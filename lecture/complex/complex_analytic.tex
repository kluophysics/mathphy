\subsection{解析函数}
\begin{definition}
若函数 $f(z)$ 在点 $z_0$ 及其邻域上处处可导, 则称 $f(z)$ 在 $z_0$ 点解析.\\
 又若 $f(z)$ 在区域 $B$ 上每一点都解析, 则称 $f(z)$ 是区域 $B$ 上的解析函数.
\end{definition} 
 可见, 函数若在某一点解析, 则必在该点可导. 反之却不一定成立. 若在全复数域上解析,我们
 称其为{\bf 完全函数}(entire function).
 若$f(z)$在某点$z_0$不可导,$z_0$称为$f(z)$的一个{\bf 奇点}(singular point).

 \begin{examplebox}{说明$f(z)=z^2$是完全函数,而$f(z)=|z|^2$的奇点数有无数个.}
    首先,$f(z) =z^2 = (x^2 + y ^2) + \imath 2 x y$,实部和虚部分别为$u(x,y) = x^2+ y^2, v(x,y)=2x y$,
连续条件和柯西-黎曼条件在全实数域均满足,可知$f(z)$在全复平面每点均可导.因此,$f(z)=z^2$是完全函数.
类似的,我们可以知道$f(z)=z^n, n\in N$的也是完全函数.\\
    接着,我们来看$f(z) = |z|^2 = (x^2 + y ^2)$,可以知道只有在$(0,0)$处满足可导条件,其他点均不解析.因此,$f(z)=|z|^2$的奇点数有无数个.
 \end{examplebox}

 解析函数实际上有着深刻的内涵.其中要义之一就是其实部和虚部(统一用$\psi$来表示)都必须满足二维的拉普拉斯方程即
 \begin{equation}
    \label{eq:Laplace_eq}
    \frac{\partial^2 \psi}{\partial x^2}+\frac{\partial^2 \psi}{\partial y^2}=0
 \end{equation}
上式可以利用柯西-黎曼条件进行验证,作为作业.$u,v$被成为调和函数或谐函数(harmonic functions)(注意不要同球谐函数spherical harmonics混淆).

第二要义就是,满足$u(x,y) = C_1$和$v(x,y)= C_2$的曲线为正交曲线族.
再次利用柯西-黎曼条件,可以验证梯度$\nabla u $和$\nabla v$正交,
\begin{equation}
    \nabla u \cdot \nabla v = \frac{\partial u}{\partial x} 
    \frac{\partial v}{\partial x}+\frac{\partial u}{\partial y} \frac{\partial v}{\partial y}=0 ,
\end{equation}
由于$\nabla u, \nabla v$代表两曲线的法向矢量,因此两曲线是正交的.

第三要义就是,当解析函数的实部(或虚部)给定,可以根据柯西-黎曼条件求解相应的虚部(或实部),进而确定该解析函数.
如已知实部,可以发现
\begin{equation}
    d v = \funcpd{v}{x} dx + \funcpd{v}{y} dy = -\funcpd{u}{y} dx + \funcpd{u}{x} dy
\end{equation}
可以验证 
$$
\funcpd{}{y} \left( - \funcpd{u}{y} \right) = \funcpd{}{x} \left( \funcpd{u}{x} \right) , 
$$
可知, $dv$是一全微分, $v = \int dv$.

\begin{examplebox}{解析函数实部为$u(x,y) = x^2 - y^2$,求解该解析函数.}
首先,可以验证$u$为调和函数.然后,利用柯西-黎曼条件可得
\begin{equation}
    \frac{\partial u}{\partial x } = 2x, \; \frac{\partial u}{\partial y } = -2y  ,
\end{equation}
因此
\begin{equation}
    d v = \frac{\partial v}{\partial x } dx + \frac{\partial v}{\partial y } dy = 2y dx + 2x dy .
\end{equation}
该积分与路径无关,可以用几种方法得到.容易得到$v(x,y) = 2xy + C$, $C$为积分常数.解析函数为$f(z)=x^2 - y^2 + \imath (2x y + C) = z^2 + \imath C$.
\end{examplebox}
    