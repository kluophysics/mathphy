\documentclass{beamer}

% \documentclass[10pt]{ctexbeamer}

\usepackage{ctex}
% % \usepackage{CJKutf8}
% % \usepackage{xeCJK}
% \usepackage{fontspec}
\usepackage[]{geometry}

% % \usepackage{CJKutf8}    % encode for Chinese
\usepackage{times}      % font for english, Times New Roman
% \usepackage{amsmath, amsfonts, amssymb} % math equations, symbols
% \usepackage[english]{babel}
% \usepackage{color}      % color content
% \usepackage{graphicx}   % import figures
% \usepackage{url}        % hyperlinks
% \usepackage{bm}         % bold type for equations
% \usepackage{hyperref}   % bookmarks
% \hypersetup{bookmarks, unicode} % unicode

% \usetheme{Madrid}
% \usecolortheme{whale} %颜色主题为
%\usetheme{Berkeley}  %演示主题为侧边导航条
% \usetheme{default}    %默认主题
% \usetheme[logo=NJUST]{ucas}
\usepackage{subfig}
\usepackage{amssymb,amsmath,mathtools}
\usepackage{amsfonts,booktabs}
\usepackage{lmodern,textcomp}
\usepackage{color}
\usepackage{tikz}
% \usepackage[latin1]{inputenc}
\usepackage{natbib}
\usepackage{multicol}

\usepackage{beamerthemesplit} % 加载主题宏包
\usetheme{Warsaw} % 选用该主题

\usepackage{graphicx}
\graphicspath{ {./tikz/} }


\def\Arg{{\mathrm{Arg}\;}}
\def\arg{{\mathrm{arg}\;}}
%Information to be included in the title page:
\title{数学物理方法}
\author[罗凯 NJUST]{罗凯}
% \institute[*]{南京理工大学理学院}
% \institute[*]{School of Science,\\
% Nanjing University of Science and Technology,\\
% kluo@njust.edu.cn}
\institute[*]{南京理工大学理学院\\
kluo@njust.edu.cn}
% \email{kluo@njust.edu.cn}
\date{2023年秋季}

\begin{document}
% \begin{CJK}{UTF8}{song}

% \AtBeginSection[]
% {

% % \begin{frame}
% %     \tableofcontents[currentsection,hideallsubsections]
% % \end{frame}

% % \begin{frame}
% %    	\ftitle{Outline 目录}		% contents for better review
% %     \tableofcontents[currentsection, currentsubsection]
% % \end{frame}
% \begin{frame}[shrink]
%     \tableofcontents[sectionstyle=show/shaded,subsectionstyle=show/shaded/hide]
% \end{frame}
% }



\begin{frame}
    \titlepage	% make the cover page here
\end{frame}

\begin{frame}[t]
    % \frametitle{目录}
    \tableofcontents
\end{frame}


\section{复数和复变函数}

\subsection{复数与复数运算}

\begin{frame}
\frametitle{复数(Complex Numbers)}

\begin{block}{定义}
    \begin{equation*}
        z = x + \imath y, \quad \imath = \sqrt{-1}
    \end{equation*} 
\end{block}
其中$x,y \in {\mathbb{R}}$。若记$z \equiv (x,y)$, 我们有
$1 = (1,0), \imath = (0, 1)$。
我们说$x,y$分别为复数$z$的{\bf 实部}(real part) 和{\bf 虚部}(imaginary part)。
\begin{columns}
\begin{column}{0.6\textwidth}
    \begin{figure}
        \includegraphics[scale=0.8]{complexplane.pdf}
        \caption{Argand diagram:复数的极坐标表示(polar representation)。}
    \end{figure}
\end{column}
\begin{column}{0.4\textwidth}   
     在极坐标中, $z=\rho e^{\imath \varphi}$, $\rho = |z|=\sqrt{x^2 + y^2}$ 为复数的模(modulus), $\varphi$为复数的辐角(argument),记作$\Arg z$。
    \begin{align*}
        & x = \rho \cos\varphi \\
        & y = \rho \sin\varphi \\
        & \varphi = \arctan y/x
    \end{align*}
\end{column}
\end{columns}
\end{frame}

% \begin{figure}
% \includegraphics[width=\linewidth*0.3]{tikz/complexplane.pdf}
% \includegraphics[width=\linewidth*0.3]{complexplane.pdf}
% \end{figure}

\begin{frame}
    \frametitle{多值问题}
$z=\rho e^{\imath \varphi}$, 由于恒等式$e^{\imath 2\pi n} = 1$, $n = 0, \pm 1, \pm 2, \dots, \in \mathbb{Z}$, 辐角$\Arg z$不能唯一确定。
它们之间相差$2\pi$的整数倍。其中满足
\begin{align*}
    0 \leq \Arg z < 2\pi,
\end{align*}
的辐角为$z$的主辐角,记为$\arg z$。$\arg z$ 为$\Arg z$的主值。
\begin{align*}
    \Arg z = \arg z + 2 n \pi (n = 0, \pm 1, \pm 2\dots).
\end{align*}

\end{frame}


\begin{frame}
    \frametitle{复数的运算}

    % \begin{columns}
    %     \begin{column}{0.7\textwidth}
        \begin{block}{共轭(conjugation)}
            \vspace*{-0.3cm}
            \begin{equation*}
                z^* = x -  \imath y
            \end{equation*}
        \end{block}
            \begin{block}{加减法(addition/subtraction)}
                \vspace*{-0.3cm}
                \begin{equation*}
                    z_1 \pm z_2 = x_1 \pm x_2 + \imath (y_1 \pm y_2)
                \end{equation*}
            \end{block}
            \begin{block}{乘法(multiplication)}
                \vspace*{-0.3cm}
                \begin{eqnarray*}
                    z_1 z_2&=& (x_1 +  \imath y_1 )  (x_2 +  \imath y_2 )
                    \\ 
                    &=& x_1 x_2 - y_1 y_2 + \imath  (y_1 x_2 + x_1 y_2)
                \end{eqnarray*}
            \end{block}
    %     \end{column}
    %     \begin{column}{0.4\textwidth}
    %     \end{column}
    % \end{columns}
    \begin{block}{除法(division)}
        \vspace*{-0.3cm}
        \begin{eqnarray*}
            \frac{z_1}{z_2} & = & \frac{x_1 +  \imath y_1 }{x_2 +  \imath y_2 }
           = \frac{x_1 +  \imath y_1 }{x_2 +  \imath y_2 } \frac{x_2 -  \imath y_2 }{x_2 -  \imath y_2 }
            \\
            & = & \frac{x_1 x_2 - y_1 y_2} {x_2^2  +  y_2^2 }  + \imath \frac{y_1 x_2 - x_1 y_2} {x_2^2  +  y_2^2 } 
        \end{eqnarray*}
    \end{block}
\end{frame}

% \begin{frame}
%     \frametitle{复数}
%     \begin{block}{复数的运算}
%         \begin{itemize}
%             \item  除法:   
%             \begin{eqnarray}
%                 \frac{z_1}{z_2} & = & \frac{x_1 +  \imath y_1 }{x_2 +  \imath y_2 }
%                 \\
%                 & = & \frac{x_1 +  \imath y_1 }{x_2 +  \imath y_2 } \frac{x_2 -  \imath y_2 }{x_2 -  \imath y_2 }
%                 \\
%                 & = & \frac{x_1 x_2 - y_1 y_2} {x_2^2  +  y_2^2 }  + \imath \frac{y_1 x_2 + x_1 y_2} {x_2^2  +  y_2^2 } 
%             \end{eqnarray}
%         \end{itemize}
%     \end{block}
% \end{frame}

\chapter{无穷级数}
\section{复数幂级数}
% \section{函数级数}

\subsection{级数的基本性质}
物理学和工程学中的一些函数常常可以用无穷级数来表示。一个很有用的例子,
\begin{equation}
    1+ x + x^2 + x^3 + \cdots = \frac{1}{1-x} \textrm{。}
\end{equation}
有了该式,我们可以处理更复杂的级数,如
\begin{equation}
    1 + a \cos \theta + a^2 \cos 2\theta + a^3 \cos 3\theta + \cdots = ? 
\end{equation}
有了前面的复数概念,我们有
\begin{equation}
    \cos \theta = \Re e^{\imath \theta} = \frac{e^{\imath \theta} +e^{-\imath \theta} }{2} \textrm{。}
\end{equation}
于是,
\begin{align*}
   & 1 + a \cos \theta + a^2 \cos 2\theta + a^3 \cos 3\theta + \cdots 
    \\  
 = &  1 + \half a e^{\imath \theta} + \half a e^{- \imath \theta} + \half a^2 e^{2\imath \theta} + \half a^2 e^{-2\imath \theta}  + \cdots 
 \\  
 =   & \half \left( 1 + a e^{\imath \theta} + a^2 e^{2\imath \theta} + \cdots \right)  
+ \half \left( 1 + a e^{- \imath \theta} + a^2 e^{-2\imath \theta}  + \cdots \right) 
\\  
= &  \half \frac{1}{1 - a e^{\imath \theta} } + \half \frac{1}{1 - a e^{-\imath \theta} }
\\  
= &  \half \left[ \frac{1-a\cos\theta + \imath a \sin \theta }{(1-a\cos\theta)^2 + (a\sin \theta)^2} + \frac{1-a\cos\theta - \imath a \sin \theta}{(1-a\cos\theta)^2 + (a\sin \theta)^2} \right]
 \\  
= &  \frac{1-a\cos\theta}{1-2 a \cos\theta + a^2} \textrm{。}
\end{align*}
现在我们要处理一个稍微复杂的级数,
\begin{equation}
    S(x) = 1 - \frac{x}{2} + \frac{x^2}{3} - \frac{x^3}{4} + \cdots
\end{equation}
为了将上式化简,我们需要把它转换成一个我们熟知的形式。由于分母比较特殊,我们想办法摆脱这些数字。为此,我们对$x S(x)$求导,得到
\begin{align}
    \frac{d}{dx} ( x S(x))  &=   \frac{d}{dx} \left( x- \frac{x^2}{2} + \frac{x^3}{3} - \frac{x^4}{4} \right)
    \nonumber \\ 
    &= 1 - x^2 + x^3 - x^4 + \cdots 
    \nonumber \\ 
    & = \frac{1}{1+x} \,
\end{align}
于是我们有,$
    x S(x) = \ln (1 + x) + C \textrm{。}$
当$x=0$, $S(x) = 1, \ln (1+x) = 0$, 有$C=0$。最终我们得到了
\begin{equation}
    \ln (1+x) = x -  \frac{x^2}{2} + \frac{x^3}{3} - \frac{x^4}{4} = \sum_{k=1}^{\infty} \frac{(-1)^{k+1} x^k}{k} \textrm{。}
\end{equation}
我们还可以利用指数的级数表示
\begin{equation}
    e^{x} = \sum_{n=0}^{\infty} \frac{x^n}{n!} = 1 + \frac{x}{1} + \frac{x^2}{2} + \frac{x^3}{3} + \cdots 
\end{equation}
来求解正余弦函数的级数表示。
\begin{enumerate}
    \item 几何级数: $ 1 + x + x^2 + x^3 + \cdots + x^n$, 对$|x|<1$,收敛于$\frac{1}{1-x}$。
    \item 调和级数: $ 1 + \half + \frac{1}{3} + \cdot +\frac{1}{k} = \sum_{k=1} \frac{1}{k}$,发散。
\end{enumerate}

值得注意的是,以上的求和表示实际上假定了$|x|<1$这个条件。容易验证,$|x|\geq 1$,级数是发散的。一般的,我们将复数的概念拓展到级数,
\begin{equation}
    s_n = \sum_{k=1}^{n} w_{k}
\end{equation}
随着,$n\to \infty$,部分求和趋于一个定值$S$,即
\begin{equation}
    \lim_{n\to \infty} s_n = S \textrm{,}
\end{equation}
我们称无穷级数 $\sum_{k=1}^{n} w_{k}${\bf 收敛}并趋于$S$。如果级数的求和趋于$\pm \infty$,级数{\bf 发散}。对于如
\begin{equation}
    \sum_{k=1}^{\infty} (-1)^k = 1 - 1 + 1 - 1 \cdots 
\end{equation}
这样的级数取值在$\pm 1$之间振荡,我们也称其发散。级数收敛的必要条件很显然是$\lim_{k\to \infty} w_k = 0$。对于级数是否收敛,在什么条件下收敛显得十分重要。
对级数收敛的充分条件寻找,派生出了各种各样的判据。

\subsection{级数的收敛判定法}

\subsubsection{柯西判据}
柯西判据(Cauchy criterion)说的是,对于$\epsilon>0$, 总存在固定的$N$使得$|s_j - s_i|< \epsilon$, 其中$i,j$是任意大于$N$的整数.也就是说,若$j>i$,
\begin{equation}
    | w_{i+1} + w_{i+2} + \cdots + w_{j} | < \epsilon ,
\end{equation}
直观地理解,也就是说某项以后的所有求和可以忽略不计,即部分求和趋于某一值,级数收敛.对于复数项级数,一样成立.如果将复数项级数的每一项都取模组成新的级数,
记为
\begin{equation}
    \sum_{k=1} |w_k| = \sum_{k=1}\sqrt { u_k^2 + v_k^2},
\end{equation}
若该级数收敛,则称原级数\textbf{绝对收敛}.绝对收敛的级数必然是收敛的.如果级数收敛,但非绝对收敛,那么我们称它为\textbf{条件收敛}.
\subsubsection{比较判定法}
如果我们有某一已知正项级数 $\sum_k a_k$收敛,若级数的每一项都满足 $0 \leq w_k \leq a_k$, 那么可以判定$\sum_k w_k$收敛.
这可以利用柯西判据进行证明.相反的, 若同发散级数$\sum_{k} a_k$比较有,$0 \leq a_k  \leq w_k$, 那么可以判定$\sum_k w_k$发散.
对于复数项级数,比较判据可以表示为$0 \leq |w_k| \leq a_k$.

\begin{examplebox}{试判定级数$\sum_{k=1} k^{-p}, p\leq 1$收敛还是发散.}
    已知调和级数$\sum_{k=1}1/k$是发散的,而 $k^{-p} > k^{-1}$, 根据比较判定法,可知该级数发散.
\end{examplebox}

\subsubsection{达朗贝尔判据}
% 级数$\sum_{n} w_n$,对足够大的$N$和与$N$无关的常数$r$,若满足条件$w_{n+1} / w_{n} \leq r < 1$,那么该级数收敛.反之,若$w_{n+1} / w_{n} \geq >= 1$,
% ,该级数发散.
达朗贝尔方法又称比值判定法(D'Alembert Ratio Criterion).
若任意项级数 $\sum_{n=1}^{\infty} w_n$ 通项满足:
$$
\lim _{n \to \infty}\left|\frac{w_{n+1}}{w_n}\right|=q ,
$$
\begin{enumerate}
    \item 当 $q<1$ 时, 级数绝对收敛;
    \item 当 $q>1$ 时, 级数发散;
    \item 当 $q=1$ 时, 此方法无效,需要其他方法判定.
\end{enumerate}

\begin{examplebox}{试判断级数$\sum_{k=1} k/2^k$的收敛性.}
    利用比值判定法
    \[ 
        \lim_{k\to \infty} \frac{w_{k+1}}{w_k}=\lim_{k\to \infty} \frac{k+1}{2^{k+1}} / \frac{k}{2^{k}}=\frac{1}{2} 
        = \lim_{k\to \infty} \frac{1}{2}\left(1+\frac{1}{k}\right) 
        = \half < 1
    \]
    可见该级数是收敛的.
\end{examplebox}
% $\frac{w_{k+1}}{w_k}=r , r 5 k$ 死关,
% 世東 $r<1$ ,那昍数收斂.
% 当 $r=1$ ,榇弅
% $$
% \frac{a_{k+1}}{a_k}=\frac{k}{k+1}<
% $$
% $$
% \begin{gathered}
% \frac{w_{k+1}}{w_k}=\frac{k+1}{2^{k+1}} / \frac{k}{2^{k+}}=\frac{1}{2} \frac{k+1}{k} \Rightarrow \frac{1}{2}<1 \\
% =\frac{1}{2}\left(1+\frac{1}{k}\right) \leqslant \frac{3}{4} \quad(k \geqslant 2)
% \end{gathered}
% $$

% \subsubsection{莱布尼兹判据(Leibniz Criterion)}
此外,还有很多其他判定方法,如对于交错级数可以利用莱布尼兹判据(Leibniz Criterion).
% \begin{enumerate}
%     \item 
% \end{enumerate}



% \end{CJK}
\end{document}