\section{复数和复变函数}

\subsection{复数与复数运算}

\begin{frame}
\frametitle{复数(Complex Numbers)}

\begin{block}{定义}
    \begin{equation*}
        z = x + \imath y, \quad \imath = \sqrt{-1}
    \end{equation*} 
\end{block}
其中$x,y \in {\mathbb{R}}$。若记$z \equiv (x,y)$, 我们有
$1 = (1,0), \imath = (0, 1)$。
我们说$x,y$分别为复数$z$的{\bf 实部}(real part) 和{\bf 虚部}(imaginary part)。
\begin{columns}
\begin{column}{0.6\textwidth}
    \begin{figure}
        \includegraphics[scale=0.8]{complexplane.pdf}
        \caption{Argand diagram:复数的极坐标表示(polar representation)。}
    \end{figure}
\end{column}
\begin{column}{0.4\textwidth}   
     在极坐标中, $z=\rho e^{\imath \varphi}$, $\rho = |z|=\sqrt{x^2 + y^2}$ 为复数的模(modulus), $\varphi$为复数的辐角(argument),记作$\Arg z$。
    \begin{align*}
        & x = \rho \cos\varphi \\
        & y = \rho \sin\varphi \\
        & \varphi = \arctan y/x
    \end{align*}
\end{column}
\end{columns}
\end{frame}

% \begin{figure}
% \includegraphics[width=\linewidth*0.3]{tikz/complexplane.pdf}
% \includegraphics[width=\linewidth*0.3]{complexplane.pdf}
% \end{figure}

\begin{frame}
    \frametitle{多值问题}
$z=\rho e^{\imath \varphi}$, 由于恒等式$e^{\imath 2\pi n} = 1$, $n = 0, \pm 1, \pm 2, \dots, \in \mathbb{Z}$, 辐角$\Arg z$不能唯一确定。
它们之间相差$2\pi$的整数倍。其中满足
\begin{align*}
    0 \leq \Arg z < 2\pi,
\end{align*}
的辐角为$z$的主辐角,记为$\arg z$。$\arg z$ 为$\Arg z$的主值。
\begin{align*}
    \Arg z = \arg z + 2 n \pi (n = 0, \pm 1, \pm 2\dots).
\end{align*}

\end{frame}


\begin{frame}
    \frametitle{复数的运算}

    % \begin{columns}
    %     \begin{column}{0.7\textwidth}
        \begin{block}{共轭(conjugation)}
            \vspace*{-0.3cm}
            \begin{equation*}
                z^* = x -  \imath y
            \end{equation*}
        \end{block}
            \begin{block}{加减法(addition/subtraction)}
                \vspace*{-0.3cm}
                \begin{equation*}
                    z_1 \pm z_2 = x_1 \pm x_2 + \imath (y_1 \pm y_2)
                \end{equation*}
            \end{block}
            \begin{block}{乘法(multiplication)}
                \vspace*{-0.3cm}
                \begin{eqnarray*}
                    z_1 z_2&=& (x_1 +  \imath y_1 )  (x_2 +  \imath y_2 )
                    \\ 
                    &=& x_1 x_2 - y_1 y_2 + \imath  (y_1 x_2 + x_1 y_2)
                \end{eqnarray*}
            \end{block}
    %     \end{column}
    %     \begin{column}{0.4\textwidth}
    %     \end{column}
    % \end{columns}
    \begin{block}{除法(division)}
        \vspace*{-0.3cm}
        \begin{eqnarray*}
            \frac{z_1}{z_2} & = & \frac{x_1 +  \imath y_1 }{x_2 +  \imath y_2 }
           = \frac{x_1 +  \imath y_1 }{x_2 +  \imath y_2 } \frac{x_2 -  \imath y_2 }{x_2 -  \imath y_2 }
            \\
            & = & \frac{x_1 x_2 - y_1 y_2} {x_2^2  +  y_2^2 }  + \imath \frac{y_1 x_2 - x_1 y_2} {x_2^2  +  y_2^2 } 
        \end{eqnarray*}
    \end{block}
\end{frame}

% \begin{frame}
%     \frametitle{复数}
%     \begin{block}{复数的运算}
%         \begin{itemize}
%             \item  除法:   
%             \begin{eqnarray}
%                 \frac{z_1}{z_2} & = & \frac{x_1 +  \imath y_1 }{x_2 +  \imath y_2 }
%                 \\
%                 & = & \frac{x_1 +  \imath y_1 }{x_2 +  \imath y_2 } \frac{x_2 -  \imath y_2 }{x_2 -  \imath y_2 }
%                 \\
%                 & = & \frac{x_1 x_2 - y_1 y_2} {x_2^2  +  y_2^2 }  + \imath \frac{y_1 x_2 + x_1 y_2} {x_2^2  +  y_2^2 } 
%             \end{eqnarray}
%         \end{itemize}
%     \end{block}
% \end{frame}
