\documentclass[11pt, a4paper]{article}
\usepackage{ctex}
%\usepackage{geometry}
\usepackage[inner=1.5cm,outer=1.5cm,top=2.5cm,bottom=2.5cm]{geometry}
\pagestyle{empty}
\usepackage{graphicx}
\usepackage{fancyhdr, lastpage, bbding, pmboxdraw}
\usepackage[usenames,dvipsnames]{color}
\definecolor{darkblue}{rgb}{0,0,.6}
\definecolor{darkred}{rgb}{.7,0,0}
\definecolor{darkgreen}{rgb}{0,.6,0}
\definecolor{red}{rgb}{.98,0,0}
\usepackage[colorlinks,pagebackref,pdfusetitle,urlcolor=darkblue,citecolor=darkblue,linkcolor=darkred,bookmarksnumbered,plainpages=false]{hyperref}
\renewcommand{\thefootnote}{\fnsymbol{footnote}}

\pagestyle{fancyplain}
\fancyhf{}
\lhead{ \fancyplain{}{Course Name} }
%\chead{ \fancyplain{}{} }
\rhead{ \fancyplain{}{\today} }
%\rfoot{\fancyplain{}{page \thepage\ of \pageref{LastPage}}}
\fancyfoot[RO, LE] {page \thepage\ of \pageref{LastPage} }
\thispagestyle{plain}

%%%%%%%%%%%% LISTING %%%
\usepackage{listings}
\usepackage{caption}
\DeclareCaptionFont{white}{\color{white}}
\DeclareCaptionFormat{listing}{\colorbox{gray}{\parbox{\textwidth}{#1#2#3}}}
\captionsetup[lstlisting]{format=listing,labelfont=white,textfont=white}
\usepackage{verbatim} % used to display code
\usepackage{fancyvrb}
\usepackage{acronym}
\usepackage{amsthm}
\VerbatimFootnotes % Required, otherwise verbatim does not work in footnotes!



\definecolor{OliveGreen}{cmyk}{0.64,0,0.95,0.40}
\definecolor{CadetBlue}{cmyk}{0.62,0.57,0.23,0}
\definecolor{lightlightgray}{gray}{0.93}



\lstset{
%language=bash,                          % Code langugage
basicstyle=\ttfamily,                   % Code font, Examples: \footnotesize, \ttfamily
keywordstyle=\color{OliveGreen},        % Keywords font ('*' = uppercase)
commentstyle=\color{gray},              % Comments font
numbers=left,                           % Line nums position
numberstyle=\tiny,                      % Line-numbers fonts
stepnumber=1,                           % Step between two line-numbers
numbersep=5pt,                          % How far are line-numbers from code
backgroundcolor=\color{lightlightgray}, % Choose background color
frame=none,                             % A frame around the code
tabsize=2,                              % Default tab size
captionpos=t,                           % Caption-position = bottom
breaklines=true,                        % Automatic line breaking?
breakatwhitespace=false,                % Automatic breaks only at whitespace?
showspaces=false,                       % Dont make spaces visible
showtabs=false,                         % Dont make tabls visible
columns=flexible,                       % Column format
morekeywords={__global__, __device__},  % CUDA specific keywords
}

%%%%%%%%%%%%%%%%%%%%%%%%%%%%%%%%%%%%
\begin{document}
\begin{center}
{\Large \textsc{数学物理方法}}
\end{center}
\begin{center}
2024秋季学期
\end{center}
%\date{September 26, 2014}

\begin{center}
\rule{6in}{0.4pt}
\begin{minipage}[t]{.75\textwidth}
\begin{tabular}{llcccll}
\textbf{授课者:} & 罗凯 & & &  & \textbf{时间:} & 周二、周四 15:50 -- 18:15 \\
\textbf{邮箱:} &  \href{mailto:kluo@njust.edu.cn}{kluo@njust.edu.cn} & & & & \textbf{地点:} & II-422.
\end{tabular}
\end{minipage}
\rule{6in}{0.4pt}
\end{center}
\vspace{.5cm}
\setlength{\unitlength}{1in}
\renewcommand{\arraystretch}{2}

\noindent\textbf{课程网址:} 
\begin{itemize}
\item \url{https://kluophysics.github.io/teaching/fall-2024/}
% \item \url{https://dlmf.nist.gov/}
\end{itemize}

% \vskip.15in
% \noindent\textbf{Office Hours:} 
% After class, or by appointment, 
% or post your questions in the forum provided 
% for this purpose on AeLP.

\vskip.15in
\noindent\textbf{主要参考书目:} %\footnotemark
This is a  restricted list of various interesting and useful books that will be touched during the course. You need to consult them occasionally.
\begin{itemize}
\item 梁昆淼,\textit{《数学物理方法(第五版)》},高等教育出版社, 2020, ISBN:978-7-04-051457-5

\item George B. Arfken, Hans J. Weber, Frank E. Harris, \textit{
 Mathematical Methods for Physicists, 7th Edition
}, Academic Press, 2013, ISBN:978-0-12-384654-9
\item 王竹溪, 郭敦仁, \textit{《特殊函数概论》},北京大学出版社, 2000, ISBN:978-7-30-104530-5
\item NIST Digital Library of Mathematical Functions, \url{https://dlmf.nist.gov/}
\item 
\end{itemize} 

% \footnotetext{Downloadable ebook versions are available on AeLP.}

\vskip.15in
\noindent\textbf{课程目标:} 培养学生的逻辑思维能力,物理建模能力,提升感受数学物理美感的能力。帮助学生树立科学的终身学习观,培养学生的自学能力和调研能力,
使学生初步具备解决简单常见物理和工程实际问题的素养。
\begin{itemize}
	\item 树立实事求是的科学态度和辩证唯物主义的世界观;
	\item 将具体物理工程问题抽象为数学问题的能力;
	\item 利用复变函数论知识理解常见物理工程问题的分析和解决方案,并习得相应的技巧。掌握常见数学物理方程的推导和求解,了解特殊函数的性质和使用;
	\item 自学和调研能力。
\end{itemize}
\vskip.15in
\noindent\textbf{课程介绍:} 数学物理方法是大学理工学科的一门重要基础课程,也是物理专业核心课。本课程的主要内容以复变函数论和数学物理方程两大部分为主,在此基础上增添了一些物理学前沿问题中的常见技巧和方法。其中,复变函数论部分主要包括复变函数、复平面上的路积分、洛朗级数展开与留数定理及其应用等章节,数学物理方程部分主要包括正交级数展开、偏微分方程与定解问题、柱坐标系与球坐标系定解问题以及格林函数方法等章节。通过本课程的学习,使学生掌握有关复变函数的基本理论,积分变换及数理方程的定解问题及其求解方法,
为进一步学习电动力学、量子力学等理论物理课程提供必要的数学基础。
\vskip.15in
\noindent\textbf{预备知识:}
高等数学中的微积分,向量分析等内容。

\vspace*{.15in}

\noindent \textbf{大致课程内容概要:}
\begin{center} 
\begin{minipage}{5in}
\begin{flushleft}
%Chapter 1 \dotfill ~$\approx$ 3 days \\
{\color{darkgreen}{\Rectangle}} 复数的定义
复数的运算 	
\\
{\color{darkgreen}{\Rectangle}} 区域的定义, 复变函数定义, 三角函数和双曲函数
\\
{\color{darkgreen}{\Rectangle}}
导数,解析函数,多值函数 
\\
{\color{red}{\Rectangle}}
复变函数积分定义和性质, 柯西积分定理  
\\
{\color{darkgreen}{\Rectangle}}
柯西积分公式,
例题和作业讲解 
\\
{\color{darkgreen}{\Rectangle}}
复数级数基本性质,
复数级数收敛判定方法
\\
{\color{red}{\Rectangle}}
幂级数,
泰勒级数,洛朗级数
\\
{\color{darkgreen}{\Rectangle}}
奇点的分类,
留数定理,
例题和作业讲解  
\\
{\color{red}{\Rectangle}}
留数定理的应用, 三角函数的积分,
积分上下限为正负无穷,
带复指数的定积分  
\\
{\color{darkgreen}{\Rectangle}}
费曼技巧,
复变量代换,
参数微分 ,
被积函数添加函数因子
\\
{\color{darkgreen}{\Rectangle}}
积分求解实例,
例题和作业讲解 
\\
{\color{darkgreen}{\Rectangle}}
变分法,
泛函,
泛函的极值,
哈密顿方程,
约束问题
\\
{\color{red}{\Rectangle}}
傅里叶级数,
傅里叶定理,
能量定理
\\
{\color{red}{\Rectangle}}
傅里叶变换,
傅里叶变换的基本性质,
高维傅里叶变换
\\
{\color{darkgreen}{\Rectangle}}
δ函数的概念,
δ函数性质,
其他,
例题和作业讲解
\\
{\color{darkgreen}{\Rectangle}}
拉普拉斯变换,
拉普拉斯变换的性质,
拉普拉斯反变换,
\\
{\color{darkgreen}{\Rectangle}}
拉普拉斯变换的应用,
求解定积分,
求解微分方程
\\
{\color{darkgreen}{\Rectangle}}
数学物理方程 ,
推导和分类,
弦振动方程,
热传导方程,
调和方程
\\
{\color{red}{\Rectangle}}
定解条件,
定解问题,
达朗贝尔公式
\\
{\color{red}{\Rectangle}}
分离变量法,
振动方程和输运方程
\\
{\color{darkgreen}{\Rectangle}}
非齐次边界条件的处理,
泊松方程
\\
{\color{red}{\Rectangle}}
特殊常微分方程,
常点邻域上的级数解法
\\
{\color{darkgreen}{\Rectangle}}
正则奇点邻域上的级数解法,
Sturm-Liouville问题
\\
{\color{red}{\Rectangle}}
特殊函数,轴对称球函数,连带勒让德函数 
\\
{\color{darkgreen}{\Rectangle}}
例题和作业讲解,
一般球函数
\\
{\color{red}{\Rectangle}}
三类柱函数,
贝塞尔方程,虚宗量贝塞尔方程
\\
{\color{darkgreen}{\Rectangle}}
泊松方程的格林函数法,
镜像法求格林函数
\\
{\color{darkgreen}{\Rectangle}}
推广的格林公式及其应用,
冲量定理法求格林函数

\end{flushleft}
\end{minipage}
\end{center}

\vspace*{.15in}
\noindent\textbf{考核方式:} 
课堂考勤和表现 (15\%),  
平时作业 (30\%), 
随堂测验 (15\%), 
期末考试 (40\%) 
%Four Projects (40\% = 4 * 10\%)

\vskip.15in
\noindent\textbf{重要日期提醒:}
\begin{center} \begin{minipage}{3.8in}
\begin{flushleft}
第四周, 国庆节放假      \dotfill 10月1-7日 没有课 \\
期末考试 (待教务处通知)   \dotfill ~ \\
%Project Deadline \dotfill ~Month Day \\
% Final Exam       \dotfill ~Dey 18, 1393  \\
\end{flushleft}
\end{minipage}
\end{center}

\vskip.15in
\noindent\textbf{注意事项:}
\begin{itemize}
	\item 请添加课程QQ群(通过搜索群号:700246241,或扫描下方二维码),
	进群后务必将备注改成实名,方便交流; 课程讲义和作业禁止外传;
	\item 有情况请假提前告知,不随意迟到, 禁止旷课;
	\item 不定期点名,课堂随机提问;
	\item 独立高质量完成作业是学习本门课程的关键。
\end{itemize}

\begin{center} \begin{minipage}{3in}
\begin{flushleft}
	\includegraphics[width=\textwidth]{qrcode.jpg}
\end{flushleft}
\end{minipage}
\end{center}


% \vskip.15in
% \noindent\textbf{注意事项:}  
% \begin{itemize}
% \item Regular attendance is essential and expected.
% \end{itemize}

% \vskip.15in
% \noindent\textbf{Academic Honesty:}   Lack of knowledge of the academic honesty policy is not a reasonable explanation for a violation.


%%%%%% THE END 
\end{document} 