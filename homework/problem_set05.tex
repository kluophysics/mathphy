\documentclass[10pt]{article}
 
\usepackage[margin=1in]{geometry} 
\usepackage{amsmath,amsthm,amssymb, graphicx, multicol, array}
\usepackage{ctex} 
\usepackage{fancyhdr}
\usepackage{fontspec}
\usepackage{fourier-orns} % 

\newcommand{\N}{\mathbb{N}}
\newcommand{\Z}{\mathbb{Z}}
 
\newenvironment{problem}[2][]{\begin{trivlist}
\item[\hskip \labelsep {\bfseries #1}\hskip \labelsep {\bfseries #2}]}{\end{trivlist}}

\renewcommand{\headrule}{%
\vspace{-8pt}\hrulefill
\raisebox{-2.1pt}{\quad\decofourleft\decotwo\decofourright\quad}\hrulefill}


\newcommand{\duedate}{10.20}
\pagestyle{fancy}

% \renewcommand{\chaptermark}[1]{\markboth{\CTEXthechapter\quad #1}{}}
% \renewcommand{\sectionmark}[1]{\markright{\CTEXthesection\quad #1}}
% \renewcommand{\sectionmark}[1]{\markleft{\CTEXthesection\quad #1}}

% \fancyhf{}
% \fancyhead[LE]{\nouppercase{\rightmark\hfill\leftmark}}
% \fancyhead[RO]{\nouppercase{\leftmark\hfill\rightmark}}

%... then configure it.
\fancyhead{} % clear all header fields
% \fancyhead[LE]{\textbf{PHYS}}
% \fancyhead[L]{\textbf{\leftmark}}
% \fancyhead[L]{\chaptermark{}}
% \fancyhead[L]{\sectionmark{}}
% \fancyhead[LR]{\textbf{\leftmark}}
\fancyhead[L]{数学物理方法11044102}
\fancyhead[R]{课后习题}

% \fancyhead[RO]{\textbf{\rightmark}}

% \fancyhead[LO]{\textbf{Kai Luo}}
% \fancyhead[LO]{罗凯}

% \fancyhead[LE]{\nouppercase{\rightmark\hfill\leftmark}}
% \fancyhead[RO]{\nouppercase{\leftmark\hfill\rightmark}}

\fancyfoot{} % clear all footer fields
% \fancyfoot[CE,CO]{\rightmark}
% \fancyfoot[LO,CE]{}
% \fancyfoot[CO,RE]{Nanjing University of Science and Technology}
\fancyfoot[C]{\thepage}
\fancyfoot[L]{截止日期\duedate} % clear all footer fields
\fancyfoot[R]{任课教师:罗凯} % clear all footer fields

% \fancyfoot[CE,O]{\thechapter}



\begin{document}
\renewcommand{\labelenumi}{(\arabic{enumi})}
\renewcommand{\labelenumii}{(\arabic{enumi}.\arabic{enumii})}



% \title{习题01}
% \author{截止日期: }
% \author{罗凯\\
% 数学物理方法}
% \date{截止日期:}
% \maketitle
 
% \begin{problem}{1.1}
% 找出满足方程\[z - 2 = 3 \frac{1+ \imath t}{1-\imath t}, -\infty < t <\infty\]的所有点$z$.
% \end{problem}




\begin{problem}{5.1}
计算积分
$$
\int_0^{2 \pi} \cos ^{2 n} x \mathrm{~d} x \text {. }
$$
\end{problem}

\begin{problem}{5.2}
  计算积分
  $$
  \int_{-\infty}^{\infty} \frac{e^{\imath m x}}{x-\imath \alpha} d x,
  $$
  和
  $$
  \int_{-\infty}^{\infty} \frac{e^{\imath m x}}{x+\imath \alpha} d x.
  $$
  其中$m>0, \Re \alpha > 0$.
  \end{problem}
  
\begin{problem}{5.3}
    使用费曼技巧计算
    $$
    I = \int_0^1 \frac{x^\alpha-1}{\log x} d x . 
    $$
  \end{problem}
   
\begin{problem}{5.4}
     使用费曼技巧计算
     $$
     \int_0^{\infty} x^2 e^{-\left(4 x^2+\frac{9}{x^2}\right)} d x .
     $$
\end{problem}

\begin{problem}{5.5}
试推导多目标变量泛函$J\left(u_1, u_2, \cdots, u_n, u_1^{\prime}, u_2^{\prime}, \cdots, u_n^{\prime} \mid x\right)$
取极值的欧拉方程是
$$
\frac{\partial L}{\partial u_i}-\frac{\partial}{\partial x}\left(\frac{\partial L}{\partial u_i^{\prime}}\right)=0 .
$$
\end{problem}


\begin{problem}{5.6}
  过二已知点 $\left(x_1, y_1\right),\left(x_2, y_2\right)$ 作一曲线, 使此曲线绕 $x$ 轴旋转所得曲面面积最小, 求曲线满足的微分方程.
\end{problem}

\begin{problem}{5.7}
请求解最速降线和等高悬链线问题的解.
它们对应的泛函表示为
$$
T[u] = \int_{x_0}^{x_1} \frac{\sqrt{1+u^{\prime 2}}}{\sqrt{2 g\left(y_0-u\right)}} d x, 
$$
和
$$
U[y(x)]= \rho g  \int_{-L / 2}^{L / 2} \sqrt{1+y^{\prime}(x)^2} \cdot y(x) d x. 
$$
\end{problem}



\begin{problem}{5.8}
在无轨道密度泛函(orbital-free density functional theory)里体系的总能量可以表示为密度$n(\mathbf{r})$的泛函$E[n]$,
$$
E[n] = T_s[n] + E_H[n] + \int dr^3 v_{ext}(\mathbf{r}) n(\mathbf{r}) + E_{xc}[n]
$$
密度满足$\int n(\mathbf{r}) dr^3 = N$这一约束, 其中$N$为总电子数.
动能部分 $T_s[n]$有常见两种近似:
\begin{enumerate}
  \item 托马斯-费米 (Thomas-Fermi) 理论里
  $$
  T_s[n] = c_{T F} \int dr^3 n^{5 / 3}(\mathbf{r});
  $$ 
  \item 冯-瓦塞克 (von Weisz\"acker)理论里
  $$
  T_s[n] =  \frac{1}{8} \int dr^3 \frac{|\nabla n(\mathbf{r})|^2}{n(\mathbf{r})}.
  $$
\end{enumerate}
经典电子云之间的作用能为Hartree能量
$$
E_H[n] = 
 \frac{1}{2} 
 \int \frac{n(\mathbf{r}) n(\mathbf{r}')}{|\mathbf{r} - \mathbf{r}'|}
 dr^3 dr'^3
$$
交换关联能$ E_{xc}[n]$这里忽略. 请对 $T_s[n]$常见两种近似下的总能量进行变分求解能量最小对应的欧拉方程.
\end{problem}
  



\end{document}