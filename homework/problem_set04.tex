\documentclass[10pt]{article}
 
\usepackage[margin=1in]{geometry} 
\usepackage{amsmath,amsthm,amssymb, graphicx, multicol, array}
\usepackage{ctex} 
\usepackage{fancyhdr}
\usepackage{fontspec}
\usepackage{fourier-orns} % 

\newcommand{\N}{\mathbb{N}}
\newcommand{\Z}{\mathbb{Z}}
 
\newenvironment{problem}[2][]{\begin{trivlist}
\item[\hskip \labelsep {\bfseries #1}\hskip \labelsep {\bfseries #2}]}{\end{trivlist}}

\renewcommand{\headrule}{%
\vspace{-8pt}\hrulefill
\raisebox{-2.1pt}{\quad\decofourleft\decotwo\decofourright\quad}\hrulefill}


\newcommand{\duedate}{9.28}
\pagestyle{fancy}

% \renewcommand{\chaptermark}[1]{\markboth{\CTEXthechapter\quad #1}{}}
% \renewcommand{\sectionmark}[1]{\markright{\CTEXthesection\quad #1}}
% \renewcommand{\sectionmark}[1]{\markleft{\CTEXthesection\quad #1}}

% \fancyhf{}
% \fancyhead[LE]{\nouppercase{\rightmark\hfill\leftmark}}
% \fancyhead[RO]{\nouppercase{\leftmark\hfill\rightmark}}

%... then configure it.
\fancyhead{} % clear all header fields
% \fancyhead[LE]{\textbf{PHYS}}
% \fancyhead[L]{\textbf{\leftmark}}
% \fancyhead[L]{\chaptermark{}}
% \fancyhead[L]{\sectionmark{}}
% \fancyhead[LR]{\textbf{\leftmark}}
\fancyhead[L]{数学物理方法11044102}
\fancyhead[R]{课后习题}

% \fancyhead[RO]{\textbf{\rightmark}}

% \fancyhead[LO]{\textbf{Kai Luo}}
% \fancyhead[LO]{罗凯}

% \fancyhead[LE]{\nouppercase{\rightmark\hfill\leftmark}}
% \fancyhead[RO]{\nouppercase{\leftmark\hfill\rightmark}}

\fancyfoot{} % clear all footer fields
% \fancyfoot[CE,CO]{\rightmark}
% \fancyfoot[LO,CE]{}
% \fancyfoot[CO,RE]{Nanjing University of Science and Technology}
\fancyfoot[C]{\thepage}
\fancyfoot[L]{截止日期\duedate} % clear all footer fields
\fancyfoot[R]{任课教师:罗凯} % clear all footer fields

% \fancyfoot[CE,O]{\thechapter}



\begin{document}
\renewcommand{\labelenumi}{(\arabic{enumi})}
\renewcommand{\labelenumii}{(\arabic{enumi}.\arabic{enumii})}



% \title{习题01}
% \author{截止日期: }
% \author{罗凯\\
% 数学物理方法}
% \date{截止日期:}
% \maketitle
 
% \begin{problem}{1.1}
% 找出满足方程\[z - 2 = 3 \frac{1+ \imath t}{1-\imath t}, -\infty < t <\infty\]的所有点$z$.
% \end{problem}


\begin{problem}{4.1}
  在指定的点 $z_0$ 的邻域上将下列函数展开为泰勒级数.
  \begin{enumerate}
    \item $\arctan z$的主值, 在 $z_0=0$;(提示:试着利用级数展开然后逐项积分.)
    \item $\ln z$ 在 $z_0=\imath$;
    \item $\ln \left(1+e^z\right)$ 在 $z_0=0$.
  \end{enumerate}
\end{problem}

\begin{problem}{4.2}
  在指定环域上(可挖去除奇点$z_0$)将下列函数展开为洛朗级数.
  \begin{enumerate}
    \item$1 / z^2(z-1)$ 在 $z_0=1$;
    \item $1 /(z-2)(z-3)$ 在 $|z|>3$;
    \item $\ln \left(1+e^z\right)$ 在 $z_0=0$;
    \item $\frac{1}{z(z+1)}$, 在 $1<|z-i|<\sqrt{2}$.
  \end{enumerate}
\end{problem}

\begin{problem}{4.3}
 求下列函数在指定点 $z_0$ 的留数:

 \begin{enumerate}
  \item  $\frac{e^{z^2}}{z-1}, z_0=1$;
  \item $\frac{e^{z^2}}{(z-1)^2}, \quad z_0=1$;
  \item $\left(\frac{z}{1-\cos z}\right)^2, z_0=0$;
  \item  $\frac{1+e^z}{z^4}, z_0=0$;
  \item $\frac{1}{z^2 \sin z}, z_0=0$.
 \end{enumerate}

\end{problem}


  
\begin{problem}{4.5}
  计算:
  \begin{itemize}
    \item 
    (i) $\int_{|z|=1} \frac{d z}{z}$;
    (ii) $\int_{|z|=1} \frac{d z}{|z|}$;
    (iii) $\int_{|z|=1} \frac{|d z|}{z}$;
    (iv) $\int_{|z|=1}\left|\frac{d z}{z}\right|$ .

    \item  $\oint_C \frac{\sin \frac{\pi z}{4}}{z^2-1} d z, C$ 分别为

    (i) $|z|=\frac{1}{2}$,
    (ii) $|z-1|=1$,
    (iii) $|z|=3$.
    
  \end{itemize}

\end{problem}


\begin{problem}{4.6}
  写出$\sin{z},\cos{z}$的级数展开,并写出各自的前4项.
  \end{problem}
  



\begin{problem}{4.7}
  判断下列级数的收敛性及绝对收敛性:
  \begin{enumerate}
    \item $\sum \frac{i^n}{\ln n}$;
    \item $\sum \frac{i^n}{n}$.
  \end{enumerate}
\end{problem}

\begin{problem}{4.8}
  确定下列级数的收敛半径(或收敛区域):
  \begin{enumerate}
    \item $\sum z^n$;
    \item $\sum \frac{1}{2^n n^n} z^n$;
    \item $\sum \frac{n !}{n^n} z^n$;
    \item $\sum n^{\ln{n}} z^n$;
    \item $\sum 2^n \sin \frac{z}{3^n}$;
    \item $\sum \frac{\ln \left(n^n\right)}{n !} z^n$.
  \end{enumerate}

\end{problem}


\begin{problem}{4.9}
  已知幂级数 $\sum a_n z^n$ 和 $\sum b_n z^n$ 的收敛半径分别为 $R_1, R_2$, 
  试讨论下列幂级数的收敛半径
  \begin{enumerate}
    \item $\sum\left(a_n+b_n\right) z^n$;
    \item $\sum a_n b_n z^n$.
  \end{enumerate}
\end{problem}

% \begin{proof}[Solution]
% Write a solution here
% \begin{align*}
% x &= y
% \end{align*}
% \begin{tabular}{| >{\centering\arraybackslash}m{1in} | >{\centering\arraybackslash}m{1in} | >{\centering\arraybackslash}m{1in} | >{\centering\arraybackslash}m{1in} |>{\centering\arraybackslash}m{1in} |}
% \hline 
%   \textbf{A} & \textbf{B} & \textbf{C} & \textbf{D} &\textbf{E} \\[8pt]
%   \hline
%   a & b & c & d & e \\[8pt]
%   \hline
%   f & g &h & i & j \\[8pt]
%   \hline
% \end{tabular}
% \end{proof}

\end{document}