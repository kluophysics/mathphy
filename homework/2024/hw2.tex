\documentclass[11pt]{article}
% \documentclass{assignment}

\usepackage[margin=1in]{geometry} 
\usepackage{amsmath,amsthm,amssymb, graphicx, multicol, array}
\usepackage{ctex} 
\usepackage{fancyhdr}
\usepackage{fontspec}
\usepackage{fourier-orns} % 

% http://kuing.infinityfreeapp.com/forum.php?mod=viewthread&tid=527&i=1
\let\oldjuhao=。
\catcode`\。=\active
\newcommand{。}{\ifmmode\text{\oldjuhao}\else\oldjuhao\fi}

\let\olddouhao=,
\catcode`\,=\active
\newcommand{,}{\ifmmode\text{\olddouhao}\else\olddouhao\fi}



\newcommand{\N}{\mathbb{N}}
\newcommand{\Z}{\mathbb{Z}}
 
\theoremstyle{remark}
\newtheorem{problem}{}


% \newenvironment{problem}[2][]{\begin{trivlist}
% \item[\hskip \labelsep {\bfseries #1}\hskip \labelsep {\bfseries #2}]}{\end{trivlist}}

\renewcommand{\headrule}{%
\vspace{-8pt}\hrulefill
\raisebox{-2.0pt}{\quad\decofourleft\decotwo\decofourright\quad}\hrulefill}


\newcommand{\duedate}{\zhdate{2024/10/10}}
\pagestyle{fancy}

% \fancyhf{}
% \fancyhead[LE]{\nouppercase{\rightmark\hfill\leftmark}}
% \fancyhead[RO]{\nouppercase{\leftmark\hfill\rightmark}}

%... then configure it.
\fancyhead{} % clear all header fields
% \fancyhead[LE]{\textbf{PHYS}}
% \fancyhead[L]{\textbf{\leftmark}}
% \fancyhead[L]{\chaptermark{}}
% \fancyhead[L]{\sectionmark{}}
% \fancyhead[LR]{\textbf{\leftmark}}
\fancyhead[L]{数学物理方法}
\zhnumsetup{style={Traditional,Financial}}
\fancyhead[R]{课后习题\zhnumber{1}}

% \fancyhead[RO]{\textbf{\rightmark}}

% \fancyhead[LO]{\textbf{Kai Luo}}
% \fancyhead[LO]{罗凯}

% \fancyhead[LE]{\nouppercase{\rightmark\hfill\leftmark}}
% \fancyhead[RO]{\nouppercase{\leftmark\hfill\rightmark}}

\fancyfoot{} % clear all footer fields
% \fancyfoot[CE,CO]{\rightmark}
% \fancyfoot[LO,CE]{}
% \fancyfoot[CO,RE]{Nanjing University of Science and Technology}
\fancyfoot[C]{\thepage}
\fancyfoot[L]{截止日期\duedate} % clear all footer fields
\fancyfoot[R]{任课教师:罗凯} % clear all footer fields

% \fancyfoot[CE,O]{\thechapter}



\begin{document}
\renewcommand{\labelenumi}{(\arabic{enumi})}
\renewcommand{\labelenumii}{(\arabic{enumi}.\arabic{enumii})}



% \title{习题01}
% \author{截止日期: }
% \author{罗凯\\
% 数学物理方法}
% \date{截止日期:}
% \maketitle
 

\begin{problem}
  证明
  $$
  |\sinh{y}| \leq |\sin(x+\imath y)| \leq |\cosh{y}|.
  $$
  提示: $\cosh^2{x} = 1+\sinh^2{x}$.
  \end{problem}

\begin{problem}
  求解方程
  $$
  2 \cosh^2 z - 3\cosh z +1 = 0.
  $$
  \end{problem}

\begin{problem}
  验证解析函数的实部和虚部满足二维拉普拉斯方程. 若将解析函数$f(z)$写成极坐标的形式$f(z) = R(r,\theta) e^{\imath \Theta(r,\theta)}$,
验证
\begin{itemize}
\item $\frac{\partial R}{\partial r}=\frac{R}{r} \frac{\partial \Theta}{\partial \theta}$,
\item $\frac{1}{r} \frac{\partial R}{\partial \theta}=-R \frac{\partial \Theta}{\partial r}$.
\end{itemize}
\end{problem}

\begin{problem}
  若将复变函数$f(z)=u+\imath v$看成是$x,y$的二元函数.再看成是$z=x+\imath y, z^*= x-\imath y$,的二元函数,试证明Cauchy-Riemann方程等价于
  $$
  \frac{\partial f}{\partial z^{*}} = 0.
  $$ 
  提示: 利用$x=(z+z^*)/2, y=(z-z^*)/(2\imath)$.
\end{problem}

\begin{problem}
  证明:若函数 $f ( z )$ 在区域 $B$内解析,其模为一常数,则函数 $f ( z )$ 本身也必为一常数.
\end{problem}

\begin{problem}
  若 $f(z)$ 和 $g(z)$ 在 $z=a$ 点解析, 且 $f(a)=g(a)=0$, 而 $g^{\prime}(a) \neq 0$, 试证:
  $$
\lim _{z \rightarrow a} \frac{f(z)}{g(z)}=\frac{f^{\prime}(a)}{g^{\prime}(a)}.
$$
\end{problem}

\begin{problem}
  找出下列函数的支点,并讨论$z$绕各支点一周回到原处函数值的变化.
  \begin{enumerate}
    \item $\sqrt[3]{1-z^3}$,
    \item $\sqrt{\frac{z-a}{z-b}}$,
    \item $\ln{(z^2 + 1)}$.
  \end{enumerate}
\end{problem}

\begin{problem}
  验证
  $$\oint_C \frac{d z}{z^2+z}=0,
  $$
  路径$C$由$|z|=R>1$确定的圆.\\

  提示: 直接应用柯西定理是错误的,需要分解因式裂项.
\end{problem}

% \begin{problem}
% \end{problem}


% \begin{problem}
%   试证明极坐标下的柯西-黎曼方程
%   $$
%     \left\{\begin{array}{l}
%     \frac{\partial u}{\partial \rho}=\frac{1}{\rho} \frac{\partial v}{\partial \varphi} ,\\
%     \frac{1}{\rho} \frac{\partial u}{\partial \varphi}=-\frac{\partial v}{\partial \rho} .
%     \end{array}\right.
%   $$
% \end{problem}

% \begin{problem}
%   已知解析函数 $f(z)$  的实部 $u(x, y)$或虚部$ v(x, y)$, 求该解析函数.
%   \begin{enumerate}
%     \item $u = e^x \sin{y}$,
%     \item $u = x^2 - y^2 + xy$, $f(0) = 0$.
%   \end{enumerate}
% \end{problem}

% \begin{proof}[Solution]
% Write a solution here
% \begin{align*}
% x &= y
% \end{align*}
% \begin{tabular}{| >{\centering\arraybackslash}m{1in} | >{\centering\arraybackslash}m{1in} | >{\centering\arraybackslash}m{1in} | >{\centering\arraybackslash}m{1in} |>{\centering\arraybackslash}m{1in} |}
% \hline 
%   \textbf{A} & \textbf{B} & \textbf{C} & \textbf{D} &\textbf{E} \\[8pt]
%   \hline
%   a & b & c & d & e \\[8pt]
%   \hline
%   f & g &h & i & j \\[8pt]
%   \hline
% \end{tabular}
% \end{proof}

\end{document}