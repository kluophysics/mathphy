\documentclass[11pt]{article}
% \documentclass{assignment}

\usepackage[margin=1in]{geometry} 
\usepackage{amsmath,amsthm,amssymb, graphicx, multicol, array}
\usepackage{ctex} 
\usepackage{fancyhdr}
\usepackage{fontspec}
\usepackage{fourier-orns} % 
 
\newcommand{\N}{\mathbb{N}}
\newcommand{\Z}{\mathbb{Z}}
 
\theoremstyle{remark}
\newtheorem{problem}{}


% \newenvironment{problem}[2][]{\begin{trivlist}
% \item[\hskip \labelsep {\bfseries #1}\hskip \labelsep {\bfseries #2}]}{\end{trivlist}}

\renewcommand{\headrule}{%
\vspace{-8pt}\hrulefill
\raisebox{-2.0pt}{\quad\decofourleft\decotwo\decofourright\quad}\hrulefill}


\newcommand{\duedate}{\zhdate{2024/12/19}}
\pagestyle{fancy}

% \fancyhf{}
% \fancyhead[LE]{\nouppercase{\rightmark\hfill\leftmark}}
% \fancyhead[RO]{\nouppercase{\leftmark\hfill\rightmark}}

%... then configure it.
\fancyhead{} % clear all header fields
% \fancyhead[LE]{\textbf{PHYS}}
% \fancyhead[L]{\textbf{\leftmark}}
% \fancyhead[L]{\chaptermark{}}
% \fancyhead[L]{\sectionmark{}}
% \fancyhead[LR]{\textbf{\leftmark}}
\fancyhead[L]{数学物理方法}
\zhnumsetup{style={Traditional,Financial}}
\fancyhead[R]{课后习题拾} 

% \fancyhead[RO]{\textbf{\rightmark}}

% \fancyhead[LO]{\textbf{Kai Luo}}
% \fancyhead[LO]{罗凯}

% \fancyhead[LE]{\nouppercase{\rightmark\hfill\leftmark}}
% \fancyhead[RO]{\nouppercase{\leftmark\hfill\rightmark}}

\fancyfoot{} % clear all footer fields
% \fancyfoot[CE,CO]{\rightmark}
% \fancyfoot[LO,CE]{}
% \fancyfoot[CO,RE]{Nanjing University of Science and Technology}
\fancyfoot[C]{\thepage}
\fancyfoot[L]{截止日期\duedate} % clear all footer fields
\fancyfoot[R]{任课教师:罗凯} % clear all footer fields

% \fancyfoot[CE,O]{\thechapter}



\begin{document}
\renewcommand{\labelenumi}{(\arabic{enumi})}
\renewcommand{\labelenumii}{(\arabic{enumi}.\arabic{enumii})}

% \begin{problem}
%   厄米多项式(Hermite polynomials) $H_n(x)$的生成函数为
%   $$
%   e^{-t^2+2 t x}=\sum_{n=0}^{\infty} H_n(x) \frac{t^n}{n !}
%   $$
%   请利用生成函数推导出下面的递推关系式,
%   \begin{itemize}
%     \item[(1)] $2 x H_n(x)-2 n H_{n-1}(x)=H_{n+1}(x)$;
%     \item[(2)] $2 n H_{n-1}(x)=H_n^{\prime}(x)$ . 
%   \end{itemize}
% \end{problem}

% \vspace{4em}

\begin{problem}
根据勒让德函数的生成函数,推导下面的递推关系式
\begin{itemize}
  \item[(1)] $
  (2 \ell+1) P_\ell(x)=P_{\ell+1}^{\prime}(x)-P_{\ell-1}^{\prime}(x),
  $
  \item[(2)]
  $
  P_{\ell+1}^{\prime}(x)=(\ell+1) P_\ell(x)+x P_\ell^{\prime}(x).
  $
\end{itemize}

\end{problem}
  
% \vspace{4em}

% \begin{problem}
%   使用递推关系(recurrence relation)或生成函数或罗德里格斯公式验证连带勒让德函数满足
%   $$
%   \begin{aligned}
%   P_{2 l}^1(0) & =0, \\
%   P_{2 l+1}^1(0) & =(-1)^{l} \frac{(2 l+1) ! !}{(2 l) ! !}.
%   \end{aligned}
%   $$
% \end{problem}




\begin{problem}
计算下列积分
\begin{itemize}
  \item[(1)] $$
  \int_{-1}^1\left(1-x^2\right) P_k^{\prime}(x) P_l^{\prime}(x) d x
  $$
  \item[(2)]
  $$
\int_{-1}^1 x P_k(x) P_{k+1}(x) d x
$$
\end{itemize}
\end{problem}

\begin{problem}
  求解下列定解问题: 
  $$\left\{\begin{array}{l}
    \nabla^2 u=0, a<r<b, \\ 
    \left.u\right|_{r=a}=u_0,\left.u\right|_{r=b}=u_0 \cos ^2 \theta
  \end{array}\right.$$
\end{problem}


\begin{problem}
  球谐函数的一般表达式为
  $$
  Y_l^m(\theta, \varphi) \equiv \sqrt{\frac{2 l+1}{4 \pi} \frac{(l-m) !}{(l+m) !}} P_l^m(\cos \theta) e^{i m \varphi}
  $$
  请求出确定以下的具体表达式$m$的取值范围,写出其中$m\geq 0$关于$\theta,\varphi$的表达式, 并转换成直角坐标系的形式
  \begin{itemize}
    \item[(1)] $Y_0^m(\theta,\varphi)$
    \item[(2)] $Y_1^m(\theta,\varphi)$
    \item[(3)] $Y_2^m(\theta,\varphi)$
  \end{itemize}
  \end{problem}

\end{document}