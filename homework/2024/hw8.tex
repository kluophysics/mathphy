\documentclass[11pt]{article}
% \documentclass{assignment}

\usepackage[margin=1in]{geometry} 
\usepackage{amsmath,amsthm,amssymb, graphicx, multicol, array}
\usepackage{ctex} 
\usepackage{fancyhdr}
\usepackage{fontspec}
\usepackage{fourier-orns} % 
 
\newcommand{\N}{\mathbb{N}}
\newcommand{\Z}{\mathbb{Z}}
 
\theoremstyle{remark}
\newtheorem{problem}{}


% \newenvironment{problem}[2][]{\begin{trivlist}
% \item[\hskip \labelsep {\bfseries #1}\hskip \labelsep {\bfseries #2}]}{\end{trivlist}}

\renewcommand{\headrule}{%
\vspace{-8pt}\hrulefill
\raisebox{-2.0pt}{\quad\decofourleft\decotwo\decofourright\quad}\hrulefill}


\newcommand{\duedate}{\zhdate{2024/12/03}}
\pagestyle{fancy}

% \fancyhf{}
% \fancyhead[LE]{\nouppercase{\rightmark\hfill\leftmark}}
% \fancyhead[RO]{\nouppercase{\leftmark\hfill\rightmark}}

%... then configure it.
\fancyhead{} % clear all header fields
% \fancyhead[LE]{\textbf{PHYS}}
% \fancyhead[L]{\textbf{\leftmark}}
% \fancyhead[L]{\chaptermark{}}
% \fancyhead[L]{\sectionmark{}}
% \fancyhead[LR]{\textbf{\leftmark}}
\fancyhead[L]{数学物理方法}
\zhnumsetup{style={Traditional,Financial}}
\fancyhead[R]{课后习题捌} 

% \fancyhead[RO]{\textbf{\rightmark}}

% \fancyhead[LO]{\textbf{Kai Luo}}
% \fancyhead[LO]{罗凯}

% \fancyhead[LE]{\nouppercase{\rightmark\hfill\leftmark}}
% \fancyhead[RO]{\nouppercase{\leftmark\hfill\rightmark}}

\fancyfoot{} % clear all footer fields
% \fancyfoot[CE,CO]{\rightmark}
% \fancyfoot[LO,CE]{}
% \fancyfoot[CO,RE]{Nanjing University of Science and Technology}
\fancyfoot[C]{\thepage}
\fancyfoot[L]{截止日期\duedate} % clear all footer fields
\fancyfoot[R]{任课教师:罗凯} % clear all footer fields

% \fancyfoot[CE,O]{\thechapter}



\begin{document}
\renewcommand{\labelenumi}{(\arabic{enumi})}
\renewcommand{\labelenumii}{(\arabic{enumi}.\arabic{enumii})}

\begin{problem}
  一长为 $\ell$ 的均匀金属细杆 (可近似看作一维的), 
  通有恒定电流。设杆的一端 $(x=0$ )温度恒为 0 , 另一端 $(x=\ell)$ 恒为 $T_0$, 
  初始时温度分布为 $\frac{T_0}{\ell} x$.
  试写出杆中温度场所满足的方程, 边界条件与初始条件。
\end{problem}

\vspace{2em}

\begin{problem}
 将下面的偏微分方程化成标准形式.
 \begin{itemize}
  \item [(1)] $
  u_{x x}+4 u_{x y}+5 u_{y y}+u_x+2 u_y=0;
  $
  \item [(2)]
  $
  u_{x x}+x u_{y y}=0.
  $

 \end{itemize}

\end{problem}
  

\begin{problem}
请利用高斯定理推导出真空中有源静电场方程,请问求解需要初始条件吗?请用文字说明.
\end{problem}

\begin{problem}
  课堂上,我们将$x,y$变量代换成$\xi, \eta$给出了$u_{xy}$的表达式,请推导验证
  $$
  u_{x x} 
  % =\left(u_{\xi \xi} \xi_x^2+u_{\xi \eta} \xi_x \eta_x+u_{\xi} \xi_{x x}\right)+\left(u_{\eta \xi} \eta_x \xi_x+u_{\eta \eta} \eta_x^2+u_\eta \eta_{x x}\right) \\
  =u_{\xi \xi} \xi_x^2+2 u_{\xi \eta} \xi_x \eta_x+u_{\eta \eta} \eta_x^2+u_{\xi} \xi_{x x}+u_\eta \eta_{x x}.
  $$.
\end{problem}
   
\begin{problem}
  整数$m$阶贝塞尔方程(Bessel's equation)
  $$
    x^2 y^{\prime \prime}+x y^{\prime}+\left(x^2-m^2\right) y = 0, 
  $$
的Laplace变换($m = 0$)为一阶常微分方程
  $$
   (p^2 + 1) f'(p) + pf(p) = 0.
  $$
   试求解$f(p)$.
\end{problem}

\begin{problem}
求$$
(2x + e^{y^2}) dx + (2xy e^{y^2} -2y ) dy = 0
$$
的通解.
\end{problem}
  



  
  


% \begin{problem}
% \end{problem}

% \begin{problem}
% \end{problem}

% \begin{problem}
% \end{problem}

\end{document}