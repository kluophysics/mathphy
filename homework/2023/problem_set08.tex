\documentclass[10pt]{article}
 
\usepackage[margin=1in]{geometry} 
\usepackage{amsmath,amsthm,amssymb, graphicx, multicol, array}
\usepackage{ctex} 
\usepackage{fancyhdr}
\usepackage{fontspec}
\usepackage{fourier-orns} % 

\newcommand{\N}{\mathbb{N}}
\newcommand{\Z}{\mathbb{Z}}
 
\newenvironment{problem}[2][]{\begin{trivlist}
\item[\hskip \labelsep {\bfseries #1}\hskip \labelsep {\bfseries #2}]}{\end{trivlist}}

\renewcommand{\headrule}{%
\vspace{-8pt}\hrulefill
\raisebox{-2.1pt}{\quad\decofourleft\decotwo\decofourright\quad}\hrulefill}


\newcommand{\duedate}{11.20}
\pagestyle{fancy}

% \renewcommand{\chaptermark}[1]{\markboth{\CTEXthechapter\quad #1}{}}
% \renewcommand{\sectionmark}[1]{\markright{\CTEXthesection\quad #1}}
% \renewcommand{\sectionmark}[1]{\markleft{\CTEXthesection\quad #1}}

% \fancyhf{}
% \fancyhead[LE]{\nouppercase{\rightmark\hfill\leftmark}}
% \fancyhead[RO]{\nouppercase{\leftmark\hfill\rightmark}}

%... then configure it.
\fancyhead{} % clear all header fields
% \fancyhead[LE]{\textbf{PHYS}}
% \fancyhead[L]{\textbf{\leftmark}}
% \fancyhead[L]{\chaptermark{}}
% \fancyhead[L]{\sectionmark{}}
% \fancyhead[LR]{\textbf{\leftmark}}
\fancyhead[L]{数学物理方法11044102}
\fancyhead[R]{课后习题}

% \fancyhead[RO]{\textbf{\rightmark}}

% \fancyhead[LO]{\textbf{Kai Luo}}
% \fancyhead[LO]{罗凯}

% \fancyhead[LE]{\nouppercase{\rightmark\hfill\leftmark}}
% \fancyhead[RO]{\nouppercase{\leftmark\hfill\rightmark}}

\fancyfoot{} % clear all footer fields
% \fancyfoot[CE,CO]{\rightmark}
% \fancyfoot[LO,CE]{}
% \fancyfoot[CO,RE]{Nanjing University of Science and Technology}
\fancyfoot[C]{\thepage}
\fancyfoot[L]{截止日期\duedate} % clear all footer fields
\fancyfoot[R]{任课教师:罗凯} % clear all footer fields

% \fancyfoot[CE,O]{\thechapter}



\begin{document}
\renewcommand{\labelenumi}{(\arabic{enumi})}
\renewcommand{\labelenumii}{(\arabic{enumi}.\arabic{enumii})}



% \title{习题01}
% \author{截止日期: }
% \author{罗凯\\
% 数学物理方法}
% \date{截止日期:}
% \maketitle
 
% \begin{problem}{1.1}
% 找出满足方程\[z - 2 = 3 \frac{1+ \imath t}{1-\imath t}, -\infty < t <\infty\]的所有点$z$.
% \end{problem}




\begin{problem}{8.1}
  一长为 $\ell$ 的均匀金属细杆 (可近似看作一维的), 
  通有恒定电流。设杆的一端 $(x=0$ )温度恒为 0 , 另一端 $(x=\ell)$ 恒为 $T_0$, 
  初始时温度分布为 $\frac{T_0}{\ell} x$.
  试写出杆中温度场所满足的方程, 边界条件与初始条件。
\end{problem}

\vspace{2em}

\begin{problem}{8.2}
 将下面的偏微分方程化成标准形式.
 \begin{itemize}
  \item [(1)] $
  u_{x x}+4 u_{x y}+5 u_{y y}+u_x+2 u_y=0;
  $
  \item [(2)]
  $
  u_{x x}+x u_{y y}=0.
  $


 \end{itemize}

\end{problem}
  

\vspace{2em}

\begin{problem}{8.3}
  整数$m$阶贝塞尔方程(Bessel's equation)
  $$
    x^2 y^{\prime \prime}+x y^{\prime}+\left(x^2-m^2\right) y = 0, 
  $$
的Laplace变换($m = 0$)为一阶常微分方程
  $$
   (p^2 + 1) f'(p) + pf(p) = 0.
  $$
   试求解$f(p)$.
\end{problem}


\vspace{2em}


\begin{problem}{8.4}
试求解常系数常微分方程
$$
y^{\prime \prime \prime}-2 y^{\prime \prime}+y^{\prime}-2 y=0 .
$$
\end{problem}

\vspace{2em}

\begin{problem}{8.5}
  在 $x_0=0$ 的邻域上用级数求解整数$m$阶贝塞尔方程.
\end{problem}

\vspace{2em}

\begin{problem}{8.6}
在 $x_0=0$ 的邻域上求解高斯方程 (超几何级数微分方程) 
$$x(x-1) y^{\prime \prime}+[(1+\alpha+\beta) x-\gamma] y^{\prime}+\alpha \beta y=0,$$
其中$\alpha, \beta, \gamma$为常数.
\end{problem} 

% \begin{problem}{8.7}
%   用$B$函数求积分
%   $$
%   \int_0^{\frac{\pi}{2}} \tan^\alpha{\theta} \,  d\theta, \quad  -1 < \alpha < 1.
%   $$
% \end{problem} 

\end{document}