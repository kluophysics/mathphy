\documentclass[10pt]{article}
 
\usepackage[margin=1in]{geometry} 
\usepackage{amsmath,amsthm,amssymb, graphicx, multicol, array}
\usepackage{ctex} 
\usepackage{fancyhdr}
\usepackage{fontspec}
\usepackage{fourier-orns} % 

\newcommand{\N}{\mathbb{N}}
\newcommand{\Z}{\mathbb{Z}}
 
\newenvironment{problem}[2][]{\begin{trivlist}
\item[\hskip \labelsep {\bfseries #1}\hskip \labelsep {\bfseries #2}]}{\end{trivlist}}

\renewcommand{\headrule}{%
\vspace{-8pt}\hrulefill
\raisebox{-2.1pt}{\quad\decofourleft\decotwo\decofourright\quad}\hrulefill}


\newcommand{\duedate}{11.02}
\pagestyle{fancy}

% \renewcommand{\chaptermark}[1]{\markboth{\CTEXthechapter\quad #1}{}}
% \renewcommand{\sectionmark}[1]{\markright{\CTEXthesection\quad #1}}
% \renewcommand{\sectionmark}[1]{\markleft{\CTEXthesection\quad #1}}

% \fancyhf{}
% \fancyhead[LE]{\nouppercase{\rightmark\hfill\leftmark}}
% \fancyhead[RO]{\nouppercase{\leftmark\hfill\rightmark}}

%... then configure it.
\fancyhead{} % clear all header fields
% \fancyhead[LE]{\textbf{PHYS}}
% \fancyhead[L]{\textbf{\leftmark}}
% \fancyhead[L]{\chaptermark{}}
% \fancyhead[L]{\sectionmark{}}
% \fancyhead[LR]{\textbf{\leftmark}}
\fancyhead[L]{数学物理方法11044102}
\fancyhead[R]{课后习题}

% \fancyhead[RO]{\textbf{\rightmark}}

% \fancyhead[LO]{\textbf{Kai Luo}}
% \fancyhead[LO]{罗凯}

% \fancyhead[LE]{\nouppercase{\rightmark\hfill\leftmark}}
% \fancyhead[RO]{\nouppercase{\leftmark\hfill\rightmark}}

\fancyfoot{} % clear all footer fields
% \fancyfoot[CE,CO]{\rightmark}
% \fancyfoot[LO,CE]{}
% \fancyfoot[CO,RE]{Nanjing University of Science and Technology}
\fancyfoot[C]{\thepage}
\fancyfoot[L]{截止日期\duedate} % clear all footer fields
\fancyfoot[R]{任课教师:罗凯} % clear all footer fields

% \fancyfoot[CE,O]{\thechapter}



\begin{document}
\renewcommand{\labelenumi}{(\arabic{enumi})}
\renewcommand{\labelenumii}{(\arabic{enumi}.\arabic{enumii})}



% \title{习题01}
% \author{截止日期: }
% \author{罗凯\\
% 数学物理方法}
% \date{截止日期:}
% \maketitle
 
% \begin{problem}{1.1}
% 找出满足方程\[z - 2 = 3 \frac{1+ \imath t}{1-\imath t}, -\infty < t <\infty\]的所有点$z$.
% \end{problem}




\begin{problem}{6.1}
  
计算矩形波$$
\begin{aligned}
& f(x)=0, \quad-\pi<x<0, \\
& f(x)=h, \quad 0<x<\pi .
\end{aligned}
$$
的傅里叶级数展开.注意观察级数展开在间断点的取值是否为预期.
\end{problem}

\begin{problem}{6.2}
  将 $\delta(x)$ 展为实数形式的傅里叶积分.
\end{problem}
  
\begin{problem}{6.3}
证明傅里叶变换$\mathcal{F}[f(x)] = F(\omega)$ 满足以下定理.
\begin{enumerate}
  \item 导数定理
  $$
      \mathcal{F} [f'(x)] = \imath \omega F(\omega)
  $$

  \item 积分定理
  $$
      \mathcal{F} [ \int^{x} f(x) dx ] = \frac{1}{\imath \omega} F(\omega)
  $$
  \item 位移定理
  $$
      \mathcal{F} [ e^{\imath \omega_0 x} f(x) ] = F(\omega - \omega_0),
  $$
  \item 卷积定理
  $$
      \mathcal{F} [f_1(x)\star f_2(x) ] = F_1(\omega) F_2(\omega).
  $$
\end{enumerate}
\end{problem}
   
\begin{problem}{6.4}
对于屏蔽效应的描述可以使用Yukawa势来表示
$$
V(\mathbf{r}) = \frac{e^{-\alpha r}}{r}
$$
其中$\alpha$描述屏蔽的强弱.
\begin{enumerate}
  \item 试在球坐标系下求出三维傅里叶变换;
  \item 当$\alpha=Z$时, 函数$e^{-Zr}$可描述类氢原子的$1s$轨道, 对其傅里叶变换关于$Z$求偏导
  $-\frac{\partial}{\partial Z} \mathcal{F} \left[ \frac{e^{-Z r}}{r}\right]$;
  \item 若是三维高斯形式$V(\mathbf{r}) = e^{-\alpha r^2}$,求其傅里叶变换.
\end{enumerate}
\end{problem}

\begin{problem}{6.5}
  \begin{enumerate}
  \item 验证$f(x)=x^{-1 / 2}$ 在Fourier傅里叶余弦和正弦变换下互易;即
  $$
  \begin{aligned}
  & \sqrt{\frac{2}{\pi}} \int_0^{\infty} x^{-1 / 2} \cos {x t} \, dx=t^{-1 / 2} \\
  & \sqrt{\frac{2}{\pi}} \int_0^{\infty} x^{-1 / 2} \sin {x t} \, dx=t^{-1 / 2}
  \end{aligned}
  $$
  \item 使用上述结果计算菲涅耳积分(Fresnel integrals)
  $$
  \int_0^{\infty} \cos \left(y^2\right) d y \text { , } \int_0^{\infty} \sin \left(y^2\right) d y
  $$
\end{enumerate}

\end{problem}




\end{document}