\documentclass[10pt]{article}
 
\usepackage[margin=1in]{geometry} 
\usepackage{amsmath,amsthm,amssymb, graphicx, multicol, array}
\usepackage{ctex} 
\usepackage{fancyhdr}
\usepackage{fontspec}
\usepackage{fourier-orns} % 

\newcommand{\N}{\mathbb{N}}
\newcommand{\Z}{\mathbb{Z}}
 
\newenvironment{problem}[2][]{\begin{trivlist}
\item[\hskip \labelsep {\bfseries #1}\hskip \labelsep {\bfseries #2}]}{\end{trivlist}}

\renewcommand{\headrule}{%
\vspace{-8pt}\hrulefill
\raisebox{-2.1pt}{\quad\decofourleft\decotwo\decofourright\quad}\hrulefill}


\newcommand{\duedate}{9.11}
\pagestyle{fancy}

% \renewcommand{\chaptermark}[1]{\markboth{\CTEXthechapter\quad #1}{}}
% \renewcommand{\sectionmark}[1]{\markright{\CTEXthesection\quad #1}}
% \renewcommand{\sectionmark}[1]{\markleft{\CTEXthesection\quad #1}}

% \fancyhf{}
% \fancyhead[LE]{\nouppercase{\rightmark\hfill\leftmark}}
% \fancyhead[RO]{\nouppercase{\leftmark\hfill\rightmark}}

%... then configure it.
\fancyhead{} % clear all header fields
% \fancyhead[LE]{\textbf{PHYS}}
% \fancyhead[L]{\textbf{\leftmark}}
% \fancyhead[L]{\chaptermark{}}
% \fancyhead[L]{\sectionmark{}}
% \fancyhead[LR]{\textbf{\leftmark}}
\fancyhead[L]{数学物理方法11044102}
\fancyhead[R]{课后习题}

% \fancyhead[RO]{\textbf{\rightmark}}

% \fancyhead[LO]{\textbf{Kai Luo}}
% \fancyhead[LO]{罗凯}

% \fancyhead[LE]{\nouppercase{\rightmark\hfill\leftmark}}
% \fancyhead[RO]{\nouppercase{\leftmark\hfill\rightmark}}

\fancyfoot{} % clear all footer fields
% \fancyfoot[CE,CO]{\rightmark}
% \fancyfoot[LO,CE]{}
% \fancyfoot[CO,RE]{Nanjing University of Science and Technology}
\fancyfoot[C]{\thepage}
\fancyfoot[L]{截止日期\duedate} % clear all footer fields
\fancyfoot[R]{任课教师:罗凯} % clear all footer fields

% \fancyfoot[CE,O]{\thechapter}



\begin{document}
\renewcommand{\labelenumi}{(\arabic{enumi})}
\renewcommand{\labelenumii}{(\arabic{enumi}.\arabic{enumii})}



% \title{习题01}
% \author{截止日期: }
% \author{罗凯\\
% 数学物理方法}
% \date{截止日期:}
% \maketitle
 
% \begin{problem}{1.1}
% 找出满足方程\[z - 2 = 3 \frac{1+ \imath t}{1-\imath t}, -\infty < t <\infty\]的所有点$z$.
% \end{problem}

\begin{problem}{1.1}
  证明
  $$
  |\sinh{y}| \leq |\sin(x+\imath y)| \leq |\cosh{y}|.
  $$
  提示: $\cosh^2{x} = 1+\sinh^2{x}$.
\end{problem}

  
\begin{problem}{1.2}
验证解析函数的实部和虚部满足二维拉普拉斯方程. 若将解析函数$f(z)$写成极坐标的形式$f(z) = R(r,\theta) e^{\imath \Theta(r,\theta)}$,
验证
\begin{itemize}
\item $\frac{\partial R}{\partial r}=\frac{R}{r} \frac{\partial \Theta}{\partial \theta}$,
\item $\frac{1}{r} \frac{\partial R}{\partial \theta}=-R \frac{\partial \Theta}{\partial r}$.
\end{itemize}

\end{problem}

\begin{problem}{1.3}
  若将复变函数$f(z)=u+\imath v$看成是$x,y$的二元函数.再看成是$z=x+\imath y, z^*= x-\imath y$,的二元函数,试证明Cauchy-Riemann方程等价于
  $$
  \frac{\partial f}{\partial z^{*}} = 0.
  $$ 
  提示: 利用$x=(z+z^*)/2, y=(z-z^*)/(2\imath)$.
\end{problem}



\begin{problem}{1.4}
  证明:若函数 $f ( z )$ 在区域 $B$内解析,其模为一常数,则函数 $f ( z )$ 本身也必为一常数.
\end{problem}

\begin{problem}{1.5}
  若 $f(z)$ 和 $g(z)$ 在 $z=a$ 点解析, 且 $f(a)=g(a)=0$, 而 $g^{\prime}(a) \neq 0$, 试证:
  $$
\lim _{z \rightarrow a} \frac{f(z)}{g(z)}=\frac{f^{\prime}(a)}{g^{\prime}(a)}.
$$
\end{problem}

\begin{problem}{1.6}
若 $u(x, y)$ 具有连续三阶偏导数, 且 $\frac{\partial^2 u}{\partial x^2}+\frac{\partial^2 u}{\partial y^2}=0$,
 证明函数 $\frac{\partial u}{\partial x}-i \frac{\partial u}{\partial y}$ 解析.
\end{problem}

\begin{problem}{1.7}
  找出下列函数的支点,并讨论$z$绕各支点一周回到原处函数值的变化.
  \begin{enumerate}
    \item $\sqrt[3]{1-z^3}$,
    \item $\sqrt{\frac{z-a}{z-b}}$,
    \item $\ln{(z^2 + 1)}$.
  \end{enumerate}

\end{problem}

\begin{problem}{1.8}
  指出多值函数$f(z) = \ln{(z-a)}$的支点及其阶数,找出其割线,并画出示意图.
\end{problem}


\begin{problem}{1.9}
  验证
  $$\oint_C \frac{d z}{z^2+z}=0,
  $$
  路径$C$由$|z|=R>1$确定的圆.\\

  提示: 直接应用柯西定理是错误的,需要分解因式裂项.

\end{problem}

% \begin{proof}[Solution]
% Write a solution here
% \begin{align*}
% x &= y
% \end{align*}
% \begin{tabular}{| >{\centering\arraybackslash}m{1in} | >{\centering\arraybackslash}m{1in} | >{\centering\arraybackslash}m{1in} | >{\centering\arraybackslash}m{1in} |>{\centering\arraybackslash}m{1in} |}
% \hline 
%   \textbf{A} & \textbf{B} & \textbf{C} & \textbf{D} &\textbf{E} \\[8pt]
%   \hline
%   a & b & c & d & e \\[8pt]
%   \hline
%   f & g &h & i & j \\[8pt]
%   \hline
% \end{tabular}
% \end{proof}

\end{document}