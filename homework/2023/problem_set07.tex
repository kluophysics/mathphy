\documentclass[10pt]{article}
 
\usepackage[margin=1in]{geometry} 
\usepackage{amsmath,amsthm,amssymb, graphicx, multicol, array}
\usepackage{ctex} 
\usepackage{fancyhdr}
\usepackage{fontspec}
\usepackage{fourier-orns} % 

\newcommand{\N}{\mathbb{N}}
\newcommand{\Z}{\mathbb{Z}}
 
\newenvironment{problem}[2][]{\begin{trivlist}
\item[\hskip \labelsep {\bfseries #1}\hskip \labelsep {\bfseries #2}]}{\end{trivlist}}

\renewcommand{\headrule}{%
\vspace{-8pt}\hrulefill
\raisebox{-2.1pt}{\quad\decofourleft\decotwo\decofourright\quad}\hrulefill}


\newcommand{\duedate}{11.13}
\pagestyle{fancy}

% \renewcommand{\chaptermark}[1]{\markboth{\CTEXthechapter\quad #1}{}}
% \renewcommand{\sectionmark}[1]{\markright{\CTEXthesection\quad #1}}
% \renewcommand{\sectionmark}[1]{\markleft{\CTEXthesection\quad #1}}

% \fancyhf{}
% \fancyhead[LE]{\nouppercase{\rightmark\hfill\leftmark}}
% \fancyhead[RO]{\nouppercase{\leftmark\hfill\rightmark}}

%... then configure it.
\fancyhead{} % clear all header fields
% \fancyhead[LE]{\textbf{PHYS}}
% \fancyhead[L]{\textbf{\leftmark}}
% \fancyhead[L]{\chaptermark{}}
% \fancyhead[L]{\sectionmark{}}
% \fancyhead[LR]{\textbf{\leftmark}}
\fancyhead[L]{数学物理方法11044102}
\fancyhead[R]{课后习题}

% \fancyhead[RO]{\textbf{\rightmark}}

% \fancyhead[LO]{\textbf{Kai Luo}}
% \fancyhead[LO]{罗凯}

% \fancyhead[LE]{\nouppercase{\rightmark\hfill\leftmark}}
% \fancyhead[RO]{\nouppercase{\leftmark\hfill\rightmark}}

\fancyfoot{} % clear all footer fields
% \fancyfoot[CE,CO]{\rightmark}
% \fancyfoot[LO,CE]{}
% \fancyfoot[CO,RE]{Nanjing University of Science and Technology}
\fancyfoot[C]{\thepage}
\fancyfoot[L]{截止日期\duedate} % clear all footer fields
\fancyfoot[R]{任课教师:罗凯} % clear all footer fields

% \fancyfoot[CE,O]{\thechapter}



\begin{document}
\renewcommand{\labelenumi}{(\arabic{enumi})}
\renewcommand{\labelenumii}{(\arabic{enumi}.\arabic{enumii})}



% \title{习题01}
% \author{截止日期: }
% \author{罗凯\\
% 数学物理方法}
% \date{截止日期:}
% \maketitle
 
% \begin{problem}{1.1}
% 找出满足方程\[z - 2 = 3 \frac{1+ \imath t}{1-\imath t}, -\infty < t <\infty\]的所有点$z$.
% \end{problem}




\begin{problem}{7.1}
  计算下列表达式的拉普拉斯变换的像函数
  \begin{enumerate}
    \item $$
    \frac{1}{t}\left(e^{\beta t}-e^{\alpha t}\right) ;
    $$
    \item $$
    \int_0^t \frac{\sin u}{u} d u;
    $$
    \item 
    $$
    \frac{1}{\sqrt{\pi t}}.
    $$
  \end{enumerate}
\end{problem}

\begin{problem}{7.2}
  计算下列表达式的拉普拉斯变换的原函数
  \begin{enumerate}
    \item $$
    \bar{f}(p)=\frac{3 p}{p^2-1} ;
    $$
    \item $$
    \frac{4 p-1}{\left(p^2+p\right)\left(4 p^2-1\right)}.
    $$
  \end{enumerate}
\end{problem}
  
\begin{problem}{7.3}
  利用拉普拉斯变换计算:  $$\int_0^{\infty} \frac{\sin x t}{t} dt. $$
\end{problem}
   
\begin{problem}{7.4}
用拉普拉斯反变换中的Bromwich(又称黎曼-梅林)积分求
$$
\frac{p}{p^2 - \omega^2}
$$
的原函数.
\end{problem}

\begin{problem}{7.5}
  利用拉普拉斯变换求解微分方程
  % $$
  % y(t)=a \sin t-2 \int_0^t y(\tau) \cos (t-\tau) d \tau .
  % $$
  $$
    \frac{d^3 y}{d t^3}+3 \frac{d^2 y}{d t^2}+3 \frac{d y}{d t}+y=6 e^{-t}, 
    \quad y(0)=\left.\frac{d y}{d t}\right|_{t=0}=\left.\frac{d^2 y}{d t^2}\right|_{t=0}=0 .
  $$
\end{problem}


\begin{problem}{7.6}
  用$\Gamma$函数求积分
  $$
  \int_0^\infty x^{\alpha -1 } e^{-x \cos{\theta}} \cos\left( x \sin{\theta} \right) dx,
  $$
  其中$\alpha > 0, -\frac{\pi}{2} < \theta < \frac{\pi}{2}$.
\end{problem} 

\begin{problem}{7.7}
  用$B$函数求积分
  $$
  \int_0^{\frac{\pi}{2}} \tan^\alpha{\theta} \,  d\theta, \quad  -1 < \alpha < 1.
  $$
\end{problem} 

\end{document}