% -*- coding: utf-8 -*-
% !TEX program = xelatex
\documentclass{njustexam}

%注意:注释掉这条命令则显示答案, 为存档用, 交教务处. 使用这条命令则为学生考试用, 交印刷厂印刷
% \answerfalse 

\begin{document}
%%%%%%%%%%%%%%%%% 以下内容根据需要更改%%%%%%%%%%%
\renewcommand{\course}{数学物理方法}                          %课程名
\renewcommand{\duration}{120}                                            %考试时长
\renewcommand{\credit}{5}                                                   %学分
\renewcommand{\syllabus}{11044102}                               %教学大纲编号
\renewcommand{\fullmark}{100}                                            %满分分值
\renewcommand{\composer}{罗凯}            %组卷教师
\renewcommand{\composedate}{\today}                   %组卷日期
\renewcommand{\validator}{}                                        %审定人
\renewcommand{\coursetype}{1}                                            % 1 为必修, 0 为选修
\renewcommand{\exammethod}{1}                                         % 1 为闭卷, 0 为开卷
\renewcommand{\testpaper}{B}                                              % A 或 B
%%%%%%%%%%%%%%%%%%%%%%%%%%%%%%%%%%%%%%

\makehead % 生成试卷表头

\makepart{填空题}{共~11~小题, 每空~1~分, 共~30~分}

\begin{problem}
    复数 $3 - 4\imath$ 的实部为\fillout{$3$}, 虚部为\fillout{$-4$}, 模为\fillout{$5$}, 主辐角为\fillout{$-\arctan\frac{4}{3}$}. 
    \end{problem}
    
    \begin{problem}
    方程 $z^4 - 16 = 0$ 的四个根分别是\fillout{$2$}, \fillout{$-2$}, \fillout{$2i$}, \fillout{$-2i$}. 
    \end{problem}
    
    \begin{problem}
    函数 $f(z)=\frac{1}{z^2(z-3)}$ 的奇点数有\fillout{$2$}个, 而 $f(z)=z^3$ 的奇点数有\fillout{$0$}个. 
    \end{problem}
    
    \begin{problem}
    某解析函数实部为 $u(x, y) = x^2 - y^2$, 则该解析函数的虚部为\fillout{$2xy$}加一常数$C$. 
    \end{problem}
    
    \begin{problem}
    级数 $\sum \left(\frac{z}{3}\right)^n$ 的收敛半径为\fillout{3}; 级数 $\sum \frac{n^2!}{n^n} z^n$ 的收敛半径为\fillout{$e^2$}. 
    \end{problem}
    
    \begin{problem}
    $f(z) = \frac{1}{z(z-3)^2}$, $z=0$是\fillout{1}阶极点, 此处留数为 \fillout{$\frac{1}{6}$}, $z=3$是\fillout{2}阶极点, 该处留数为\fillout{$-\frac{1}{9}$}. 
    \end{problem}
    
    \begin{problem}
    函数$f(t) = e^{-st}$ 和$g(t)=\cosh{st}$, 其中$s$为常数, 它们的拉普拉斯变换分别为 \fillout{$\frac{1}{p+s}$}和\fillout{$\frac{p}{p^2 - s^2}$}.
    \end{problem}
    
    \begin{problem}
    典型的数理方程分为\fillout{波动方程}、\fillout{输运方程}和\fillout{稳定场方程}, 它们大致对应数学上二阶偏微分方程的分类, 即双曲型、抛物型和椭圆型. 
    \end{problem}
    
    \begin{problem}
    定解问题由\fillout{泛定}方程和\fillout{定解}条件组成. 
    \end{problem}
    
    \begin{problem}
    $\Gamma(z)$的取值: $\Gamma(1)=$\fillout{$1$}, $\Gamma(5)=$\fillout{$24$}, $\Gamma(\frac{1}{2})= $\fillout{$\sqrt{\pi}$}.
    \end{problem}
    
    \begin{problem}
    勒让德函数$P_\ell(x)$:
    $P_0(x) = $\fillout{$1$}, 连带勒让德函数$P_\ell^m(x)$:$P^0_2(x) = $\fillout{$\frac{1}{2}(3x^2- 1)$}. 
    \end{problem}
    
    \begin{problem}
    整数阶贝塞尔函数 $J_m(x)$:$J_1(x=0)=$ \fillout{$0$}, $ J_m (x=0)=$ \fillout{$0$}($m>1$).
    \end{problem}
    
    

\makepart{选择题}{共~10~小题, 每题~2~分, 共~20~分}

\begin{problem}
    如果$f(z)$在$z_0$处有一个二阶极点, 其留数为:
    \pickout{D}
    \begin{abcd}
    \item $\text{Res}[f(z),  z_0] = \frac{1}{\lim_{z \to z_0} (z - z_0)^2 f(z)}$
    \item $\text{Res}[f(z),  z_0] = \frac{1}{f''(z_0)}$
    \item $\text{Res}[f(z),  z_0] = f''(z_0)$
    \item $\text{Res}[f(z),  z_0] = \lim_{z \to z_0} \frac{d}{dz} \left( (z - z_0)^2 f(z) \right)$
    \end{abcd}
    \end{problem}
    
    \begin{problem}
    在描述热传导现象时,哪个方程最适合描述非稳态情况?\pickout{B}
    \begin{abcd}
    \item 波动方程
    \item 热传导方程
    \item 泊松方程
    \item 拉普拉斯方程
    \end{abcd}
    \end{problem}
    
    \begin{problem}
    对于一根自由边界的弦上的波动,哪个方程描述波函数$u(x, t)$的行为?\pickout{B}
    \begin{abcd}
    \item 热传导方程
    \item 波动方程
    \item 泊松方程
    \item 拉普拉斯方程
    \end{abcd}
    \end{problem}
    
    \begin{problem}
    对于函数$F(\omega)$的傅里叶变换表示为$f(t) = \mathcal{F}^{-1}[F(\omega)]$, 其中$\mathcal{F}^{-1}$表示傅里叶逆变换. 下面哪个等式是正确的?\pickout{A}
    \begin{abcd}
    \item $f(t) = \int_{-\infty}^{\infty} F(\omega) e^{\imath\omega t} d\omega$
    \item $f(t) = \frac{1}{2\pi} \int_{0}^{\infty} F(\omega) e^{\imath\omega t} d\omega$
    \item $f(t) = \mathcal{F}^{-1}[F(t)]$
    \item $f(t) = \frac{1}{\sqrt{2\pi}} \int_{-\infty}^{\infty} F(\omega) e^{\imath\omega t} d\omega$
    \end{abcd}
    \end{problem}
    
    \begin{problem}
    若贝塞尔函数$J_n(x)$的生成函数为$e^{\frac{1}{2} x (t - \frac{1}{t})}$, 对于非负整数$n$, 其满足的递推关系是?\pickout{B}
    \begin{abcd}
    \item $J_{n+1}(x) = \frac{2n}{x} J_n(x) - J_{n-1}(x)$
    \item $J_{n+1}(x) = J_n(x) - \frac{2n}{x} J_{n-1}(x)$
    \item $J_{n+1}(x) = 2 J_n(x) - J_{n-1}(x)$
    \item $J_{n+1}(x) = J_n(x) + \frac{2n}{x} J_{n-1}(x)$
    \end{abcd}
    \end{problem}

    \begin{problem}
        球谐函数 $Y_{\ell m}(\theta, \phi)$ 的正交性质可以表示为:
        \pickout{D}
        \begin{abcd}
        \item $\int_{0}^{2\pi} \int_{0}^{\pi} Y_{\ell m}(\theta, \phi) Y_{\ell' m'}(\theta, \phi) \sin \theta d\theta d\phi = \delta_{\ell \ell'} \delta_{m m'}$
        \item $\int_{0}^{2\pi} \int_{0}^{\pi} Y_{\ell m}(\theta, \phi) Y_{\ell' m'}(\theta, \phi) \sin \theta d\theta d\phi = \delta_{\ell \ell'}$
        \item $\int_{0}^{2\pi} \int_{0}^{\pi} Y_{\ell m}(\theta, \phi) Y_{\ell' m'}(\theta, \phi) \sin \theta d\theta d\phi = \delta_{m m'}$
        \item $\int_{0}^{2\pi} \int_{0}^{\pi} Y_{\ell m}(\theta, \phi) Y_{\ell' m'}(\theta, \phi) \sin \theta d\theta d\phi = 1$
        \end{abcd}
        \end{problem}

        \begin{problem}{(2分)}
            贝塞尔函数 $J_n(x)$ 的正交性质可以表示为:
            \pickout{C}
            \begin{abcd}
            \item $\int_{0}^{\infty} x J_n(x) J_m(x) dx = \delta_{nm}$
            \item $\int_{0}^{\infty} x J_n(x) J_m(x) dx = 0$, 对于 $n \neq m$
            \item $\int_{0}^{\infty} x J_n(x) J_m(x) dx = \frac{1}{2} \delta_{nm}$
            \item $\int_{0}^{\infty} x J_n(x) J_m(x) dx = \frac{\pi}{2} \delta_{nm}$
            \end{abcd}
            \end{problem}
    
            \begin{problem}{(2分)}
                连带勒让德函数 $P_n^m(x)$ 的正交性质可以表示为:
                \pickout{C}
                \begin{abcd}
                \item $\int_{-1}^{1} P_n^m(x) P_{n'}^m(x) dx = \frac{2}{2n+1} \delta_{nn'}$
                \item $\int_{-1}^{1} P_n^m(x) P_{n'}^m(x) dx = \frac{2}{2n+1} \delta_{mm'}$
                \item $\int_{-1}^{1} P_n^m(x) P_{n'}^{m'}(x) dx = \frac{2}{2n+1} \delta_{nn'} \delta_{mm'}$
                \item $\int_{-1}^{1} P_n^m(x) P_{n'}^{m'}(x) dx = 1$
                \end{abcd}
                \end{problem}
                
                \begin{problem}{(2分)}
                球谐函数 $Y_{\ell m}(\theta, \phi)$ 与连带勒让德函数 $P_{\ell}^m(\cos \theta)$ 的关系是:
                \pickout{A}
                \begin{abcd}
                \item $Y_{\ell m}(\theta, \phi) = \sqrt{\frac{2\ell+1}{4\pi} \frac{(\ell-m)!}{(\ell+m)!}} P_{\ell}^m(\cos \theta) e^{im\phi}$
                \item $Y_{\ell m}(\theta, \phi) = \sqrt{\frac{2\ell+1}{4\pi}} P_{\ell}^m(\cos \theta) e^{im\phi}$
                \item $Y_{\ell m}(\theta, \phi) = \sqrt{\frac{2\ell+1}{4\pi} \frac{(\ell+m)!}{(\ell-m)!}} P_{\ell}^m(\cos \theta) e^{im\phi}$
                \item $Y_{\ell m}(\theta, \phi) = \sqrt{\frac{2\ell+1}{4\pi}} P_{\ell}^m(\cos \theta) e^{-im\phi}$
                \end{abcd}
                \end{problem}
                
                \begin{problem}{(2分)}
                在球坐标系下,拉普拉斯方程的解可以表示为球谐函数的线性组合,这是因为球谐函数是拉普拉斯方程的:
                \pickout{D}
                \begin{abcd}
                \item 特解
                \item 通解
                \item 齐次解
                \item 本征函数
                \end{abcd}
                \end{problem}

                \begin{problem}{(2分)}
                    平面波 $e^{i\mathbf{k} \cdot \mathbf{r}}$ 可以用球面波展开为:
                    \pickout{A}
                    \begin{abcd}
                    \item $e^{i\mathbf{k} \cdot \mathbf{r}} = \sum_{\ell=0}^{\infty} i^{\ell} (2\ell+1) j_{\ell}(kr) P_{\ell}(\cos \theta)$
                    \item $e^{i\mathbf{k} \cdot \mathbf{r}} = \sum_{\ell=0}^{\infty} i^{\ell} j_{\ell}(kr) P_{\ell}(\cos \theta)$
                    \item $e^{i\mathbf{k} \cdot \mathbf{r}} = \sum_{\ell=0}^{\infty} (2\ell+1) j_{\ell}(kr) P_{\ell}(\cos \theta)$
                    \item $e^{i\mathbf{k} \cdot \mathbf{r}} = \sum_{\ell=0}^{\infty} i^{\ell} j_{\ell}(kr) Y_{\ell}^{m}(\theta, \phi)$
                    \end{abcd}
                    \end{problem}
                    
                    \begin{problem}{(2分)}
                    在球坐标系下,平面波 $e^{i\mathbf{k} \cdot \mathbf{r}}$ 的展开式中,$j_{\ell}(kr)$ 是:
                    \pickout{B}
                    \begin{abcd}
                    \item 第一类贝塞尔函数
                    \item 第一类球贝塞尔函数
                    \item 第二类贝塞尔函数
                    \item 第二类球贝塞尔函数
                    \end{abcd}
                    \end{problem}
                    
                    \begin{problem}{(2分)}
                    平面波 $e^{i\mathbf{k} \cdot \mathbf{r}}$ 的球面波展开式中,$P_{\ell}(\cos \theta)$ 是:
                    \pickout{C}
                    \begin{abcd}
                    \item 勒让德多项式
                    \item 连带勒让德函数
                    \item 球谐函数
                    \item 贝塞尔函数
                    \end{abcd}
                    \end{problem}

                \begin{problem}{(2分)}
                    变分法中的欧拉-拉格朗日方程是用于求解什么问题的?
                    \pickout{D}
                    \begin{abcd}
                    \item 最小二乘法问题
                    \item 线性规划问题
                    \item 非线性规划问题
                    \item 泛函极值问题
                    \end{abcd}
                    \end{problem}
                    
                    \begin{problem}{(2分)}
                    在变分法中,泛函 $J[y] = \int_{a}^{b} F(x, y, y') dx$ 的欧拉-拉格朗日方程是:
                    \pickout{A}
                    \begin{abcd}
                    \item $\frac{\partial F}{\partial y} - \frac{d}{dx} \left( \frac{\partial F}{\partial y'} \right) = 0$
                    \item $\frac{\partial F}{\partial y} + \frac{d}{dx} \left( \frac{\partial F}{\partial y'} \right) = 0$
                    \item $\frac{\partial F}{\partial y} - \frac{d}{dx} \left( \frac{\partial F}{\partial y} \right) = 0$
                    \item $\frac{\partial F}{\partial y'} - \frac{d}{dx} \left( \frac{\partial F}{\partial y} \right) = 0$
                    \end{abcd}
                    \end{problem}
                    
                    \begin{problem}{(2分)}
                    在变分法中,如果泛函 $J[y] = \int_{a}^{b} F(x, y, y') dx$ 的被积函数 $F$ 不显含 $x$,则该泛函的欧拉-拉格朗日方程可以简化为:
                    \pickout{B}
                    \begin{abcd}
                    \item $F - y' \frac{\partial F}{\partial y'} = C$
                    \item $F + y' \frac{\partial F}{\partial y'} = C$
                    \item $F - y \frac{\partial F}{\partial y} = C$
                    \item $F + y \frac{\partial F}{\partial y} = C$
                    \end{abcd}
                    \end{problem}
                    
                    \begin{problem}{(2分)}
                    在变分法中,如果泛函 $J[y] = \int_{a}^{b} F(x, y, y') dx$ 的被积函数 $F$ 不显含 $y$,则该泛函的欧拉-拉格朗日方程可以简化为:
                    \pickout{A}
                    \begin{abcd}
                    \item $\frac{d}{dx} \left( \frac{\partial F}{\partial y'} \right) = 0$
                    \item $\frac{d}{dx} \left( \frac{\partial F}{\partial y} \right) = 0$
                    \item $\frac{\partial F}{\partial y'} = 0$
                    \item $\frac{\partial F}{\partial y} = 0$
                    \end{abcd}
                    \end{problem}
                    
                    \begin{problem}{(2分)}
                    在变分法中,如果泛函 $J[y] = \int_{a}^{b} F(x, y, y') dx$ 的被积函数 $F$ 不显含 $y'$,则该泛函的欧拉-拉格朗日方程可以简化为:
                    \pickout{C}
                    \begin{abcd}
                    \item $\frac{\partial F}{\partial y} = 0$
                    \item $\frac{\partial F}{\partial y'} = 0$
                    \item $F = C$
                    \item $\frac{d}{dx} \left( F \right) = 0$
                    \end{abcd}
                    \end{problem}

\makepart{计算证明题}{共~50~分}%共~6~小题, 每小题~8~分, 共~48~分}

\begin{problem}{(10分)}
    利用留数定理计算积分 
    $$I = \int_{-\infty}^{\infty} \frac{d x}{(x^2 + 1)(x^2+4)}. $$
    \end{problem}
    \vspace{1em}
    \begin{solution}
    被积函数$f(z) =\frac{1}{(z^2+1)(z^2+4)}$ 有四个一阶极点: 
    $z_1=i, z_2=-i, z_3=2i, z_4=-2i$. 
    不难看出, 上半平面只有 $z_1=i$ 和 $z_3=2i$ 两个极点. 
    因此根据留数定理,
    $$
    I=\int_{-\infty}^{\infty} \frac{d x}{(x^2+1)(x^2+4)}=
    2 \pi i\left[\left.\operatorname{Res} f(z)\right|_{z=i}+\left.\operatorname{Res} f(z)\right|_{z=2i}\right]
    $$
    分别求上式中的留数, 
    $$ 
    \left.\operatorname{Res} f(z)\right|_{z=i}=\lim _{z \rightarrow i}(z-i) \frac{1}{(z^2+1)(z^2+4)}=\frac{1}{2 i} \frac{1}{3^2-1^2} 
    $$ 
    类似地有,
    $$\left.\operatorname{Res} f(z)\right|_{z=2i}=\frac{1}{4 i} \frac{1}{1^2-2^2}$$ 
    于是,我们得到
    \begin{eqnarray*}
      I &=&2 \pi i\left[\frac{1}{4 i} \frac{1}{(2^2-1^2)}+\frac{1}{2 i} \frac{1}{1^2-2^2}\right]   \\
      &=&\pi \frac{1}{1^2-2^2}\left(\frac{1}{2}-\frac{1}{4}\right) \\
      &=&\frac{\pi}{2 \cdot 1 \cdot (2+1)}  \text{ }
    \end{eqnarray*} 
    \end{solution}
    

    \begin{problem}{(10分)}
        将函数$$f(z) = \frac{1}{(z-1)(z-2)}$$ 
        在 $|z|>2$的环域展开为洛朗级数. 
        \end{problem}
        \vfill
        \begin{solution}
        函数$f(z)$可以写成
        $$
        \frac{1}{(z-1)(z-2)}=\frac{1}{(z-2)}-\frac{1}{(z-1)} 
        $$
        由 $|z|>2$ 得到 $\left|\frac{2}{z}\right|<1, \left|\frac{1}{z}\right|<1$,  
        $$
        \frac{1}{z-2} = \frac{1}{z} \frac{1}{1-\frac{2}{z}} = \frac{1}{z} \sum_{n=0} \left(\frac{2}{z}\right)^n,  
        $$
        类似的可以得到 
        $$\frac{1}{z-1} = \frac{1}{z} \frac{1}{1-\frac{1}{z}} = \frac{1}{z} \sum_{n=0} \left(\frac{1}{z}\right)^n,  
        $$
        最终我们有 
        $$f(z) = \sum_{n=0}^{\infty}\left(2^n-1^n\right) z^{-n-1}.  $$
        \end{solution}


        \begin{problem}{(10分)}
            用$\Gamma$函数表示积分
            $$
            I(\alpha, \theta) = \int_0^\infty x^{ \alpha - 1}  e^{-x \cos{\theta}} \cos\left( x \sin{\theta} \right) dx, 
            $$
            其中$ -\frac{\pi}{2} < \theta < \frac{\pi}{2}$. 
            求得$I(3, \frac{\pi}{6})$的值.
            \end{problem}  
            \vfill
            \begin{solution}
            对一般的$\alpha$, 依习题解答,\\
            $I\?=\int_0^{\infty} x^{\alpha-1} e^{-x \cos \theta} \cos (x \sin \theta) \dx$ \par
            \+ $=\int_0^{\infty} x^{\alpha-1} e^{-x \cos \theta} \operatorname{Re}\left(e^{i x \sin \theta}\right) \dx $ \par
            \+ $=\operatorname{Re} \int_0^{\infty} x^{\alpha-1} e^{-x(\cos \theta-i \sin \theta)} \dx$ \par
            \+ $=\operatorname{Re} \int_0^{\infty} x^{\alpha-1} e^{-x e^{-i \theta}} \dx$ \par
            作变量代换, 令 $x e^{-i \theta}=y$ $\Rightarrow$ $x=e^{i \theta} y,  \dx=e^{i \theta} \dy$ \par
            于是,  $I \?=\operatorname{Re} \int_0^{\infty}\left(e^{i \theta} y\right)^{\alpha-1} e^{-y} e^{i \theta} \dy $ \par
            \+ $ =\operatorname{Re}\left(e^{i \theta}\right)^\alpha \int_0^{\infty} y^{\alpha-1} e^{-y} \dy $ \par
            \+ $ =\operatorname{Re}\left(e^{i \alpha \theta}\right) \Gamma(\alpha)$ \par
            \+ $ = \cos {(\theta \alpha)} \Gamma(\alpha).  $ 
            将$\alpha = 3, \theta = \frac{\pi}{6} $ 代入得
            $$I(3, \frac{\pi}{6})=  \cos{ \frac{\pi}{2}}  \Gamma(3) = 0. $$
            \end{solution}





\vfill
\newpage
\makedata{可能用到的数据和公式} %附录数据
柯西公式(Cauchy's formula)
\[
  f^{(n)}(z) = \frac{n!}{2\pi \imath} \oint_C \frac{f(\zeta)}{(\zeta - z)^{n+1}} d \zeta. 
  \label{eq:cauchy_formula_nth_derivative}
\]

\bigskip
函数$f(t)$的拉普拉斯变换$\bar{f}(p)$为
\[
    \bar{f}(p) = \mathcal{L} \{ f(t) \} = \int_0 ^{\infty} e^{-pt} f(t) \dt . 
\]

\bigskip

双曲正余函数(hyperbolic sine/cosine function)表达为
\begin{align*}
    \cosh z&= \frac{e^{z} + e^{ - z} }{2} , 
    \\
    \sinh z &= \frac{e^{z} - e^{ - z} }{2} . 
\end{align*}
% \bigskip

伽玛函数(Gamma function)
\[
  \Gamma(z) \equiv \int_{0}^{\infty} e^{-t} t^{z-1} d t,  \quad \operatorname{Re} z>0 . 
\]
\bigskip

贝塔函数(Beta function)
\[
    B(p,  q) = \int_0^1 t^{p -1} (1-t)^{q-1} \dt,  \operatorname{Re} p> 0,  \operatorname{Re} q >0. 
    \label{eq:beta_def1}
\]
\bigskip

% \begin{tabularx}{\linewi\dth}{*{4}{>{$}X<{$}}}
% \hline
% \Phi_0(0. 5)=0. 6915 & \Phi_0(1)=0. 8413 & \Phi_0(2)=0. 9773 & \Phi_0(2. 5)=0. 9938 \\
% t_{0. 01}(8)=3. 355 & t_{0. 01}(9)=3. 250 & t_{0. 01}(15)=2. 947 & t_{0. 01}(16)=2. 921 \\
% \chi_{0. 005}^2(8)=22. 0 & \chi_{0. 005}^2(9)=23. 6 & \chi_{0. 005}^2(15)=32. 8 & \chi_{0. 005}^2(16)=34. 3 \\
% \chi_{0. 995}^2(8)=1. 34 & \chi_{0. 995}^2(9)=1. 73 & \chi_{0. 995}^2(15)=4. 60 & \chi_{0. 995}^2(16)=5. 14 \\
% \hline
% \end{tabularx}

% 两个自变数 $x$ 和 $y$ 的二阶线性偏微分方程的通用形式
% \[
%     a_{11} u_{x x}+2 a_{12} u_{x y}+a_{22} u_{y y}+b_1 u_x+b_2 u_y+c u+f=0. 
%     \label{eq:two_variable_diff_equation}
% \]
% \bigskip

球坐标系下的拉普拉斯算符
\[
 \Delta = \frac{1}{r^2} \frac{\partial}{\partial r} \left( r^2 \frac{\partial }{\partial r} \right)
  + \frac{1}{r^2\sin \theta} \frac{\partial}{\partial \theta} \left( \sin\theta \frac{\partial}{\partial \theta} \right)
  + \frac{1}{r^2\sin^2 \theta} \frac{\partial^2}{\partial^2 \varphi} .
\]
分离变量得到轴对称($m=0$)
情况下解的形式为
\[
  u(r,  \theta) = \sum_{l=0}^{\infty} \left( A_l r^l + \frac{B_l}{r^{l+1}} \right) P_{l} (\cos \theta)
\]

\bigskip

$l$ 阶勒让德多项式(Legendre polynomial)的罗德里格斯表达式(Rodrigues's formula) 
$$P_l(x)=\frac{1}{2^l l !} \frac{d^l}{d x^l}\left(x^2-1\right)^l$$
\end{document}
