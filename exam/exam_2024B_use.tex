% -*- coding: utf-8 -*-
% !TEX program = xelatex
\documentclass{njustexam}

%注意:注释掉这条命令则显示答案, 为存档用, 交教务处. 使用这条命令则为学生考试用, 交印刷厂印刷
\answerfalse 

\begin{document}
%%%%%%%%%%%%%%%%% 以下内容根据需要更改%%%%%%%%%%%
\renewcommand{\course}{数学物理方法-补考}                          %课程名
\renewcommand{\duration}{120}                                            %考试时长
\renewcommand{\credit}{5}                                                   %学分
\renewcommand{\syllabus}{11044102}                               %教学大纲编号
\renewcommand{\fullmark}{100}                                            %满分分值
\renewcommand{\composer}{罗凯}            %组卷教师
\renewcommand{\composedate}{\today}                   %组卷日期
\renewcommand{\validator}{}                                        %审定人
\renewcommand{\coursetype}{1}                                            % 1 为必修, 0 为选修
\renewcommand{\exammethod}{1}                                         % 1 为闭卷, 0 为开卷
\renewcommand{\testpaper}{B}                                              % A 或 B
%%%%%%%%%%%%%%%%%%%%%%%%%%%%%%%%%%%%%%

\makehead % 生成试卷表头

\makepart{填空题}{共~11~小题, 每空~1~分, 共~30~分}


\begin{problem}
  复数 $3 - 4\imath$ 的实部为\fillout{$3$}, 虚部为\fillout{$-4$}, 模为\fillout{$5$}, 主辐角为\fillout{$-\arctan\frac{4}{3}$}. 
  \end{problem}
  
  \begin{problem}
  方程 $z^4 - 16 = 0$ 的四个根分别是\fillout{$2$}, \fillout{$-2$}, \fillout{$2i$}, \fillout{$-2i$}. 
  \end{problem}
  
  \begin{problem}
  函数 $f(z)=\frac{1}{z^2(z-3)}$ 的奇点数有\fillout{$2$}个, 而 $f(z)=z^3$ 的奇点数有\fillout{$0$}个. 
  \end{problem}
  
  \begin{problem}
  某解析函数实部为 $u(x, y) = x^2 - y^2$, 则该解析函数的虚部为\fillout{$2xy$}加一常数$C$. 
  \end{problem}
  
  \begin{problem}
  级数 $\sum \left(\frac{z}{3}\right)^n$ 的收敛半径为\fillout{3}; 级数 $\sum \frac{n^2!}{n^n} z^n$ 的收敛半径为\fillout{$e^2$}. 
  \end{problem}
  
  \begin{problem}
  $f(z) = \frac{1}{z(z-3)^2}$, $z=0$是\fillout{1}阶极点, 此处留数为 \fillout{$\frac{1}{6}$}, $z=3$是\fillout{2}阶极点, 该处留数为\fillout{$-\frac{1}{9}$}. 
  \end{problem}
  
  \begin{problem}
  函数$f(t) = e^{-st}$ 和$g(t)=\cosh{st}$, 其中$s$为常数, 它们的拉普拉斯变换分别为 \fillout{$\frac{1}{p+s}$}和\fillout{$\frac{p}{p^2 - s^2}$}.
  \end{problem}
  
  \begin{problem}
  典型的数理方程分为\fillout{波动方程}、\fillout{输运方程}和\fillout{稳定场方程}, 它们大致对应数学上二阶偏微分方程的分类, 即双曲型、抛物型和椭圆型. 
  \end{problem}
  
  \begin{problem}
  定解问题由\fillout{泛定}方程和\fillout{定解}条件组成. 
  \end{problem}
  
  \begin{problem}
  $\Gamma(z)$的取值: $\Gamma(1)=$\fillout{$1$}, $\Gamma(5)=$\fillout{$24$}, $\Gamma(\frac{1}{2})= $\fillout{$\sqrt{\pi}$}.
  \end{problem}
  
  \begin{problem}
  勒让德函数$P_\ell(x)$:
  $P_0(x) = $\fillout{$1$}, 连带勒让德函数$P_\ell^m(x)$: $P^0_2(x) = $\fillout{$\frac{1}{2}(3x^2- 1)$},
  $P^1_1(x) = $\fillout{$\sqrt{1-x^2}$}. 
  \end{problem}
  
  % \begin{problem}
  % 整数阶贝塞尔函数 $J_m(x)$:$J_1(x=0)=$ \fillout{$0$}, $ J_m (x=0)=$ \fillout{$0$}($m>1$).
  % \end{problem}
  
  

\makepart{选择题}{共~10~小题, 每题~2~分, 共~20~分}

\begin{problem}
  如果$f(z)$在$z_0$处有一个二阶极点, 其留数为:
  \pickout{D}
  \begin{abcd}
  \item $\text{Res}[f(z),  z_0] = \frac{1}{\lim_{z \to z_0} (z - z_0)^2 f(z)}$
  \item $\text{Res}[f(z),  z_0] = \frac{1}{f''(z_0)}$
  \item $\text{Res}[f(z),  z_0] = f''(z_0)$
  \item $\text{Res}[f(z),  z_0] = \lim_{z \to z_0} \frac{d}{dz} \left( (z - z_0)^2 f(z) \right)$
  \end{abcd}
  \end{problem}

  % 留数定理的基本表达式:

% 留数定理的基本表达式是:

% \item $\oint_C f(z) , dz = 2\pi i \text{Res}[f(z),  z_0]$

% \item $\oint_C f(z) , dz = \frac{2\pi}{i} \text{Res}[f(z),  z_0]$

% \item $\oint_C f(z) , dz = \pi i \text{Res}[f(z),  z_0]$

% \item $\oint_C f(z) , dz = \frac{\pi}{i} \text{Res}[f(z),  z_0]$

\begin{problem}
  在变分法中,泛函 $J[y] = \int_{a}^{b} F(x, y, y') dx$ 的欧拉-拉格朗日方程是:
  \pickout{A}
  \begin{abcd}
  \item $\frac{\partial F}{\partial y} - \frac{d}{dx} \left( \frac{\partial F}{\partial y'} \right) = 0$
  \item $\frac{\partial F}{\partial y} + \frac{d}{dx} \left( \frac{\partial F}{\partial y'} \right) = 0$
  \item $\frac{\partial F}{\partial y} - \frac{d}{dx} \left( \frac{\partial F}{\partial y} \right) = 0$
  \item $\frac{\partial F}{\partial y'} - \frac{d}{dx} \left( \frac{\partial F}{\partial y} \right) = 0$
  \end{abcd}
  \end{problem}



  % \begin{problem}
  %   贝塞尔函数 $J_n(x)$ 的正交性质可以表示为:
  %   \pickout{C}
  %   \begin{abcd}
  %   \item $\int_{0}^{\infty} x J_n(x) J_m(x) dx = \delta_{nm}$
  %   \item $\int_{0}^{\infty} x J_n(x) J_m(x) dx = 0$, 对于 $n \neq m$
  %   \item $\int_{0}^{\infty} x J_n(x) J_m(x) dx = \frac{1}{2} \delta_{nm}$
  %   \item $\int_{0}^{\infty} x J_n(x) J_m(x) dx = \frac{\pi}{2} \delta_{nm}$
  %   \end{abcd}
  %   \end{problem}
   
    \begin{problem}
      在球坐标系下,拉普拉斯方程的解可以表示为球谐函数的线性组合,这是因为球谐函数是拉普拉斯方程的:
      \pickout{D}
      \begin{abcd}
      \item 特解
      \item 通解
      \item 齐次解
      \item 本征函数
      \end{abcd}
      \end{problem}

% \begin{problem}
%   下面哪一个网址与我们学习的数学物理方法特殊函数息息相关? \pickout{B}
%   \begin{abcd}
%     \item https://mathworld.wolfram.com/
%     \item https://dlmf.nist.gov
%     \item https://csrc.nist.gov
%     \item https://projecteuclid.org/
%   \end{abcd}

% \end{problem}

% \begin{problem}
%   如果$f(z)$在$z_0$处有一个一阶极点, 其留数为:
%   \pickout{D}
% \begin{abcd}
%   \item $\text{Res}[f(z),  z_0] = \frac{1}{\lim_{z \to z_0} (z - z_0)f(z)}$
%   \item $\text{Res}[f(z),  z_0] = \frac{1}{f'(z_0)}$
%   \item $\text{Res}[f(z),  z_0] = f'(z_0)$
%    \item $\text{Res}[f(z),  z_0] = \lim_{z \to z_0} (z - z_0)f(z)$
% \end{abcd}
% \end{problem}
% 极点和留数关系:

% 留数定理的应用:

% 使用留数定理计算积分$\int_{-\infty}^{\infty} \frac{1}{1+x^2} , dx$的结果是:

% \item $\frac{\pi}{2}$

% \item $\pi$

% \item $2\pi$

% \item $0$



% \begin{problem}
% 在描述扩散现象时, 哪个方程最适合描述非稳态情况? \pickout{A}
%   \begin{abcd}
%     \item 热传导方程
%     \item 拉普拉斯方程   
%     \item 泊松方程
%     \item 波动方程
%   \end{abcd}
% \end{problem}

% \begin{problem}
%   对于一根固定边界的弦上的波动, 哪个方程描述波函数$u(x,  t)$的行为?
%   \pickout{C}
%   \begin{abcd}
%     \item 热传导方程 
%     \item 泊松方程
%     \item 波动方程
%     \item 拉普拉斯方程
%     \end{abcd}
% \end{problem}

% \begin{problem}
%   在信号处理中, 哪个方法常常用于将一个信号从时域转换到频域?
%   \pickout{A}
%   \begin{abcd}
%     \item 傅里叶变换 
%     \item 拉普拉斯变换
%     \item 小波变换
%     \item 保角变换
%   \end{abcd}
% \end{problem}

% \begin{problem}
%   对于函数$F(\omega)$的傅里叶变换表示为$f(t) = \mathcal{F}^{-1}[F(\omega)]$, 其中$\mathcal{F}^{-1}$表示傅里叶逆变换. 下面哪个等式是正确的?
%   \pickout{D}
%   \begin{abcd}
%     \item  $f(t) = \int_{-\infty}^{\infty} F(\omega) e^{-\imath\omega t} d\omega$

%     \item  $f(t) = \frac{1}{2\pi} \int_{0}^{\infty} F(\omega) e^{-\imath\omega t} d\omega$

    
%     \item $f(t) = \mathcal{F}^{-1}[F(t)]$
    
%     \item $f(t) = \frac{1}{\sqrt{2\pi}} \int_{-\infty}^{\infty} F(\omega) e^{-\imath\omega t} d\omega$

%   \end{abcd}

% \end{problem}


% \begin{problem}
%   单位阶跃函数$\Theta(t)$的拉普拉斯变换是:
%   \pickout{A}
%   \begin{abcd}
%     \item $\frac{1}{s}$
%     \item $\frac{1}{s^2}$
%     \item $\frac{1}{s^2} + \frac{1}{s}$
%     \item $\frac{1}{s} - \frac{1}{s^2}$
%   \end{abcd}

% \end{problem}

% \begin{problem}
% 复数$z$可以写成幅度和相位的形式$z = re^{i\theta}$, 其中$r$是幅度, $\theta$是相位. 
% 如果要展开$\sin(\theta)$, 下面哪个级数是正确的? \pickout{D}
% \begin{abcd}
%   \item $\sum_{n=0}^{\infty} \frac{(-1)^n \theta^{2n}}{(2n)!}$
%   \item $\sum_{n=1}^{\infty} \frac{(-1)^n \theta^n}{n!}$
%   \item $\sum_{n=1}^{\infty} \frac{(-1)^n \theta^{n+1}}{n!}$
%   \item $\sum_{n=0}^{\infty} \frac{(-1)^n \theta^{2n+1}}{(2n+1)!}$

% \end{abcd}
% \end{problem}
\begin{problem}
% 伽玛函数的性质:

伽玛函数$\Gamma(z)$的性质中, 以下哪个是正确的? \pickout{C}
\begin{abcd}


\item $\Gamma(z+1) = \frac{\Gamma(z)}{z}$

\item $\Gamma(z+1) = \frac{1}{z\Gamma(z)}$
\item $\Gamma(z+1) = z\Gamma(z)$
\item $\Gamma(z+1) = z+\Gamma(z)$
\end{abcd}
\end{problem}
% 伽玛函数的性质:

% 伽玛函数$\Gamma(z)$的性质中, 以下哪个是正确的?

% \item $\Gamma(z+1) = z\Gamma(z)$

% \item $\Gamma(z+1) = \frac{\Gamma(z)}{z}$

% \item $\Gamma(z+1) = \frac{1}{z\Gamma(z)}$

% \item $\Gamma(z+1) = z+\Gamma(z)$

% \begin{problem}
%   关于贝塔函数$B(p,  q)$的说法正确的是:
%   \pickout{C}
% \begin{abcd}
%   \item $B(p,  q) = \int_{0}^{\infty} t^{p+q-1}e^{-t} \dt$
%   \item $B(p,  q) = \int_{0}^{1} t^{p-1}(1-t)^{q-1} \dt$
%   \item $B(p,  q) = \frac{\Gamma(p)\Gamma(q)}{\Gamma(p+q)}$
%   \item $B(p,  q) = \frac{\Gamma(p+q)}{\Gamma(p)\Gamma(q)}$
% \end{abcd}
% \end{problem}

% \begin{problem}
%   对于二阶线性偏微分方程以下说法正确的是: \pickout{D}
%   \pickout{A}
%   \begin{abcd}
%     \item 傅里叶变换 
%     \item 拉普拉斯变换
%     \item 小波变换
%     \item 保角变换
%   \end{abcd}
% \end{problem}

% \begin{problem}
  % 伽玛函数的性质:
  
%   伽玛函数$\Gamma(z)$的性质中, 以下哪个是正确的? \pickout{A}
%   \begin{abcd}

% \item $\Gamma(z) = \frac{B(z,  1)}{z}$

% \item $\Gamma(z) = \frac{B(z,  1)}{z-1}$

% \item $\Gamma(z) = \frac{B(z,  1)}{z+1}$

% \item $\Gamma(z) = \frac{B(z,  1)}{z+2}$
%   \end{abcd}
%   \end{problem}
% 伽玛函数和贝塔函数的关系:

% 伽玛函数$\Gamma(z)$和贝塔函数$B(p,  q)$之间的关系是:

% \item $\Gamma(z) = \frac{B(z,  1)}{z}$

% \item $\Gamma(z) = \frac{B(z,  1)}{z-1}$

% \item $\Gamma(z) = \frac{B(z,  1)}{z+1}$

% \item $\Gamma(z) = \frac{B(z,  1)}{z+2}$


\begin{problem}
  若贝塞尔函数$J_n(x)$的生成函数为$e^{\frac{1}{2} x (t - \frac{1}{t})}$, 对于非负整数$n$, 其满足的递推关系是?
  \pickout{A}
  \begin{abcd}
    \item  $J_{n+1}(x) = \frac{2n}{x} J_n(x) - J_{n-1}(x)$
    \item  $J_{n+1}(x) = J_n(x) - \frac{2n}{x} J_{n-1}(x)$
    \item  $J_{n+1}(x) = 2 J_n(x) - J_{n-1}(x)$
    \item  $J_{n+1}(x) = J_n(x) + \frac{2n}{x} J_{n-1}(x)$
  \end{abcd}
\end{problem}

% \begin{problem}
%   连带勒让德函数 $P_n^m(x)$ 的正交性质可以表示为:
%   \pickout{C}
%   \begin{abcd}
%   \item $\int_{-1}^{1} P_n^m(x) P_{n'}^m(x) dx = \frac{2}{2n+1} \delta_{nn'}$
%   \item $\int_{-1}^{1} P_n^m(x) P_{n'}^m(x) dx = \frac{2}{2n+1} \delta_{mm'}$
%   \item $\int_{-1}^{1} P_n^m(x) P_{n'}^{m'}(x) dx = \frac{2}{2n+1} \delta_{nn'} \delta_{mm'}$
%   \item $\int_{-1}^{1} P_n^m(x) P_{n'}^{m'}(x) dx = 1$
%   \end{abcd}
%   \end{problem}

\begin{problem}
  勒让德多项式$P_n(x)$的正交性质可以表示为:
  \pickout{B}
  \begin{abcd}
    \item $\int_{-1}^{1} P_n(x) dx = 1$

    \item $\int_{-1}^{1} P_n(x) P_m(x) dx = 0$, 对于$n \neq m$
    
    \item $\int_{0}^{\pi} P_n(x) dx = 0$
    
    \item $\int_{-1}^{1} P_n(x) P_m(x) dx = \delta_{nm} + 1$
  \end{abcd}
\end{problem}

\begin{problem}
  $Y_{10}(\theta,  \phi)$的表达式是:
  \pickout{A}
\begin{abcd}
  \item $\sqrt{\frac{3}{4\pi}} \cos \theta$
  \item $\sqrt{\frac{3}{4\pi}} \sin \theta$
  \item $\sqrt{\frac{3}{8\pi}} \sin \theta e^{i\phi}$
  \item $\sqrt{\frac{3}{8\pi}} \cos \theta e^{i\phi}$
\end{abcd}
\end{problem}

% \begin{problem}
% 对于$x$趋于无穷大时, 第一类贝塞尔函数$J_n(x)$的渐近行为是:
% \begin{abcd}
%   \item $J_n(x) \sim \sqrt{\frac{2}{\pi n}} \cos\left(x - \frac{\pi}{4}n - \frac{\pi}{2}\right)$
%   \item $J_n(x) \sim \sqrt{\frac{2}{\pi n}} \cos\left(x - \frac{\pi}{4}n + \frac{\pi}{4}\right)$
%   \item $J_n(x) \sim \sqrt{\frac{1}{\pi n}} \cos\left(x - \frac{\pi}{4}n + \frac{\pi}{4}\right)$
%   \item $J_n(x) \sim \sqrt{\frac{1}{2\pi n}} \cos\left(x - \frac{\pi}{4}n + \frac{\pi}{4}\right)$
% \end{abcd}
% \end{problem}


\begin{problem}
  对于$n$为正整数时, $x$趋于零时, 第一类贝塞尔函数$J_n(x)$的极限是:
  \pickout{B}

  \begin{abcd}
\item $\lim_{x \to 0} J_n(x) = 1$

\item $\lim_{x \to 0} J_n(x) = 0$

\item $\lim_{x \to 0} J_n(x) = (-1)^n$

\item $\lim_{x \to 0} J_n(x) = \infty$
 \end{abcd}
\end{problem}


\begin{problem}
  球贝塞尔函数满足哪个微分方程? 
  \pickout{C}

  \begin{abcd}
    \item 热传导方程
    \item 薛定谔方程
    \item 亥姆霍兹方程
    \item 泊松方程
  \end{abcd}
\end{problem}

% \begin{problem}
%   在一个轴对称的电场中, 电势 $V(r,  \theta)$ 的分布可以用球谐函数展开表示. 
%   若电场满足拉普拉斯方程, 
%   此时, 电势 $V(r,  \theta)$ 的展开系数中, 与角度 $\theta$ 有关的部分应当包含: \pickout{D}
%   \begin{abcd}
%   % \item 虚宗量贝塞尔函数 $i^l J_l(kr)$
%   % \item 贝塞尔函数 $J_l(kr)$
%   % \item 球谐函数 $Y_{lm}(\theta,  \phi)$
%   % \item 勒让德函数 $P_l(\cos\theta)$
%   \item 虚宗量贝塞尔函数
%   \item 贝塞尔函数 
%   \item 球谐函数 
%   \item 勒让德函数
%   \end{abcd}
% \end{problem}

\begin{problem}
  球谐函数 $Y_{\ell m}(\theta, \phi)$ 与连带勒让德函数 $P_{\ell}^m(\cos \theta)$ 的关系是:
  \pickout{A}
  \begin{abcd}
  \item $Y_{\ell m}(\theta, \phi) = \sqrt{\frac{2\ell+1}{4\pi} \frac{(\ell-m)!}{(\ell+m)!}} P_{\ell}^m(\cos \theta) e^{im\phi}$
  \item $Y_{\ell m}(\theta, \phi) = \sqrt{\frac{2\ell+1}{4\pi}} P_{\ell}^m(\cos \theta) e^{im\phi}$
  \item $Y_{\ell m}(\theta, \phi) = \sqrt{\frac{2\ell+1}{4\pi} \frac{(\ell+m)!}{(\ell-m)!}} P_{\ell}^m(\cos \theta) e^{im\phi}$
  \item $Y_{\ell m}(\theta, \phi) = \sqrt{\frac{2\ell+1}{4\pi}} P_{\ell}^m(\cos \theta) e^{-im\phi}$
  \end{abcd}
  \end{problem}
  
  
  % \begin{problem}
  %   球谐函数 $Y_{\ell m}(\theta, \phi)$ 的正交性质可以表示为:
  %   \pickout{D}
  %   \begin{abcd}
  %   \item $\int_{0}^{2\pi} \int_{0}^{\pi} Y_{\ell m}(\theta, \phi) Y_{\ell' m'}(\theta, \phi) \sin \theta d\theta d\phi = \delta_{\ell \ell'} \delta_{m m'}$
  %   \item $\int_{0}^{2\pi} \int_{0}^{\pi} Y_{\ell m}(\theta, \phi) Y_{\ell' m'}(\theta, \phi) \sin \theta d\theta d\phi = \delta_{\ell \ell'}$
  %   \item $\int_{0}^{2\pi} \int_{0}^{\pi} Y_{\ell m}(\theta, \phi) Y_{\ell' m'}(\theta, \phi) \sin \theta d\theta d\phi = \delta_{m m'}$
  %   \item $\int_{0}^{2\pi} \int_{0}^{\pi} Y_{\ell m}(\theta, \phi) Y_{\ell' m'}(\theta, \phi) \sin \theta d\theta d\phi = 1$
  %   \end{abcd}
  %   \end{problem}

% \begin{problem}
%   在一个照明公司, 设计师们希望通过调整灯具的灯罩形状来实现更均匀的光照分布. 他们发现球谐函数可以描述灯罩的光照分布. 如果他们希望获得最均匀的照明效果, 最可能使用的球谐函数是:
%   \pickout{A}
%   \begin{abcd}
%   \item $Y_{0}^0(\theta,  \phi)$
  
%   \item $Y_{1}^{-1}(\theta,  \phi)$
  
%   \item $Y_{2}^2(\theta,  \phi)$
  
%   \item $Y_{3}^1(\theta,  \phi)$
%   \end{abcd}
%   \end{problem}

  % \begin{problem}
  %   光学工程师设计了一种用于薄膜涂层的新材料, 以改善透明度和折射率. 为了最佳地控制光的传播, 他们最可能使用的球谐函数是:
  %   \pickout{D}
  %   \begin{abcd}

  %   \item $Y_{10}(\theta,  \phi)$
    
  %   \item $Y_{11}(\theta,  \phi)$
    
  %   \item $Y_{21}(\theta,  \phi)$
    
  %   \item $Y_{32}(\theta,  \phi)$
  %  \end{abcd}

  % \end{problem}
% \item $J_n(x) \sim \frac{x^n}{n!}$

% \item $J_n(x) \sim \frac{x^n}{\sqrt{n}}$

% \item $J_n(x) \sim (-1)^n\frac{x^{2n}}{(2n)!}$

% \item $J_n(x) \sim \frac{(-1)^n}{n!}x^n$


\makepart{计算证明题}{共~50~分}%共~6~小题, 每小题~8~分, 共~48~分}

% \begin{problem}{(6分)}
%   证明
%   $$
%   |\sinh{y}| \leq |\sin(x+\imath y)| \leq |\cosh{y}|. 
%   $$
% \end{problem}\text{\score{10}}

% \renewcommand{\solutionname}{证} % 将“解"字改为“证"字
% \begin{solution}
%   \everymath{\displaystyle}%
%   原式 \? $=\int\e^{2x}\, \sec^2 x\dx+2\int\e^{2x}\, \tan x\dx$ \score{2}
%   \+ $=\int\e^{2x}\, \d(\tan x)+ 2\int\e^{2x}\, \tan x\dx$ \score{4}
%   \+ $=\e^{2x}\, \tan x - 2\int\e^{2x}\, \tan x\dx+ 2\int\e^{2x}\, \tan x\dx$ \score{6}
%   \+ $=\e^{2x}\, \tan x + C$ \score{8}
% \end{solution}

% \begin{problem}{(6分)}
%   设 $\Psi(t,  x)=e^{ 2 t x-t^2 }$,  $t$是复变数,  证明: 
%   $$
%   \left. \frac{\partial^n \Psi(t,  x)}{\partial t^n}\right|_{t=0}=(-1)^n e^{x^2} \frac{d^n}{\dx^n} e^{-x^2}
%   $$
%   % 提示: 对回路积分进行积分变数的代换 $\xi=z-x$. 
% \end{problem}

% \begin{problem}{(5分)}
%   证明傅里叶变换$\mathcal{F}[f(x)] = F(\omega)$ 满足位移定理
%   $$
%       \mathcal{F} [ e^{\imath \omega_0 x} f(x) ] = F(\omega - \omega_0).
%   $$
% \end{problem}
% \begin{solution}
% \end{solution}


\begin{problem}{(10分)}
  利用留数定理计算积分 
  $$I = \int_{-\infty}^{\infty} \frac{d x}{(x^2 + 1)(x^2+4)}. $$
  \end{problem}
\smallskip

  \begin{solution}
  被积函数$f(z) =\frac{1}{(z^2+1)(z^2+4)}$ 有四个一阶极点: 
  $z_1=i, z_2=-i, z_3=2i, z_4=-2i$. 
  不难看出, 上半平面只有 $z_1=i$ 和 $z_3=2i$ 两个极点.  \score{4}
  因此根据留数定理,
  $$
  I=\int_{-\infty}^{\infty} \frac{d x}{(x^2+1)(x^2+4)}=
  2 \pi i\left[\left.\operatorname{Res} f(z)\right|_{z=i}+\left.\operatorname{Res} f(z)\right|_{z=2i}\right]
  $$ 
  分别求上式中的留数, 
  $$ 
  \left.\operatorname{Res} f(z)\right|_{z=i}=\lim _{z \rightarrow i}(z-i) \frac{1}{(z^2+1)(z^2+4)}=\frac{1}{2 i} \frac{1}{3}  \score{6}
  $$ 
  类似地有,
  $$\left.\operatorname{Res} f(z)\right|_{z=2i}=\frac{1}{4 i} \frac{1}{-3} \score{8}$$  
  于是,我们得到
$$ I =2 \pi i\left[ - \frac{1}{4 i} \frac{1}{3}+\frac{1}{2 i} \frac{1}{3}\right] = \frac{\pi}{6}.    \score{10} $$
\end{solution}
\bigskip


\begin{problem}{(10分)}
将函数$$f(z) = \frac{1}{(z-1)(z-2)}$$ 
在 $|z|>2$的环域展开为洛朗级数. 
\end{problem}
\smallskip

\begin{solution}
函数$f(z)$可以写成
$$
\frac{1}{(z-1)(z-2)}=\frac{1}{(z-2)}-\frac{1}{(z-1)}   \score{2}
$$
由 $|z|>2$ 得到 $\left|\frac{2}{z}\right|<1, \left|\frac{1}{z}\right|<1$,    \score{4}
$$
\frac{1}{z-2} = \frac{1}{z} \frac{1}{1-\frac{2}{z}} = \frac{1}{z} \sum_{n=0} \left(\frac{2}{z}\right)^n,  \score{6}
$$
类似的可以得到 
$$\frac{1}{z-1} = \frac{1}{z} \frac{1}{1-\frac{1}{z}} = \frac{1}{z} \sum_{n=0} \left(\frac{1}{z}\right)^n,  \score{8}
$$
最终我们有 
$$f(z) = \sum_{n=0}^{\infty}\left(2^n-1^n\right) z^{-n-1}. \score{10} $$
\end{solution}
\bigskip



% \begin{problem}{(6分)}
%   计算下面的围道积分
%   $$\oint_{C} \frac{\sin \frac{\pi z}{4}}{z^2-1} dz,  $$
%    其中$C$为$|z-1|=1$. 
% \end{problem}
% \vfill

% \begin{solution}
% % \everymath{\displaystyle}%
% \? 被积函数$f(z) = \frac{\sin \frac{\pi z}{4}}{z^2-1}$含两个奇点$z_1=1,  z_2=-1$. \\ 
% \+ 路径$C$只包含其中$z_1=1$这一奇点.  \score{2}
% \+ 不难得到$f(z)$在$z=z_1$处的留数为 $$\text{Res} f(z) |_ {z\to z_1} = \lim_{z \to z_1} (z-z_1) f(z) =\frac{\sqrt{2}}{4}\score{4}$$
% \+ 因此根据留数定理, 最终结果为
% $$ 2\pi \imath \frac{\sqrt{2}}{4} = \frac{\pi\imath}{\sqrt{2}}\score{6}$$ 
% \end{solution}
% \bigskip


% \begin{problem}{(6分)}
%   求 $|\sin z|$ 在闭区域 $0 \leq \operatorname{Re} z \leq 2 \pi,  0 \leq \Im z \leq 2 \pi$ 中的最大值. 
% \end{problem}

% \clearpage
\begin{problem}{(10分)}
用$\Gamma$函数表示积分
$$
I(\alpha, \theta) = \int_0^\infty x^{ \alpha - 1}  e^{-x \cos{\theta}} \cos\left( x \sin{\theta} \right) dx, 
$$
其中$ -\frac{\pi}{2} < \theta < \frac{\pi}{2}$. 
求得$I(3, \frac{\pi}{6})$的值.
\end{problem}  
\smallskip

\begin{solution}
对一般的$\alpha$, 依习题解答,\\
$I\?=\int_0^{\infty} x^{\alpha-1} e^{-x \cos \theta} \cos (x \sin \theta) \dx$ \par
\+ $=\int_0^{\infty} x^{\alpha-1} e^{-x \cos \theta} \operatorname{Re}\left(e^{i x \sin \theta}\right) \dx $ \par
\+ $=\operatorname{Re} \int_0^{\infty} x^{\alpha-1} e^{-x(\cos \theta-i \sin \theta)} \dx$ \par
\+ $=\operatorname{Re} \int_0^{\infty} x^{\alpha-1} e^{-x e^{-i \theta}} \dx$ \score{3} \par 
作变量代换, 令 $x e^{-i \theta}=y$ $\Rightarrow$ $x=e^{i \theta} y,  \dx=e^{i \theta} \dy$ \score{5} \par 
于是,  $I \?=\operatorname{Re} \int_0^{\infty}\left(e^{i \theta} y\right)^{\alpha-1} e^{-y} e^{i \theta} \dy $ \par
\+ $ =\operatorname{Re}\left(e^{i \theta}\right)^\alpha \int_0^{\infty} y^{\alpha-1} e^{-y} \dy $ \par
\+ $ =\operatorname{Re}\left(e^{i \alpha \theta}\right) \Gamma(\alpha)$  \par
\+ $ = \cos {(\theta \alpha)} \Gamma(\alpha).    $ \score{8}
将$\alpha = 3, \theta = \frac{\pi}{6} $ 代入得
$$I(3, \frac{\pi}{6})=  \cos{ \frac{\pi}{2}}  \Gamma(3) = 0.  \score{10}$$  
\end{solution}
\bigskip



% \begin{problem}{(7分)}
%   计算拉普拉斯变换的原函数
%   $$
%     % \frac{4 p-1}{\left(p^2+p\right)\left(4 p^2-1\right)}. 
%     \bar{f}(p)=\frac{2 p}{p^2-1}. 
%   $$
% \end{problem}
% \vfill


% \begin{solution}
%   函数$\bar{f}(p)$可以写成
%   $$
%   \bar{f}(p) = \frac{1}{p+1} + \frac{1}{p-1}  \score{3}
%   $$
%   由拉普拉斯变换的定义不难得到$$\mathcal{L}^{-1} \left[ \frac{1}{p-s} \right] = e^{st}, $$
%   故得到
%   $$\mathcal{L}^{-1} \left[ \frac{1}{p+1}\right] = e^{-t} $$
%   和
%   $$\mathcal{L}^{-1} \left[ \frac{1}{p-1}\right] = e^{+t} \score{5}  $$
%   根据双曲余弦函数的定义, 最终我们有原函数
%   $$f(t) =\mathcal{L}^{-1} \left[ \bar{f}(p) \right] = 2\cosh t.  \score{7}$$
% \end{solution}\bigskip
  
% \begin{problem}{(8分)}
%   将函数$$f(z) = \frac{1}{(z-2)(z-3)}$$ 
%   在 $|z|>3$的环域展开为洛朗级数. 
% \end{problem}
%   \vfill
% \begin{solution}
%   % \everymath{\displaystyle}%
%   \? 函数$f(z)$可以写成
%   $$
%   \frac{1}{(z-2)(z-3)}=\frac{1}{(z-3)}-\frac{1}{(z-2)} \score{2}
%   $$
%   \+由 $|z|>3$ 得到 $\left|\frac{3}{z}\right|<1, \left|\frac{2}{z}\right|<1$,  \score{3}
%   \+ $$\frac{1}{z-3} = \frac{1}{z} \frac{1}{1-\frac{3}{z}} = \frac{1}{z} \sum_{n=0} \left(\frac{3}{z}\right)^n,  \score{5} $$
%   \+类似的可以得到 $$\frac{1}{z-2} = \frac{1}{z} \frac{1}{1-\frac{2}{z}} = \frac{1}{z} \sum_{n=0} \left(\frac{2}{z}\right)^n,  \score{7}$$
%   \+ 最终我们有 
%   $$f(z) = \sum_{n=0}^{\infty}\left(3^n-2^n\right) z^{-n-1}.  \score{8}$$
% \end{solution}\bigskip



% \begin{problem}
% 计算矩形波$$
% \begin{aligned}
% & f(x)=0,  \quad-\pi<x<0,  \\
% & f(x)=h,  \quad 0<x<\pi . 
% \end{aligned}
% $$
% 的傅里叶级数展开. 
% \end{problem}






% \begin{problem}{(8分)}
%   计算积分 $$\int_0^{2 \pi} \frac{\dx}{(a+b \cos x)^2},  \quad a>b>0$$. 
% \end{problem}

% \begin{solution}
 
%   $ I\?=\int_0^{\infty} x^{\alpha-1} e^{-x \cos \theta} \cos (x \sin \theta) \dx$ \par
%      \+ $=\int_0^{\infty} x^{\alpha-1} e^{-x \cos \theta} \operatorname{Re}\left(e^{i x \sin \theta}\right) \dx $ \par
%      \+ $=\operatorname{Re} \int_0^{\infty} x^{\alpha-1} e^{-x(\cos \theta-i \sin \theta)} \dx$ \par
%     \+ $=\operatorname{Re} \int_0^{\infty} x^{\alpha-1} e^{-x e^{-i \theta}} \dx$ \score{3}
%    令 $x e^{-i \theta}=y$ 做变量代换$\Rightarrow$ $x=e^{i \theta} y,  \dx=e^{i \theta} \dy$  \score{5}\newline
%    于是,  $I \?=\operatorname{Re} \int_0^{\infty}\left(e^{i \theta} y\right)^{\alpha-1} e^{-y} e^{i \theta} \dy $ \par
%          \+ $ =\operatorname{Re}\left(e^{i \theta}\right)^\alpha \int_0^{\infty} y^{\alpha-1} e^{-y} \dy $ \par
%          \+ $ =\operatorname{Re}\left(e^{i \alpha \theta}\right) \Gamma(\alpha)$ \par
%          \+ $ = \cos {(\theta \alpha)} \Gamma(\alpha)  $\score{8}
% \end{solution}\bigskip
% \vfill

% \begin{problem}{(8分)}
%   将下面的偏微分方程化成标准形式, 
%    $$
%   %  u_{x x}+4 u_{x y}+5 u_{y y}+u_x+2 u_y=0. 
%    u_{x x}+4 u_{x y}+5 u_{y y}=0. 
%    $$

%  \end{problem}

% \vfill

\begin{problem}{(10分)}
求谐振子常微分方程$$\frac{d^2y}{d x^2 } + \omega^2 y = 0,$$
\end{problem}
在$x=0$处为偶函数的级数解.
\smallskip

\begin{solution}
  由$p(x) = 1, q(x) = \omega^2$,可知$x=0$为常微分方程的常点. \score{2}
  解可以写成
  $y(x)=\sum_{k=0}^{\infty} a_k\left(x-x_0\right)^k$.
其导数为
$$y^{\prime}(x)=\sum_{k=0}^{\infty} a_k \cdot k\left(x-x_0\right)^{k-1},$$
二阶导数为
$$y^{\prime \prime}(x)=\sum_{k=0}^{\infty} a_k k(k-1)\left(x-x_0\right)^{k-2}$$% \score{4}
带入原方程得到\smallskip
$$y^n+w^2 y=\sum_{k=0}^{\infty} a_k \cdot k(k-1)\left(x-x_0\right)^{k-2}+w^2 \sum_{k=0}^{\infty} a_k\left(x-x_0\right)^k=0. \score{4}
$$
下面分析各项的系数有
\?$$x^0: \quad a_2 \cdot 2 \cdot 1+w_2 a_0=0 ,$$
\+$$x^1: a_3 \cdot 3 \cdot 2+w^2 a_1=0,$$
\+$$x^2: \quad a_4 \cdot 4 \cdot 3+w^1 a_2=0, $$ 
\+$$x^3: a_5 \cdot 5.4+w^2 a_3=0. \score{6} $$

由此得到
$a_{j+2}=\frac{-\omega^2}{(j+2)(j+1)} a_j$
 由偶函数要求知$a_1=0$, 从而$a_3=0, a_5=0, \cdots$ \score{8}
 对于偶数阶项, 有
$$a_2=-\frac{w^2}{2\cdot1 } a_0 = -\frac{w^2}{2!} a_0,  
 \quad a_4=-\frac{w^2 }{4 \cdot 3} a_2=+\frac{w^4}{4 !} a_0,
 \quad a_6=-\frac{w^2}{6 \cdot 5} a_4=-\frac{w 6}{6 !} a_0 .
 $$
则解为
$$y  \left.=a_0\left[1-\frac{(\omega x)^2}{2 !}+\frac{(\omega x}{4}\right)^4-\frac{\omega x x^6}{6 !}+\omega\right] 
 =a_0 \cos \omega x . \score{10}$$
\end{solution}
\bigskip

\begin{problem}{(10分)}
求解下列定解问题: 
$\left\{\begin{array}{l}
  \nabla^2 u=0,  \quad 0\leq r \leq a \\ 
  \left. u\right|_{r=a}=\lambda_0 \cos^2 \theta,  \\
  \left. u\right|_{r=0} \text{有限}. 
  %\left. u\right|_{r=b}=\lambda_0 \cos ^2 \theta. 
\end{array}\right. $
\end{problem} 
\bigskip

\begin{solution}
由边界条件可知该问题为轴对称情况
通解形式为
$$  u(r,  \theta) = \sum_{l=0}^{\infty} \left( A_l r^l + \frac{B_l}{r^{l+1}} \right) P_{l} (\cos \theta). 
$$
由$r=0$处有限的边界条件可知$B_l=0$.  \score{2}\\
因此, 对$r=a$处的边界条件带入, 得
$$
\sum_{l=0}^{\infty}  A_l r^l  P_{l} (\cos \theta) 
= \lambda_0 \cos^2 \theta = \lambda_0 x^2 \score{4} 
$$ 
勒让德多项式中$$P_0(x) = 1,  P_1(x) = x,  P_2 (x) = \frac{1}{2}(3x^2 - 1)$$ \\
不难得到 $$x^2 = \frac{2}{3} P_2(x) + \frac{1}{3} P_0(x), \score{6}$$  \\
比较两边系数, 得
$$A_0 = \frac{\lambda_0}{3}, \\
 A_2 = \frac{2 \lambda_0}{3a^2}, \\
  A_l = 0 (l\neq 0,  2).  \score{8} $$ 
这样, 
$$u(r, \theta) = \frac{\lambda_0}{3} + \frac{2 \lambda_0}{3a^2} r^2 P_2(\cos \theta).  \score{10} $$ 
\end{solution}


% \begin{problem}{(10分)}
%   求解下列定解问题: 
%   $\left\{\begin{array}{l}
%     \nabla^2 u=0,  r \geq a \\ 
%     \left. u\right|_{r=a}=u_0 \cos^2 \theta,  \\
%     \left. u\right|_{r=\infty } \text{有限}. 
%     %\left. u\right|_{r=b}=u_0 \cos ^2 \theta. 
%   \end{array}\right. $
% \end{problem} 
% \vfill


% \begin{solution}
% 由边界条件可知该问题为轴对称情况
% 通解形式为
% $$  u(r,  \theta) = \sum_{l=0}^{\infty} \left( A_l r^l + \frac{B_l}{r^{l+1}} \right) P_{l} (\cos \theta). 
% $$
% 由$r=\infty $处有限的边界条件可知$A_l=0$.  \score{2}\\
% 因此, 对$r=a$处的边界条件带入, 得
% $$
% \sum_{l=0}^{\infty}  \frac{B_l}{a^{l+1}}  P_{l} (\cos \theta) 
% = u_0 \cos^2 \theta = u_0 x^2 \score{4} 
% $$ 
% 勒让德多项式中$$P_0(x) = 1,  P_1(x) = x,  P_2 (x) = \frac{1}{2}(3x^2 - 1)$$ \\
% 不难得到 $$x^2 = \frac{2}{3} P_2(x) + \frac{1}{3} P_0(x), \score{6}$$  \\
% 比较两边系数, 得
% $$B_0 = \frac{u_0}{3} a, \\
%  B_2 = \frac{2 u_0 a^3}{3}, \\
%   b_l = 0 (l\neq 0,  2).  \score{8} $$ 
% 这样, 
% % $$u(r, \theta) = \frac{u_0}{3} + \frac{2 u_0}{3a^2} r^2 P_2(\cos \theta).  \score{10} $$ 
% $$u(r, \theta) = \frac{u_0}{3}  \frac{a}{r}+ \frac{2 u_0}{3}  \frac{a^3}{r^3} P_2(\cos \theta).  \score{10} $$ 
% \end{solution}\bigskip



% \vspace{15em}

% \newpage
% \clearpage
% \bigskip
\newpage
\makedata{可能用到的数据和公式} %附录数据
柯西公式(Cauchy's formula)
\[
  f^{(n)}(z) = \frac{n!}{2\pi \imath} \oint_C \frac{f(\zeta)}{(\zeta - z)^{n+1}} d \zeta. 
  \label{eq:cauchy_formula_nth_derivative}
\]

\smallskip
函数$f(t)$的拉普拉斯变换$\bar{f}(p)$为
\[
    \bar{f}(p) = \mathcal{L} \{ f(t) \} = \int_0 ^{\infty} e^{-pt} f(t) \dt . 
\]

\smallskip

双曲正余函数(hyperbolic sine/cosine function)表达为
\begin{align*}
    \cosh z&= \frac{e^{z} + e^{ - z} }{2} , 
    \\
    \sinh z &= \frac{e^{z} - e^{ - z} }{2} . 
\end{align*}
% \smallskip

伽玛函数(Gamma function)
\[
  \Gamma(z) \equiv \int_{0}^{\infty} e^{-t} t^{z-1} d t,  \quad \operatorname{Re} z>0 . 
\]
\smallskip

贝塔函数(Beta function)
\[
    B(p,  q) = \int_0^1 t^{p -1} (1-t)^{q-1} \dt,  \operatorname{Re} p> 0,  \operatorname{Re} q >0. 
    \label{eq:beta_def1}
\]
\smallskip

% \begin{tabularx}{\linewi\dth}{*{4}{>{$}X<{$}}}
% \hline
% \Phi_0(0. 5)=0. 6915 & \Phi_0(1)=0. 8413 & \Phi_0(2)=0. 9773 & \Phi_0(2. 5)=0. 9938 \\
% t_{0. 01}(8)=3. 355 & t_{0. 01}(9)=3. 250 & t_{0. 01}(15)=2. 947 & t_{0. 01}(16)=2. 921 \\
% \chi_{0. 005}^2(8)=22. 0 & \chi_{0. 005}^2(9)=23. 6 & \chi_{0. 005}^2(15)=32. 8 & \chi_{0. 005}^2(16)=34. 3 \\
% \chi_{0. 995}^2(8)=1. 34 & \chi_{0. 995}^2(9)=1. 73 & \chi_{0. 995}^2(15)=4. 60 & \chi_{0. 995}^2(16)=5. 14 \\
% \hline
% \end{tabularx}

% 两个自变数 $x$ 和 $y$ 的二阶线性偏微分方程的通用形式
% \[
%     a_{11} u_{x x}+2 a_{12} u_{x y}+a_{22} u_{y y}+b_1 u_x+b_2 u_y+c u+f=0. 
%     \label{eq:two_variable_diff_equation}
% \]
% \smallskip

球坐标系下的拉普拉斯算符
\[
 \Delta = \frac{1}{r^2} \frac{\partial}{\partial r} \left( r^2 \frac{\partial }{\partial r} \right)
  + \frac{1}{r^2\sin \theta} \frac{\partial}{\partial \theta} \left( \sin\theta \frac{\partial}{\partial \theta} \right)
  + \frac{1}{r^2\sin^2 \theta} \frac{\partial^2}{\partial^2 \varphi} .
\]
分离变量得到轴对称($m=0$)
情况下解的形式为
\[
  u(r,  \theta) = \sum_{l=0}^{\infty} \left( A_l r^l + \frac{B_l}{r^{l+1}} \right) P_{l} (\cos \theta)
\]

\smallskip

$l$ 阶勒让德多项式(Legendre polynomial)的罗德里格斯表达式(Rodrigues's formula) 
$$P_l(x)=\frac{1}{2^l l !} \frac{d^l}{d x^l}\left(x^2-1\right)^l$$
\end{document}
