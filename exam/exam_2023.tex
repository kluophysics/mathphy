% -*- coding: utf-8 -*-
% !TEX program = xelatex
\documentclass{njustexam}

%注意:注释掉这条命令则显示答案, 为存档用, 交教务处. 使用这条命令则为学生考试用, 交印刷厂印刷
\answerfalse 

\begin{document}
%%%%%%%%%%%%%%%%% 以下内容根据需要更改%%%%%%%%%%%
\renewcommand{\course}{数学物理方法}                          %课程名
\renewcommand{\duration}{120}                                            %考试时长
\renewcommand{\credit}{5}                                                   %学分
\renewcommand{\syllabus}{11044102}                               %教学大纲编号
\renewcommand{\fullmark}{100}                                            %满分分值
\renewcommand{\composer}{罗凯}            %组卷教师
\renewcommand{\composedate}{\today}                   %组卷日期
\renewcommand{\validator}{}                                        %审定人
\renewcommand{\coursetype}{1}                                            % 1 为必修, 0 为选修
\renewcommand{\exammethod}{1}                                         % 1 为闭卷, 0 为开卷
\renewcommand{\testpaper}{A}                                              % A 或 B
%%%%%%%%%%%%%%%%%%%%%%%%%%%%%%%%%%%%%%

\makehead % 生成试卷表头

\makepart{填空题}{共~11~小题, 每空~1~分, 共~30~分}

\begin{problem}
 复数 $e^{1+\imath}$的实部为\fillout{$e\cos{1}$},  虚部为\fillout{$e\sin{1}$},  
 模为\fillout{$e$}, 辐角为\fillout{$1$}. 
\end{problem}


\begin{problem}
方程 $z^3 + \imath = 0$的
的三个根分别是\fillout{$\imath$}, 
\fillout{$-\frac{\sqrt{3}}{2} - \frac{\imath}{2}$}, 
\fillout{$ \frac{\sqrt{3}}{2} - \frac{\imath}{2}$}. 
\end{problem}


\begin{problem}
  以复数 $z_0$为圆心,  以任意小正实数 $\varepsilon$ 为半径作一圆,  则圆内所有 点的集合称为 $z_0$ 的\fillout{邻域}. 
\end{problem}


\begin{problem}
$f(z)=z^2$的奇点数有\fillout{零}个, 而$f(z)=|z|^2$的奇点数有\fillout{无数}个. 
\end{problem}


\begin{problem}
某解析函数实部为$u(x, y) = x^2 - y^2$, 则该解析函数的虚部为\fillout{$2xy$}加一常数$C$. 
\end{problem}

\begin{problem}
  级数$\sum z^n$的收敛半径为\fillout{1};
  级数$\sum \frac{n !}{n^n} z^n$的收敛半径为\fillout{e}. 
  % 级数$\sum \frac{n !}{n^n} z^n$的收敛半径为\fillout{e};
  % 级数$\sum n^{\ln{n}} z^n$的收敛半径为\fillout{1}. 
  % 如果复幂级数$\sum_{n=0}^{\infty} d_n z^n$的收敛半径为$R = 3$, 
  % 那么它在复平面上的绝对收敛区域是 $\{z \in \mathbb{C} \mid |z| < \fillout{} \}$. 
 \end{problem}

\begin{problem}
 $ f(z) = \frac{1}{z( z- 2)^3}$,  $z=0$是\fillout{1}阶极点, 此处留数为 \fillout{$-\frac{1}{8}$}, 
 $z=2$是\fillout{3}阶极点,  该处留数为\fillout{$\frac{1}{8}$}. 
\end{problem}

\begin{problem}
典型的数理方程分为波动方程、输运方程和稳定场方程, 
它们大致对应数学上二阶偏微分方程的分类, 即\fillout{双曲型}、\fillout{抛物型}和\fillout{椭圆型}. 
\end{problem}

\begin{problem}
  定解问题由\fillout{泛定}方程和\fillout{定解}条件组成, 后者分为\fillout{初始}条件和\fillout{边界}条件. 
\end{problem}

\begin{problem}
$\Gamma(z)$的取值: $\Gamma(n)=$\fillout{$(n-1)!$},  
% $\Gamma(\frac{1}{2})=$\fillout{$\sqrt{\pi}$},  %, , 
$\Gamma(5)=$\fillout{$24$}, 
 $\Gamma(\frac{3}{2})=$\fillout{$\frac{\sqrt{\pi}}{2}$}. 
\end{problem}

\begin{problem}
连带勒让德函数$P_\ell^m(x)$中$m$%最大可取的值为\fillout{$\ell$}, 
最小可取值为\fillout{$-\ell$}, 
$P_1^0(x) = $\fillout{x},  $P_1^1(x) = $\fillout{$\sqrt{1-x^2}$}. 
\end{problem}

% \begin{problem}
% 部分三类柱函数的渐近行为
% $\lim_{x\to 0} J_0(x)=$ \fillout{1}, 
%  $\lim_{x\to 0} J_\nu(x)=$ \fillout{0}($\nu>0$), 
%  $\lim_{x\to 0} N_\nu(x)=$ \fillout{$\infty$}. 
% \end{problem}

% \begin{problem}
% 球谐函数$Y_l^m(\theta,  \varphi)$的常见表达式, 
% $Y_0^0(\theta,  \varphi)=$\fillout{},  
% $Y_1^0(\theta,  \varphi)=$\fillout{}, 
% $Y_1^1(\theta,  \varphi)=$\fillout{}. 
% \end{problem}
  

\makepart{选择题}{共~15~小题, 每题~2~分, 共~30~分}

\begin{problem}
下面哪一个笑话{\bf{是}}课上同学分享的 \pickout{B}
  \begin{abcd}
  \item 为什么复数很容易感到孤独?因为他们总是在二维空间里, 而我们生活在三维世界.    
  \item 为什么实变函数不喜欢参加派对?因为它们只会带着实数!没有虚情假意!
  \item 有一天, $\imath$和$-1$在草地上漫步. $\imath$说:"我觉得今天很平凡." $-1$回答:"是的, 真实又实在."
  \item 复数和实数一起比赛跑步, 结果实数总是赢. 为什么?因为复数总是在原地转圈(虚部). 
  \end{abcd}
  \end{problem}

  % 留数定理的基本表达式:

% 留数定理的基本表达式是:

% \item $\oint_C f(z) , dz = 2\pi i \text{Res}[f(z),  z_0]$

% \item $\oint_C f(z) , dz = \frac{2\pi}{i} \text{Res}[f(z),  z_0]$

% \item $\oint_C f(z) , dz = \pi i \text{Res}[f(z),  z_0]$

% \item $\oint_C f(z) , dz = \frac{\pi}{i} \text{Res}[f(z),  z_0]$

\begin{problem}
  下面哪一个网址与我们学习的数学物理方法特殊函数息息相关? \pickout{B}
  \begin{abcd}
    \item https://mathworld.wolfram.com/
    \item https://dlmf.nist.gov
    \item https://csrc.nist.gov
    \item https://projecteuclid.org/
  \end{abcd}

\end{problem}

\begin{problem}
  如果$f(z)$在$z_0$处有一个一阶极点, 其留数为:
  \pickout{A}
\begin{abcd}
  \item $\text{Res}[f(z),  z_0] = \lim_{z \to z_0} (z - z_0)f(z)$
  \item $\text{Res}[f(z),  z_0] = \frac{1}{\lim_{z \to z_0} (z - z_0)f(z)}$
  \item $\text{Res}[f(z),  z_0] = \frac{1}{f'(z_0)}$
  \item $\text{Res}[f(z),  z_0] = f'(z_0)$
\end{abcd}
\end{problem}
% 极点和留数关系:



% 留数定理的应用:

% 使用留数定理计算积分$\int_{-\infty}^{\infty} \frac{1}{1+x^2} , dx$的结果是:

% \item $\frac{\pi}{2}$

% \item $\pi$

% \item $2\pi$

% \item $0$



% \begin{problem}
% 在描述扩散现象时, 哪个方程最适合描述非稳态情况? \pickout{C}
% \begin{abcd}
%   \item 拉普拉斯方程   
%   \item Poisson方程
%   \item 热传导方程
%   \item 波动方程
%   \end{abcd}
% \end{problem}

\begin{problem}
  对于一根固定边界的弦上的波动, 哪个方程描述波函数$u(x,  t)$的行为?
  \pickout{C}
  \begin{abcd}
    \item 热传导方程 
    \item 泊松方程
    \item 波动方程
    \item 拉普拉斯方程
    \end{abcd}
\end{problem}

\begin{problem}
  在信号处理中, 哪个方法常常用于将一个信号从时域转换到频域?
  \pickout{A}
  \begin{abcd}
    \item 傅里叶变换 
    \item 拉普拉斯变换
    \item 小波变换
    \item 保角变换
  \end{abcd}
\end{problem}

% \begin{problem}
%   对于函数$F(\omega)$的傅里叶变换表示为$f(t) = \mathcal{F}^{-1}[F(\omega)]$, 其中$\mathcal{F}^{-1}$表示傅里叶逆变换. 下面哪个等式是正确的?
%   \pickout{D}
%   \begin{abcd}
%     \item  $f(t) = \int_{-\infty}^{\infty} F(\omega) e^{-\imath\omega t} d\omega$

%     \item  $f(t) = \frac{1}{2\pi} \int_{0}^{\infty} F(\omega) e^{-\imath\omega t} d\omega$

    
%     \item $f(t) = \mathcal{F}^{-1}[F(t)]$
    
%     \item $f(t) = \frac{1}{\sqrt{2\pi}} \int_{-\infty}^{\infty} F(\omega) e^{-\imath\omega t} d\omega$

%   \end{abcd}

% \end{problem}


\begin{problem}
  单位阶跃函数$\Theta(t)$的拉普拉斯变换是:
  \pickout{A}
  \begin{abcd}
    \item $\frac{1}{s}$
    \item $\frac{1}{s^2}$
    \item $\frac{1}{s^2} + \frac{1}{s}$
    \item $\frac{1}{s} - \frac{1}{s^2}$
  \end{abcd}

\end{problem}

\begin{problem}
复数$z$可以写成幅度和相位的形式$z = re^{i\theta}$, 其中$r$是幅度, $\theta$是相位. 
如果要展开$\sin(\theta)$, 下面哪个级数是正确的? \pickout{D}
\begin{abcd}
  \item $\sum_{n=0}^{\infty} \frac{(-1)^n \theta^{2n}}{(2n)!}$
  \item $\sum_{n=1}^{\infty} \frac{(-1)^n \theta^n}{n!}$
  \item $\sum_{n=1}^{\infty} \frac{(-1)^n \theta^{n+1}}{n!}$
  \item $\sum_{n=0}^{\infty} \frac{(-1)^n \theta^{2n+1}}{(2n+1)!}$

\end{abcd}
\end{problem}


% 伽玛函数的性质:

% 伽玛函数$\Gamma(z)$的性质中, 以下哪个是正确的?

% \item $\Gamma(z+1) = z\Gamma(z)$

% \item $\Gamma(z+1) = \frac{\Gamma(z)}{z}$

% \item $\Gamma(z+1) = \frac{1}{z\Gamma(z)}$

% \item $\Gamma(z+1) = z+\Gamma(z)$

\begin{problem}
  关于贝塔函数$B(p,  q)$的说法正确的是:
  \pickout{C}
\begin{abcd}
  \item $B(p,  q) = \int_{0}^{\infty} t^{p+q-1}e^{-t} \dt$
  \item $B(p,  q) = \int_{0}^{1} t^{p-1}(1-t)^{q-1} \dt$
  \item $B(p,  q) = \frac{\Gamma(p)\Gamma(q)}{\Gamma(p+q)}$
  \item $B(p,  q) = \frac{\Gamma(p+q)}{\Gamma(p)\Gamma(q)}$
\end{abcd}
\end{problem}

% \begin{problem}
%   对于二阶线性偏微分方程以下说法正确的是: \pickout{D}
%   \pickout{A}
%   \begin{abcd}
%     \item 傅里叶变换 
%     \item 拉普拉斯变换
%     \item 小波变换
%     \item 保角变换
%   \end{abcd}
% \end{problem}

% 伽玛函数和贝塔函数的关系:

% 伽玛函数$\Gamma(z)$和贝塔函数$B(p,  q)$之间的关系是:

% \item $\Gamma(z) = \frac{B(z,  1)}{z}$

% \item $\Gamma(z) = \frac{B(z,  1)}{z-1}$

% \item $\Gamma(z) = \frac{B(z,  1)}{z+1}$

% \item $\Gamma(z) = \frac{B(z,  1)}{z+2}$


\begin{problem}
  对于非负整数$n$, 哪个是贝塞尔函数$J_n(x)$的递推关系?
  \pickout{C}
  \begin{abcd}
    \item  $J_{n+1}(x) = J_n(x) - \frac{2n}{x} J_{n-1}(x)$
    \item  $J_{n+1}(x) = 2 J_n(x) - J_{n-1}(x)$
    \item  $J_{n+1}(x) = \frac{2n}{x} J_n(x) - J_{n-1}(x)$
    \item  $J_{n+1}(x) = J_n(x) + \frac{2n}{x} J_{n-1}(x)$
  \end{abcd}
\end{problem}

\begin{problem}
  勒让德多项式$P_n(x)$的正交性质可以表示为:
  \pickout{B}
  \begin{abcd}
    \item $\int_{-1}^{1} P_n(x) dx = 1$

    \item $\int_{-1}^{1} P_n(x) P_m(x) dx = 0$, 对于$n \neq m$
    
    \item $\int_{0}^{\pi} P_n(x) dx = 0$
    
    \item $\int_{-1}^{1} P_n(x) P_m(x) dx = \delta_{nm}$
  \end{abcd}
\end{problem}

\begin{problem}
  $Y_{1}^{0}(\theta,  \phi)$的表达式是:
  \pickout{A}
\begin{abcd}
  \item $\sqrt{\frac{3}{4\pi}} \cos \theta$
  \item $\sqrt{\frac{3}{4\pi}} \sin \theta$
  \item $\sqrt{\frac{3}{8\pi}} \sin \theta e^{i\phi}$
  \item $\sqrt{\frac{3}{8\pi}} \cos \theta e^{i\phi}$
\end{abcd}
\end{problem}

% \begin{problem}
% 对于$x$趋于无穷大时, 第一类贝塞尔函数$J_n(x)$的渐近行为是:
% \begin{abcd}
%   \item $J_n(x) \sim \sqrt{\frac{2}{\pi n}} \cos\left(x - \frac{\pi}{4}n - \frac{\pi}{2}\right)$
%   \item $J_n(x) \sim \sqrt{\frac{2}{\pi n}} \cos\left(x - \frac{\pi}{4}n + \frac{\pi}{4}\right)$
%   \item $J_n(x) \sim \sqrt{\frac{1}{\pi n}} \cos\left(x - \frac{\pi}{4}n + \frac{\pi}{4}\right)$
%   \item $J_n(x) \sim \sqrt{\frac{1}{2\pi n}} \cos\left(x - \frac{\pi}{4}n + \frac{\pi}{4}\right)$
% \end{abcd}
% \end{problem}


\begin{problem}
  对于$n$为正整数时, $x$趋于零时, 第一类贝塞尔函数$J_n(x)$的极限是:
  \pickout{B}

  \begin{abcd}
\item $\lim_{x \to 0} J_n(x) = 1$

\item $\lim_{x \to 0} J_n(x) = 0$

\item $\lim_{x \to 0} J_n(x) = (-1)^n$

\item $\lim_{x \to 0} J_n(x) = \infty$
 \end{abcd}
\end{problem}


\begin{problem}
  球贝塞尔函数满足哪个微分方程? 
  \pickout{B}

  \begin{abcd}
    \item 热传导方程
    \item 亥姆霍兹方程
    \item 薛定谔方程
    \item 泊松方程
  \end{abcd}
\end{problem}

\begin{problem}
  在一个球对称的电场中, 电势 $V(r,  \theta)$ 的分布可以用球谐函数展开表示. 
  若电场满足拉普拉斯方程, 
  此时, 电势 $V(r,  \theta)$ 的展开系数中, 与角度 $\theta$ 有关的部分应当包含: \pickout{D}
  \begin{abcd}
  \item 虚宗量贝塞尔函数 $i^l J_l(kr)$
  \item 贝塞尔函数 $J_l(kr)$
  \item 球谐函数 $Y_{lm}(\theta,  \phi)$
  \item 勒让德函数 $P_l(\cos\theta)$

  \end{abcd}
\end{problem}



% \begin{problem}
%   在一个照明公司, 设计师们希望通过调整灯具的灯罩形状来实现更均匀的光照分布. 他们发现球谐函数可以描述灯罩的光照分布. 如果他们希望获得最均匀的照明效果, 最可能使用的球谐函数是:
%   \pickout{A}
%   \begin{abcd}
%   \item $Y_{0}^0(\theta,  \phi)$
  
%   \item $Y_{1}^{-1}(\theta,  \phi)$
  
%   \item $Y_{2}^2(\theta,  \phi)$
  
%   \item $Y_{3}^1(\theta,  \phi)$
%   \end{abcd}
%   \end{problem}

  \begin{problem}
    光学工程师设计了一种用于薄膜涂层的新材料, 以改善透明度和折射率. 为了最佳地控制光的传播, 他们最可能使用的球谐函数是:
    \pickout{D}
    \begin{abcd}

    \item $Y_{1}^0(\theta,  \phi)$
    
    \item $Y_{1}^{-1}(\theta,  \phi)$
    
    \item $Y_{3}^1(\theta,  \phi)$
    
    \item $Y_{3}^2(\theta,  \phi)$
   \end{abcd}

  \end{problem}
% \item $J_n(x) \sim \frac{x^n}{n!}$

% \item $J_n(x) \sim \frac{x^n}{\sqrt{n}}$

% \item $J_n(x) \sim (-1)^n\frac{x^{2n}}{(2n)!}$

% \item $J_n(x) \sim \frac{(-1)^n}{n!}x^n$

\makepart{计算证明题}{共~40~分}%共~6~小题, 每小题~8~分, 共~48~分}

% \begin{problem}{(6分)}
%   证明
%   $$
%   |\sinh{y}| \leq |\sin(x+\imath y)| \leq |\cosh{y}|. 
%   $$
% \end{problem}

% \renewcommand{\solutionname}{证} % 将“解"字改为“证"字
% \begin{solution}
%   \everymath{\displaystyle}%
%   原式 \? $=\int\e^{2x}\, \sec^2 x\dx+2\int\e^{2x}\, \tan x\dx$ \score{2}
%   \+ $=\int\e^{2x}\, \d(\tan x)+ 2\int\e^{2x}\, \tan x\dx$ \score{4}
%   \+ $=\e^{2x}\, \tan x - 2\int\e^{2x}\, \tan x\dx+ 2\int\e^{2x}\, \tan x\dx$ \score{6}
%   \+ $=\e^{2x}\, \tan x + C$ \score{8}
% \end{solution}

% \begin{problem}{(6分)}
%   设 $\Psi(t,  x)=e^{ 2 t x-t^2 }$,  $t$是复变数,  证明: 
%   $$
%   \left. \frac{\partial^n \Psi(t,  x)}{\partial t^n}\right|_{t=0}=(-1)^n e^{x^2} \frac{d^n}{\dx^n} e^{-x^2}
%   $$
%   % 提示: 对回路积分进行积分变数的代换 $\xi=z-x$. 
% \end{problem}

\begin{problem}{(6分)}
  计算下面的围道积分
  $$\oint_{C} \frac{\sin \frac{\pi z}{4}}{z^2-1} dz,  $$
   其中$C$为$|z-1|=1$. 
\end{problem}
\vfill

\begin{solution}
\everymath{\displaystyle}%
\? 被积函数$f(z) = \frac{\sin \frac{\pi z}{4}}{z^2-1}$含两个奇点$z_1=1,  z_2=-1$. \\ 
\+ 路径$C$只包含其中$z_1=1$这一奇点.  \score{2}
\+ 不难得到$f(z)$在$z=z_1$处的留数为 $$\text{Res} f(z) |_ {z\to z_1} = \lim_{z \to z_1} (z-z_1) f(z) =\frac{\sqrt{2}}{4}\score{4}$$
\+ 因此根据留数定理, 最终结果为
$$ 2\pi \imath \frac{\sqrt{2}}{4} = \frac{\pi\imath}{\sqrt{2}}\score{6}$$ 
\end{solution}




% \begin{problem}{(6分)}
%   求 $|\sin z|$ 在闭区域 $0 \leq \Re z \leq 2 \pi,  0 \leq \Im z \leq 2 \pi$ 中的最大值. 
% \end{problem}
\begin{problem}{(6分)}
  计算拉普拉斯变换的原函数
  $$
    % \frac{4 p-1}{\left(p^2+p\right)\left(4 p^2-1\right)}. 
    \bar{f}(p)=\frac{2 p}{p^2-1}. 
  $$
\end{problem}
\vfill

\begin{solution}
  函数$\bar{f}(p)$可以写成
  $$
  \bar{f}(p) = \frac{3}{2} \left[ \frac{1}{p+1} + \frac{1}{p-1}  \right] \score{2}
  $$
  由拉普拉斯变换的定义不难得到$$\mathcal{L}^{-1} \left[ \frac{1}{p-s} \right] = e^{st}, $$
  故$\mathcal{L}^{-1} \left[ \frac{1}{p+1}\right] = e^{-t} $
  并且$\mathcal{L}^{-1} \left[ \frac{1}{p-1}\right] = e^{+t} $\score{4} 
  最终我们有 原函数
  $$f(t) =\mathcal{L}^{-1} \left[ \bar{f}(p) \right] = 3\cosh t.  \score{6}$$
\end{solution}
  

  
\begin{problem}{(6分)}
    用$\Gamma$函数求积分
    $$
    \int_0^\infty x^{\alpha -1 } e^{-x \cos{\theta}} \cos\left( x \sin{\theta} \right) dx, 
    $$
    其中$\alpha > 0,  -\frac{\pi}{2} < \theta < \frac{\pi}{2}$. 
  \end{problem} 
    
\vfill

\begin{solution}
  
$ I\?=\int_0^{\infty} x^{\alpha-1} e^{-x \cos \theta} \cos (x \sin \theta) \dx$ \par
    \+ $=\int_0^{\infty} x^{\alpha-1} e^{-x \cos \theta} \operatorname{Re}\left(e^{i x \sin \theta}\right) \dx $ \par
    \+ $=\operatorname{Re} \int_0^{\infty} x^{\alpha-1} e^{-x(\cos \theta-i \sin \theta)} \dx$ \par
  \+ $=\operatorname{Re} \int_0^{\infty} x^{\alpha-1} e^{-x e^{-i \theta}} \dx$ \score{2}
  令 $x e^{-i \theta}=y$ 做变量代换$\Rightarrow$ $x=e^{i \theta} y,  \dx=e^{i \theta} \dy$  \score{4}\newline
  于是,  $I \?=\operatorname{Re} \int_0^{\infty}\left(e^{i \theta} y\right)^{\alpha-1} e^{-y} e^{i \theta} \dy $ \par
        \+ $ =\operatorname{Re}\left(e^{i \theta}\right)^\alpha \int_0^{\infty} y^{\alpha-1} e^{-y} \dy $ \par
        \+ $ =\operatorname{Re}\left(e^{i \alpha \theta}\right) \Gamma(\alpha)$ \par
        \+ $ = \cos {(\theta \alpha)} \Gamma(\alpha)  $\score{6}
\end{solution}




\begin{problem}{(8分)}
将函数$$f(z) = \frac{1}{(z-2)(z-3)}$$ 
在 $|z|>3$的环域展开为洛朗级数. 
\end{problem}
  
  
\vfill

\begin{solution}
  \everymath{\displaystyle}%
  \? 函数$f(z)$可以写成
  $$
  \frac{1}{(z-2)(z-3)}=\frac{1}{(z-3)}-\frac{1}{(z-2)} \score{2}
  $$
  \+由 $|z|>3$ 得到 $\left|\frac{3}{z}\right|<1, \left|\frac{2}{z}\right|<1$,  \score{3}
  \+ $$\frac{1}{z-3} = \frac{1}{z} \frac{1}{1-\frac{3}{z}} = \frac{1}{z} \sum_{n=0} \left(\frac{3}{z}\right)^n,  \score{5} $$
  \+类似的可以得到 $$\frac{1}{z-2} = \frac{1}{z} \frac{1}{1-\frac{2}{z}} = \frac{1}{z} \sum_{n=0} \left(\frac{2}{z}\right)^n,  \score{7}$$
  \+ 最终我们有 
  $$f(z) = \sum_{n=0}^{\infty}\left(3^n-2^n\right) z^{-n-1}.  \score{8}$$
\end{solution}


% \begin{problem}
% 计算矩形波$$
% \begin{aligned}
% & f(x)=0,  \quad-\pi<x<0,  \\
% & f(x)=h,  \quad 0<x<\pi . 
% \end{aligned}
% $$
% 的傅里叶级数展开. 
% \end{problem}






% \begin{problem}{(8分)}
%   计算积分 $$\int_0^{2 \pi} \frac{\dx}{(a+b \cos x)^2},  \quad a>b>0$$. 
% \end{problem}

% \begin{solution}
 
%   $ I\?=\int_0^{\infty} x^{\alpha-1} e^{-x \cos \theta} \cos (x \sin \theta) \dx$ \par
%      \+ $=\int_0^{\infty} x^{\alpha-1} e^{-x \cos \theta} \operatorname{Re}\left(e^{i x \sin \theta}\right) \dx $ \par
%      \+ $=\operatorname{Re} \int_0^{\infty} x^{\alpha-1} e^{-x(\cos \theta-i \sin \theta)} \dx$ \par
%     \+ $=\operatorname{Re} \int_0^{\infty} x^{\alpha-1} e^{-x e^{-i \theta}} \dx$ \score{3}
%    令 $x e^{-i \theta}=y$ 做变量代换$\Rightarrow$ $x=e^{i \theta} y,  \dx=e^{i \theta} \dy$  \score{5}\newline
%    于是,  $I \?=\operatorname{Re} \int_0^{\infty}\left(e^{i \theta} y\right)^{\alpha-1} e^{-y} e^{i \theta} \dy $ \par
%          \+ $ =\operatorname{Re}\left(e^{i \theta}\right)^\alpha \int_0^{\infty} y^{\alpha-1} e^{-y} \dy $ \par
%          \+ $ =\operatorname{Re}\left(e^{i \alpha \theta}\right) \Gamma(\alpha)$ \par
%          \+ $ = \cos {(\theta \alpha)} \Gamma(\alpha)  $\score{8}
% \end{solution}
% \vfill

% \begin{problem}{(8分)}
%   将下面的偏微分方程化成标准形式, 
%    $$
%   %  u_{x x}+4 u_{x y}+5 u_{y y}+u_x+2 u_y=0. 
%    u_{x x}+4 u_{x y}+5 u_{y y}=0. 
%    $$

%  \end{problem}



\begin{problem}{(10分)}
  求解下列定解问题: 
  $\left\{\begin{array}{l}
    \nabla^2 u=0,  \quad 0\leq r < a \\ 
    \left. u\right|_{r=a}=u_0 \cos^2 \theta,  \\
    \left. u\right|_{r=0} \text{有限}. 
    %\left. u\right|_{r=b}=u_0 \cos ^2 \theta. 
  \end{array}\right. $
\end{problem} 
\vfill


\begin{solution}
由边界条件可知该问题为轴对称情况
通解形式为
$$  u(r,  \theta) = \sum_{l=0}^{\infty} \left( A_l r^l + \frac{B_l}{r^{l+1}} \right) P_{l} (\cos \theta). 
$$
由$r=0$处有限的边界条件可知$B_l=0$.  \score{2}\\
因此, 对$r=a$处的边界条件带入, 得
$$
\sum_{l=0}^{\infty}  A_l r^l  P_{l} (\cos \theta) 
= u_0 \cos^2 \theta = u_0 x^2 \score{4} 
$$ 
勒让德多项式中$P_0(x) = 1,  P_1(x) = x,  P_2 (x) = \frac{1}{2}(3x^2 - 1)$ \\
其中 $x^2 = \frac{2}{3} P_2(x) + \frac{1}{3} P_0(x)$,   \score{6}\\
比较两边系数, 得
$A_0 = \frac{u_0}{3},  A_2 = \frac{2 u_0}{3a^2},  A_l = 0 (l\neq 0,  2)$.  \score{8} \\

这样, 
$$u(r, \theta) = \frac{u_0}{3} + \frac{2 u_0}{3a^2} r^2 P_2(\cos \theta).  \score{10} $$ 
\end{solution}

% \vfill
\newpage
\makedata{可能用到的数据和公式} %附录数据
柯西公式(Cauchy's formula)
\begin{equation*}
  f^{(n)}(z) = \frac{n!}{2\pi \imath} \oint_C \frac{f(\zeta)}{(\zeta - z)^{n+1}} d \zeta. 
  \label{eq:cauchy_formula_nth_derivative}
\end{equation*}

\bigskip
函数$f(t)$的拉普拉斯变换$\bar{f}(p)$为
\begin{equation*}
    \bar{f}(p) = \mathcal{L} \{ f(t) \} = \int_0 ^{\infty} e^{-pt} f(t) \dt . 
\end{equation*}

\bigskip

双曲正余函数(hyperbolic sine/cosine function)表达为
\begin{align*}
    \cosh z&= \frac{e^{z} + e^{ - z} }{2} , 
    \\
    \sinh z &= \frac{e^{z} - e^{ - z} }{2} . 
\end{align*}
% \bigskip

伽玛函数(Gamma function)
\begin{equation*}
  \Gamma(z) \equiv \int_{0}^{\infty} e^{-t} t^{z-1} d t,  \quad \Re z>0 . 
\end{equation*}
\bigskip

贝塔函数(Beta function)
\begin{equation*}
    B(p,  q) = \int_0^1 t^{p -1} (1-t)^{q-1} \dt,  \Re p> 0,  \Re q >0. 
    \label{eq:beta_def1}
\end{equation*}
\bigskip

% \begin{tabularx}{\linewi\dth}{*{4}{>{$}X<{$}}}
% \hline
% \Phi_0(0. 5)=0. 6915 & \Phi_0(1)=0. 8413 & \Phi_0(2)=0. 9773 & \Phi_0(2. 5)=0. 9938 \\
% t_{0. 01}(8)=3. 355 & t_{0. 01}(9)=3. 250 & t_{0. 01}(15)=2. 947 & t_{0. 01}(16)=2. 921 \\
% \chi_{0. 005}^2(8)=22. 0 & \chi_{0. 005}^2(9)=23. 6 & \chi_{0. 005}^2(15)=32. 8 & \chi_{0. 005}^2(16)=34. 3 \\
% \chi_{0. 995}^2(8)=1. 34 & \chi_{0. 995}^2(9)=1. 73 & \chi_{0. 995}^2(15)=4. 60 & \chi_{0. 995}^2(16)=5. 14 \\
% \hline
% \end{tabularx}

% 两个自变数 $x$ 和 $y$ 的二阶线性偏微分方程的通用形式
% \begin{equation*}
%     a_{11} u_{x x}+2 a_{12} u_{x y}+a_{22} u_{y y}+b_1 u_x+b_2 u_y+c u+f=0. 
%     \label{eq:two_variable_diff_equation}
% \end{equation*}
% \bigskip

球坐标系下的拉普拉斯算符
\begin{equation*}
 \Delta = \frac{1}{r^2} \frac{\partial}{\partial r} \left( r^2 \frac{\partial }{\partial r} \right)
  + \frac{1}{r^2\sin \theta} \frac{\partial}{\partial \theta} \left( \sin\theta \frac{\partial}{\partial \theta} \right)
  + \frac{1}{r^2\sin^2 \theta} \frac{\partial^2}{\partial^2 \varphi} .
\end{equation*}
分离变量得到轴对称($m=0$)
情况下解的形式为
\begin{equation*}
  u(r,  \theta) = \sum_{l=0}^{\infty} \left( A_l r^l + \frac{B_l}{r^{l+1}} \right) P_{l} (\cos \theta)
\end{equation*}

\bigskip

$l$ 阶勒让德多项式(Legendre polynomial)的罗德里格斯表达式(Rodrigues's formula) 
$$P_l(x)=\frac{1}{2^l l !} \frac{d^l}{d x^l}\left(x^2-1\right)^l$$
\end{document}
