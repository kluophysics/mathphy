% -*- coding: utf-8 -*-
% !TEX program = xelatex
\documentclass{njustexam}

\usepackage{wrapfig}

%注意:注释掉这条命令则显示答案, 为存档用, 交教务处. 使用这条命令则为学生考试用, 交印刷厂印刷
% \answerfalse 

\begin{document}
%%%%%%%%%%%%%%%%% 以下内容根据需要更改%%%%%%%%%%%
\renewcommand{\course}{大学物理II-模拟}                          %课程名
\renewcommand{\duration}{120}                                            %考试时长
\renewcommand{\credit}{5}                                                   %学分
\renewcommand{\syllabus}{11220804-0}                               %教学大纲编号
\renewcommand{\fullmark}{100}                                            %满分分值
\renewcommand{\composer}{罗凯}            %组卷教师
\renewcommand{\composedate}{\today}                   %组卷日期
\renewcommand{\validator}{}                                        %审定人
\renewcommand{\coursetype}{1}                                            % 1 为必修, 0 为选修
\renewcommand{\exammethod}{1}                                         % 1 为闭卷, 0 为开卷
\renewcommand{\testpaper}{A}                                              % A 或 B
%%%%%%%%%%%%%%%%%%%%%%%%%%%%%%%%%%%%%%

\makehead % 生成试卷表头

\makepart{填空题}{共~8~小题, 每空~1~分, 共~20~分}

% 电磁学:

% 在真空中,光速的数值为 \underline{\hspace{2cm}} m/s。
% 静电场中,库仑力与电荷之间的关系由 \underline{\hspace{2cm}} 定律描述。
% 安培环路定理描述了 \underline{\hspace{2cm}} 和磁场之间的关系。
% 法拉第电磁感应定律描述了磁场变化引起的 \underline{\hspace{2cm}}。
% 电场的高斯定理表明,电场通过一个 \underline{\hspace{2cm}} 的闭合曲面的通量等于这个曲面内的电荷总量除以真空介电常数 $\varepsilon_0$。
% 光学:

% 光的波长与频率之间的关系由 \underline{\hspace{2cm}} 公式描述。
% 在光的折射现象中,折射率定义为光在介质中的速度与真空中的速度的 \underline{\hspace{2cm}}。
% 光的全反射现象发生在光线从折射率较大的介质射向 \underline{\hspace{2cm}} 折射率的介质时。
% 某个透镜的焦距是它的 \underline{\hspace{2cm}} 和介质折射率的函数。
% 杨氏双缝干涉实验中,光的干涉现象是由光的 \underline{\hspace{2cm}} 性质引起的。
% 现代物理:

% 量子力学中,波函数描述了粒子的 \underline{\hspace{2cm}}。
% 量子力学的基本方程是薛定谔方程,它描述了粒子的 \underline{\hspace{2cm}}。
% 物质波的频率与粒子的动量之间的关系由 \underline{\hspace{2cm}} 关系给出。
% 质能方程 $E=mc^2$ 揭示了质量与能量之间的 \underline{\hspace{2cm}}。
% 狭义相对论中,光速是一个 \underline{\hspace{2cm}} 常数。



\begin{problem}
  现代物理的两个支柱理论是\fillout{量子力学}和\fillout{相对论}。
\end{problem}


\begin{problem}
磁感应强度的单位是\fillout{特斯拉 (Tesla)}或\fillout{高斯 (Gauss)}. 
\end{problem}


\begin{problem}
薄膜干涉分为\fillout{等厚}干涉和\fillout{等倾}干涉;牛顿环属于\fillout{等厚}干涉,对应的图案为\fillout{同心圆环}. 
\end{problem}

\begin{problem}
  常见光的衍射分为 \fillout{菲涅耳衍射}(即球面光波的衍射)和 \fillout{夫琅和费衍射}(即平面光波的衍射)。
 \end{problem}

\begin{problem}
爱因斯坦因(Albert Einstein, 1879-1955)成功解释了\fillout{光电效应} 获得1921年诺贝尔物理学奖。该解释采用了
\fillout{普朗克(Planck)}的量子假说。
\end{problem}

\begin{problem}
感应电动势分为\fillout{动生}电动势和\fillout{感生}电动势。 
\end{problem}

% \begin{problem}
%   均匀磁场的磁感强度垂直于半径为r的圆面。今以该圆周为边线,作一半球面S(如图所示),则通过S面的磁通量的大小为: \fillout {$\pi r^2 B$}.
%   % \begin{figure}[h]
%     % \centering
%     \includegraphics{Picture1.png}
%   % \end{figure}
% \end{problem}


% \begin{problem}
%   如图所示,电流从a点分两路通过对称的圆环形分路,汇合于b点,若ca、bd 都沿环的径向,则在环形分路的环心处的磁感应强度为 : \fillout {0}.
%   \includegraphics{Picture2.png}
% \end{problem}


\begin{problem}
  物质波是由\fillout{德布罗意}提出的,后来被实验验证,包括\fillout{电子衍射实验或汤姆逊}实验。
  薛定谔方程的波函数$\Psi$的统计解释由\fillout{玻恩}给出,其模的平方$|\Psi|^2$表示\fillout{概率密度}。 
\end{problem}


\begin{problem}
  一般介质的折射率比空气折射率要\fillout{大},介质中光速比真空中光速\fillout{小}
\end{problem}

\makepart{选择题}{共~20~小题, 每题~2~分, 共~40~分}


\begin{problem}

  本门课程《大学物理II》课堂时间是:\pickout{D}
  \begin{abcd}
    \item 周一和周三上午
    \item 周一和周三下午
    \item 周二和周四上午
    \item 周二和周四下午
  \end{abcd}
\end{problem} 

\begin{problem}

  电流在空间产生磁场遵循的物理定律是:\pickout{B}
  \begin{abcd}
    \item 安培定律
    \item 毕奥-萨伐尔定律
    \item 欧姆定律
    \item 法拉第电磁感应定律
  \end{abcd}
  \end{problem} 

\begin{problem}

  安培环路定理描述了哪两个物理量之间的关系?\pickout{D}
  \begin{abcd}
\item 电荷和电场
\item 电流和电势差
\item 电场和磁场
\item 电流和磁场
  \end{abcd}
  \end{problem}

  \begin{problem}

    法拉第电磁感应定律描述了磁场变化引起的:\pickout{A}
    \begin{abcd}
      \item 电势差
      \item 电场强度
      \item 电荷产生
      \item 电流感应
    \end{abcd}
    \end{problem}

\begin{problem}

  洛仑兹力不会:\pickout{A}
  \begin{abcd}
    \item 改变带电粒子的速率;  
    \item 改变带电粒子的动量;
    \item 改变带电粒子的速度;  
    \item 改变带电粒子的加速度。
  \end{abcd}
  \end{problem}


  \begin{problem}

      对于磁铁来说,磁畴(domain) 指的是:\pickout{C}
      \begin{abcd}
        \item 磁铁极间的区域
        \item 受磁场影响的磁铁周围空间
        \item 磁铁内部的区域,其中单个原子的磁极是对齐的
        \item 磁性材料开采的区域
      \end{abcd}
  \end{problem}


  \begin{problem}

    关于顺磁介质的磁导率下列说法正确的是:
    \pickout{C}
    \begin{abcd}
      \item 比真空的磁导率略小
      \item 比真空的磁导率略大
      \item 远小于真空的磁导率
      \item 远大于真空的磁导率
    \end{abcd}
\end{problem}

\begin{problem}

  用细导线均匀密绕成的长为l,半径为a(l>>a),总匝数为N的螺线管通以稳恒电流I,当管内充满磁导率为的均匀磁介质后,管中任意一点:\pickout{D}
  \begin{abcd}
    \item 磁感应强度大小为 $B=\mu_0 \mu_r N I$

    \item 磁感应强度大小为 $B=\mu_r N I / l $

    \item 磁场强度大小为 $H=\mu_0 N I / l$

    \item 磁场强度大小为 $H=N I / l$
  \end{abcd}
\end{problem}


\begin{problem}
  在稳恒磁场中, 有磁介质存在时的安培环路定理的积分形式是 : \pickout{B}
  \begin{abcd}
    \item  $\oint_L \vec{B} \cdot d \vec{l}=\sum_{\left(L_{\text {内 }}\right)} I$
    \item  $\oint_L \vec{H} \cdot d \vec{l}=\sum_{\left(L_{\text {内 }}\right)} I$
    \item  $\oint_L \vec{H} \cdot d \vec{l}=\mu_0 \sum_{\left(L_内\right)} I$
    \item  $\oint_L \vec{H} \cdot d \vec{l}=I_0+\iint_S \frac{\partial \vec{D}}{\partial t} \cdot d \vec{S}$
  \end{abcd}

\end{problem}


\begin{problem}
  半径为圆线圈置于磁感强度为的均匀磁场中,线圈平面与磁场方向垂直,线圈电阻为;当把线圈转动使其法向与夹角时,线圈中已通过电量与线圈面积及转动时间的关系是: 
: \pickout{A}
  \begin{abcd}
    \item  与线圈面积成正比,与时间无关; 
    \item  与线圈面积成正比,与时间成正比;
    \item  与线圈面积成反比,与时间成正比;
    \item  与线圈面积成反比,与时间无关;
  \end{abcd}

\end{problem}




\begin{problem}

    与霍耳效应(Hall Effect)的应用\textbf{不}符合的是:\pickout{A}
    \begin{abcd}
      \item 计算电子的电荷量
      \item 判断半导体的类型
      \item 测量磁场
      \item 测量载流子浓度
    \end{abcd}
\end{problem}


\begin{problem}

  矢量叉乘(cross product)遵循的规则是:\pickout{B}
  \begin{abcd}
    \item 左手螺旋规则
    \item 右手螺旋规则
    \item 反常左手螺旋规则
    \item 反常右手螺旋规则
  \end{abcd}
\end{problem}


    \begin{problem}

    杨氏双缝干涉实验中,光的干涉现象是由光的:\pickout{B}
    \begin{abcd}
      \item 散射性质
      \item 波动性质
      \item 反射性质
      \item 透射性质
    \end{abcd}
    \end{problem}



\begin{problem}
一束波长为 $\lambda$ 的单色平行光垂直照射到宽的 $a$ 的单琏 $\mathrm{AB}$ 上, 
    若屏上的 $\mathrm{P}$ 为第三级明纹,则单缝 $A B$ 边缘 $A 、 B$ 两处光线之间的光程差为: \pickout{4}
    \begin{abcd}
      \item $3 \lambda$
      \item $6 \lambda$
      \item $5 \lambda / 2$
      \item $7 \lambda / 2$
    \end{abcd}
\end{problem}
  
    \begin{problem}

      量子力学中,波函数描述了粒子的:\pickout{D}
      \begin{abcd}
        \item 质量
        \item 能量
        \item 速度
        \item 状态
      \end{abcd}
      \end{problem}

      \begin{problem}

        物质波的频率与粒子的动量之间的关系由哪个关系给出?\pickout{B}
        \begin{abcd}
          \item 康普顿效应
          \item 德布罗意关系
          \item 波尔理论
          \item 光电效应
        \end{abcd}
        \end{problem}

\begin{problem}

  质能方程 $E=mc^2$ 揭示了质量与能量之间的:\pickout{B}
  \begin{abcd}
    \item 相等关系
    \item 变换关系
    \item 成正比关系
    \item 无关系
  \end{abcd}
  \end{problem}

  \begin{problem}
    能量为5.0eV的光子入射到某金属表面,测得光电子的最大初动能是1.5eV,为了使该金属能产生光电效应,则入射光子的最低能量为  \pickout{C}                                               
        \begin{abcd}
          \item  1.5eV 
          \item  2.5eV  
          \item  3.5eV   
          \item  5.0eV
        \end{abcd}
        \end{problem}

\begin{problem}

    杨氏双缝实验中,设想用完全相同但偏振化方向相互垂直的偏振片各盖一缝,则屏幕上 :\pickout{D}
    \begin{abcd}
      \item 条纹形状不变,光强变小.
      \item 条纹形状不变,光强也不变.
      \item 条纹移动, 光强减弱.
      \item 看不见干涉条纹.
    \end{abcd}
\end{problem}


\begin{problem}
  一束平行入射面振动的线偏振光以起偏角入到某介质表面, 则反射光与折射光的偏振情况是 \pickout{D}
  \begin{abcd}
    \item 反射光与折射光都是平行入射面振动的线偏光.
    \item 反射光是垂直入射面振动的线偏光, 折射光是平行入射面振动的线偏光.
    \item 反射光是平行入射面振动的线偏光, 折射光是垂直入射面振动的线偏光.
    \item 折射光是平行入射面振动的线偏光, 看不见反射光.
  \end{abcd}
\end{problem}




\makepart{计算题}{共~40~分}%共~6~小题, 每小题~8~分, 共~48~分}

% \begin{problem}{(6分)}
%   证明
%   $$
%   |\sinh{y}| \leq |\sin(x+\imath y)| \leq |\cosh{y}|. 
%   $$
% \end{problem}

% \renewcommand{\solutionname}{证} % 将“解"字改为“证"字
% \begin{solution}
%   \everymath{\displaystyle}%
%   原式 \? $=\int\e^{2x}\, \sec^2 x\dx+2\int\e^{2x}\, \tan x\dx$ \score{2}
%   \+ $=\int\e^{2x}\, \d(\tan x)+ 2\int\e^{2x}\, \tan x\dx$ \score{4}
%   \+ $=\e^{2x}\, \tan x - 2\int\e^{2x}\, \tan x\dx+ 2\int\e^{2x}\, \tan x\dx$ \score{6}
%   \+ $=\e^{2x}\, \tan x + C$ \score{8}
% \end{solution}

% \begin{problem}{(6分)}
%   设 $\Psi(t,  x)=e^{ 2 t x-t^2 }$,  $t$是复变数,  证明: 
%   $$
%   \left. \frac{\partial^n \Psi(t,  x)}{\partial t^n}\right|_{t=0}=(-1)^n e^{x^2} \frac{d^n}{\dx^n} e^{-x^2}
%   $$
%   % 提示: 对回路积分进行积分变数的代换 $\xi=z-x$. 
% \end{problem}

\begin{problem}{(10分)}
  一螺绕环,横截面的半径为a,中心线的半径为R ,R $\gg$ a ,其上由表面绝缘的导线均匀地密绕两个线圈,一个$N_1$匝,另一个$N_2$匝。求两线圈的互感M。
\end{problem}
\vfill

\begin{solution}
% \everymath{\displaystyle}%
\? 假设 1 线圈通电流为 $I_1$. \\ 
\+ 则 $$B_1=\frac{\mu_0 N I_1}{2 \pi R} \score{4}$$  
\+ 根据互感的定义$$
M=M_{21}=\frac{\Psi_{21}}{I_1}=\frac{N_2 \frac{\mu_0 N_1 I}{2 \pi R} \pi a^2}{I_1}
$$
\+  最终结果为
$$ M =\frac{\mu_0 N_1 N_2}{2 R} a^2\score{6}$$ 
\end{solution}


% \begin{problem}{(6分)}
%   求 $|\sin z|$ 在闭区域 $0 \leq \operatorname{Re} z \leq 2 \pi,  0 \leq \Im z \leq 2 \pi$ 中的最大值. 
% \end{problem}
\begin{problem}{(10分)}
  电流I均匀地流过一根截面半径为R的长直铜导线。在导线内部取一平面S,一边为轴线,另一边在导线外壁上,长度为L,试求  
  \begin{enumerate}
    \item 磁感应强度分布
    \item 通过面S的磁通量
  \end{enumerate}
% \begin{wrapfigure}{c}{0.2\textwidth}
%   \vspace{-20pt}
%   \begin{center}
%     \includegraphics[width=0.2\textwidth]{Picture4.png}
%   \end{center}
%   \vspace{-20pt}
%   \caption{A gull}
%   \vspace{-10pt}
% \end{wrapfigure}

  % \begin{wrapfigure}{r}{0.2\textwidth}
  %   \begin{center}
    \begin{flushright}
      \includegraphics[width=0.2\textwidth]{Picture4.png}
    \end{flushright}
  %   \end{center}
  %   \caption{A gull}
  % \end{wrapfigure}

  % \begin{figure}



  % \includegraphics[]{Picture4.png}
  % \end{figure}
\end{problem}
\vfill

\begin{solution}
  函数$\bar{f}(p)$可以写成
  $$
  \bar{f}(p) = \frac{3}{2} \left[ \frac{1}{p+1} + \frac{1}{p-1}  \right] \score{3}
  $$
  由拉普拉斯变换的定义不难得到$$\mathcal{L}^{-1} \left[ \frac{1}{p-s} \right] = e^{st}, $$
  故得到
  $$\mathcal{L}^{-1} \left[ \frac{1}{p+1}\right] = e^{-t} $$
  和
  $$\mathcal{L}^{-1} \left[ \frac{1}{p-1}\right] = e^{+t} \score{5}  $$
  根据双曲余弦函数的定义, 最终我们有原函数
  $$f(t) =\mathcal{L}^{-1} \left[ \bar{f}(p) \right] = 3\cosh t.  \score{7}$$
\end{solution}
  
\begin{problem}{(10分)}
  如图所示, 一磁导率为 $\mu_1$ 的无限长磁介质圆柱体, 其半径为 $R_1$, 其中通有电流 $I$, 且电流沿横截面均匀分布。
  在该磁介质圆柱的外面有一半径为 $R_2$ 的无限长同轴圆柱面,二者之间充满磁导率为 $\mu_2$ 的均匀磁介质, 圆柱面外为真空。在圆柱面上通有相反方向的电流 $I$ 。
  试求:
 \begin{enumerate}[label=(\roman*) ]
    \item  圆柱体内 $\left(r<R_1\right)$ 任一点的磁感应强度 $\boldsymbol{B}$;
    \item 圆柱体外与圆柱面内 $\left(R_1<r<R_2\right)$ 任一点的磁感应强度 $\boldsymbol{B}$;
    \item 圆柱面外 $\left(r>R_2\right)$ 任一点的磁感应强度 $\boldsymbol{B}$ 。
 \end{enumerate}
  \begin{flushright}
    \includegraphics[width=0.15\textwidth]{Picture5.png}
  \end{flushright}
\end{problem}
  \vfill
  

\begin{solution}
  % \everymath{\displaystyle}%
  \? 函数$f(z)$可以写成
  $$
  \frac{1}{(z-2)(z-3)}=\frac{1}{(z-3)}-\frac{1}{(z-2)} \score{2}
  $$
  \+由 $|z|>3$ 得到 $\left|\frac{3}{z}\right|<1, \left|\frac{2}{z}\right|<1$,  \score{3}
  \+ $$\frac{1}{z-3} = \frac{1}{z} \frac{1}{1-\frac{3}{z}} = \frac{1}{z} \sum_{n=0} \left(\frac{3}{z}\right)^n,  \score{5} $$
  \+类似的可以得到 $$\frac{1}{z-2} = \frac{1}{z} \frac{1}{1-\frac{2}{z}} = \frac{1}{z} \sum_{n=0} \left(\frac{2}{z}\right)^n,  \score{7}$$
  \+ 最终我们有 
  $$f(z) = \sum_{n=0}^{\infty}\left(3^n-2^n\right) z^{-n-1}.  \score{8}$$
\end{solution}
  
\begin{problem}{(9分)}
    用$\Gamma$函数求积分
    $$
    \int_0^\infty x^{ -\frac{1}{2} } e^{-x \cos{\theta}} \cos\left( x \sin{\theta} \right) dx, 
    $$
    % 其中$\alpha > 0,  -\frac{\pi}{2} < \theta < \frac{\pi}{2}$. 
    其中$ -\frac{\pi}{2} < \theta < \frac{\pi}{2}$. 
\end{problem} 
  
\vfill

\begin{solution}
  对一般的$\alpha$, 依习题解答,\\
  $I\?=\int_0^{\infty} x^{\alpha-1} e^{-x \cos \theta} \cos (x \sin \theta) \dx$ \par
    \+ $=\int_0^{\infty} x^{\alpha-1} e^{-x \cos \theta} \operatorname{Re}\left(e^{i x \sin \theta}\right) \dx $ \par
    \+ $=\operatorname{Re} \int_0^{\infty} x^{\alpha-1} e^{-x(\cos \theta-i \sin \theta)} \dx$ \par
  \+ $=\operatorname{Re} \int_0^{\infty} x^{\alpha-1} e^{-x e^{-i \theta}} \dx .$ \score{3}
  作变量代换, 令 $x e^{-i \theta}=y$ $\Rightarrow$ $x=e^{i \theta} y,  \dx=e^{i \theta} \dy$  \score{5}\newline
  于是,  $I \?=\operatorname{Re} \int_0^{\infty}\left(e^{i \theta} y\right)^{\alpha-1} e^{-y} e^{i \theta} \dy $ \score{6}\par
        \+ $ =\operatorname{Re}\left(e^{i \theta}\right)^\alpha \int_0^{\infty} y^{\alpha-1} e^{-y} \dy $ \par
        \+ $ =\operatorname{Re}\left(e^{i \alpha \theta}\right) \Gamma(\alpha)$ \par
        \+ $ = \cos {(\theta \alpha)} \Gamma(\alpha).  $\score{8}
  将$\alpha = \frac{1}{2}$ 代入得
  $$I = \sqrt{\pi} \cos{(\frac{\theta}{2})}.\score{9}$$
\end{solution}



% \begin{problem}
% 计算矩形波$$
% \begin{aligned}
% & f(x)=0,  \quad-\pi<x<0,  \\
% & f(x)=h,  \quad 0<x<\pi . 
% \end{aligned}
% $$
% 的傅里叶级数展开. 
% \end{problem}






% \begin{problem}{(8分)}
%   计算积分 $$\int_0^{2 \pi} \frac{\dx}{(a+b \cos x)^2},  \quad a>b>0$$. 
% \end{problem}

% \begin{solution}
 
%   $ I\?=\int_0^{\infty} x^{\alpha-1} e^{-x \cos \theta} \cos (x \sin \theta) \dx$ \par
%      \+ $=\int_0^{\infty} x^{\alpha-1} e^{-x \cos \theta} \operatorname{Re}\left(e^{i x \sin \theta}\right) \dx $ \par
%      \+ $=\operatorname{Re} \int_0^{\infty} x^{\alpha-1} e^{-x(\cos \theta-i \sin \theta)} \dx$ \par
%     \+ $=\operatorname{Re} \int_0^{\infty} x^{\alpha-1} e^{-x e^{-i \theta}} \dx$ \score{3}
%    令 $x e^{-i \theta}=y$ 做变量代换$\Rightarrow$ $x=e^{i \theta} y,  \dx=e^{i \theta} \dy$  \score{5}\newline
%    于是,  $I \?=\operatorname{Re} \int_0^{\infty}\left(e^{i \theta} y\right)^{\alpha-1} e^{-y} e^{i \theta} \dy $ \par
%          \+ $ =\operatorname{Re}\left(e^{i \theta}\right)^\alpha \int_0^{\infty} y^{\alpha-1} e^{-y} \dy $ \par
%          \+ $ =\operatorname{Re}\left(e^{i \alpha \theta}\right) \Gamma(\alpha)$ \par
%          \+ $ = \cos {(\theta \alpha)} \Gamma(\alpha)  $\score{8}
% \end{solution}
% \vfill

% \begin{problem}{(8分)}
%   将下面的偏微分方程化成标准形式, 
%    $$
%   %  u_{x x}+4 u_{x y}+5 u_{y y}+u_x+2 u_y=0. 
%    u_{x x}+4 u_{x y}+5 u_{y y}=0. 
%    $$

%  \end{problem}



\begin{problem}{(10分)}
  求解下列定解问题: 
  $\left\{\begin{array}{l}
    \nabla^2 u=0,  \quad 0\leq r \leq a \\ 
    \left. u\right|_{r=a}=u_0 \cos^2 \theta,  \\
    \left. u\right|_{r=0} \text{有限}. 
    %\left. u\right|_{r=b}=u_0 \cos ^2 \theta. 
  \end{array}\right. $
\end{problem} 
\vfill


\begin{solution}
由边界条件可知该问题为轴对称情况
通解形式为
$$  u(r,  \theta) = \sum_{l=0}^{\infty} \left( A_l r^l + \frac{B_l}{r^{l+1}} \right) P_{l} (\cos \theta). 
$$
由$r=0$处有限的边界条件可知$B_l=0$.  \score{2}\\
因此, 对$r=a$处的边界条件带入, 得
$$
\sum_{l=0}^{\infty}  A_l r^l  P_{l} (\cos \theta) 
= u_0 \cos^2 \theta = u_0 x^2 \score{4} 
$$ 
勒让德多项式中$$P_0(x) = 1,  P_1(x) = x,  P_2 (x) = \frac{1}{2}(3x^2 - 1)$$ \\
不难得到 $$x^2 = \frac{2}{3} P_2(x) + \frac{1}{3} P_0(x), \score{6}$$  \\
比较两边系数, 得
$$A_0 = \frac{u_0}{3}, \\
 A_2 = \frac{2 u_0}{3a^2}, \\
  A_l = 0 (l\neq 0,  2).  \score{8} $$ 
这样, 
$$u(r, \theta) = \frac{u_0}{3} + \frac{2 u_0}{3a^2} r^2 P_2(\cos \theta).  \score{10} $$ 
\end{solution}

\vfill
\newpage
\makedata{可能用到的数据和公式} %附录数据
柯西公式(Cauchy's formula)
\[
  f^{(n)}(z) = \frac{n!}{2\pi \imath} \oint_C \frac{f(\zeta)}{(\zeta - z)^{n+1}} d \zeta. 
  \label{eq:cauchy_formula_nth_derivative}
\]

\bigskip
函数$f(t)$的拉普拉斯变换$\bar{f}(p)$为
\[
    \bar{f}(p) = \mathcal{L} \{ f(t) \} = \int_0 ^{\infty} e^{-pt} f(t) \dt . 
\]

\bigskip

双曲正余函数(hyperbolic sine/cosine function)表达为
\begin{align*}
    \cosh z&= \frac{e^{z} + e^{ - z} }{2} , 
    \\
    \sinh z &= \frac{e^{z} - e^{ - z} }{2} . 
\end{align*}
% \bigskip

伽玛函数(Gamma function)
\[
  \Gamma(z) \equiv \int_{0}^{\infty} e^{-t} t^{z-1} d t,  \quad \operatorname{Re} z>0 . 
\]
\bigskip

贝塔函数(Beta function)
\[
    B(p,  q) = \int_0^1 t^{p -1} (1-t)^{q-1} \dt,  \operatorname{Re} p> 0,  \operatorname{Re} q >0. 
    \label{eq:beta_def1}
\]
\bigskip

球坐标系下的拉普拉斯算符
\[
 \Delta = \frac{1}{r^2} \frac{\partial}{\partial r} \left( r^2 \frac{\partial }{\partial r} \right)
  + \frac{1}{r^2\sin \theta} \frac{\partial}{\partial \theta} \left( \sin\theta \frac{\partial}{\partial \theta} \right)
  + \frac{1}{r^2\sin^2 \theta} \frac{\partial^2}{\partial^2 \varphi} .
\]
分离变量得到轴对称($m=0$)
情况下解的形式为
\[
  u(r,  \theta) = \sum_{l=0}^{\infty} \left( A_l r^l + \frac{B_l}{r^{l+1}} \right) P_{l} (\cos \theta)
\]

\bigskip

$l$ 阶勒让德多项式(Legendre polynomial)的罗德里格斯表达式(Rodrigues's formula) 
$$P_l(x)=\frac{1}{2^l l !} \frac{d^l}{d x^l}\left(x^2-1\right)^l$$
\end{document}
