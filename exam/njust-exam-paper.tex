% -*- coding: utf-8 -*-
% !TEX program = xelatex
\documentclass{njustexam}

%注意:注释掉这条命令则显示答案,为存档用,交教务处。使用这条命令则为学生考试用,交印刷厂印刷
%\answerfalse 

\begin{document}
%%%%%%%%%%%%%%%%% 以下内容根据需要更改%%%%%%%%%%%
\renewcommand{\course}{我是课程名}                          %课程名
\renewcommand{\duration}{100}                                            %考试时长
\renewcommand{\credit}{3}                                                   %学分
\renewcommand{\syllabus}{12345678-9}                               %教学大纲编号
\renewcommand{\fullmark}{100}                                            %满分分值
\renewcommand{\composer}{张三、李四、王五}            %组卷教师
\renewcommand{\composedate}{2021年4月28日}                   %组卷日期
\renewcommand{\validator}{赵六}                                        %审定人
\renewcommand{\coursetype}{1}                                            % 1 为必修,0 为选修
\renewcommand{\exammethod}{1}                                         % 1 为闭卷,0 为开卷
\renewcommand{\testpaper}{A}                                              % A 或 B
%%%%%%%%%%%%%%%%%%%%%%%%%%%%%%%%%%%%%%

\makehead % 生成试卷表头

\makepart{填空题}{共~6~小题,每小题~3~分,共~18~分}

\begin{problem}
设常数$k>0$,函数$f(x)=\ln x-\dfrac{x}{\e}+k$在$(0,+\infty)$内零点的个数为 \fillout{$2$}.
\end{problem}


\begin{problem}
设$\va=(2,1,2)$,$\vb=(4,-1,10)$,$\vc=\vb-\lambda\va$,且$\va\bot\vc$,则$\lambda=$ \fillout{$3$}.
\end{problem}


\begin{problem}
已知二阶行列式 $\left|\begin{array}{cc}
  1 & 2\\
  - 3 & x
\end{array}\right|=0$,则 $x=$ \fillout{$-6$}.
\end{problem}


\begin{problem}
向量组 $\alpha_1=(1,1,0), \alpha_2=(0,1,1), \alpha_3=(1,0,1)$,
则将向量 $\beta=(4, 5, 3)$ 表示为 $\alpha_1, \alpha_2, \alpha_3$
的线性组合为 $\beta=$ \fillout{$3\alpha_1+2\alpha_2+\alpha_3$}.
\end{problem}


\begin{problem}
已知随机变量$\xi$的期望和方差各为$E\xi=3, D\xi=2$, 则$E\xi^2=$ \fillout{$11$}.
\end{problem}


\begin{problem}
已知$\xi$和$\eta$相互独立且$\xi\sim N(1,4), \eta\sim N(2,5)$,则$\xi-2\eta\sim$ \fillout{$N(-3,24)$}.
\end{problem}


\makepart{单选题}{共~6~小题,每小题~3~分,共~18~分}

\begin{problem}
在下列等式中,正确的结果是\pickout{C}
\begin{abcd}
\item $\int f'(x)\dx=f(x)$
\item $\int \d f(x)=f(x)$
\item $\frac{\d}{\dx}\big(\int f(x)\dx\big)=f(x)$
\item $\d\big(\int f(x)\dx\big)=f(x)$
\end{abcd}
\end{problem}

\bigskip

\begin{problem}
假设$F(x)$是连续函数$f(x)$的一个原函数,则必有\pickout{A}
\begin{abcd}
\item $F(x)$是偶函数 $\Leftrightarrow$ $f(x)$是奇函数
\item $F(x)$是奇函数 $\Leftrightarrow$ $f(x)$是偶函数
\item $F(x)$是周期函数 $\Leftrightarrow$ $f(x)$是周期函数
\item $F(x)$是单调函数 $\Leftrightarrow$ $f(x)$是单调函数
\end{abcd}
\end{problem}

\bigskip

\begin{problem}
设矩阵 $A = \left(\begin{array}{ccc}
  1 & 1 & 0\\
  1 & x & 0\\
  0 & 0 & 1
\end{array}\right)$ 其中两个特征值为 $\lambda_1 = 1$ 和 $\lambda_2
= 2$,则 $x=$ \pickout{B}
\begin{abcd}
\item $2$
\item $1$
\item $0$
\item $-1$
\end{abcd}
\end{problem}

\bigskip

\begin{problem}
二次型 $f = 4 x_1^2 - 2 x_1 x_2 + 6 x_2^2$ 对应的矩阵等于 \pickout{C}
\begin{abcd}
\item $\left(\begin{array}{cc}
  4 & - 2\\
  - 2 & 6
\end{array}\right)$
\item $\left(\begin{array}{cc}
  2 & - 2\\
  - 2 & 3
\end{array}\right)$
\item $\left(\begin{array}{cc}
  4 & - 1\\
  - 1 & 6
\end{array}\right)$
\item $\left(\begin{array}{cc}
  2 & - 1\\
  - 1 & 3
\end{array}\right)$
\end{abcd}
\end{problem}

\bigskip

\begin{problem}
下列说法\CJKunderline{不正确}的是\pickout{B}
\begin{abcd}
\item 大数定律说明了大量相互独立且同分布的随机变量的均值的稳定性
\item 大数定律说明大量相互独立且同分布的随机变量的均值近似于正态分布
\item 中心极限定理说明了大量相互独立且同分布的随机变量的和的稳定性
\item 中心极限定理说明大量相互独立且同分布的随机变量的和近似于正态分布
\end{abcd}
\end{problem}

\bigskip

\begin{problem}
对总体$X$和样本$(X_1,\cdots,X_n)$的说法哪个是\CJKunderline{不正确}的\pickout{D}
\begin{abcd}
\item 总体是随机变量
\item 样本是$n$元随机变量
\item $X_1, \cdots, X_n$相互独立
\item $X_1 = X_2 =\cdots = X_n$
\end{abcd}
\end{problem}


\makepart{计算题}{共~6~小题,每小题~8~分,共~48~分}

\begin{problem}
求不定积分$\displaystyle\int\e^{2x}\,(\tan x+1)^2\dx$。
\end{problem}

\smallskip

\begin{solution}
\everymath{\displaystyle}%
原式 \? $=\int\e^{2x}\,\sec^2 x\dx+2\int\e^{2x}\,\tan x\dx$ \score{2}
\+ $=\int\e^{2x}\,\d(\tan x)+ 2\int\e^{2x}\,\tan x\dx$ \score{4}
\+ $=\e^{2x}\,\tan x - 2\int\e^{2x}\,\tan x\dx+ 2\int\e^{2x}\,\tan x\dx$ \score{6}
\+ $=\e^{2x}\,\tan x + C$ \score{8}
\end{solution}

\bigskip

\begin{problem}
求过点$A(1,2,-1), B(2,3,0),C(3,3,2)$ 的三角形$\triangle ABC$ 的面积和它们确定的平面方程.
\end{problem}

\smallskip

\begin{solution}
由题设$\overrightarrow{AB}=(1,1,1),\overrightarrow{AC}=(2,1,3)$, \score{2}
故$\overrightarrow{AB}\times \overrightarrow{AC}=\begin{vmatrix}
\vec{i}&\vec{j} &\vec{k}\\
1&1&1\\
2&1&3\\
\end{vmatrix}=(2,-1,-1)$, \score{4}
三角形$\triangle ABC$ 的面积为$S_{\triangle ABC}=\dfrac{1}{2}\big|\overrightarrow{AB}\times
\overrightarrow{AC}\big|=\dfrac{1}{2}\sqrt{6}.$ \score{6}
所求平面的方程为$2(x-2)-(y-3)-z=0$, 即$2x-y-z-1=0$ \score{8}			
\end{solution}

\bigskip

\begin{problem}
计算四阶行列式 $A = \left|\begin{array}{cccc}
  0 & 1 & 2 & 3\\
  1 & 2 & 3 & 0\\
  2 & 3 & 0 & 1\\
  3 & 0 & 1 & 2
\end{array}\right|$ 的值.
\end{problem}

\smallskip

\begin{solution}
$A \? = \left|\begin{array}{cccc}
    0 & 1 & 2 & 3\\
    1 & 2 & 3 & 0\\
    2 & 3 & 0 & 1\\
    3 & 0 & 1 & 2
  \end{array}\right| = \left|\begin{array}{cccc}
    0 & 1 & 2 & 3\\
    1 & 2 & 3 & 0\\
    0 & - 1 & - 6 & 1\\
    0 & - 6 & - 8 & 2
  \end{array}\right| = 1 \cdot (- 1)^{2 + 1} \left|\begin{array}{ccc}
    1 & 2 & 3\\
    - 1 & - 6 & 1\\
    - 6 & - 8 & 2
  \end{array}\right|$ \score{4}
\+ $= -\left|\begin{array}{ccc}
    1 & 2 & 3\\
    0 & - 4 & 4\\
    0 & 4 & 20
  \end{array}\right| = - \left|\begin{array}{cc}
    - 4 & 4\\
    4 & 20
  \end{array}\right| = -(-4\cdot20-4\cdot4) = 96$ \score{8}
\end{solution}

\bigskip

\begin{problem}
利用配方法,将二次型 $f = x_1^2 + 2 x_1 x_2 - 6 x_1 x_3 + 2 x_2^2 - 12
x_2 x_3 + 9 x^2_3$ 化为标准形 $f = d_1 y^2_1 + d_2 y^2_2 + d_3 y^2_3$ .
\end{problem}

\smallskip

\begin{solution}
$f \? = x_1^2 + 2 x_1 x_2 - 6 x_1 x_3 + 2 x_2^2 - 12 x_2 x_3 + 9 x^2_3$ \par
  \+ $= x_1^2 + 2 x_1 (x_2 - 3 x_3) + (x_2 - 3 x_3)^2 + x_2^2 - 6 x_2 x_3 $ \par
  \+ $= (x_1 + x_2 - 3 x_3)^2 + x_2^2 - 6 x_2 x_3$ \score{3}
  \+ $= (x_1 + x_2 - 3 x_3)^2 + x_2^2 - 2 x_2 \cdot 3 x_3 + (3 x_3)^2 - 9x_3^2$ \par
  \+ $= (x_1 + x_2 - 3 x_3)^2 + (x_2 - 3 x_3)^2 - 9 x_3^2$ \score{6}
令$y_1 = x_1 + x_2 - 3 x_3, y_2 = x_2 - 3 x_3, y_3 = x_3$, \newline
则$f = y_1^2 + y_2^2 - 9y_3^2$为标准形.\score{8}
\end{solution}

\bigskip

\begin{problem}
设每发炮弹命中飞机的概率是0.2且相互独立,现在发射100发炮弹.\par
(1) 用切贝谢夫不等式估计命中数目$\xi$在10发到30发之间的概率.\par
(2) 用中心极限定理估计命中数目$\xi$在10发到30发之间的概率.
\end{problem}

\smallskip

\begin{solution}
$E\xi = n p = 100 \cdot 0.2 = 20, D\xi = n p q = 100 \cdot 0.2 \cdot 0.8 = 16$. \score{2}
(1) $P (10 < \xi < 30) = P (|\xi - E\xi| < 10) \ge 1 - \frac{D\xi}{10^2}
     = 1 - \frac{16}{100} = 0.84$. \score{4}
(2) $P (10 < \xi < 30) \? \approx \Phi_0\left(\frac{30 - 20}{\sqrt{16}}\right)
         - \Phi_0\left(\frac{10 - 20}{\sqrt{16}}\right)$ \score{6}
      \+ $= 2 \Phi_0(2.5) - 1 = 2 \cdot 0.9938 - 1 =0.9876$ \score{8}
\end{solution}

\bigskip

\begin{problem}
从正态总体$N(\mu,\sigma^2)$中抽出样本容量为16的样本,算得其平均数为3160,标准差为100.
试检验假设$H_0:\mu=3140$是否成立($\alpha = 0.01$).
\end{problem}

\smallskip

\begin{solution}
(1) 待检假设 $H_0 : \mu = 3140$. \score{1}
(2) 选取统计量 $T = \frac{\widebar{X}-\mu}{S / \sqrt{n}} \sim t(n-1)$. \score{3}
(3) 查表得到 $t_{\alpha} = t_{\alpha} (n - 1) = t_{0.01} (15) =2.947$. \score{5}
(4) 计算统计值 $t = \frac{\widebar{x} - \mu_0}{s/\sqrt{n}} =\frac{3160-3140}{100/4} = 0.8$.\score{7}
(5) 由于 $| t | < t_{\alpha}$, 故接受 $H_0$, 即假设成立. \score{8}
\end{solution}

\bigskip


\makepart{证明题}{共~2~小题,每小题~8~分,共~16~分}

\renewcommand{\solutionname}{证} % 将“解”字改为“证”字

\begin{problem}
设数列$\{x_n\}$满足$x_1=\sqrt2$,$x_{n+1}=\sqrt{2+x_n}$.证明数列收敛,并求出极限.
\end{problem}

\smallskip

\begin{solution}
(1) 事实上,由于$x_1<2$,且$x_k<2$时
$$x_{k+1}=\sqrt{2+x_k}<\sqrt{2+2}=2,$$
由数学归纳法知对所有$n$都有$x_n<2$,即数列有上界.
又由于
$$\frac{x_{n+1}}{x_n}=\sqrt{\frac{2}{x_n^2}+\frac{1}{x_n}}>\sqrt{\frac{2}{2^2}+\frac{1}{2}}=1,$$
所以数列单调增加.由极限存在准则II,数列必定收敛.\score{4}
(2) 设数列的极限为$A$,对递推公式两边同时取极限得到
$$A=\sqrt{2+A}.$$
解得$A=2$,即数列$\{x_n\}$的极限为$2$.\score{8}
\end{solution}

\bigskip

\begin{problem}
设数列$\{x_n\}$满足$x_1=\sqrt2$,$x_{n+1}=\sqrt{2+x_n}$.证明数列收敛,并求出极限.
\end{problem}

\smallskip

\begin{solution}
(1) 事实上,由于$x_1<2$,且$x_k<2$时
$$x_{k+1}=\sqrt{2+x_k}<\sqrt{2+2}=2,$$
由数学归纳法知对所有$n$都有$x_n<2$,即数列有上界.
又由于
$$\frac{x_{n+1}}{x_n}=\sqrt{\frac{2}{x_n^2}+\frac{1}{x_n}}>\sqrt{\frac{2}{2^2}+\frac{1}{2}}=1,$$
所以数列单调增加.由极限存在准则II,数列必定收敛.\score{4}
(2) 设数列的极限为$A$,对递推公式两边同时取极限得到
$$A=\sqrt{2+A}.$$
解得$A=2$,即数列$\{x_n\}$的极限为$2$.\score{8}
\end{solution}

\bigskip

\begin{problem}
设数列$\{x_n\}$满足$x_1=\sqrt2$,$x_{n+1}=\sqrt{2+x_n}$.证明数列收敛,并求出极限.
\end{problem}

\smallskip

\begin{solution}
(1) 事实上,由于$x_1<2$,且$x_k<2$时
$$x_{k+1}=\sqrt{2+x_k}<\sqrt{2+2}=2,$$
由数学归纳法知对所有$n$都有$x_n<2$,即数列有上界.
又由于
$$\frac{x_{n+1}}{x_n}=\sqrt{\frac{2}{x_n^2}+\frac{1}{x_n}}>\sqrt{\frac{2}{2^2}+\frac{1}{2}}=1,$$
所以数列单调增加.由极限存在准则II,数列必定收敛.\score{4}
(2) 设数列的极限为$A$,对递推公式两边同时取极限得到
$$A=\sqrt{2+A}.$$
解得$A=2$,即数列$\{x_n\}$的极限为$2$.\score{8}
\end{solution}

\bigskip

\begin{problem}
设数列$\{x_n\}$满足$x_1=\sqrt2$,$x_{n+1}=\sqrt{2+x_n}$.证明数列收敛,并求出极限.
\end{problem}

\smallskip

\begin{solution}
(1) 事实上,由于$x_1<2$,且$x_k<2$时
$$x_{k+1}=\sqrt{2+x_k}<\sqrt{2+2}=2,$$
由数学归纳法知对所有$n$都有$x_n<2$,即数列有上界.
又由于
$$\frac{x_{n+1}}{x_n}=\sqrt{\frac{2}{x_n^2}+\frac{1}{x_n}}>\sqrt{\frac{2}{2^2}+\frac{1}{2}}=1,$$
所以数列单调增加.由极限存在准则II,数列必定收敛.\score{4}
(2) 设数列的极限为$A$,对递推公式两边同时取极限得到
$$A=\sqrt{2+A}.$$
解得$A=2$,即数列$\{x_n\}$的极限为$2$.\score{8}
\end{solution}

\bigskip

\begin{problem}
设数列$\{x_n\}$满足$x_1=\sqrt2$,$x_{n+1}=\sqrt{2+x_n}$.证明数列收敛,并求出极限.
\end{problem}

\smallskip

\begin{solution}
(1) 事实上,由于$x_1<2$,且$x_k<2$时
$$x_{k+1}=\sqrt{2+x_k}<\sqrt{2+2}=2,$$
由数学归纳法知对所有$n$都有$x_n<2$,即数列有上界.
又由于
$$\frac{x_{n+1}}{x_n}=\sqrt{\frac{2}{x_n^2}+\frac{1}{x_n}}>\sqrt{\frac{2}{2^2}+\frac{1}{2}}=1,$$
所以数列单调增加.由极限存在准则II,数列必定收敛.\score{4}
(2) 设数列的极限为$A$,对递推公式两边同时取极限得到
$$A=\sqrt{2+A}.$$
解得$A=2$,即数列$\{x_n\}$的极限为$2$.\score{8}
\end{solution}

\bigskip

\begin{problem}
设数列$\{x_n\}$满足$x_1=\sqrt2$,$x_{n+1}=\sqrt{2+x_n}$.证明数列收敛,并求出极限.
\end{problem}

\smallskip

\begin{solution}
(1) 事实上,由于$x_1<2$,且$x_k<2$时
$$x_{k+1}=\sqrt{2+x_k}<\sqrt{2+2}=2,$$
由数学归纳法知对所有$n$都有$x_n<2$,即数列有上界.
又由于
$$\frac{x_{n+1}}{x_n}=\sqrt{\frac{2}{x_n^2}+\frac{1}{x_n}}>\sqrt{\frac{2}{2^2}+\frac{1}{2}}=1,$$
所以数列单调增加.由极限存在准则II,数列必定收敛.\score{4}
(2) 设数列的极限为$A$,对递推公式两边同时取极限得到
$$A=\sqrt{2+A}.$$
解得$A=2$,即数列$\{x_n\}$的极限为$2$.\score{8}
\end{solution}

\bigskip

\begin{problem}
设数列$\{x_n\}$满足$x_1=\sqrt2$,$x_{n+1}=\sqrt{2+x_n}$.证明数列收敛,并求出极限.
\end{problem}

\smallskip

\begin{solution}
(1) 事实上,由于$x_1<2$,且$x_k<2$时
$$x_{k+1}=\sqrt{2+x_k}<\sqrt{2+2}=2,$$
由数学归纳法知对所有$n$都有$x_n<2$,即数列有上界.
又由于
$$\frac{x_{n+1}}{x_n}=\sqrt{\frac{2}{x_n^2}+\frac{1}{x_n}}>\sqrt{\frac{2}{2^2}+\frac{1}{2}}=1,$$
所以数列单调增加.由极限存在准则II,数列必定收敛.\score{4}
(2) 设数列的极限为$A$,对递推公式两边同时取极限得到
$$A=\sqrt{2+A}.$$
解得$A=2$,即数列$\{x_n\}$的极限为$2$.\score{8}
\end{solution}

\bigskip

\begin{problem}
设数列$\{x_n\}$满足$x_1=\sqrt2$,$x_{n+1}=\sqrt{2+x_n}$.证明数列收敛,并求出极限.
\end{problem}

\smallskip

\begin{solution}
(1) 事实上,由于$x_1<2$,且$x_k<2$时
$$x_{k+1}=\sqrt{2+x_k}<\sqrt{2+2}=2,$$
由数学归纳法知对所有$n$都有$x_n<2$,即数列有上界.
又由于
$$\frac{x_{n+1}}{x_n}=\sqrt{\frac{2}{x_n^2}+\frac{1}{x_n}}>\sqrt{\frac{2}{2^2}+\frac{1}{2}}=1,$$
所以数列单调增加.由极限存在准则II,数列必定收敛.\score{4}
(2) 设数列的极限为$A$,对递推公式两边同时取极限得到
$$A=\sqrt{2+A}.$$
解得$A=2$,即数列$\{x_n\}$的极限为$2$.\score{8}
\end{solution}

\bigskip

\begin{problem}
设数列$\{x_n\}$满足$x_1=\sqrt2$,$x_{n+1}=\sqrt{2+x_n}$.证明数列收敛,并求出极限.
\end{problem}

\smallskip

\begin{solution}
(1) 事实上,由于$x_1<2$,且$x_k<2$时
$$x_{k+1}=\sqrt{2+x_k}<\sqrt{2+2}=2,$$
由数学归纳法知对所有$n$都有$x_n<2$,即数列有上界.
又由于
$$\frac{x_{n+1}}{x_n}=\sqrt{\frac{2}{x_n^2}+\frac{1}{x_n}}>\sqrt{\frac{2}{2^2}+\frac{1}{2}}=1,$$
所以数列单调增加.由极限存在准则II,数列必定收敛.\score{4}
(2) 设数列的极限为$A$,对递推公式两边同时取极限得到
$$A=\sqrt{2+A}.$$
解得$A=2$,即数列$\{x_n\}$的极限为$2$.\score{8}
\end{solution}

\bigskip

\begin{problem}
设事件$A$和$B$相互独立,证明$A$和$\widebar{B}$相互独立.
\end{problem}

\smallskip

\begin{solution}
\? $P (A \cdot \widebar{B}) = P (A - B) = P (A - A B)$ \score{2}
\< $= P (A) - P (A B) = P (A) - P (A) P (B)$ \score{4}
\< $= P (A) (1 - P (B)) = P (A) P (\widebar{B})$ \score{6}
所以$A$和$\widebar{B}$相互独立.\score{8}
\end{solution}

\bigskip

\makedata{一些可能用到的数据} %附录数据

\begin{tabularx}{\linewidth}{*{4}{>{$}X<{$}}}
\hline
\Phi_0(0.5)=0.6915 & \Phi_0(1)=0.8413 & \Phi_0(2)=0.9773 & \Phi_0(2.5)=0.9938 \\
t_{0.01}(8)=3.355 & t_{0.01}(9)=3.250 & t_{0.01}(15)=2.947 & t_{0.01}(16)=2.921 \\
\chi_{0.005}^2(8)=22.0 & \chi_{0.005}^2(9)=23.6 & \chi_{0.005}^2(15)=32.8 & \chi_{0.005}^2(16)=34.3 \\
\chi_{0.995}^2(8)=1.34 & \chi_{0.995}^2(9)=1.73 & \chi_{0.995}^2(15)=4.60 & \chi_{0.995}^2(16)=5.14 \\
\hline
\end{tabularx}

\end{document}
